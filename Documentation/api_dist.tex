% Options for packages loaded elsewhere
% Options for packages loaded elsewhere
\PassOptionsToPackage{unicode}{hyperref}
\PassOptionsToPackage{hyphens}{url}
\PassOptionsToPackage{dvipsnames,svgnames,x11names}{xcolor}
%
\documentclass[
  letterpaper,
  DIV=11,
  numbers=noendperiod]{scrartcl}
\usepackage{xcolor}
\usepackage{amsmath,amssymb}
\setcounter{secnumdepth}{-\maxdimen} % remove section numbering
\usepackage{iftex}
\ifPDFTeX
  \usepackage[T1]{fontenc}
  \usepackage[utf8]{inputenc}
  \usepackage{textcomp} % provide euro and other symbols
\else % if luatex or xetex
  \usepackage{unicode-math} % this also loads fontspec
  \defaultfontfeatures{Scale=MatchLowercase}
  \defaultfontfeatures[\rmfamily]{Ligatures=TeX,Scale=1}
\fi
\usepackage{lmodern}
\ifPDFTeX\else
  % xetex/luatex font selection
\fi
% Use upquote if available, for straight quotes in verbatim environments
\IfFileExists{upquote.sty}{\usepackage{upquote}}{}
\IfFileExists{microtype.sty}{% use microtype if available
  \usepackage[]{microtype}
  \UseMicrotypeSet[protrusion]{basicmath} % disable protrusion for tt fonts
}{}
\makeatletter
\@ifundefined{KOMAClassName}{% if non-KOMA class
  \IfFileExists{parskip.sty}{%
    \usepackage{parskip}
  }{% else
    \setlength{\parindent}{0pt}
    \setlength{\parskip}{6pt plus 2pt minus 1pt}}
}{% if KOMA class
  \KOMAoptions{parskip=half}}
\makeatother
% Make \paragraph and \subparagraph free-standing
\makeatletter
\ifx\paragraph\undefined\else
  \let\oldparagraph\paragraph
  \renewcommand{\paragraph}{
    \@ifstar
      \xxxParagraphStar
      \xxxParagraphNoStar
  }
  \newcommand{\xxxParagraphStar}[1]{\oldparagraph*{#1}\mbox{}}
  \newcommand{\xxxParagraphNoStar}[1]{\oldparagraph{#1}\mbox{}}
\fi
\ifx\subparagraph\undefined\else
  \let\oldsubparagraph\subparagraph
  \renewcommand{\subparagraph}{
    \@ifstar
      \xxxSubParagraphStar
      \xxxSubParagraphNoStar
  }
  \newcommand{\xxxSubParagraphStar}[1]{\oldsubparagraph*{#1}\mbox{}}
  \newcommand{\xxxSubParagraphNoStar}[1]{\oldsubparagraph{#1}\mbox{}}
\fi
\makeatother

\usepackage{color}
\usepackage{fancyvrb}
\newcommand{\VerbBar}{|}
\newcommand{\VERB}{\Verb[commandchars=\\\{\}]}
\DefineVerbatimEnvironment{Highlighting}{Verbatim}{commandchars=\\\{\}}
% Add ',fontsize=\small' for more characters per line
\usepackage{framed}
\definecolor{shadecolor}{RGB}{241,243,245}
\newenvironment{Shaded}{\begin{snugshade}}{\end{snugshade}}
\newcommand{\AlertTok}[1]{\textcolor[rgb]{0.68,0.00,0.00}{#1}}
\newcommand{\AnnotationTok}[1]{\textcolor[rgb]{0.37,0.37,0.37}{#1}}
\newcommand{\AttributeTok}[1]{\textcolor[rgb]{0.40,0.45,0.13}{#1}}
\newcommand{\BaseNTok}[1]{\textcolor[rgb]{0.68,0.00,0.00}{#1}}
\newcommand{\BuiltInTok}[1]{\textcolor[rgb]{0.00,0.23,0.31}{#1}}
\newcommand{\CharTok}[1]{\textcolor[rgb]{0.13,0.47,0.30}{#1}}
\newcommand{\CommentTok}[1]{\textcolor[rgb]{0.37,0.37,0.37}{#1}}
\newcommand{\CommentVarTok}[1]{\textcolor[rgb]{0.37,0.37,0.37}{\textit{#1}}}
\newcommand{\ConstantTok}[1]{\textcolor[rgb]{0.56,0.35,0.01}{#1}}
\newcommand{\ControlFlowTok}[1]{\textcolor[rgb]{0.00,0.23,0.31}{\textbf{#1}}}
\newcommand{\DataTypeTok}[1]{\textcolor[rgb]{0.68,0.00,0.00}{#1}}
\newcommand{\DecValTok}[1]{\textcolor[rgb]{0.68,0.00,0.00}{#1}}
\newcommand{\DocumentationTok}[1]{\textcolor[rgb]{0.37,0.37,0.37}{\textit{#1}}}
\newcommand{\ErrorTok}[1]{\textcolor[rgb]{0.68,0.00,0.00}{#1}}
\newcommand{\ExtensionTok}[1]{\textcolor[rgb]{0.00,0.23,0.31}{#1}}
\newcommand{\FloatTok}[1]{\textcolor[rgb]{0.68,0.00,0.00}{#1}}
\newcommand{\FunctionTok}[1]{\textcolor[rgb]{0.28,0.35,0.67}{#1}}
\newcommand{\ImportTok}[1]{\textcolor[rgb]{0.00,0.46,0.62}{#1}}
\newcommand{\InformationTok}[1]{\textcolor[rgb]{0.37,0.37,0.37}{#1}}
\newcommand{\KeywordTok}[1]{\textcolor[rgb]{0.00,0.23,0.31}{\textbf{#1}}}
\newcommand{\NormalTok}[1]{\textcolor[rgb]{0.00,0.23,0.31}{#1}}
\newcommand{\OperatorTok}[1]{\textcolor[rgb]{0.37,0.37,0.37}{#1}}
\newcommand{\OtherTok}[1]{\textcolor[rgb]{0.00,0.23,0.31}{#1}}
\newcommand{\PreprocessorTok}[1]{\textcolor[rgb]{0.68,0.00,0.00}{#1}}
\newcommand{\RegionMarkerTok}[1]{\textcolor[rgb]{0.00,0.23,0.31}{#1}}
\newcommand{\SpecialCharTok}[1]{\textcolor[rgb]{0.37,0.37,0.37}{#1}}
\newcommand{\SpecialStringTok}[1]{\textcolor[rgb]{0.13,0.47,0.30}{#1}}
\newcommand{\StringTok}[1]{\textcolor[rgb]{0.13,0.47,0.30}{#1}}
\newcommand{\VariableTok}[1]{\textcolor[rgb]{0.07,0.07,0.07}{#1}}
\newcommand{\VerbatimStringTok}[1]{\textcolor[rgb]{0.13,0.47,0.30}{#1}}
\newcommand{\WarningTok}[1]{\textcolor[rgb]{0.37,0.37,0.37}{\textit{#1}}}

\usepackage{longtable,booktabs,array}
\usepackage{calc} % for calculating minipage widths
% Correct order of tables after \paragraph or \subparagraph
\usepackage{etoolbox}
\makeatletter
\patchcmd\longtable{\par}{\if@noskipsec\mbox{}\fi\par}{}{}
\makeatother
% Allow footnotes in longtable head/foot
\IfFileExists{footnotehyper.sty}{\usepackage{footnotehyper}}{\usepackage{footnote}}
\makesavenoteenv{longtable}
\usepackage{graphicx}
\makeatletter
\newsavebox\pandoc@box
\newcommand*\pandocbounded[1]{% scales image to fit in text height/width
  \sbox\pandoc@box{#1}%
  \Gscale@div\@tempa{\textheight}{\dimexpr\ht\pandoc@box+\dp\pandoc@box\relax}%
  \Gscale@div\@tempb{\linewidth}{\wd\pandoc@box}%
  \ifdim\@tempb\p@<\@tempa\p@\let\@tempa\@tempb\fi% select the smaller of both
  \ifdim\@tempa\p@<\p@\scalebox{\@tempa}{\usebox\pandoc@box}%
  \else\usebox{\pandoc@box}%
  \fi%
}
% Set default figure placement to htbp
\def\fps@figure{htbp}
\makeatother





\setlength{\emergencystretch}{3em} % prevent overfull lines

\providecommand{\tightlist}{%
  \setlength{\itemsep}{0pt}\setlength{\parskip}{0pt}}



 


\KOMAoption{captions}{tableheading}
\usepackage{amsmath}
\usepackage{amssymb}
\makeatletter
\@ifpackageloaded{caption}{}{\usepackage{caption}}
\AtBeginDocument{%
\ifdefined\contentsname
  \renewcommand*\contentsname{Table of contents}
\else
  \newcommand\contentsname{Table of contents}
\fi
\ifdefined\listfigurename
  \renewcommand*\listfigurename{List of Figures}
\else
  \newcommand\listfigurename{List of Figures}
\fi
\ifdefined\listtablename
  \renewcommand*\listtablename{List of Tables}
\else
  \newcommand\listtablename{List of Tables}
\fi
\ifdefined\figurename
  \renewcommand*\figurename{Figure}
\else
  \newcommand\figurename{Figure}
\fi
\ifdefined\tablename
  \renewcommand*\tablename{Table}
\else
  \newcommand\tablename{Table}
\fi
}
\@ifpackageloaded{float}{}{\usepackage{float}}
\floatstyle{ruled}
\@ifundefined{c@chapter}{\newfloat{codelisting}{h}{lop}}{\newfloat{codelisting}{h}{lop}[chapter]}
\floatname{codelisting}{Listing}
\newcommand*\listoflistings{\listof{codelisting}{List of Listings}}
\makeatother
\makeatletter
\makeatother
\makeatletter
\@ifpackageloaded{caption}{}{\usepackage{caption}}
\@ifpackageloaded{subcaption}{}{\usepackage{subcaption}}
\makeatother
\usepackage{bookmark}
\IfFileExists{xurl.sty}{\usepackage{xurl}}{} % add URL line breaks if available
\urlstyle{same}
\hypersetup{
  pdftitle={Distributions},
  colorlinks=true,
  linkcolor={blue},
  filecolor={Maroon},
  citecolor={Blue},
  urlcolor={Blue},
  pdfcreator={LaTeX via pandoc}}


\title{Distributions}
\author{}
\date{}
\begin{document}
\maketitle


\section{Sampled from distributions}\label{sampled-from-distributions}

\texttt{Utils.np\_dists.UnifiedDist} is a class to unify various
distribution methods and provide a consistent interface for sampling and
inference.

\subsubsection{Asymmetric Laplace}\label{asymmetric-laplace}

Samples from an Asymmetric Laplace distribution.

The Asymmetric Laplace distribution is a generalization of the Laplace
distribution, where the two sides of the distribution are scaled
differently. It is defined by a location parameter (loc), a scale
parameter (scale), and an asymmetry parameter (asymmetry).

\[
f(x) = \frac{\text{asymmetry}}{\text{scale}(\text{asymmetry}^2+1)} \exp\left(-\frac{\text{asymmetry}}{\text{scale}}( \text{loc} - x)\right) \\
\text{if } x < \text{loc} \\ 
\frac{\text{asymmetry}}{\text{scale}(\text{asymmetry}^2+1)} \exp\left(-\frac{1}{\text{scale} \cdot \text{asymmetry}}(x - \text{loc})\right) \\             
\text{if } x \ge \text{loc}
\]

\paragraph{Args:}\label{args}

\begin{Shaded}
\begin{Highlighting}[]
\NormalTok{bi.dist.asymmetric\_laplace(}
\NormalTok{loc}\OperatorTok{=}\FloatTok{0.0}\NormalTok{,}
\NormalTok{scale}\OperatorTok{=}\FloatTok{1.0}\NormalTok{,}
\NormalTok{asymmetry}\OperatorTok{=}\FloatTok{1.0}\NormalTok{,}
\NormalTok{validate\_args}\OperatorTok{=}\VariableTok{None}\NormalTok{,}
\NormalTok{name}\OperatorTok{=}\StringTok{\textquotesingle{}x\textquotesingle{}}\NormalTok{,}
\NormalTok{obs}\OperatorTok{=}\VariableTok{None}\NormalTok{,}
\NormalTok{mask}\OperatorTok{=}\VariableTok{None}\NormalTok{,}
\NormalTok{sample}\OperatorTok{=}\VariableTok{False}\NormalTok{,}
\NormalTok{seed}\OperatorTok{=}\DecValTok{0}\NormalTok{,}
\NormalTok{shape}\OperatorTok{=}\NormalTok{(),}
\NormalTok{event}\OperatorTok{=}\DecValTok{0}\NormalTok{,}
\NormalTok{create\_obj}\OperatorTok{=}\VariableTok{False}\NormalTok{,}
\NormalTok{)}
\end{Highlighting}
\end{Shaded}

\begin{itemize}
\item
  \emph{loc} (jnp.ndarray or float): Location parameter of the
  distribution.
\item
  \emph{scale} (jnp.ndarray or float): Scale parameter of the
  distribution.
\item
  \emph{asymmetry} (jnp.ndarray or float): Asymmetry parameter of the
  distribution.
\item
  \emph{shape} (tuple): A multi-purpose argument for shaping. When
  \texttt{sample=False} (model building), this is used with
  \texttt{.expand(shape)} to set the distribution's batch shape. When
  \texttt{sample=True} (direct sampling), this is used as
  \texttt{sample\_shape} to draw a raw JAX array of the given shape.
\item
  \emph{event} (int): The number of batch dimensions to reinterpret as
  event dimensions (used in model building).
\item
  \emph{mask} (jnp.ndarray, bool): Optional boolean array to mask
  observations.
\item
  \emph{create\_obj} (bool): If True, returns the raw BI distribution
  object instead of creating a sample site. This is essential for
  building complex distributions like \texttt{MixtureSameFamily}.
\item
  \emph{sample} (bool, optional): A control-flow argument. If
  \texttt{True}, the function will directly sample a raw JAX array from
  the distribution, bypassing the BI model context. If \texttt{False},
  it will create a \texttt{BI.sample} site within a model. Defaults to
  \texttt{False}.
\item
  \emph{obs} (jnp.ndarray, optional): The observed value for this random
  variable. If provided, the sample site is conditioned on this value,
  and the function returns the observed value. If \texttt{None}, the
  site is treated as a latent (unobserved) random variable. Defaults to
  \texttt{None}.
\item
  \emph{name} (str, optional): The name of the sample site in a BI
  model. This is used to uniquely identify the random variable. Defaults
  to `x'.
\item
  validate\_args (bool, optional): Whether to enable validation of
  distribution parameters. Defaults to \texttt{None}.
\end{itemize}

\paragraph{Returns:}\label{returns}

\begin{itemize}
\item
  When \texttt{sample=False}: A BI AsymmetricLaplace distribution object
  (for model building).
\item
  When \texttt{sample=True}: A JAX array of samples drawn from the
  BetaBinomial distribution (for direct sampling).
\item
  When \texttt{create\_obj=True}: The raw BI AsymmetricLaplace
  distribution object (for advanced use cases).
\end{itemize}

\paragraph{Example Usage:}\label{example-usage}

\begin{Shaded}
\begin{Highlighting}[]
\ImportTok{from}\NormalTok{ BI }\ImportTok{import}\NormalTok{ bi}
\NormalTok{m }\OperatorTok{=}\NormalTok{ bi(}\StringTok{\textquotesingle{}cpu\textquotesingle{}}\NormalTok{)}
\NormalTok{m.dist.asymmetric\_laplace(loc}\OperatorTok{=}\FloatTok{0.0}\NormalTok{, scale}\OperatorTok{=}\FloatTok{1.0}\NormalTok{, asymmetry}\OperatorTok{=}\FloatTok{1.0}\NormalTok{, sample}\OperatorTok{=}\VariableTok{True}\NormalTok{)}
\end{Highlighting}
\end{Shaded}

\paragraph{Wrapper of:}\label{wrapper-of}

https://num.pyro.ai/en/stable/distributions.html\#asymmetriclaplace

\begin{center}\rule{0.5\linewidth}{0.5pt}\end{center}

\subsubsection{Asymmetric Laplace
Quantile}\label{asymmetric-laplace-quantile}

Samples from an AsymmetricLaplaceQuantile distribution.

This distribution is an alternative parameterization of the
AsymmetricLaplace distribution, commonly used in Bayesian quantile
regression. It utilizes a \texttt{quantile} parameter to define the
balance between the left- and right-hand sides of the distribution,
representing the proportion of probability density that falls to the
left-hand side.

\[   
f(x) = \frac{1}{2 \sigma} \exp\left(-\frac{|x - \mu|}{\sigma} \frac{1}{q-1}\right) \left(1 - \frac{1}{2q}\right)
\]

\paragraph{Args:}\label{args-1}

\begin{Shaded}
\begin{Highlighting}[]
\NormalTok{bi.dist.asymmetric\_laplace\_quantile(}
\NormalTok{loc}\OperatorTok{=}\FloatTok{0.0}\NormalTok{,}
\NormalTok{scale}\OperatorTok{=}\FloatTok{1.0}\NormalTok{,}
\NormalTok{quantile}\OperatorTok{=}\FloatTok{0.5}\NormalTok{,}
\NormalTok{validate\_args}\OperatorTok{=}\VariableTok{None}\NormalTok{,}
\NormalTok{name}\OperatorTok{=}\StringTok{\textquotesingle{}x\textquotesingle{}}\NormalTok{,}
\NormalTok{obs}\OperatorTok{=}\VariableTok{None}\NormalTok{,}
\NormalTok{mask}\OperatorTok{=}\VariableTok{None}\NormalTok{,}
\NormalTok{sample}\OperatorTok{=}\VariableTok{False}\NormalTok{,}
\NormalTok{seed}\OperatorTok{=}\DecValTok{0}\NormalTok{,}
\NormalTok{shape}\OperatorTok{=}\NormalTok{(),}
\NormalTok{event}\OperatorTok{=}\DecValTok{0}\NormalTok{,}
\NormalTok{create\_obj}\OperatorTok{=}\VariableTok{False}\NormalTok{,}
\NormalTok{)}
\end{Highlighting}
\end{Shaded}

\begin{itemize}
\item
  \emph{loc} (float): The location parameter of the distribution.
\item
  \emph{sample} (float): The scale parameter of the distribution.
\end{itemize}

quantile (float): The quantile parameter, representing the proportion of
probability density to the left of the median. Must be between 0 and 1.

\begin{itemize}
\item
  \emph{shape} (tuple): A multi-purpose argument for shaping. When
  \texttt{sample=False} (model building), this is used with
  \texttt{.expand(shape)} to set the distribution's batch shape. When
  \texttt{sample=True} (direct sampling), this is used as
  \texttt{sample\_shape} to draw a raw JAX array of the given shape.
\item
  \emph{event} (int): The number of batch dimensions to reinterpret as
  event dimensions (used in model building).
\item
  \emph{mask} (jnp.ndarray, bool): Optional boolean array to mask
  observations.
\item
  \emph{create\_obj} (bool): If True, returns the raw BI distribution
  object instead of creating a sample site. This is essential for
  building complex distributions like \texttt{MixtureSameFamily}.
\item
  \emph{sample} (bool, optional): A control-flow argument. If
  \texttt{True}, the function will directly sample a raw JAX array from
  the distribution, bypassing the BI model context. If \texttt{False},
  it will create a \texttt{BI.sample} site within a model. Defaults to
  \texttt{False}.
\item
  \emph{obs} (jnp.ndarray, optional): The observed value for this random
  variable. If provided, the sample site is conditioned on this value,
  and the function returns the observed value. If \texttt{None}, the
  site is treated as a latent (unobserved) random variable. Defaults to
  \texttt{None}.
\item
  \emph{name} (str, optional): The name of the sample site in a BI
  model. This is used to uniquely identify the random variable. Defaults
  to `x'.
\end{itemize}

\paragraph{Returns:}\label{returns-1}

\begin{itemize}
\item
  When \texttt{sample=False}: A BI AsymmetricLaplaceQuantile
  distribution object (for model building).
\item
  When \texttt{sample=True}: A JAX array of samples drawn from the
  BetaBinomial distribution (for direct sampling).
\item
  When \texttt{create\_obj=True}: The raw BI AsymmetricLaplaceQuantile
  distribution object (for advanced use cases).
\end{itemize}

\paragraph{Example Usage:}\label{example-usage-1}

\begin{Shaded}
\begin{Highlighting}[]
\ImportTok{from}\NormalTok{ BI }\ImportTok{import}\NormalTok{ bi}
\NormalTok{m }\OperatorTok{=}\NormalTok{ bi(}\StringTok{\textquotesingle{}cpu\textquotesingle{}}\NormalTok{)}
\NormalTok{m.dist.asymmetric\_laplace\_quantile(loc}\OperatorTok{=}\FloatTok{0.0}\NormalTok{, scale}\OperatorTok{=}\FloatTok{1.0}\NormalTok{, quantile}\OperatorTok{=}\FloatTok{0.5}\NormalTok{, sample}\OperatorTok{=}\VariableTok{True}\NormalTok{)}
\end{Highlighting}
\end{Shaded}

\paragraph{Wrapper of:}\label{wrapper-of-1}

https://num.pyro.ai/en/stable/distributions.html\#asymmetriclaplacequantile

\begin{center}\rule{0.5\linewidth}{0.5pt}\end{center}

\subsubsection{Bernoulli}\label{bernoulli}

The Bernoulli distribution models a single trial with two possible
outcomes: success or failure. It is parameterized by the probability of
success, often denoted as `p'.

\[   P(X=1) = p \\
P(X=0) = 1 - p
\]

\paragraph{Args:}\label{args-2}

\begin{Shaded}
\begin{Highlighting}[]
\NormalTok{bi.dist.bernoulli(}
\NormalTok{probs}\OperatorTok{=}\VariableTok{None}\NormalTok{,}
\NormalTok{logits}\OperatorTok{=}\VariableTok{None}\NormalTok{,}
\NormalTok{validate\_args}\OperatorTok{=}\VariableTok{None}\NormalTok{,}
\NormalTok{name}\OperatorTok{=}\StringTok{\textquotesingle{}x\textquotesingle{}}\NormalTok{,}
\NormalTok{obs}\OperatorTok{=}\VariableTok{None}\NormalTok{,}
\NormalTok{mask}\OperatorTok{=}\VariableTok{None}\NormalTok{,}
\NormalTok{sample}\OperatorTok{=}\VariableTok{False}\NormalTok{,}
\NormalTok{seed}\OperatorTok{=}\DecValTok{0}\NormalTok{,}
\NormalTok{shape}\OperatorTok{=}\NormalTok{(),}
\NormalTok{event}\OperatorTok{=}\DecValTok{0}\NormalTok{,}
\NormalTok{create\_obj}\OperatorTok{=}\VariableTok{False}\NormalTok{,}
\NormalTok{)}
\end{Highlighting}
\end{Shaded}

\begin{itemize}
\item
  \emph{probs} (jnp.ndarray, optional): Probability of success for each
  Bernoulli trial. Must be between 0 and 1. logits (jnp.ndarray,
  optional): Log-odds of success for each Bernoulli trial.
  \texttt{probs\ =\ sigmoid(logits)}.
\item
  \emph{shape} (tuple): A multi-purpose argument for shaping. When
  \texttt{sample=False} (model building), this is used with
  \texttt{.expand(shape)} to set the distribution's batch shape. When
  \texttt{sample=True} (direct sampling), this is used as
  \texttt{sample\_shape} to draw a raw JAX array of the given shape.
\item
  \emph{event} (int): The number of batch dimensions to reinterpret as
  event dimensions (used in model building).
\item
  \emph{mask} (jnp.ndarray, bool, optional): Optional boolean array to
  mask observations.
\item
  \emph{create\_obj} (bool, optional): If True, returns the raw BI
  distribution object instead of creating a sample site. This is
  essential for building complex distributions like
  \texttt{MixtureSameFamily}.
\item
  \emph{sample} (bool, optional): A control-flow argument. If
  \texttt{True}, the function will directly sample a raw JAX array from
  the distribution, bypassing the BI model context. If \texttt{False},
  it will create a \texttt{BI.sample} site within a model. Defaults to
  \texttt{False}.
\item
  \emph{seed} (int, optional): An integer used to generate a JAX PRNGKey
  for reproducible sampling when \texttt{sample=True}. {[}7{]} This
  argument has no effect when \texttt{sample=False}, as randomness is
  handled by BI's inference engine. Defaults to 0.
\item
  \emph{obs} (jnp.ndarray, optional): The observed value for this random
  variable. If provided, the sample site is conditioned on this value,
  and the function returns the observed value. If \texttt{None}, the
  site is treated as a latent (unobserved) random variable. Defaults to
  \texttt{None}.
\item
  \emph{name} (str, optional): The name of the sample site in a BI
  model. This is used to uniquely identify the random variable. Defaults
  to `x'.
\end{itemize}

\paragraph{Returns:}\label{returns-2}

BI Bernoulli distribution object (for model building) when
\texttt{sample=False}. JAX array of samples drawn from the Bernoulli
distribution (for direct sampling) when \texttt{sample=True}. The raw BI
distribution object (for advanced use cases) when
\texttt{create\_obj=True}.

\paragraph{Example Usage:}\label{example-usage-2}

\begin{Shaded}
\begin{Highlighting}[]
\ImportTok{from}\NormalTok{ BI }\ImportTok{import}\NormalTok{ bi}
\NormalTok{m }\OperatorTok{=}\NormalTok{ bi(}\StringTok{\textquotesingle{}cpu\textquotesingle{}}\NormalTok{)}
\NormalTok{m.dist.bernoulli(probs}\OperatorTok{=}\FloatTok{0.7}\NormalTok{, sample}\OperatorTok{=}\VariableTok{True}\NormalTok{)}
\end{Highlighting}
\end{Shaded}

\paragraph{Wrapper of:
https://num.pyro.ai/en/stable/distributions.html\#bernoulli}\label{wrapper-of-httpsnum.pyro.aienstabledistributions.htmlbernoulli}

\begin{center}\rule{0.5\linewidth}{0.5pt}\end{center}

\subsubsection{Bernoulli Logits}\label{bernoulli-logits}

Samples from a Bernoulli distribution parameterized by logits.

The Bernoulli distribution models a single binary event (success or
failure), parameterized by the log-odds ratio of success. The
probability of success is given by the sigmoid function applied to the
logit.

\[
P(x) = \sigma(logits)
\]

\paragraph{Args:}\label{args-3}

\begin{Shaded}
\begin{Highlighting}[]
\NormalTok{bi.dist.bernoulli\_logits(}
\NormalTok{logits}\OperatorTok{=}\VariableTok{None}\NormalTok{,}
\NormalTok{validate\_args}\OperatorTok{=}\VariableTok{None}\NormalTok{,}
\NormalTok{name}\OperatorTok{=}\StringTok{\textquotesingle{}x\textquotesingle{}}\NormalTok{,}
\NormalTok{obs}\OperatorTok{=}\VariableTok{None}\NormalTok{,}
\NormalTok{mask}\OperatorTok{=}\VariableTok{None}\NormalTok{,}
\NormalTok{sample}\OperatorTok{=}\VariableTok{False}\NormalTok{,}
\NormalTok{seed}\OperatorTok{=}\DecValTok{0}\NormalTok{,}
\NormalTok{shape}\OperatorTok{=}\NormalTok{(),}
\NormalTok{event}\OperatorTok{=}\DecValTok{0}\NormalTok{,}
\NormalTok{create\_obj}\OperatorTok{=}\VariableTok{False}\NormalTok{,}
\NormalTok{)}
\end{Highlighting}
\end{Shaded}

\begin{itemize}
\item
  \emph{logits} (jnp.ndarray, optional): Log-odds ratio of success. Must
  be real-valued.
\item
  \emph{shape} (tuple): A multi-purpose argument for shaping. When
  \texttt{sample=False} (model building), this is used with
  \texttt{.expand(shape)} to set the distribution's batch shape. When
  \texttt{sample=True} (direct sampling), this is used as
  \texttt{sample\_shape} to draw a raw JAX array of the given shape.
\item
  \emph{event} (int, optional): The number of batch dimensions to
  reinterpret as
\item
  \emph{event} dimensions (used in model building). Defaults to 0.
\item
  \emph{mask} (jnp.ndarray, bool, optional): Optional boolean array to
  mask observations. Defaults to None.
\item
  \emph{create\_obj} (bool, optional): If True, returns the raw BI
  distribution object instead of creating a sample site. This is
  essential for building complex distributions like
  \texttt{MixtureSameFamily}. Defaults to False.
\item
  \emph{sample} (bool, optional): A control-flow argument. If
  \texttt{True}, the function will directly sample a raw JAX array from
  the distribution, bypassing the BI model context. If \texttt{False},
  it will create a \texttt{BI.sample} site within a model. Defaults to
  \texttt{False}.
\item
  \emph{obs} (jnp.ndarray, optional): The observed value for this random
  variable. If provided, the sample site is conditioned on this value,
  and the function returns the observed value. If \texttt{None}, the
  site is treated as a latent (unobserved) random variable. Defaults to
  \texttt{None}.
\item
  \emph{name} (str, optional): The name of the sample site in a BI
  model. This is used to uniquely identify the random variable. Defaults
  to `x'.
\end{itemize}

\paragraph{Returns:}\label{returns-3}

\begin{itemize}
\item
  When \texttt{sample=False}: A BI BernoulliLogits distribution object
  (for model building).
\item
  When \texttt{sample=True}: A JAX array of samples drawn from the
  BernoulliLogits distribution (for direct sampling).
\item
  When \texttt{create\_obj=True}: The raw BI distribution object (for
  advanced use cases).
\end{itemize}

\paragraph{Example Usage:}\label{example-usage-3}

\begin{Shaded}
\begin{Highlighting}[]
\ImportTok{from}\NormalTok{ BI }\ImportTok{import}\NormalTok{ bi}
\NormalTok{m }\OperatorTok{=}\NormalTok{ bi(}\StringTok{\textquotesingle{}cpu\textquotesingle{}}\NormalTok{)}
\NormalTok{m.dist.bernoulli\_logits(logits}\OperatorTok{=}\NormalTok{jnp.array([}\FloatTok{0.2}\NormalTok{, }\DecValTok{1}\NormalTok{, }\DecValTok{2}\NormalTok{]), sample}\OperatorTok{=}\VariableTok{True}\NormalTok{)}
\end{Highlighting}
\end{Shaded}

\paragraph{Wrapper of:}\label{wrapper-of-2}

https://num.pyro.ai/en/stable/distributions.html\#bernoulli-logits

\begin{center}\rule{0.5\linewidth}{0.5pt}\end{center}

\subsubsection{Bernoulli Probs}\label{bernoulli-probs}

Samples from a Bernoulli distribution parameterized by probabilities.

The Bernoulli distribution models the probability of success in a single
trial, where the outcome is binary (success or failure). It is
characterized by a single parameter, \texttt{probs}, representing the
probability of success.

\[
P(X=1) = p
\]

where: p is the probability of success (0 \textless= p \textless= 1)

\paragraph{Args:}\label{args-4}

\begin{Shaded}
\begin{Highlighting}[]
\NormalTok{bi.dist.bernoulli\_probs(}
\NormalTok{probs,}
\NormalTok{validate\_args}\OperatorTok{=}\VariableTok{None}\NormalTok{,}
\NormalTok{name}\OperatorTok{=}\StringTok{\textquotesingle{}x\textquotesingle{}}\NormalTok{,}
\NormalTok{obs}\OperatorTok{=}\VariableTok{None}\NormalTok{,}
\NormalTok{mask}\OperatorTok{=}\VariableTok{None}\NormalTok{,}
\NormalTok{sample}\OperatorTok{=}\VariableTok{False}\NormalTok{,}
\NormalTok{seed}\OperatorTok{=}\DecValTok{0}\NormalTok{,}
\NormalTok{shape}\OperatorTok{=}\NormalTok{(),}
\NormalTok{event}\OperatorTok{=}\DecValTok{0}\NormalTok{,}
\NormalTok{create\_obj}\OperatorTok{=}\VariableTok{False}\NormalTok{,}
\NormalTok{)}
\end{Highlighting}
\end{Shaded}

\begin{itemize}
\item
  \emph{probs} (jnp.ndarray): The probability of success for each
  Bernoulli trial. Must be between 0 and 1.
\item
  \emph{shape} (tuple): A multi-purpose argument for shaping. When
  \texttt{sample=False} (model building), this is used with
  \texttt{.expand(shape)} to set the distribution's batch shape. When
  \texttt{sample=True} (direct sampling), this is used as
  \texttt{sample\_shape} to draw a raw JAX array of the given shape.
\item
  \emph{event} (int): The number of batch dimensions to reinterpret as
  event dimensions (used in model building).
\item
  \emph{mask} (jnp.ndarray, bool): Optional boolean array to mask
  observations.
\item
  \emph{create\_obj} (bool): If True, returns the raw BI distribution
  object instead of creating a sample site. This is essential for
  building complex distributions like \texttt{MixtureSameFamily}.
\item
  \emph{sample} (bool, optional): A control-flow argument. If
  \texttt{True}, the function will directly sample a raw JAX array from
  the distribution, bypassing the BI model context. If \texttt{False},
  it will create a \texttt{BI.sample} site within a model. Defaults to
  \texttt{False}.
\item
  \emph{obs} (jnp.ndarray, optional): The observed value for this random
  variable. If provided, the sample site is conditioned on this value,
  and the function returns the observed value. If \texttt{None}, the
  site is treated as a latent (unobserved) random variable. Defaults to
  \texttt{None}.
\item
  \emph{name} (str, optional): The name of the sample site in a BI
  model. This is used to uniquely identify the random variable. Defaults
  to `x'.
\end{itemize}

\paragraph{Returns:}\label{returns-4}

\begin{itemize}
\item
  When \texttt{sample=False}: A BI BernoulliProbs distribution object
  (for model building).
\item
  When \texttt{sample=True}: A JAX array of samples drawn from the
  BernoulliLogits distribution (for direct sampling).
\item
  When \texttt{create\_obj=True}: The raw BI BernoulliProbs object (for
  advanced use cases).
\end{itemize}

\paragraph{Example Usage:}\label{example-usage-4}

\begin{Shaded}
\begin{Highlighting}[]
\ImportTok{from}\NormalTok{ BI }\ImportTok{import}\NormalTok{ bi}
\NormalTok{m }\OperatorTok{=}\NormalTok{ bi(}\StringTok{\textquotesingle{}cpu\textquotesingle{}}\NormalTok{)}
\NormalTok{m.dist.bernoulli\_probs(probs}\OperatorTok{=}\NormalTok{jnp.array([}\FloatTok{0.2}\NormalTok{, }\FloatTok{0.7}\NormalTok{, }\FloatTok{0.5}\NormalTok{]), sample}\OperatorTok{=}\VariableTok{True}\NormalTok{)}
\end{Highlighting}
\end{Shaded}

\paragraph{Wrapper of:}\label{wrapper-of-3}

https://num.pyro.ai/en/stable/distributions.html\#bernoulliprobs

\begin{center}\rule{0.5\linewidth}{0.5pt}\end{center}

\subsubsection{Beta}\label{beta}

Samples from a Beta distribution, defined on the interval {[}0, 1{]}.
The Beta distribution is a versatile distribution often used to model
probabilities or proportions. It is parameterized by two positive shape
parameters, often referred to as concentration parameters in the BI
context.

\[
f(x) = \frac{x^{\alpha - 1} (1 - x)^{\beta - 1}}{B(\alpha, \beta)}
\]

where \(\alpha\) and \(\beta\) are the concentration parameters, and
\(B(x, y)\) is the Beta function.

\paragraph{Args:}\label{args-5}

\begin{Shaded}
\begin{Highlighting}[]
\NormalTok{bi.dist.beta(}
\NormalTok{concentration1,}
\NormalTok{concentration0,}
\NormalTok{validate\_args}\OperatorTok{=}\VariableTok{None}\NormalTok{,}
\NormalTok{name}\OperatorTok{=}\StringTok{\textquotesingle{}x\textquotesingle{}}\NormalTok{,}
\NormalTok{obs}\OperatorTok{=}\VariableTok{None}\NormalTok{,}
\NormalTok{mask}\OperatorTok{=}\VariableTok{None}\NormalTok{,}
\NormalTok{sample}\OperatorTok{=}\VariableTok{False}\NormalTok{,}
\NormalTok{seed}\OperatorTok{=}\DecValTok{0}\NormalTok{,}
\NormalTok{shape}\OperatorTok{=}\NormalTok{(),}
\NormalTok{event}\OperatorTok{=}\DecValTok{0}\NormalTok{,}
\NormalTok{create\_obj}\OperatorTok{=}\VariableTok{False}\NormalTok{,}
\NormalTok{)}
\end{Highlighting}
\end{Shaded}

\begin{itemize}
\item
  \emph{concentration1} (jnp.ndarray): The first concentration parameter
  (shape parameter). Must be positive.
\item
  \emph{concentration0} (jnp.ndarray): The second concentration
  parameter (shape parameter). Must be positive.
\item
  \emph{shape} (tuple): A multi-purpose argument for shaping. When
  \texttt{sample=False} (model building), this is used with
  \texttt{.expand(shape)} to set the distribution's batch shape. When
  \texttt{sample=True} (direct sampling), this is used as
  \texttt{sample\_shape} to draw a raw JAX array of the given shape.
\item
  \emph{event} (int): The number of batch dimensions to reinterpret as
  event dimensions (used in model building).
\item
  \emph{mask} (jnp.ndarray, bool): Optional boolean array to mask
  observations.
\item
  \emph{create\_obj} (bool): If True, returns the raw BI distribution
  object instead of creating a sample site. This is essential for
  building complex distributions like \texttt{MixtureSameFamily}.
\item
  \emph{sample} (bool, optional): A control-flow argument. If
  \texttt{True}, the function will directly sample a raw JAX array from
  the distribution, bypassing the BI model context. If \texttt{False},
  it will create a \texttt{BI.sample} site within a model. Defaults to
  \texttt{False}.
\item
  \emph{obs} (jnp.ndarray, optional): The observed value for this random
  variable. If provided, the sample site is conditioned on this value,
  and the function returns the observed value. If \texttt{None}, the
  site is treated as a latent (unobserved) random variable. Defaults to
  \texttt{None}.
\item
  \emph{name} (str, optional): The name of the sample site in a BI
  model. This is used to uniquely identify the random variable. Defaults
  to `x'.
\end{itemize}

\paragraph{Returns:}\label{returns-5}

\begin{itemize}
\item
  When \texttt{sample=False}: A BI Beta distribution object (for model
  building).
\item
  When \texttt{sample=True}: A JAX array of samples drawn from the
  BernoulliLogits distribution (for direct sampling).
\item
  When \texttt{create\_obj=True}: The raw BI Beta object (for advanced
  use cases).
\end{itemize}

\paragraph{\#\#\#\# Example Usage:}\label{example-usage-5}

\begin{Shaded}
\begin{Highlighting}[]
\ImportTok{from}\NormalTok{ BI }\ImportTok{import}\NormalTok{ bi}
\NormalTok{m }\OperatorTok{=}\NormalTok{ bi(}\StringTok{\textquotesingle{}cpu\textquotesingle{}}\NormalTok{)}
\NormalTok{m.dist.beta(concentration1}\OperatorTok{=}\FloatTok{1.0}\NormalTok{, concentration0}\OperatorTok{=}\FloatTok{1.0}\NormalTok{, sample}\OperatorTok{=}\VariableTok{True}\NormalTok{)}
\end{Highlighting}
\end{Shaded}

\paragraph{Wrapper of:}\label{wrapper-of-4}

https://num.pyro.ai/en/stable/distributions.html\#beta

\begin{center}\rule{0.5\linewidth}{0.5pt}\end{center}

\subsubsection{BetaBinomial}\label{betabinomial}

Samples from a BetaBinomial distribution, a compound distribution where
the probability of success in a binomial experiment is drawn from a Beta
distribution. This models situations where the underlying probability of
success is not fixed but varies according to a prior belief represented
by the Beta distribution.

\[   
P(X=k) = \binom{n}{k} \frac{\Gamma(\alpha + k)}{\Gamma(\alpha + \beta + n - k)} \frac{\Gamma(\beta + n - k)}{\Gamma(\beta)}
\]

\paragraph{Args:}\label{args-6}

\begin{Shaded}
\begin{Highlighting}[]
\NormalTok{bi.dist.beta\_binomial(}
\NormalTok{concentration1,}
\NormalTok{concentration0,}
\NormalTok{total\_count}\OperatorTok{=}\DecValTok{1}\NormalTok{,}
\NormalTok{validate\_args}\OperatorTok{=}\VariableTok{None}\NormalTok{,}
\NormalTok{name}\OperatorTok{=}\StringTok{\textquotesingle{}x\textquotesingle{}}\NormalTok{,}
\NormalTok{obs}\OperatorTok{=}\VariableTok{None}\NormalTok{,}
\NormalTok{mask}\OperatorTok{=}\VariableTok{None}\NormalTok{,}
\NormalTok{sample}\OperatorTok{=}\VariableTok{False}\NormalTok{,}
\NormalTok{seed}\OperatorTok{=}\DecValTok{0}\NormalTok{,}
\NormalTok{shape}\OperatorTok{=}\NormalTok{(),}
\NormalTok{event}\OperatorTok{=}\DecValTok{0}\NormalTok{,}
\NormalTok{create\_obj}\OperatorTok{=}\VariableTok{False}\NormalTok{,}
\NormalTok{)}
\end{Highlighting}
\end{Shaded}

\begin{itemize}
\item
  \emph{concentration1} (jnp.ndarray): The first concentration parameter
  (alpha) of the Beta distribution.
\item
  \emph{concentration0} (jnp.ndarray): The second concentration
  parameter (beta) of the Beta distribution.
\item
  \emph{total\_count} (jnp.ndarray): The number of Bernoulli trials in
  the Binomial part of the distribution.
\item
  \emph{shape} (tuple): A multi-purpose argument for shaping. When
  \texttt{sample=False} (model building), this is used with
  \texttt{.expand(shape)} to set the distribution's batch shape. When
  \texttt{sample=True} (direct sampling), this is used as
  \texttt{sample\_shape} to draw a raw JAX array of the given shape.
\item
  \emph{event} (int): The number of batch dimensions to reinterpret as
  event dimensions (used in model building).
\item
  \emph{mask} (jnp.ndarray, bool): Optional boolean array to mask
  observations.
\item
  \emph{create\_obj} (bool): If True, returns the raw BI distribution
  object instead of creating a sample site. This is essential for
  building complex distributions like \texttt{MixtureSameFamily}.
\item
  \emph{sample} (bool, optional): A control-flow argument. If
  \texttt{True}, the function will directly sample a raw JAX array from
  the distribution, bypassing the BI model context. If \texttt{False},
  it will create a \texttt{BI.sample} site within a model. Defaults to
  \texttt{False}.
\item
  \emph{obs} (jnp.ndarray, optional): The observed value for this random
  variable. If provided, the sample site is conditioned on this value,
  and the function returns the observed value. If \texttt{None}, the
  site is treated as a latent (unobserved) random variable. Defaults to
  \texttt{None}.
\item
  \emph{name} (str, optional): The name of the sample site in a BI
  model. This is used to uniquely identify the random variable. Defaults
  to `x'.
\end{itemize}

\paragraph{Returns:}\label{returns-6}

\begin{itemize}
\item
  When \texttt{sample=False}: A BI BetaBinomial distribution object (for
  model building).
\item
  When \texttt{sample=True}: A JAX array of samples drawn from the
  BetaBinomial distribution (for direct sampling).
\item
  When \texttt{create\_obj=True}: The raw BI BetaBinomial distribution
  object (for advanced use cases).
\end{itemize}

\paragraph{Example Usage:}\label{example-usage-6}

\begin{Shaded}
\begin{Highlighting}[]
\ImportTok{from}\NormalTok{ BI }\ImportTok{import}\NormalTok{ bi}
\NormalTok{m }\OperatorTok{=}\NormalTok{ bi(}\StringTok{\textquotesingle{}cpu\textquotesingle{}}\NormalTok{)}
\NormalTok{m.dist.beta\_binomial(concentration1}\OperatorTok{=}\FloatTok{1.0}\NormalTok{, concentration0}\OperatorTok{=}\FloatTok{1.0}\NormalTok{, total\_count}\OperatorTok{=}\DecValTok{10}\NormalTok{, sample}\OperatorTok{=}\VariableTok{True}\NormalTok{)}
\end{Highlighting}
\end{Shaded}

\paragraph{Wrapper of:}\label{wrapper-of-5}

https://num.pyro.ai/en/stable/distributions.html\#betabinomial

\begin{center}\rule{0.5\linewidth}{0.5pt}\end{center}

\subsubsection{Beta Proportion}\label{beta-proportion}

The BetaProportion distribution is a reparameterization of the
conventional Beta distribution in terms of a the variate mean and a
precision parameter. It's useful for modeling rates and proportions.

\[
f(x) = \frac{x^{\alpha - 1} (1 - x)^{\beta - 1}}{B(\alpha, \beta)}
\]

\paragraph{Args:}\label{args-7}

\begin{Shaded}
\begin{Highlighting}[]
\NormalTok{bi.dist.beta\_proportion(}
\NormalTok{mean,}
\NormalTok{concentration,}
\NormalTok{validate\_args}\OperatorTok{=}\VariableTok{None}\NormalTok{,}
\NormalTok{name}\OperatorTok{=}\StringTok{\textquotesingle{}x\textquotesingle{}}\NormalTok{,}
\NormalTok{obs}\OperatorTok{=}\VariableTok{None}\NormalTok{,}
\NormalTok{mask}\OperatorTok{=}\VariableTok{None}\NormalTok{,}
\NormalTok{sample}\OperatorTok{=}\VariableTok{False}\NormalTok{,}
\NormalTok{seed}\OperatorTok{=}\DecValTok{0}\NormalTok{,}
\NormalTok{shape}\OperatorTok{=}\NormalTok{(),}
\NormalTok{event}\OperatorTok{=}\DecValTok{0}\NormalTok{,}
\NormalTok{create\_obj}\OperatorTok{=}\VariableTok{False}\NormalTok{,}
\NormalTok{)}
\end{Highlighting}
\end{Shaded}

\begin{itemize}
\item
  \emph{mean} (jnp.ndarray): The mean of the BetaProportion
  distribution,must be between 0 and 1.
\item
  \emph{concentration} (jnp.ndarray): The concentration parameter of the
  BetaProportion distribution.
\item
  \emph{shape} (tuple): A multi-purpose argument for shaping. When
  \texttt{sample=False} (model building), this is used with
  \texttt{.expand(shape)} to set the distribution's batch shape. When
  \texttt{sample=True} (direct sampling), this is used as
  \texttt{sample\_shape} to draw a raw JAX array of the given shape.
\item
  \emph{event} (int): The number of batch dimensions to reinterpret as
  event dimensions (used in model building).
\item
  \emph{mask} (jnp.ndarray, bool): Optional boolean array to mask
  observations.
\item
  \emph{create\_obj} (bool): If True, returns the raw BI distribution
  object instead of creating a sample site. This is essential for
  building complex distributions like \texttt{MixtureSameFamily}.
\item
  \emph{sample} (bool, optional): A control-flow argument. If
  \texttt{True}, the function will directly sample a raw JAX array from
  the distribution, bypassing the BI model context. If \texttt{False},
  it will create a \texttt{BI.sample} site within a model. Defaults to
  \texttt{False}.
\item
  \emph{obs} (jnp.ndarray, optional): The observed value for this random
  variable. If provided, the sample site is conditioned on this value,
  and the function returns the observed value. If \texttt{None}, the
  site is treated as a latent (unobserved) random variable. Defaults to
  \texttt{None}.
\item
  \emph{name} (str, optional): The name of the sample site in a BI
  model. This is used to uniquely identify the random variable. Defaults
  to `x'.
\end{itemize}

\paragraph{Returns:}\label{returns-7}

\begin{itemize}
\item
  When \texttt{sample=False}: A BI BetaProportion distribution object
  (for model building).
\item
  When \texttt{sample=True}: A JAX array of samples drawn from the
  BetaBinomial distribution (for direct sampling).
\item
  When \texttt{create\_obj=True}: The raw BI BetaProportion distribution
  object (for advanced use cases).
\end{itemize}

\paragraph{Example Usage:}\label{example-usage-7}

\begin{Shaded}
\begin{Highlighting}[]
\ImportTok{from}\NormalTok{ BI }\ImportTok{import}\NormalTok{ bi}
\NormalTok{m }\OperatorTok{=}\NormalTok{ bi(}\StringTok{\textquotesingle{}cpu\textquotesingle{}}\NormalTok{)}
\NormalTok{samples }\OperatorTok{=}\NormalTok{ m.dist.beta\_proportion(mean}\OperatorTok{=}\FloatTok{0.5}\NormalTok{, concentration}\OperatorTok{=}\FloatTok{2.0}\NormalTok{, sample}\OperatorTok{=}\VariableTok{True}\NormalTok{, shape}\OperatorTok{=}\NormalTok{(}\DecValTok{1000}\NormalTok{,))}
\end{Highlighting}
\end{Shaded}

\paragraph{Wrapper of:}\label{wrapper-of-6}

https://num.pyro.ai/en/stable/distributions.html\#beta\_proportion

\begin{center}\rule{0.5\linewidth}{0.5pt}\end{center}

\subsubsection{Binomial}\label{binomial}

The Binomial distribution models the number of successes in a sequence
of independent Bernoulli trials. It represents the probability of
obtaining exactly \emph{k} successes in \emph{n} trials, where each
trial has a probability \emph{p} of success.

\[   P(X = k) = \binom{n}{k} p^k (1-p)^{n-k}
\]

\paragraph{Args:}\label{args-8}

\begin{Shaded}
\begin{Highlighting}[]
\NormalTok{bi.dist.binomial(}
\NormalTok{total\_count}\OperatorTok{=}\DecValTok{1}\NormalTok{,}
\NormalTok{probs}\OperatorTok{=}\VariableTok{None}\NormalTok{,}
\NormalTok{logits}\OperatorTok{=}\VariableTok{None}\NormalTok{,}
\NormalTok{validate\_args}\OperatorTok{=}\VariableTok{None}\NormalTok{,}
\NormalTok{name}\OperatorTok{=}\StringTok{\textquotesingle{}x\textquotesingle{}}\NormalTok{,}
\NormalTok{obs}\OperatorTok{=}\VariableTok{None}\NormalTok{,}
\NormalTok{mask}\OperatorTok{=}\VariableTok{None}\NormalTok{,}
\NormalTok{sample}\OperatorTok{=}\VariableTok{False}\NormalTok{,}
\NormalTok{seed}\OperatorTok{=}\DecValTok{0}\NormalTok{,}
\NormalTok{shape}\OperatorTok{=}\NormalTok{(),}
\NormalTok{event}\OperatorTok{=}\DecValTok{0}\NormalTok{,}
\NormalTok{create\_obj}\OperatorTok{=}\VariableTok{False}\NormalTok{,}
\NormalTok{)}
\end{Highlighting}
\end{Shaded}

\begin{itemize}
\item
  \emph{total\_count} (int): The number of trials \emph{n}.
\item
  \emph{probs} (jnp.ndarray, optional): The probability of success
  \emph{p} for each trial. Must be between 0 and 1.
\item
  \emph{logits} (jnp.ndarray, optional): The log-odds of success for
  each trial. \texttt{probs\ =\ jax.nn.sigmoid(logits)}.
\item
  \emph{shape} (tuple): A multi-purpose argument for shaping. When
  \texttt{sample=False} (model building), this is used with
  \texttt{.expand(shape)} to set the distribution's batch shape. When
  \texttt{sample=True} (direct sampling), this is used as
  \texttt{sample\_shape} to draw a raw JAX array of the given shape.
\item
  \emph{event} (int): The number of batch dimensions to reinterpret as
  event dimensions (used in model building).
\item
  \emph{mask} (jnp.ndarray, bool, optional): Optional boolean array to
  mask observations.
\item
  \emph{create\_obj} (bool, optional): If True, returns the raw BI
  distribution object instead of creating a sample site. This is
  essential for building complex distributions like
  \texttt{MixtureSameFamily}.
\item
  \emph{sample} (bool, optional): A control-flow argument. If
  \texttt{True}, the function will directly sample a raw JAX array from
  the distribution, bypassing the BI model context. If \texttt{False},
  it will create a \texttt{BI.sample} site within a model. Defaults to
  \texttt{False}.
\item
  \emph{seed} (int, optional): An integer used to generate a JAX PRNGKey
  for reproducible sampling when \texttt{sample=True}. {[}7{]} This
  argument has no effect when \texttt{sample=False}, as randomness is
  handled by BI's inference engine. Defaults to 0.
\item
  \emph{obs} (jnp.ndarray, optional): The observed value for this random
  variable. If provided, the sample site is conditioned on this value,
  and the function returns the observed value. If \texttt{None}, the
  site is treated as a latent (unobserved) random variable. Defaults to
  \texttt{None}.
\item
  \emph{name} (str, optional): The name of the sample site in a BI
  model. This is used to uniquely identify the random variable. Defaults
  to `x'.
\end{itemize}

\paragraph{Returns:}\label{returns-8}

Binomial distribution object (for model building) when
\texttt{sample=False}. JAX array of samples drawn from the Binomial
distribution (for direct sampling) when \texttt{sample=True}. The raw BI
distribution object (for advanced use cases) when
\texttt{create\_obj=True}.

\paragraph{Example Usage:}\label{example-usage-8}

\begin{Shaded}
\begin{Highlighting}[]
\ImportTok{from}\NormalTok{ BI }\ImportTok{import}\NormalTok{ bi}
\NormalTok{m }\OperatorTok{=}\NormalTok{ bi(}\StringTok{\textquotesingle{}cpu\textquotesingle{}}\NormalTok{)}
\NormalTok{m.dist.binomial(total\_count}\OperatorTok{=}\DecValTok{10}\NormalTok{, probs}\OperatorTok{=}\FloatTok{0.5}\NormalTok{, sample}\OperatorTok{=}\VariableTok{True}\NormalTok{)}
\end{Highlighting}
\end{Shaded}

\paragraph{Wrapper of:
https://num.pyro.ai/en/stable/distributions.html\#binomial}\label{wrapper-of-httpsnum.pyro.aienstabledistributions.htmlbinomial}

\begin{center}\rule{0.5\linewidth}{0.5pt}\end{center}

\subsubsection{Binomial Logits}\label{binomial-logits}

The BinomialLogits distribution represents a binomial distribution
parameterized by logits. It is useful when the probability of success is
not directly known but is instead expressed as logits, which are the
natural logarithm of the odds ratio.

\[
P(X=k) = \binom{n}{k} \frac{e^{logits_k}}{1 + e^{logits_k}}
\]

\paragraph{Args:}\label{args-9}

\begin{Shaded}
\begin{Highlighting}[]
\NormalTok{bi.dist.binomial\_logits(}
\NormalTok{logits,}
\NormalTok{total\_count}\OperatorTok{=}\DecValTok{1}\NormalTok{,}
\NormalTok{validate\_args}\OperatorTok{=}\VariableTok{None}\NormalTok{,}
\NormalTok{name}\OperatorTok{=}\StringTok{\textquotesingle{}x\textquotesingle{}}\NormalTok{,}
\NormalTok{obs}\OperatorTok{=}\VariableTok{None}\NormalTok{,}
\NormalTok{mask}\OperatorTok{=}\VariableTok{None}\NormalTok{,}
\NormalTok{sample}\OperatorTok{=}\VariableTok{False}\NormalTok{,}
\NormalTok{seed}\OperatorTok{=}\DecValTok{0}\NormalTok{,}
\NormalTok{shape}\OperatorTok{=}\NormalTok{(),}
\NormalTok{event}\OperatorTok{=}\DecValTok{0}\NormalTok{,}
\NormalTok{create\_obj}\OperatorTok{=}\VariableTok{False}\NormalTok{,}
\NormalTok{)}
\end{Highlighting}
\end{Shaded}

logits (jnp.ndarray): Log-odds of each success. total\_count (int):
Number of trials.

\begin{itemize}
\item
  \emph{shape} (tuple): A multi-purpose argument for shaping. When
  \texttt{sample=False} (model building), this is used with
  \texttt{.expand(shape)} to set the distribution's batch shape. When
  \texttt{sample=True} (direct sampling), this is used as
  \texttt{sample\_shape} to draw a raw JAX array of the given shape.
\item
  \emph{event} (int): The number of batch dimensions to reinterpret as
  event dimensions (used in model building).
\item
  \emph{mask} (jnp.ndarray, bool): Optional boolean array to mask
  observations.
\item
  \emph{create\_obj} (bool): If True, returns the raw BI distribution
  object instead of creating a sample site. This is essential for
  building complex distributions like \texttt{MixtureSameFamily}.
\item
  \emph{sample} (bool, optional): A control-flow argument. If
  \texttt{True}, the function will directly sample a raw JAX array from
  the distribution, bypassing the BI model context. If \texttt{False},
  it will create a \texttt{BI.sample} site within a model. Defaults to
  \texttt{False}.
\item
  \emph{obs} (jnp.ndarray, optional): The observed value for this random
  variable. If provided, the sample site is conditioned on this value,
  and the function returns the observed value. If \texttt{None}, the
  site is treated as a latent (unobserved) random variable. Defaults to
  \texttt{None}.
\item
  \emph{name} (str, optional): The name of the sample site in a BI
  model. This is used to uniquely identify the random variable. Defaults
  to `x'.
\end{itemize}

\paragraph{Returns:}\label{returns-9}

\begin{itemize}
\item
  When \texttt{sample=False}: A BI BinomialLogits distribution object
  (for model building).
\item
  When \texttt{sample=True}: A JAX array of samples drawn from the
  BernoulliLogits distribution (for direct sampling).
\item
  When \texttt{create\_obj=True}: The raw BI BinomialLogits object (for
  advanced use cases).
\end{itemize}

\paragraph{Example Usage:}\label{example-usage-9}

\begin{Shaded}
\begin{Highlighting}[]
\ImportTok{from}\NormalTok{ BI }\ImportTok{import}\NormalTok{ bi}
\NormalTok{m }\OperatorTok{=}\NormalTok{ bi(}\StringTok{\textquotesingle{}cpu\textquotesingle{}}\NormalTok{)}
\NormalTok{m.dist.binomial\_logits(logits}\OperatorTok{=}\NormalTok{jnp.zeros(}\DecValTok{10}\NormalTok{), total\_count}\OperatorTok{=}\DecValTok{5}\NormalTok{, sample}\OperatorTok{=}\VariableTok{True}\NormalTok{)}
\end{Highlighting}
\end{Shaded}

\paragraph{Wrapper of:}\label{wrapper-of-7}

https://num.pyro.ai/en/stable/distributions.html\#binomialllogits

\begin{center}\rule{0.5\linewidth}{0.5pt}\end{center}

\subsubsection{Binomial Probs}\label{binomial-probs}

Samples from a Binomial distribution with specified probabilities for
each trial.

The Binomial distribution models the number of successes in a sequence
of independent Bernoulli trials, where each trial has the same
probability of success.

\[   
P(k) = \binom{n}{k} p^k (1-p)^{n-k}
\]

\paragraph{Args:}\label{args-10}

\begin{Shaded}
\begin{Highlighting}[]
\NormalTok{bi.dist.binomial\_probs(}
\NormalTok{probs,}
\NormalTok{total\_count}\OperatorTok{=}\DecValTok{1}\NormalTok{,}
\NormalTok{validate\_args}\OperatorTok{=}\VariableTok{None}\NormalTok{,}
\NormalTok{name}\OperatorTok{=}\StringTok{\textquotesingle{}x\textquotesingle{}}\NormalTok{,}
\NormalTok{obs}\OperatorTok{=}\VariableTok{None}\NormalTok{,}
\NormalTok{mask}\OperatorTok{=}\VariableTok{None}\NormalTok{,}
\NormalTok{sample}\OperatorTok{=}\VariableTok{False}\NormalTok{,}
\NormalTok{seed}\OperatorTok{=}\DecValTok{0}\NormalTok{,}
\NormalTok{shape}\OperatorTok{=}\NormalTok{(),}
\NormalTok{event}\OperatorTok{=}\DecValTok{0}\NormalTok{,}
\NormalTok{create\_obj}\OperatorTok{=}\VariableTok{False}\NormalTok{,}
\NormalTok{)}
\end{Highlighting}
\end{Shaded}

\begin{itemize}
\item
  \emph{probs} (jnp.ndarray): The probability of success for each trial.
  Must be between 0 and 1.
\item
  \emph{total\_count} (int): The number of trials in each sequence.
\item
  \emph{shape} (tuple): A multi-purpose argument for shaping. When
  \texttt{sample=False} (model building), this is used with
  \texttt{.expand(shape)} to set the distribution's batch shape. When
  \texttt{sample=True} (direct sampling), this is used as
  \texttt{sample\_shape} to draw a raw JAX array of the given shape.
\item
  \emph{event} (int): The number of batch dimensions to reinterpret as
  event dimensions (used in model building).
\item
  \emph{mask} (jnp.ndarray, bool): Optional boolean array to mask
  observations.
\item
  \emph{create\_obj} (bool): If True, returns the raw BI distribution
  object instead of creating a sample site. This is essential for
  building complex distributions like \texttt{MixtureSameFamily}.
\item
  \emph{sample} (bool, optional): A control-flow argument. If
  \texttt{True}, the function will directly sample a raw JAX array from
  the distribution, bypassing the BI model context. If \texttt{False},
  it will create a \texttt{BI.sample} site within a model. Defaults to
  \texttt{False}.
\item
  \emph{obs} (jnp.ndarray, optional): The observed value for this random
  variable. If provided, the sample site is conditioned on this value,
  and the function returns the observed value. If \texttt{None}, the
  site is treated as a latent (unobserved) random variable. Defaults to
  \texttt{None}.
\item
  \emph{name} (str, optional): The name of the sample site in a BI
  model. This is used to uniquely identify the random variable. Defaults
  to `x'.
\end{itemize}

\paragraph{Returns:}\label{returns-10}

\begin{itemize}
\item
  When \texttt{sample=False}: A BI BinomialProbs distribution object
  (for model building).
\item
  When \texttt{sample=True}: A JAX array of samples drawn from the
  BernoulliLogits distribution (for direct sampling).
\item
  When \texttt{create\_obj=True}: The raw BI BinomialLogits object (for
  advanced use cases).
\end{itemize}

\paragraph{Example Usage:}\label{example-usage-10}

\begin{Shaded}
\begin{Highlighting}[]
\ImportTok{from}\NormalTok{ BI }\ImportTok{import}\NormalTok{ bi}
\NormalTok{m }\OperatorTok{=}\NormalTok{ bi(}\StringTok{\textquotesingle{}cpu\textquotesingle{}}\NormalTok{)}
\NormalTok{m.dist.binomial\_probs(probs}\OperatorTok{=}\FloatTok{0.5}\NormalTok{, total\_count}\OperatorTok{=}\DecValTok{10}\NormalTok{, sample}\OperatorTok{=}\VariableTok{True}\NormalTok{)}
\end{Highlighting}
\end{Shaded}

\paragraph{Wrapper of:}\label{wrapper-of-8}

https://num.pyro.ai/en/stable/distributions.html\#binomialprobs

\begin{center}\rule{0.5\linewidth}{0.5pt}\end{center}

\subsubsection{Conditional Autoregressive
(CAR)}\label{conditional-autoregressive-car}

The CAR distribution models a vector of variables where each variable is
a linear combination of its neighbors in a graph.

\[   
p(x) = \prod_{i=1}^{K} \mathcal{N}(x_i | \mu_i, \Sigma_i)
\]

where \(\mu_i\) is a function of the values of the neighbors of site
\(i\) and \(\Sigma_i\) is the variance of site \(i\).

.. note::

The CAR distribution is a special case of the multivariate normal
distribution. It is used to model spatial data, such as temperature or
precipitation.

\paragraph{Args:}\label{args-11}

\begin{Shaded}
\begin{Highlighting}[]
\NormalTok{bi.dist.car(}
\NormalTok{loc,}
\NormalTok{correlation,}
\NormalTok{conditional\_precision,}
\NormalTok{adj\_matrix,}
\NormalTok{is\_sparse}\OperatorTok{=}\VariableTok{False}\NormalTok{,}
\NormalTok{validate\_args}\OperatorTok{=}\VariableTok{None}\NormalTok{,}
\NormalTok{name}\OperatorTok{=}\StringTok{\textquotesingle{}x\textquotesingle{}}\NormalTok{,}
\NormalTok{obs}\OperatorTok{=}\VariableTok{None}\NormalTok{,}
\NormalTok{mask}\OperatorTok{=}\VariableTok{None}\NormalTok{,}
\NormalTok{sample}\OperatorTok{=}\VariableTok{False}\NormalTok{,}
\NormalTok{seed}\OperatorTok{=}\DecValTok{0}\NormalTok{,}
\NormalTok{shape}\OperatorTok{=}\NormalTok{(),}
\NormalTok{event}\OperatorTok{=}\DecValTok{0}\NormalTok{,}
\NormalTok{create\_obj}\OperatorTok{=}\VariableTok{False}\NormalTok{,}
\NormalTok{)}
\end{Highlighting}
\end{Shaded}

\begin{itemize}
\item
  \emph{loc} (Union{[}float, Array{]}): Mean of the distribution.
\item
  \emph{correlation} (Union{[}float, Array{]}): Correlation between
  variables.
\item
  \emph{conditional\_precision} (Union{[}float, Array{]}): Precision of
  the distribution.
\item
  \emph{adj\_matrix} (Union{[}Array, scipy.sparse.spmatrix{]}):
  Adjacency matrix defining the graph. is\_sparse (bool): Whether the
  adjacency matrix is sparse. Defaults to False.
\item
  *validate\_args/ (bool): Whether to validate arguments. Defaults to
  None.
\item
  \emph{sample} (bool, optional): A control-flow argument. If
  \texttt{True}, the function will directly sample a raw JAX array from
  the distribution, bypassing the BI model context. If \texttt{False},
  it will create a \texttt{BI.sample} site within a model. Defaults to
  \texttt{False}.
\item
  \emph{obs} (jnp.ndarray, optional): The observed value for this random
  variable. If provided, the sample site is conditioned on this value,
  and the function returns the observed value. If \texttt{None}, the
  site is treated as a latent (unobserved) random variable. Defaults to
  \texttt{None}.
\item
  \emph{name} (str, optional): The name of the sample site in a BI
  model. This is used to uniquely identify the random variable. Defaults
  to `x'.
\end{itemize}

\paragraph{Returns:}\label{returns-11}

\begin{itemize}
\item
  When \texttt{sample=False}: A BI CAR distribution object (for model
  building).
\item
  When \texttt{sample=True}: A JAX array of samples drawn from the
  BernoulliLogits distribution (for direct sampling).
\item
  When \texttt{create\_obj=True}: The raw BI CAR object (for advanced
  use cases).
\end{itemize}

\begin{center}\rule{0.5\linewidth}{0.5pt}\end{center}

\subsubsection{Categorical
distribution.}\label{categorical-distribution.}

The Categorical distribution, also known as the multinomial
distribution, describes the probability of different outcomes from a
finite set of possibilities. It is commonly used to model discrete
choices or classifications.

\[   
P(k) = \frac{e^{\log(p_k)}}{\sum_{j=1}^{K} e^{\log(p_j)}}
\]

where \(p_k\) is the probability of outcome \(k\), and the sum is over
all possible outcomes.

\paragraph{Args:}\label{args-12}

\begin{Shaded}
\begin{Highlighting}[]
\NormalTok{bi.dist.categorical(}
\NormalTok{probs}\OperatorTok{=}\VariableTok{None}\NormalTok{,}
\NormalTok{logits}\OperatorTok{=}\VariableTok{None}\NormalTok{,}
\NormalTok{validate\_args}\OperatorTok{=}\VariableTok{None}\NormalTok{,}
\NormalTok{name}\OperatorTok{=}\StringTok{\textquotesingle{}x\textquotesingle{}}\NormalTok{,}
\NormalTok{obs}\OperatorTok{=}\VariableTok{None}\NormalTok{,}
\NormalTok{mask}\OperatorTok{=}\VariableTok{None}\NormalTok{,}
\NormalTok{sample}\OperatorTok{=}\VariableTok{False}\NormalTok{,}
\NormalTok{seed}\OperatorTok{=}\DecValTok{0}\NormalTok{,}
\NormalTok{shape}\OperatorTok{=}\NormalTok{(),}
\NormalTok{event}\OperatorTok{=}\DecValTok{0}\NormalTok{,}
\NormalTok{create\_obj}\OperatorTok{=}\VariableTok{False}\NormalTok{,}
\NormalTok{)}
\end{Highlighting}
\end{Shaded}

\begin{itemize}
\item
  \emph{probs} (jnp.ndarray): A 1D array of probabilities for each
  category. Must sum to 1.
\item
  \emph{shape} (tuple): A multi-purpose argument for shaping. When
  \texttt{sample=False} (model building), this is used with
  \texttt{.expand(shape)} to set the distribution's batch shape. When
  \texttt{sample=True} (direct sampling), this is used as
  \texttt{sample\_shape} to draw a raw JAX array of the given shape.
\item
  \emph{event} (int): The number of batch dimensions to reinterpret as
  event dimensions (used in model building).
\item
  \emph{mask} (jnp.ndarray, bool): Optional boolean array to mask
  observations.
\item
  \emph{create\_obj} (bool): If True, returns the raw BI distribution
  object instead of creating a sample site. This is essential for
  building complex distributions like \texttt{MixtureSameFamily}.
\item
  \emph{sample} (bool, optional): A control-flow argument. If
  \texttt{True}, the function will directly sample a raw JAX array from
  the distribution, bypassing the BI model context. If \texttt{False},
  it will create a \texttt{BI.sample} site within a model. Defaults to
  \texttt{False}.
\item
  \emph{seed} (int, optional): An integer used to generate a JAX PRNGKey
  for reproducible sampling when \texttt{sample=True}. {[}7{]} This
  argument has no effect when \texttt{sample=False}, as randomness is
  handled by BI's inference engine. Defaults to 0.
\item
  \emph{obs} (jnp.ndarray, optional): The observed value for this random
  variable. If provided, the sample site is conditioned on this value,
  and the function returns the observed value. If \texttt{None}, the
  site is treated as a latent (unobserved) random variable. Defaults to
  \texttt{None}.
\item
  \emph{name} (str, optional): The name of the sample site in a BI
  model. This is used to uniquely identify the random variable. Defaults
  to `x'.
\end{itemize}

\paragraph{Returns:}\label{returns-12}

\begin{itemize}
\item
  When \texttt{sample=False}: A BI Categorical distribution object (for
  model building).
\item
  When \texttt{sample=True}: A JAX array of samples drawn from the
  Categorical distribution (for direct sampling).
\item
  When \texttt{create\_obj=True}: The raw BI distribution object (for
  advanced use cases).
\end{itemize}

\paragraph{Example Usage:}\label{example-usage-11}

\begin{Shaded}
\begin{Highlighting}[]
\ImportTok{from}\NormalTok{ BI }\ImportTok{import}\NormalTok{ bi}
\NormalTok{m }\OperatorTok{=}\NormalTok{ bi(}\StringTok{\textquotesingle{}cpu\textquotesingle{}}\NormalTok{)}
\NormalTok{m.dist.categorical(probs}\OperatorTok{=}\NormalTok{jnp.array([}\FloatTok{0.2}\NormalTok{, }\FloatTok{0.3}\NormalTok{, }\FloatTok{0.5}\NormalTok{]), sample}\OperatorTok{=}\VariableTok{True}\NormalTok{)}
\end{Highlighting}
\end{Shaded}

\paragraph{Wrapper of:
https://num.pyro.ai/en/stable/distributions.html\#categorical}\label{wrapper-of-httpsnum.pyro.aienstabledistributions.htmlcategorical}

\begin{center}\rule{0.5\linewidth}{0.5pt}\end{center}

\subsubsection{Categorical Logits}\label{categorical-logits}

Samples from a Categorical distribution with logits. This distribution
represents a discrete probability distribution over a finite set of
outcomes, where the probabilities are determined by the logits. The
probability of each outcome is given by the softmax function applied to
the logits.

\[
P(k) = \frac{e^{logits_k}}{\sum_{j=1}^{K} e^{logits_j}}
\]

\paragraph{Args:}\label{args-13}

\begin{Shaded}
\begin{Highlighting}[]
\NormalTok{bi.dist.categorical\_logits(}
\NormalTok{logits,}
\NormalTok{validate\_args}\OperatorTok{=}\VariableTok{None}\NormalTok{,}
\NormalTok{name}\OperatorTok{=}\StringTok{\textquotesingle{}x\textquotesingle{}}\NormalTok{,}
\NormalTok{obs}\OperatorTok{=}\VariableTok{None}\NormalTok{,}
\NormalTok{mask}\OperatorTok{=}\VariableTok{None}\NormalTok{,}
\NormalTok{sample}\OperatorTok{=}\VariableTok{False}\NormalTok{,}
\NormalTok{seed}\OperatorTok{=}\DecValTok{0}\NormalTok{,}
\NormalTok{shape}\OperatorTok{=}\NormalTok{(),}
\NormalTok{event}\OperatorTok{=}\DecValTok{0}\NormalTok{,}
\NormalTok{create\_obj}\OperatorTok{=}\VariableTok{False}\NormalTok{,}
\NormalTok{)}
\end{Highlighting}
\end{Shaded}

\begin{itemize}
\item
  \emph{logits} (jnp.ndarray): Log-odds of each category.
\item
  \emph{shape} (tuple): A multi-purpose argument for shaping. When
  \texttt{sample=False} (model building), this is used with
  \texttt{.expand(shape)} to set the distribution's batch shape. When
  \texttt{sample=True} (direct sampling), this is used as
  \texttt{sample\_shape} to draw a raw JAX array of the given shape.
\item
  \emph{event} (int): The number of batch dimensions to reinterpret as
  event dimensions (used in model building).
\item
  \emph{mask} (jnp.ndarray, bool): Optional boolean array to mask
  observations.
\item
  \emph{create\_obj} (bool): If True, returns the raw BI distribution
  object instead of creating a sample site. This is essential for
  building complex distributions like \texttt{MixtureSameFamily}.
\item
  \emph{sample} (bool, optional): A control-flow argument. If
  \texttt{True}, the function will directly sample a raw JAX array from
  the distribution, bypassing the BI model context. If \texttt{False},
  it will create a \texttt{BI.sample} site within a model. Defaults to
  \texttt{False}.
\item
  \emph{obs} (jnp.ndarray, optional): The observed value for this random
  variable. If provided, the sample site is conditioned on this value,
  and the function returns the observed value. If \texttt{None}, the
  site is treated as a latent (unobserved) random variable. Defaults to
  \texttt{None}.
\item
  \emph{name} (str, optional): The name of the sample site in a BI
  model. This is used to uniquely identify the random variable. Defaults
  to `x'.
\end{itemize}

\paragraph{Returns:}\label{returns-13}

\begin{itemize}
\item
  When \texttt{sample=False}: A BI CategoricalLogits distribution object
  (for model building).
\item
  When \texttt{sample=True}: A JAX array of samples drawn from the
  BernoulliLogits distribution (for direct sampling).
\item
  When \texttt{create\_obj=True}: The raw BI CategoricalLogits object
  (for advanced use cases).
\end{itemize}

\paragraph{Example Usage:}\label{example-usage-12}

\begin{Shaded}
\begin{Highlighting}[]
\ImportTok{from}\NormalTok{ BI }\ImportTok{import}\NormalTok{ bi}
\NormalTok{m }\OperatorTok{=}\NormalTok{ bi(}\StringTok{\textquotesingle{}cpu\textquotesingle{}}\NormalTok{)}
\NormalTok{m.dist.categorical\_logits(logits}\OperatorTok{=}\NormalTok{jnp.zeros(}\DecValTok{5}\NormalTok{), sample}\OperatorTok{=}\VariableTok{True}\NormalTok{)}
\end{Highlighting}
\end{Shaded}

\paragraph{Wrapper of:}\label{wrapper-of-9}

https://num.pyro.ai/en/stable/distributions.html\#categoricallogits

\begin{center}\rule{0.5\linewidth}{0.5pt}\end{center}

\subsubsection{Categorical Probs
distribution.}\label{categorical-probs-distribution.}

Samples from a Categorical distribution.

The Categorical distribution is a discrete probability distribution that
represents the probability of each outcome from a finite set of
possibilities. It is often used to model the outcome of a random process
with a fixed number of possible outcomes, such as the roll of a die or
the selection of an item from a list.

\[   
P(x) = \frac{probs_i}{\sum_{k=1}^{K} probs_k}
\]

\paragraph{Args:}\label{args-14}

\begin{Shaded}
\begin{Highlighting}[]
\NormalTok{bi.dist.categorical\_probs(}
\NormalTok{probs,}
\NormalTok{validate\_args}\OperatorTok{=}\VariableTok{None}\NormalTok{,}
\NormalTok{name}\OperatorTok{=}\StringTok{\textquotesingle{}x\textquotesingle{}}\NormalTok{,}
\NormalTok{obs}\OperatorTok{=}\VariableTok{None}\NormalTok{,}
\NormalTok{mask}\OperatorTok{=}\VariableTok{None}\NormalTok{,}
\NormalTok{sample}\OperatorTok{=}\VariableTok{False}\NormalTok{,}
\NormalTok{seed}\OperatorTok{=}\DecValTok{0}\NormalTok{,}
\NormalTok{shape}\OperatorTok{=}\NormalTok{(),}
\NormalTok{event}\OperatorTok{=}\DecValTok{0}\NormalTok{,}
\NormalTok{create\_obj}\OperatorTok{=}\VariableTok{False}\NormalTok{,}
\NormalTok{)}
\end{Highlighting}
\end{Shaded}

\begin{itemize}
\item
  \emph{probs} (jnp.ndarray): Probabilities for each category. Must sum
  to 1.
\item
  \emph{shape} (tuple): A multi-purpose argument for shaping. When
  \texttt{sample=False} (model building), this is used with
  \texttt{.expand(shape)} to set the distribution's batch shape. When
  \texttt{sample=True} (direct sampling), this is used as
  \texttt{sample\_shape} to draw a raw JAX array of the given shape.
\item
  \emph{event} (int): The number of batch dimensions to reinterpret as
  event dimensions (used in model building).
\item
  \emph{mask} (jnp.ndarray, bool): Optional boolean array to mask
  observations.
\item
  \emph{create\_obj} (bool): If True, returns the raw BI distribution
  object instead of creating a sample site. This is essential for
  building complex distributions like \texttt{MixtureSameFamily}.
\item
  \emph{sample} (bool, optional): A control-flow argument. If
  \texttt{True}, the function will directly sample a raw JAX array from
  the distribution, bypassing the BI model context. If \texttt{False},
  it will create a \texttt{BI.sample} site within a model. Defaults to
  \texttt{False}.
\item
  \emph{obs} (jnp.ndarray, optional): The observed value for this random
  variable. If provided, the sample site is conditioned on this value,
  and the function returns the observed value. If \texttt{None}, the
  site is treated as a latent (unobserved) random variable. Defaults to
  \texttt{None}.
\item
  \emph{name} (str, optional): The name of the sample site in a BI
  model. This is used to uniquely identify the random variable. Defaults
  to `x'.
\end{itemize}

\paragraph{Returns:}\label{returns-14}

\begin{itemize}
\item
  When \texttt{sample=False}: A BI CategoricalProbs distribution object
  (for model building).
\item
  When \texttt{sample=True}: A JAX array of samples drawn from the
  BernoulliLogits distribution (for direct sampling).
\item
  When \texttt{create\_obj=True}: The raw BI CategoricalProbs object
  (for advanced use cases).
\end{itemize}

\paragraph{Example Usage:}\label{example-usage-13}

\begin{Shaded}
\begin{Highlighting}[]
\ImportTok{from}\NormalTok{ BI }\ImportTok{import}\NormalTok{ bi}
\NormalTok{m }\OperatorTok{=}\NormalTok{ bi(}\StringTok{\textquotesingle{}cpu\textquotesingle{}}\NormalTok{)}
\NormalTok{m.dist.categorical\_probs(probs}\OperatorTok{=}\NormalTok{jnp.array([}\FloatTok{0.2}\NormalTok{, }\FloatTok{0.3}\NormalTok{, }\FloatTok{0.5}\NormalTok{]), sample}\OperatorTok{=}\VariableTok{True}\NormalTok{)}
\end{Highlighting}
\end{Shaded}

\paragraph{Wrapper of:}\label{wrapper-of-10}

https://num.pyro.ai/en/stable/distributions.html\#categoricalprobs

\begin{center}\rule{0.5\linewidth}{0.5pt}\end{center}

\subsubsection{Cauchy}\label{cauchy}

Samples from a Cauchy

The Cauchy distribution, also known as the Lorentz distribution, is a
continuous probability distribution that arises frequently in various
fields, including physics and statistics. It is characterized by its
heavy tails, which extend indefinitely.

\[
f(x) = \frac{1}{\pi \gamma} \left[ \frac{\gamma^2}{(x - \mu)^2 + \gamma^2} \right]
\]

\paragraph{Args:}\label{args-15}

\begin{Shaded}
\begin{Highlighting}[]
\NormalTok{bi.dist.cauchy(}
\NormalTok{loc}\OperatorTok{=}\FloatTok{0.0}\NormalTok{,}
\NormalTok{scale}\OperatorTok{=}\FloatTok{1.0}\NormalTok{,}
\NormalTok{validate\_args}\OperatorTok{=}\VariableTok{None}\NormalTok{,}
\NormalTok{name}\OperatorTok{=}\StringTok{\textquotesingle{}x\textquotesingle{}}\NormalTok{,}
\NormalTok{obs}\OperatorTok{=}\VariableTok{None}\NormalTok{,}
\NormalTok{mask}\OperatorTok{=}\VariableTok{None}\NormalTok{,}
\NormalTok{sample}\OperatorTok{=}\VariableTok{False}\NormalTok{,}
\NormalTok{seed}\OperatorTok{=}\DecValTok{0}\NormalTok{,}
\NormalTok{shape}\OperatorTok{=}\NormalTok{(),}
\NormalTok{event}\OperatorTok{=}\DecValTok{0}\NormalTok{,}
\NormalTok{create\_obj}\OperatorTok{=}\VariableTok{False}\NormalTok{,}
\NormalTok{)}
\end{Highlighting}
\end{Shaded}

\begin{itemize}
\item
  \emph{loc} (jnp.ndarray or float, optional): Location parameter.
  Defaults to 0.0.
\item
  \emph{sample} (jnp.ndarray or float, optional): Scale parameter. Must
  be positive. Defaults to 1.0.
\item
  \emph{shape} (tuple): A multi-purpose argument for shaping. When
  \texttt{sample=False} (model building), this is used with
  \texttt{.expand(shape)} to set the distribution's batch shape. When
  \texttt{sample=True} (direct sampling), this is used as
  \texttt{sample\_shape} to draw a raw JAX array of the given shape.
\item
  \emph{event} (int, optional): The number of batch dimensions to
  reinterpret as event dimensions (used in model building). Defaults to
  None.
\item
  \emph{mask} (jnp.ndarray, bool, optional): Optional boolean array to
  mask observations. Defaults to None.
\item
  \emph{create\_obj} (bool, optional): If True, returns the raw BI
  distribution object instead of creating a sample site. This is
  essential for building complex distributions like
  \texttt{MixtureSameFamily}. Defaults to False.
\item
  \emph{sample} (bool, optional): A control-flow argument. If
  \texttt{True}, the function will directly sample a raw JAX array from
  the distribution, bypassing the BI model context. If \texttt{False},
  it will create a \texttt{BI.sample} site within a model. Defaults to
  \texttt{False}.
\item
  \emph{obs} (jnp.ndarray, optional): The observed value for this random
  variable. If provided, the sample site is conditioned on this value,
  and the function returns the observed value. If \texttt{None}, the
  site is treated as a latent (unobserved) random variable. Defaults to
  \texttt{None}.
\item
  \emph{name} (str, optional): The name of the sample site in a BI
  model. This is used to uniquely identify the random variable. Defaults
  to `x'.
\end{itemize}

\paragraph{Returns:}\label{returns-15}

\begin{itemize}
\item
  When \texttt{sample=False}: A BI Cauchy distribution object (for model
  building).
\item
  When \texttt{sample=True}: A JAX array of samples drawn from the
  Cauchy distribution (for direct sampling).
\item
  When \texttt{create\_obj=True}: The raw BI distribution object (for
  advanced use cases).
\end{itemize}

\paragraph{Example Usage:}\label{example-usage-14}

\begin{Shaded}
\begin{Highlighting}[]
\ImportTok{from}\NormalTok{ BI }\ImportTok{import}\NormalTok{ bi}
\NormalTok{m }\OperatorTok{=}\NormalTok{ bi(}\StringTok{\textquotesingle{}cpu\textquotesingle{}}\NormalTok{)}
\NormalTok{m.dist.cauchy(loc}\OperatorTok{=}\FloatTok{0.0}\NormalTok{, scale}\OperatorTok{=}\FloatTok{1.0}\NormalTok{, sample}\OperatorTok{=}\VariableTok{True}\NormalTok{)}
\end{Highlighting}
\end{Shaded}

\paragraph{Wrapper of:}\label{wrapper-of-11}

https://num.pyro.ai/en/stable/distributions.html\#cauchy

\begin{center}\rule{0.5\linewidth}{0.5pt}\end{center}

\subsubsection{Chi-squared}\label{chi-squared}

The Chi-squared distribution is a continuous probability distribution
that arises frequently in hypothesis testing, particularly in ANOVA and
chi-squared tests. It is defined by a single positive parameter, degrees
of freedom (df), which determines the shape of the distribution.

\[   
p(x; df) = \frac{1}{2^{df/2} \Gamma(df/2)} x^{df/2 - 1} e^{-x/2}
\]

\paragraph{Args:}\label{args-16}

\begin{Shaded}
\begin{Highlighting}[]
\NormalTok{bi.dist.chi2(}
\NormalTok{df,}
\NormalTok{validate\_args}\OperatorTok{=}\VariableTok{None}\NormalTok{,}
\NormalTok{name}\OperatorTok{=}\StringTok{\textquotesingle{}x\textquotesingle{}}\NormalTok{,}
\NormalTok{obs}\OperatorTok{=}\VariableTok{None}\NormalTok{,}
\NormalTok{mask}\OperatorTok{=}\VariableTok{None}\NormalTok{,}
\NormalTok{sample}\OperatorTok{=}\VariableTok{False}\NormalTok{,}
\NormalTok{seed}\OperatorTok{=}\DecValTok{0}\NormalTok{,}
\NormalTok{shape}\OperatorTok{=}\NormalTok{(),}
\NormalTok{event}\OperatorTok{=}\DecValTok{0}\NormalTok{,}
\NormalTok{create\_obj}\OperatorTok{=}\VariableTok{False}\NormalTok{,}
\NormalTok{)}
\end{Highlighting}
\end{Shaded}

df (jnp.ndarray): Degrees of freedom. Must be positive.

\begin{itemize}
\item
  \emph{shape} (tuple): A multi-purpose argument for shaping. When
  \texttt{sample=False} (model building), this is used with
  \texttt{.expand(shape)} to set the distribution's batch shape. When
  \texttt{sample=True} (direct sampling), this is used as
  \texttt{sample\_shape} to draw a raw JAX array of the given shape.
\item
  \emph{event} (int): The number of batch dimensions to reinterpret as
  event dimensions (used in model building).
\item
  \emph{mask} (jnp.ndarray, bool): Optional boolean array to mask
  observations.
\item
  \emph{create\_obj} (bool): If True, returns the raw BI distribution
  object instead of creating a sample site. This is essential for
  building complex distributions like \texttt{MixtureSameFamily}.
\item
  \emph{sample} (bool, optional): A control-flow argument. If
  \texttt{True}, the function will directly sample a raw JAX array from
  the distribution, bypassing the BI model context. If \texttt{False},
  it will create a \texttt{BI.sample} site within a model. Defaults to
  \texttt{False}.
\item
  \emph{obs} (jnp.ndarray, optional): The observed value for this random
  variable. If provided, the sample site is conditioned on this value,
  and the function returns the observed value. If \texttt{None}, the
  site is treated as a latent (unobserved) random variable. Defaults to
  \texttt{None}.
\item
  \emph{name} (str, optional): The name of the sample site in a BI
  model. This is used to uniquely identify the random variable. Defaults
  to `x'.
\end{itemize}

\paragraph{Returns:}\label{returns-16}

\begin{itemize}
\item
  When \texttt{sample=False}: A BI Chi2 distribution object (for model
  building).
\item
  When \texttt{sample=True}: A JAX array of samples drawn from the
  BernoulliLogits distribution (for direct sampling).
\item
  When \texttt{create\_obj=True}: The raw BI Chi2 object (for advanced
  use cases).
\end{itemize}

\paragraph{Example Usage:}\label{example-usage-15}

\begin{Shaded}
\begin{Highlighting}[]
\ImportTok{from}\NormalTok{ BI }\ImportTok{import}\NormalTok{ bi}
\NormalTok{m }\OperatorTok{=}\NormalTok{ bi(}\StringTok{\textquotesingle{}cpu)}
\ErrorTok{m.dist.chi2}\NormalTok{(df}\OperatorTok{=}\FloatTok{3.0}\NormalTok{, sample }\OperatorTok{=} \VariableTok{True}\NormalTok{)}
\end{Highlighting}
\end{Shaded}

\paragraph{Wrapper of:}\label{wrapper-of-12}

https://num.pyro.ai/en/stable/distributions.html\#chi2

\begin{center}\rule{0.5\linewidth}{0.5pt}\end{center}

\subsubsection{Circulant Normal Multivariate
normal}\label{circulant-normal-multivariate-normal}

Circulant Normal Multivariate normal distribution with covariance matrix
\(\mathbf{C}\) that is positive-definite and circulant {[}1{]}, i.e.,
has periodic boundary conditions. The density of a sample
\(\mathbf{x}\in\mathbb{R}^n\) is the standard multivariate normal
density

\[
p\left(\mathbf{x}\mid\boldsymbol{\mu},\mathbf{C}\right) =
\frac{\left(\mathrm{det}\,\mathbf{C}\right)^{-1/2}}{\left(2\pi\right)^{n / 2}}
\exp\left(-\frac{1}{2}\left(\mathbf{x}-\boldsymbol{\mu}\right)^\intercal
\mathbf{C}^{-1}\left(\mathbf{x}-\boldsymbol{\mu}\right)\right),
\]

where \(\mathrm{det}\) denotes the determinant and \(^\intercal\) the
transpose. Circulant matrices can be diagnolized efficiently using the
discrete Fourier transform {[}1{]}, allowing the log likelihood to be
evaluated in \(n \log n\) time for \(n\) observations {[}2{]}.

\begin{itemize}
\item
  \emph{loc:} Mean of the distribution \(\boldsymbol{\mu}\).
\item
  \emph{covariance\_row:} First row of the circulant covariance matrix
  \(\boldsymbol{C}\). Because of periodic boundary conditions, the
  covariance matrix is fully determined by its first row (see
  :func:\texttt{jax.scipy.linalg.toeplitz} for further details).
\item
  \emph{covariance\_rfft:} Real part of the real fast Fourier transform
  of :code:\texttt{covariance\_row}, the first row of the circulant
  covariance matrix \(\boldsymbol{C}\).
\item
  \emph{sample} (bool, optional): A control-flow argument. If
  \texttt{True}, the function will directly sample a raw JAX array from
  the distribution, bypassing the BI model context. If \texttt{False},
  it will create a \texttt{BI.sample} site within a model. Defaults to
  \texttt{False}.
\item
  \emph{obs} (jnp.ndarray, optional): The observed value for this random
  variable. If provided, the sample site is conditioned on this value,
  and the function returns the observed value. If \texttt{None}, the
  site is treated as a latent (unobserved) random variable. Defaults to
  \texttt{None}.
\item
  \emph{name} (str, optional): The name of the sample site in a BI
  model. This is used to uniquely identify the random variable. Defaults
  to `x'.
\end{itemize}

References:

\begin{enumerate}
\def\labelenumi{\arabic{enumi}.}
\item
  Wikipedia. (n.d.). Circulant matrix. Retrieved March 6, 2025, from
  https://en.wikipedia.org/wiki/Circulant\_matrix
\item
  Wood, A. T. A., \& Chan, G. (1994). Simulation of Stationary Gaussian
  Processes in \(\left[0, 1\right]^d\). \emph{Journal of Computational
  and Graphical Statistics}, 3(4), 409--432.
  https://doi.org/10.1080/10618600.1994.10474655
\end{enumerate}

\paragraph{Args:}\label{args-17}

\begin{Shaded}
\begin{Highlighting}[]
\NormalTok{bi.dist.circulant\_normal(}
\NormalTok{loc: jax.Array,}
\NormalTok{covariance\_row: jax.Array }\OperatorTok{=} \VariableTok{None}\NormalTok{,}
\NormalTok{covariance\_rfft: jax.Array }\OperatorTok{=} \VariableTok{None}\NormalTok{,}
\NormalTok{validate\_args}\OperatorTok{=}\VariableTok{None}\NormalTok{,}
\NormalTok{name}\OperatorTok{=}\StringTok{\textquotesingle{}x\textquotesingle{}}\NormalTok{,}
\NormalTok{obs}\OperatorTok{=}\VariableTok{None}\NormalTok{,}
\NormalTok{mask}\OperatorTok{=}\VariableTok{None}\NormalTok{,}
\NormalTok{sample}\OperatorTok{=}\VariableTok{False}\NormalTok{,}
\NormalTok{seed}\OperatorTok{=}\DecValTok{0}\NormalTok{,}
\NormalTok{shape}\OperatorTok{=}\NormalTok{(),}
\NormalTok{event}\OperatorTok{=}\DecValTok{0}\NormalTok{,}
\NormalTok{create\_obj}\OperatorTok{=}\VariableTok{False}\NormalTok{,}
\NormalTok{)}
\end{Highlighting}
\end{Shaded}

\begin{itemize}
\item
  \emph{loc} : jnp.ndarray Mean of the distribution
  \(\boldsymbol{\mu}\).
\item
  \emph{covariance\_row} : jnp.ndarray, optional. First row of the
  circulant covariance matrix \(\mathbf{C}\). Defaults to None.
\item
  \emph{covariance\_rfft} : jnp.ndarray, optional Real part of the real
  fast Fourier transform of :code:\texttt{covariance\_row}. Defaults to
  None.
\end{itemize}

\paragraph{Returns:}\label{returns-17}

\begin{itemize}
\item
  When \texttt{sample=False}: A BI Circulant Normal Distribution
  distribution object (for model building).
\item
  When \texttt{sample=True}: A JAX array of samples drawn from the
  BernoulliLogits distribution (for direct sampling).
\item
  When \texttt{create\_obj=True}: The raw BI Circulant Normal
  Distribution object (for advanced use cases).
\end{itemize}

\paragraph{Wrapper of:
https://num.pyro.ai/en/stable/distributions.html\#normal}\label{wrapper-of-httpsnum.pyro.aienstabledistributions.htmlnormal}

\begin{center}\rule{0.5\linewidth}{0.5pt}\end{center}

\subsubsection{Delta}\label{delta}

The Delta distribution, also known as a point mass distribution, assigns
probability 1 to a single point and 0 elsewhere. It's useful for
representing deterministic variables or as a building block for more
complex distributions.

\[   P(x = v) = 1
\]

\paragraph{Args:}\label{args-18}

\begin{Shaded}
\begin{Highlighting}[]
\NormalTok{bi.dist.delta(}
\NormalTok{v}\OperatorTok{=}\FloatTok{0.0}\NormalTok{,}
\NormalTok{log\_density}\OperatorTok{=}\FloatTok{0.0}\NormalTok{,}
\NormalTok{event\_dim}\OperatorTok{=}\DecValTok{0}\NormalTok{,}
\NormalTok{validate\_args}\OperatorTok{=}\VariableTok{None}\NormalTok{,}
\NormalTok{name}\OperatorTok{=}\StringTok{\textquotesingle{}x\textquotesingle{}}\NormalTok{,}
\NormalTok{obs}\OperatorTok{=}\VariableTok{None}\NormalTok{,}
\NormalTok{mask}\OperatorTok{=}\VariableTok{None}\NormalTok{,}
\NormalTok{sample}\OperatorTok{=}\VariableTok{False}\NormalTok{,}
\NormalTok{seed}\OperatorTok{=}\DecValTok{0}\NormalTok{,}
\NormalTok{shape}\OperatorTok{=}\NormalTok{(),}
\NormalTok{event}\OperatorTok{=}\DecValTok{0}\NormalTok{,}
\NormalTok{create\_obj}\OperatorTok{=}\VariableTok{False}\NormalTok{,}
\NormalTok{)}
\end{Highlighting}
\end{Shaded}

v (jnp.ndarray): The location of the point mass. log\_density (float,
optional): The log probability density of the point mass. This is
primarily for creating distributions that are non-normalized or for
specific advanced use cases. For a standard delta distribution, this
should be 0. Defaults to 0.0.\\
event\_dim (int, optional): The number of rightmost dimensions of
\texttt{v} to interpret as event dimensions. Defaults to 0. -
\emph{shape} (tuple): A multi-purpose argument for shaping. When
\texttt{sample=False} (model building), this is used with
\texttt{.expand(shape)} to set the distribution's batch shape. When
\texttt{sample=True} (direct sampling), this is used as
\texttt{sample\_shape} to draw a raw JAX array of the given shape.

\begin{itemize}
\item
  \emph{event} (int): The number of batch dimensions to reinterpret as
  event dimensions (used in model building).
\item
  \emph{mask} (jnp.ndarray, bool): Optional boolean array to mask
  observations.
\item
  \emph{create\_obj} (bool): If True, returns the raw BI distribution
  object instead of creating a sample site. This is essential for
  building complex distributions like \texttt{MixtureSameFamily}.
\item
  \emph{sample} (bool, optional): A control-flow argument. If
  \texttt{True}, the function will directly sample a raw JAX array from
  the distribution, bypassing the BI model context. If \texttt{False},
  it will create a \texttt{BI.sample} site within a model. Defaults to
  \texttt{False}.
\item
  \emph{obs} (jnp.ndarray, optional): The observed value for this random
  variable. If provided, the sample site is conditioned on this value,
  and the function returns the observed value. If \texttt{None}, the
  site is treated as a latent (unobserved) random variable. Defaults to
  \texttt{None}.
\item
  \emph{name} (str, optional): The name of the sample site in a BI
  model. This is used to uniquely identify the random variable. Defaults
  to `x'.
\end{itemize}

\paragraph{Returns:}\label{returns-18}

\begin{itemize}
\item
  When \texttt{sample=False}: A BI Circulant Delta Distribution
  distribution object (for model building).
\item
  When \texttt{sample=True}: A JAX array of samples drawn from the
  BernoulliLogits distribution (for direct sampling).
\item
  When \texttt{create\_obj=True}: The raw BI Circulant Delta
  Distribution object (for advanced use cases).
\end{itemize}

\paragraph{Example Usage:}\label{example-usage-16}

\begin{Shaded}
\begin{Highlighting}[]
\ImportTok{from}\NormalTok{ BI }\ImportTok{import}\NormalTok{ bi}
\NormalTok{m }\OperatorTok{=}\NormalTok{ bi(}\StringTok{\textquotesingle{}cpu\textquotesingle{}}\NormalTok{)}
\NormalTok{m.dist.delta(v}\OperatorTok{=}\FloatTok{0.0}\NormalTok{, sample}\OperatorTok{=}\VariableTok{True}\NormalTok{)}
\end{Highlighting}
\end{Shaded}

\paragraph{Wrapper of:}\label{wrapper-of-13}

https://num.pyro.ai/en/stable/distributions.html\#delta

\begin{center}\rule{0.5\linewidth}{0.5pt}\end{center}

\subsubsection{Dirichlet}\label{dirichlet}

Samples from a Dirichlet distribution.

The Dirichlet distribution is a multivariate generalization of the Beta
distribution. It is a probability distribution over a simplex, which is
a set of vectors where each element is non-negative and sums to one. It
is often used as a prior distribution for categorical distributions.

\[   P(x_1, ..., x_K) = \frac{\Gamma(\sum_{i=1}^K \alpha_i)}{\prod_{i=1}^K \Gamma(\alpha_i)} \prod_{i=1}^K x_i^{\alpha_i - 1}
\]

\paragraph{Args:}\label{args-19}

\begin{Shaded}
\begin{Highlighting}[]
\NormalTok{bi.dist.dirichlet(}
\NormalTok{concentration,}
\NormalTok{validate\_args}\OperatorTok{=}\VariableTok{None}\NormalTok{,}
\NormalTok{name}\OperatorTok{=}\StringTok{\textquotesingle{}x\textquotesingle{}}\NormalTok{,}
\NormalTok{obs}\OperatorTok{=}\VariableTok{None}\NormalTok{,}
\NormalTok{mask}\OperatorTok{=}\VariableTok{None}\NormalTok{,}
\NormalTok{sample}\OperatorTok{=}\VariableTok{False}\NormalTok{,}
\NormalTok{seed}\OperatorTok{=}\DecValTok{0}\NormalTok{,}
\NormalTok{shape}\OperatorTok{=}\NormalTok{(),}
\NormalTok{event}\OperatorTok{=}\DecValTok{0}\NormalTok{,}
\NormalTok{create\_obj}\OperatorTok{=}\VariableTok{False}\NormalTok{,}
\NormalTok{)}
\end{Highlighting}
\end{Shaded}

concentration (jnp.ndarray): The concentration parameter(s) of the
Dirichlet distribution. Must be a positive array.

\begin{itemize}
\item
  \emph{shape} (tuple): A multi-purpose argument for shaping. When
  \texttt{sample=False} (model building), this is used with
  \texttt{.expand(shape)} to set the distribution's batch shape. When
  \texttt{sample=True} (direct sampling), this is used as
  \texttt{sample\_shape} to draw a raw JAX array of the given shape.
\item
  \emph{event} (int): The number of batch dimensions to reinterpret as
  event dimensions (used in model building).
\item
  \emph{mask} (jnp.ndarray, bool): Optional boolean array to mask
  observations.
\item
  \emph{create\_obj} (bool): If True, returns the raw BI distribution
  object instead of creating a sample site. This is essential for
  building complex distributions like \texttt{MixtureSameFamily}.
\item
  \emph{sample} (bool, optional): A control-flow argument. If
  \texttt{True}, the function will directly sample a raw JAX array from
  the distribution, bypassing the BI model context. If \texttt{False},
  it will create a \texttt{BI.sample} site within a model. Defaults to
  \texttt{False}.
\item
  \emph{obs} (jnp.ndarray, optional): The observed value for this random
  variable. If provided, the sample site is conditioned on this value,
  and the function returns the observed value. If \texttt{None}, the
  site is treated as a latent (unobserved) random variable. Defaults to
  \texttt{None}.
\item
  \emph{name} (str, optional): The name of the sample site in a BI
  model. This is used to uniquely identify the random variable. Defaults
  to `x'.
\end{itemize}

\paragraph{Returns:}\label{returns-19}

\begin{itemize}
\item
  When \texttt{sample=False}: A BI Dirichlet Distribution distribution
  object (for model building).
\item
  When \texttt{sample=True}: A JAX array of samples drawn from the
  BernoulliLogits distribution (for direct sampling).
\item
  When \texttt{create\_obj=True}: The raw BI Dirichlet Distribution
  object (for advanced use cases).
\end{itemize}

\paragraph{Example Usage:}\label{example-usage-17}

\begin{Shaded}
\begin{Highlighting}[]
\ImportTok{from}\NormalTok{ BI }\ImportTok{import}\NormalTok{ bi}
\NormalTok{m }\OperatorTok{=}\NormalTok{ bi(}\StringTok{\textquotesingle{}cpu\textquotesingle{}}\NormalTok{)}
\NormalTok{m.dist.dirichlet(concentration}\OperatorTok{=}\NormalTok{jnp.array([}\FloatTok{1.0}\NormalTok{, }\FloatTok{1.0}\NormalTok{, }\FloatTok{1.0}\NormalTok{]), sample}\OperatorTok{=}\VariableTok{True}\NormalTok{)}
\end{Highlighting}
\end{Shaded}

\paragraph{Wrapper of:
https://num.pyro.ai/en/stable/distributions.html\#dirichlet}\label{wrapper-of-httpsnum.pyro.aienstabledistributions.htmldirichlet}

\begin{center}\rule{0.5\linewidth}{0.5pt}\end{center}

\subsubsection{Dirichlet-Multinomial}\label{dirichlet-multinomial}

Creates a Dirichlet-Multinomial compound distribution, which is a
Multinomial distribution with a Dirichlet prior on its probabilities. It
is often used in Bayesian statistics to model count data where the
proportions of categories are uncertain.

The probability mass function is given by:

\[
P(\mathbf{x} | \boldsymbol{\alpha}, n) = \frac{n!}{\prod_{i=1}^k x_i!} \frac{\Gamma(\sum_{i=1}^k \alpha_i)}{\Gamma(n + \sum_{i=1}^k \alpha_i)} \prod_{i=1}^k \frac{\Gamma(x_i + \alpha_i)}{\Gamma(\alpha_i)}
\]where \(\mathbf{x}\) is a vector of counts, \(n\) is the total number
of trials (\texttt{total\_count}), and \(\boldsymbol{\alpha}\) is the
\texttt{concentration} parameter vector for the Dirichlet prior.

\paragraph{Args:}\label{args-20}

\begin{Shaded}
\begin{Highlighting}[]
\NormalTok{bi.dist.dirichlet\_multinomial(}
\NormalTok{concentration,}
\NormalTok{total\_count}\OperatorTok{=}\DecValTok{1}\NormalTok{,}
\NormalTok{validate\_args}\OperatorTok{=}\VariableTok{None}\NormalTok{,}
\NormalTok{name}\OperatorTok{=}\StringTok{\textquotesingle{}x\textquotesingle{}}\NormalTok{,}
\NormalTok{obs}\OperatorTok{=}\VariableTok{None}\NormalTok{,}
\NormalTok{mask}\OperatorTok{=}\VariableTok{None}\NormalTok{,}
\NormalTok{sample}\OperatorTok{=}\VariableTok{False}\NormalTok{,}
\NormalTok{seed}\OperatorTok{=}\DecValTok{0}\NormalTok{,}
\NormalTok{shape}\OperatorTok{=}\NormalTok{(),}
\NormalTok{event}\OperatorTok{=}\DecValTok{0}\NormalTok{,}
\NormalTok{create\_obj}\OperatorTok{=}\VariableTok{False}\NormalTok{,}
\NormalTok{)}
\end{Highlighting}
\end{Shaded}

concentration (jnp.ndarray): The concentration parameter (alpha) for the
Dirichlet prior. Values must be positive. The last dimension is
interpreted as the number of categories. total\_count (int, jnp.ndarray,
optional): The total number of trials (n). This must be a non-negative
integer. Defaults to 1. validate\_args (bool, optional): Whether to
enable validation of distribution parameters. Defaults to \texttt{None}.
- \emph{name} (str, optional): The name of the sample site in a BI
model. This is used to uniquely identify the random variable. Defaults
to `x'. - \emph{obs} (jnp.ndarray, optional): The observed value for
this random variable. If provided, the sample site is conditioned on
this value, and the function returns the observed value. If
\texttt{None}, the site is treated as a latent (unobserved) random
variable. Defaults to \texttt{None}.

\begin{itemize}
\item
  \emph{mask} (jnp.ndarray, bool, optional): Optional boolean array to
  mask observations. If provided, events with a \texttt{True} mask will
  be conditioned on \texttt{obs}, while the remaining events will be
  treated as latent variables. Defaults to \texttt{None}.
\item
  \emph{sample} (bool, optional): A control-flow argument. If
  \texttt{True}, the function will directly sample a raw JAX array from
  the distribution, bypassing the BI model context. If \texttt{False},
  it will create a \texttt{BI.sample} site within a model. Defaults to
  \texttt{False}.
\item
  \emph{seed} (int, optional): An integer used to generate a JAX PRNGKey
  for reproducible sampling when \texttt{sample=True}. This argument has
  no effect when \texttt{sample=False}. Defaults to 0.
\item
  \emph{shape} (tuple, optional): A multi-purpose argument for shaping.
  When \texttt{sample=False} (model building), this is used with
  \texttt{.expand(shape)} to set the distribution's batch shape. When
  \texttt{sample=True} (direct sampling), this is used as
  \texttt{sample\_shape} to draw a raw JAX array of the given shape.
\item
  \emph{event} (int, optional): The number of batch dimensions to
  reinterpret as event dimensions (used in model building).
\item
  \emph{create\_obj} (bool, optional): If True, returns the raw BI
  distribution object instead of creating a sample site. This is
  essential for building complex distributions like
  \texttt{MixtureSameFamily}. Defaults to \texttt{False}.
\end{itemize}

\paragraph{Returns:}\label{returns-20}

\begin{itemize}
\item
  When \texttt{sample=False}: A BI dirichlet\_multinomial Distribution
  distribution object (for model building).
\item
  When \texttt{sample=True}: A JAX array of samples drawn from the
  BernoulliLogits distribution (for direct sampling).
\item
  When \texttt{create\_obj=True}: The raw BI dirichlet\_multinomial
  Distribution object (for advanced use cases).
\end{itemize}

\paragraph{Example Usage:}\label{example-usage-18}

\begin{Shaded}
\begin{Highlighting}[]
\ImportTok{from}\NormalTok{ BI }\ImportTok{import}\NormalTok{ bi}
\ImportTok{import}\NormalTok{ jax.numpy }\ImportTok{as}\NormalTok{ jnp}
\NormalTok{m }\OperatorTok{=}\NormalTok{ bi(}\StringTok{\textquotesingle{}cpu\textquotesingle{}}\NormalTok{)}

\CommentTok{\# Direct sampling}
\CommentTok{\# Sample a single vector of counts for 10 trials from 3 categories}
\NormalTok{counts }\OperatorTok{=}\NormalTok{ m.dist.dirichlet\_multinomial(concentration}\OperatorTok{=}\NormalTok{jnp.array([}\FloatTok{1.0}\NormalTok{, }\FloatTok{1.0}\NormalTok{, }\FloatTok{1.0}\NormalTok{]),total\_count}\OperatorTok{=}\DecValTok{10}\NormalTok{,sample}\OperatorTok{=}\VariableTok{True}\NormalTok{)}

\CommentTok{\# Usage within a model}
\KeywordTok{def}\NormalTok{ my\_model(obs\_data}\OperatorTok{=}\VariableTok{None}\NormalTok{):}
\CommentTok{\# Define a prior on the concentration parameter}
\NormalTok{alpha }\OperatorTok{=}\NormalTok{ m.dist.half\_cauchy(scale}\OperatorTok{=}\NormalTok{jnp.ones(}\DecValTok{5}\NormalTok{), name}\OperatorTok{=}\StringTok{\textquotesingle{}alpha\textquotesingle{}}\NormalTok{, shape}\OperatorTok{=}\NormalTok{(}\DecValTok{5}\NormalTok{,))}

\CommentTok{\# Model observed counts}
\ControlFlowTok{with}\NormalTok{ m.plate(}\StringTok{\textquotesingle{}data\textquotesingle{}}\NormalTok{, }\BuiltInTok{len}\NormalTok{(obs\_data)):}
\NormalTok{y }\OperatorTok{=}\NormalTok{ m.dist.dirichlet\_multinomial(}
\NormalTok{concentration}\OperatorTok{=}\NormalTok{alpha,}
\NormalTok{total\_count}\OperatorTok{=}\DecValTok{100}\NormalTok{,}
\NormalTok{name}\OperatorTok{=}\StringTok{\textquotesingle{}y\textquotesingle{}}\NormalTok{,}
\NormalTok{obs}\OperatorTok{=}\NormalTok{obs\_data}
\NormalTok{)}
\end{Highlighting}
\end{Shaded}

\paragraph{Wrapper of:}\label{wrapper-of-14}

https://num.pyro.ai/en/stable/distributions.html\#dirichletmultinomial

\begin{center}\rule{0.5\linewidth}{0.5pt}\end{center}

\subsubsection{Discrete Uniform}\label{discrete-uniform}

Samples from a Discrete Uniform distribution.

The Discrete Uniform distribution defines a uniform distribution over a
range of integers. It is characterized by a lower bound (\texttt{low})
and an upper bound (\texttt{high}), inclusive.

\[   P(X = k) = \frac{1}{high - low + 1}, \quad k \in \{low, low+1, ..., high\}
\]

\paragraph{Args:}\label{args-21}

\begin{Shaded}
\begin{Highlighting}[]
\NormalTok{bi.dist.discrete\_uniform(}
\NormalTok{low}\OperatorTok{=}\DecValTok{0}\NormalTok{,}
\NormalTok{high}\OperatorTok{=}\DecValTok{1}\NormalTok{,}
\NormalTok{validate\_args}\OperatorTok{=}\VariableTok{None}\NormalTok{,}
\NormalTok{name}\OperatorTok{=}\StringTok{\textquotesingle{}x\textquotesingle{}}\NormalTok{,}
\NormalTok{obs}\OperatorTok{=}\VariableTok{None}\NormalTok{,}
\NormalTok{mask}\OperatorTok{=}\VariableTok{None}\NormalTok{,}
\NormalTok{sample}\OperatorTok{=}\VariableTok{False}\NormalTok{,}
\NormalTok{seed}\OperatorTok{=}\DecValTok{0}\NormalTok{,}
\NormalTok{shape}\OperatorTok{=}\NormalTok{(),}
\NormalTok{event}\OperatorTok{=}\DecValTok{0}\NormalTok{,}
\NormalTok{create\_obj}\OperatorTok{=}\VariableTok{False}\NormalTok{,}
\NormalTok{)}
\end{Highlighting}
\end{Shaded}

low (jnp.ndarray): The lower bound of the uniform range, inclusive. high
(jnp.ndarray): The upper bound of the uniform range, inclusive.

\begin{itemize}
\item
  \emph{shape} (tuple): A multi-purpose argument for shaping. When
  \texttt{sample=False} (model building), this is used with
  \texttt{.expand(shape)} to set the distribution's batch shape. When
  \texttt{sample=True} (direct sampling), this is used as
  \texttt{sample\_shape} to draw a raw JAX array of the given shape.
\item
  \emph{event} (int): The number of batch dimensions to reinterpret as
  event dimensions (used in model building).
\item
  \emph{mask} (jnp.ndarray, bool): Optional boolean array to mask
  observations.
\item
  \emph{create\_obj} (bool): If True, returns the raw BI distribution
  object instead of creating a sample site. This is essential for
  building complex distributions like \texttt{MixtureSameFamily}.
\item
  \emph{sample} (bool, optional): A control-flow argument. If
  \texttt{True}, the function will directly sample a raw JAX array from
  the distribution, bypassing the BI model context. If \texttt{False},
  it will create a \texttt{BI.sample} site within a model. Defaults to
  \texttt{False}.
\item
  \emph{obs} (jnp.ndarray, optional): The observed value for this random
  variable. If provided, the sample site is conditioned on this value,
  and the function returns the observed value. If \texttt{None}, the
  site is treated as a latent (unobserved) random variable. Defaults to
  \texttt{None}.
\item
  \emph{name} (str, optional): The name of the sample site in a BI
  model. This is used to uniquely identify the random variable. Defaults
  to `x'.
\end{itemize}

\paragraph{Returns:}\label{returns-21}

\begin{itemize}
\item
  When \texttt{sample=False}: A BI DiscreteUniform Distribution
  distribution object (for model building).
\item
  When \texttt{sample=True}: A JAX array of samples drawn from the
  BernoulliLogits distribution (for direct sampling).
\item
  When \texttt{create\_obj=True}: The raw BI dirichlet\_multinomial
  Distribution object (for advanced use cases).
\end{itemize}

\paragraph{Example Usage:}\label{example-usage-19}

\begin{Shaded}
\begin{Highlighting}[]
\ImportTok{from}\NormalTok{ BI }\ImportTok{import}\NormalTok{ bi}
\NormalTok{m }\OperatorTok{=}\NormalTok{ bi(}\StringTok{\textquotesingle{}cpu\textquotesingle{}}\NormalTok{)}
\NormalTok{m.dist.discrete\_uniform(low}\OperatorTok{=}\DecValTok{0}\NormalTok{, high}\OperatorTok{=}\DecValTok{5}\NormalTok{, sample}\OperatorTok{=}\VariableTok{True}\NormalTok{)}
\end{Highlighting}
\end{Shaded}

\paragraph{Wrapper of:}\label{wrapper-of-15}

https://num.pyro.ai/en/stable/distributions.html\#discreteuniform

\begin{center}\rule{0.5\linewidth}{0.5pt}\end{center}

\subsubsection{Doubly Truncated Power
Law}\label{doubly-truncated-power-law}

This distribution represents a continuous power law with a finite
support bounded between \texttt{low} and \texttt{high}, and with an
exponent \texttt{alpha}. It is normalized over the interval
\texttt{{[}low,\ high{]}} to ensure the area under the density function
is 1.

The probability density function (PDF) is defined as:

\[
f(x;\,\alpha,a,b) = \frac{x^{\alpha}}{Z(\alpha,a,b)}, 
\quad x \in [a,b]
\]

where the normalization constant \(Z(\alpha,a,b)\) is given by

\[
Z(\alpha,a,b) =
\begin{cases}
\log(b) - \log(a), & \text{if } \alpha = -1, \\
\dfrac{b^{1+\alpha} - a^{1+\alpha}}{1+\alpha}, & \text{otherwise}.
\end{cases}
\]

This distribution is useful for modeling data that follows a power-law
behavior but is naturally bounded due to measurement or theoretical
constraints (e.g., finite-size systems).

\paragraph{Args:}\label{args-22}

\begin{Shaded}
\begin{Highlighting}[]
\NormalTok{bi.dist.doubly\_truncated\_power\_law(}
\NormalTok{alpha,}
\NormalTok{low,}
\NormalTok{high,}
\NormalTok{validate\_args}\OperatorTok{=}\VariableTok{None}\NormalTok{,}
\NormalTok{name}\OperatorTok{=}\StringTok{\textquotesingle{}x\textquotesingle{}}\NormalTok{,}
\NormalTok{obs}\OperatorTok{=}\VariableTok{None}\NormalTok{,}
\NormalTok{mask}\OperatorTok{=}\VariableTok{None}\NormalTok{,}
\NormalTok{sample}\OperatorTok{=}\VariableTok{False}\NormalTok{,}
\NormalTok{seed}\OperatorTok{=}\DecValTok{0}\NormalTok{,}
\NormalTok{shape}\OperatorTok{=}\NormalTok{(),}
\NormalTok{event}\OperatorTok{=}\DecValTok{0}\NormalTok{,}
\NormalTok{create\_obj}\OperatorTok{=}\VariableTok{False}\NormalTok{,}
\NormalTok{)}
\end{Highlighting}
\end{Shaded}

alpha (float or array-like): Power-law exponent.

low (float or array-like): Lower bound of the distribution (must be
\(≥\) 0).

high (float or array-like): Upper bound of the distribution (must be
\(>\) 0).

\begin{itemize}
\tightlist
\item
  \emph{shape} (tuple, optional): The shape of the output tensor.
  Defaults to None.
\end{itemize}

validate\_args (bool, optional): Whether to validate the arguments.
Defaults to True.

\begin{itemize}
\item
  \emph{sample} (bool, optional): A control-flow argument. If
  \texttt{True}, the function will directly sample a raw JAX array from
  the distribution, bypassing the BI model context. If \texttt{False},
  it will create a \texttt{BI.sample} site within a model. Defaults to
  \texttt{False}.
\item
  \emph{obs} (jnp.ndarray, optional): The observed value for this random
  variable. If provided, the sample site is conditioned on this value,
  and the function returns the observed value. If \texttt{None}, the
  site is treated as a latent (unobserved) random variable. Defaults to
  \texttt{None}.
\item
  \emph{name} (str, optional): The name of the sample site in a BI
  model. This is used to uniquely identify the random variable. Defaults
  to `x'.
\end{itemize}

\paragraph{Returns:}\label{returns-22}

\begin{itemize}
\item
  When \texttt{sample=False}: A BI doubly\_truncated\_power\_law
  Distribution distribution object (for model building).
\item
  When \texttt{sample=True}: A JAX array of samples drawn from the
  BernoulliLogits distribution (for direct sampling).
\item
  When \texttt{create\_obj=True}: The raw BI
  doubly\_truncated\_power\_law Distribution object (for advanced use
  cases).
\end{itemize}

\paragraph{Example Usage:}\label{example-usage-20}

\begin{Shaded}
\begin{Highlighting}[]
\ImportTok{from}\NormalTok{ BI }\ImportTok{import}\NormalTok{ bi}
\NormalTok{m }\OperatorTok{=}\NormalTok{ bi(}\StringTok{\textquotesingle{}cpu\textquotesingle{}}\NormalTok{)        }
\NormalTok{m.dist.doubly\_truncated\_power\_law(low}\OperatorTok{=}\FloatTok{0.1}\NormalTok{, high}\OperatorTok{=}\FloatTok{10.0}\NormalTok{, alpha}\OperatorTok{=}\FloatTok{2.0}\NormalTok{, sample}\OperatorTok{=}\VariableTok{True}\NormalTok{)}
\end{Highlighting}
\end{Shaded}

\begin{center}\rule{0.5\linewidth}{0.5pt}\end{center}

\subsubsection{Euler--Maruyama}\label{eulermaruyama}

Euler--Maruyama methode is a method for the approximate numerical
solution of a stochastic differential equation (SDE). It simulates the
solution to an SDE by iteratively applying the Euler method to each time
step, incorporating a random perturbation to account for the diffusion
term.

\[
dX_t = f(X_t, t) dt + g(X_t, t) dW_t
\]where: - \(X_t\) is the state of the system at time \(t\). -
\(f(X_t, t)\) is the drift coefficient. - \(g(X_t, t)\) is the diffusion
coefficient. - \(dW_t\) is a Wiener process (Brownian motion).

\paragraph{Args:}\label{args-23}

\begin{Shaded}
\begin{Highlighting}[]
\NormalTok{bi.dist.euler\_maruyama(}
\NormalTok{t,}
\NormalTok{sde\_fn,}
\NormalTok{init\_dist,}
\NormalTok{validate\_args}\OperatorTok{=}\VariableTok{None}\NormalTok{,}
\NormalTok{name}\OperatorTok{=}\StringTok{\textquotesingle{}x\textquotesingle{}}\NormalTok{,}
\NormalTok{obs}\OperatorTok{=}\VariableTok{None}\NormalTok{,}
\NormalTok{mask}\OperatorTok{=}\VariableTok{None}\NormalTok{,}
\NormalTok{sample}\OperatorTok{=}\VariableTok{False}\NormalTok{,}
\NormalTok{seed}\OperatorTok{=}\DecValTok{0}\NormalTok{,}
\NormalTok{shape}\OperatorTok{=}\NormalTok{(),}
\NormalTok{event}\OperatorTok{=}\DecValTok{0}\NormalTok{,}
\NormalTok{create\_obj}\OperatorTok{=}\VariableTok{False}\NormalTok{,}
\NormalTok{)}
\end{Highlighting}
\end{Shaded}

t (jnp.ndarray): Discretized time steps.

sde\_fn (callable): A function that takes the current state and time as
input and returns the drift and diffusion coefficients. init\_dist
(Distribution): The initial distribution of the system.

\begin{itemize}
\tightlist
\item
  \emph{shape} (tuple, optional): The shape of the output tensor.
  Defaults to None.
\end{itemize}

sample\_shape (tuple, optional): The shape of the samples to draw.
Defaults to None.

validate\_args (bool, optional): Whether to validate the arguments.
Defaults to True.

\begin{itemize}
\item
  \emph{sample} (bool, optional): A control-flow argument. If
  \texttt{True}, the function will directly sample a raw JAX array from
  the distribution, bypassing the BI model context. If \texttt{False},
  it will create a \texttt{BI.sample} site within a model. Defaults to
  \texttt{False}.
\item
  \emph{obs} (jnp.ndarray, optional): The observed value for this random
  variable. If provided, the sample site is conditioned on this value,
  and the function returns the observed value. If \texttt{None}, the
  site is treated as a latent (unobserved) random variable. Defaults to
  \texttt{None}.
\item
  \emph{name} (str, optional): The name of the sample site in a BI
  model. This is used to uniquely identify the random variable. Defaults
  to `x'.
\end{itemize}

\paragraph{Returns:}\label{returns-23}

jnp.ndarray: Samples drawn from the Euler--Maruyama distribution.

\paragraph{Example Usage:}\label{example-usage-21}

\begin{Shaded}
\begin{Highlighting}[]
\ImportTok{from}\NormalTok{ BI }\ImportTok{import}\NormalTok{ bi}
\NormalTok{m }\OperatorTok{=}\NormalTok{ bi(}\StringTok{\textquotesingle{}cpu\textquotesingle{}}\NormalTok{)}
\NormalTok{m.dist.euler\_maruyama(t}\OperatorTok{=}\NormalTok{jnp.array([}\FloatTok{0.0}\NormalTok{, }\FloatTok{0.1}\NormalTok{, }\FloatTok{0.2}\NormalTok{]), sde\_fn}\OperatorTok{=}\KeywordTok{lambda}\NormalTok{ x, t: (x, }\FloatTok{1.0}\NormalTok{), init\_dist}\OperatorTok{=}\NormalTok{m.dist.normal(}\FloatTok{0.0}\NormalTok{, }\FloatTok{1.0}\NormalTok{, create\_obj}\OperatorTok{=}\VariableTok{True}\NormalTok{), sample }\OperatorTok{=} \VariableTok{True}\NormalTok{)}
\end{Highlighting}
\end{Shaded}

\begin{center}\rule{0.5\linewidth}{0.5pt}\end{center}

\subsubsection{Exponential}\label{exponential}

The Exponential distribution is a continuous probability distribution
that models the time until an event occurs in a Poisson process, where
events occur continuously and independently at a constant average rate.
It is often used to model the duration of events, such as the time until
a machine fails or the length of a phone call.

\[   f(x) = \lambda e^{-\lambda x} \text{ for } x \geq 0
\]

\paragraph{Args:}\label{args-24}

\begin{Shaded}
\begin{Highlighting}[]
\NormalTok{bi.dist.exponential(}
\NormalTok{rate}\OperatorTok{=}\FloatTok{1.0}\NormalTok{,}
\NormalTok{validate\_args}\OperatorTok{=}\VariableTok{None}\NormalTok{,}
\NormalTok{name}\OperatorTok{=}\StringTok{\textquotesingle{}x\textquotesingle{}}\NormalTok{,}
\NormalTok{obs}\OperatorTok{=}\VariableTok{None}\NormalTok{,}
\NormalTok{mask}\OperatorTok{=}\VariableTok{None}\NormalTok{,}
\NormalTok{sample}\OperatorTok{=}\VariableTok{False}\NormalTok{,}
\NormalTok{seed}\OperatorTok{=}\DecValTok{0}\NormalTok{,}
\NormalTok{shape}\OperatorTok{=}\NormalTok{(),}
\NormalTok{event}\OperatorTok{=}\DecValTok{0}\NormalTok{,}
\NormalTok{create\_obj}\OperatorTok{=}\VariableTok{False}\NormalTok{,}
\NormalTok{)}
\end{Highlighting}
\end{Shaded}

rate (jnp.ndarray): The rate parameter, \(\lambda\). Must be positive.

\begin{itemize}
\item
  \emph{shape} (tuple): A multi-purpose argument for shaping. When
  \texttt{sample=False} (model building), this is used with
  \texttt{.expand(shape)} to set the distribution's batch shape. When
  \texttt{sample=True} (direct sampling), this is used as
  \texttt{sample\_shape} to draw a raw JAX array of the given shape.
\item
  \emph{event} (int): The number of batch dimensions to reinterpret as
  event dimensions (used in model building).
\item
  \emph{mask} (jnp.ndarray, bool): Optional boolean array to mask
  observations.
\item
  \emph{create\_obj} (bool): If True, returns the raw BI distribution
  object instead of creating a sample site. This is essential for
  building complex distributions like \texttt{MixtureSameFamily}.
\item
  \emph{sample} (bool, optional): A control-flow argument. If
  \texttt{True}, the function will directly sample a raw JAX array from
  the distribution, bypassing the BI model context. If \texttt{False},
  it will create a \texttt{BI.sample} site within a model. Defaults to
  \texttt{False}.
\item
  \emph{obs} (jnp.ndarray, optional): The observed value for this random
  variable. If provided, the sample site is conditioned on this value,
  and the function returns the observed value. If \texttt{None}, the
  site is treated as a latent (unobserved) random variable. Defaults to
  \texttt{None}.
\item
  \emph{name} (str, optional): The name of the sample site in a BI
  model. This is used to uniquely identify the random variable. Defaults
  to `x'.
\end{itemize}

\paragraph{Returns:}\label{returns-24}

\begin{itemize}
\tightlist
\item
  When \texttt{sample=False}: A BI Exponential distribution object (for
  model building).
\item
  When \texttt{sample=True}: A JAX array of samples drawn from the
  Exponential distribution (for direct sampling).
\item
  When \texttt{create\_obj=True}: The raw BI distribution object (for
  advanced use cases).
\end{itemize}

\paragraph{Example Usage:}\label{example-usage-22}

\begin{Shaded}
\begin{Highlighting}[]
\ImportTok{from}\NormalTok{ BI }\ImportTok{import}\NormalTok{ bi}
\NormalTok{m }\OperatorTok{=}\NormalTok{ bi(}\StringTok{\textquotesingle{}cpu\textquotesingle{}}\NormalTok{)}
\NormalTok{m.dist.exponential(rate}\OperatorTok{=}\FloatTok{1.0}\NormalTok{, sample}\OperatorTok{=}\VariableTok{True}\NormalTok{)}
\end{Highlighting}
\end{Shaded}

\paragraph{Wrapper of:}\label{wrapper-of-16}

https://num.pyro.ai/en/stable/distributions.html\#exponential

\begin{center}\rule{0.5\linewidth}{0.5pt}\end{center}

\subsubsection{Folded}\label{folded}

Samples from a Folded distribution, which is the absolute value of a
base univariate distribution. This distribution reflects the base
distribution across the origin, effectively taking the absolute value of
each sample.

\[
p(x) = \sum_{k=-\infty}^{\infty} p(x - 2k)
\]

\paragraph{Args:}\label{args-25}

\begin{Shaded}
\begin{Highlighting}[]
\NormalTok{bi.dist.folded\_distribution(}
\NormalTok{base\_dist,}
\NormalTok{validate\_args}\OperatorTok{=}\VariableTok{None}\NormalTok{,}
\NormalTok{name}\OperatorTok{=}\StringTok{\textquotesingle{}x\textquotesingle{}}\NormalTok{,}
\NormalTok{obs}\OperatorTok{=}\VariableTok{None}\NormalTok{,}
\NormalTok{mask}\OperatorTok{=}\VariableTok{None}\NormalTok{,}
\NormalTok{sample}\OperatorTok{=}\VariableTok{False}\NormalTok{,}
\NormalTok{seed}\OperatorTok{=}\DecValTok{0}\NormalTok{,}
\NormalTok{shape}\OperatorTok{=}\NormalTok{(),}
\NormalTok{event}\OperatorTok{=}\DecValTok{0}\NormalTok{,}
\NormalTok{create\_obj}\OperatorTok{=}\VariableTok{False}\NormalTok{,}
\NormalTok{)}
\end{Highlighting}
\end{Shaded}

\begin{itemize}
\item
  \emph{loc} (float, optional): Location parameter of the base
  distribution. Defaults to 0.0.
\item
  \emph{sample} (float, optional): Scale parameter of the base
  distribution. Defaults to 1.0.
\item
  \emph{shape} (tuple): A multi-purpose argument for shaping. When
  \texttt{sample=False} (model building), this is used with
  \texttt{.expand(shape)} to set the distribution's batch shape. When
  \texttt{sample=True} (direct sampling), this is used as
  \texttt{sample\_shape} to draw a raw JAX array of the given shape.
\item
  \emph{event} (int): The number of batch dimensions to reinterpret as
  event dimensions (used in model building).
\item
  \emph{mask} (jnp.ndarray, bool): Optional boolean array to mask
  observations.
\item
  \emph{create\_obj} (bool): If True, returns the raw BI distribution
  object instead of creating a sample site. This is essential for
  building complex distributions like \texttt{MixtureSameFamily}.
\item
  \emph{sample} (bool, optional): A control-flow argument. If
  \texttt{True}, the function will directly sample a raw JAX array from
  the distribution, bypassing the BI model context. If \texttt{False},
  it will create a \texttt{BI.sample} site within a model. Defaults to
  \texttt{False}.
\item
  \emph{obs} (jnp.ndarray, optional): The observed value for this random
  variable. If provided, the sample site is conditioned on this value,
  and the function returns the observed value. If \texttt{None}, the
  site is treated as a latent (unobserved) random variable. Defaults to
  \texttt{None}.
\item
  \emph{name} (str, optional): The name of the sample site in a BI
  model. This is used to uniquely identify the random variable. Defaults
  to `x'.
\end{itemize}

\paragraph{Returns:}\label{returns-25}

BI FoldedDistribution distribution object (for model building) when
\texttt{sample=False}. JAX array of samples drawn from the
FoldedDistribution distribution (for direct sampling) when
\texttt{sample=True}. The raw BI distribution object (for advanced use
cases) when \texttt{create\_obj=True}.

\paragraph{Example Usage:}\label{example-usage-23}

\begin{Shaded}
\begin{Highlighting}[]
\ImportTok{from}\NormalTok{ BI }\ImportTok{import}\NormalTok{ bi}
\NormalTok{m }\OperatorTok{=}\NormalTok{ bi(}\StringTok{\textquotesingle{}cpu\textquotesingle{}}\NormalTok{)}
\NormalTok{m.dist.folded\_distribution(m.dist.normal(loc}\OperatorTok{=}\FloatTok{0.0}\NormalTok{, scale}\OperatorTok{=}\FloatTok{1.0}\NormalTok{, create\_obj }\OperatorTok{=} \VariableTok{True}\NormalTok{), sample}\OperatorTok{=}\VariableTok{True}\NormalTok{)}
\end{Highlighting}
\end{Shaded}

\paragraph{Wrapper of:}\label{wrapper-of-17}

https://num.pyro.ai/en/stable/distributions.html\#foldeddistribution

\begin{center}\rule{0.5\linewidth}{0.5pt}\end{center}

\subsubsection{Gamma}\label{gamma}

Samples from a Gamma distribution.

The Gamma distribution is a continuous probability distribution that
arises frequently in Bayesian statistics, particularly in prior
distributions for variance parameters. It is defined by two positive
shape parameters, concentration (k) and rate (theta).

\[
f(x) = \frac{\beta^{\alpha}}{\Gamma(\alpha)} x^{\alpha-1} e^{-\beta x}, \quad x > 0
\]

\paragraph{Args:}\label{args-26}

\begin{Shaded}
\begin{Highlighting}[]
\NormalTok{bi.dist.gamma(}
\NormalTok{concentration,}
\NormalTok{rate}\OperatorTok{=}\FloatTok{1.0}\NormalTok{,}
\NormalTok{validate\_args}\OperatorTok{=}\VariableTok{None}\NormalTok{,}
\NormalTok{name}\OperatorTok{=}\StringTok{\textquotesingle{}x\textquotesingle{}}\NormalTok{,}
\NormalTok{obs}\OperatorTok{=}\VariableTok{None}\NormalTok{,}
\NormalTok{mask}\OperatorTok{=}\VariableTok{None}\NormalTok{,}
\NormalTok{sample}\OperatorTok{=}\VariableTok{False}\NormalTok{,}
\NormalTok{seed}\OperatorTok{=}\DecValTok{0}\NormalTok{,}
\NormalTok{shape}\OperatorTok{=}\NormalTok{(),}
\NormalTok{event}\OperatorTok{=}\DecValTok{0}\NormalTok{,}
\NormalTok{create\_obj}\OperatorTok{=}\VariableTok{False}\NormalTok{,}
\NormalTok{)}
\end{Highlighting}
\end{Shaded}

concentration (jnp.ndarray): The shape parameter of the Gamma
distribution (k \textgreater{} 0).

rate (jnp.ndarray): The rate parameter of the Gamma distribution (theta
\textgreater{} 0).

\begin{itemize}
\item
  \emph{shape} (tuple): A multi-purpose argument for shaping. When
  \texttt{sample=False} (model building), this is used with
  \texttt{.expand(shape)} to set the distribution's batch shape. When
  \texttt{sample=True} (direct sampling), this is used as
  \texttt{sample\_shape} to draw a raw JAX array of the given shape.
\item
  \emph{event} (int): The number of batch dimensions to reinterpret as
  event dimensions (used in model building).
\item
  \emph{mask} (jnp.ndarray, bool): Optional boolean array to mask
  observations.
\item
  \emph{create\_obj} (bool): If True, returns the raw BI distribution
  object instead of creating a sample site. This is essential for
  building complex distributions like \texttt{MixtureSameFamily}.
\item
  \emph{sample} (bool, optional): A control-flow argument. If
  \texttt{True}, the function will directly sample a raw JAX array from
  the distribution, bypassing the BI model context. If \texttt{False},
  it will create a \texttt{BI.sample} site within a model. Defaults to
  \texttt{False}.
\item
  \emph{obs} (jnp.ndarray, optional): The observed value for this random
  variable. If provided, the sample site is conditioned on this value,
  and the function returns the observed value. If \texttt{None}, the
  site is treated as a latent (unobserved) random variable. Defaults to
  \texttt{None}.
\item
  \emph{name} (str, optional): The name of the sample site in a BI
  model. This is used to uniquely identify the random variable. Defaults
  to `x'.
\end{itemize}

\paragraph{Returns:}\label{returns-26}

Gamma: A BI Gamma distribution object (for model building).

jnp.ndarray: A JAX array of samples drawn from the Gamma distribution
(for direct sampling).

Gamma: The raw BI distribution object (for advanced use cases).

\paragraph{Example Usage:}\label{example-usage-24}

\begin{Shaded}
\begin{Highlighting}[]
\ImportTok{from}\NormalTok{ BI }\ImportTok{import}\NormalTok{ bi}
\NormalTok{m }\OperatorTok{=}\NormalTok{ bi(}\StringTok{\textquotesingle{}cpu\textquotesingle{}}\NormalTok{)}
\NormalTok{m.dist.gamma(concentration}\OperatorTok{=}\FloatTok{2.0}\NormalTok{, rate}\OperatorTok{=}\FloatTok{0.5}\NormalTok{, sample}\OperatorTok{=}\VariableTok{True}\NormalTok{)}
\end{Highlighting}
\end{Shaded}

\paragraph{Wrapper of:}\label{wrapper-of-18}

https://num.pyro.ai/en/stable/distributions.html\#gamma

\begin{center}\rule{0.5\linewidth}{0.5pt}\end{center}

\subsubsection{Gamma Poisson}\label{gamma-poisson}

A compound distribution comprising of a gamma-poisson pair, also
referred to as a gamma-poisson mixture. The \texttt{rate} parameter for
the :class:\texttt{\textasciitilde{}numpyro.distributions.Poisson}
distribution is unknown and randomly drawn from a
:class:\texttt{\textasciitilde{}numpyro.distributions.Gamma}
distribution.

\[   P(X = x) = \int_0^\infty \frac{1}{x} \exp(-x \lambda) \frac{1}{\Gamma(\alpha)} x^{\alpha - 1} e^{-x \beta} dx
\]

\paragraph{Args:}\label{args-27}

\begin{Shaded}
\begin{Highlighting}[]
\NormalTok{bi.dist.gamma\_poisson(}
\NormalTok{concentration,}
\NormalTok{rate}\OperatorTok{=}\FloatTok{1.0}\NormalTok{,}
\NormalTok{validate\_args}\OperatorTok{=}\VariableTok{None}\NormalTok{,}
\NormalTok{name}\OperatorTok{=}\StringTok{\textquotesingle{}x\textquotesingle{}}\NormalTok{,}
\NormalTok{obs}\OperatorTok{=}\VariableTok{None}\NormalTok{,}
\NormalTok{mask}\OperatorTok{=}\VariableTok{None}\NormalTok{,}
\NormalTok{sample}\OperatorTok{=}\VariableTok{False}\NormalTok{,}
\NormalTok{seed}\OperatorTok{=}\DecValTok{0}\NormalTok{,}
\NormalTok{shape}\OperatorTok{=}\NormalTok{(),}
\NormalTok{event}\OperatorTok{=}\DecValTok{0}\NormalTok{,}
\NormalTok{create\_obj}\OperatorTok{=}\VariableTok{False}\NormalTok{,}
\NormalTok{)}
\end{Highlighting}
\end{Shaded}

concentration (jnp.ndarray): Shape parameter (alpha) of the Gamma
distribution. rate (jnp.ndarray): Rate parameter (beta) for the Gamma
distribution.

\begin{itemize}
\item
  \emph{shape} (tuple): A multi-purpose argument for shaping. When
  \texttt{sample=False} (model building), this is used with
  \texttt{.expand(shape)} to set the distribution's batch shape. When
  \texttt{sample=True} (direct sampling), this is used as
  \texttt{sample\_shape} to draw a raw JAX array of the given shape.
\item
  \emph{event} (int): The number of batch dimensions to reinterpret as
  event dimensions (used in model building).
\item
  \emph{mask} (jnp.ndarray, bool): Optional boolean array to mask
  observations.
\item
  \emph{create\_obj} (bool): If True, returns the raw BI distribution
  object instead of creating a sample site. This is essential for
  building complex distributions like \texttt{MixtureSameFamily}.
\item
  \emph{sample} (bool, optional): A control-flow argument. If
  \texttt{True}, the function will directly sample a raw JAX array from
  the distribution, bypassing the BI model context. If \texttt{False},
  it will create a \texttt{BI.sample} site within a model. Defaults to
  \texttt{False}.
\item
  \emph{obs} (jnp.ndarray, optional): The observed value for this random
  variable. If provided, the sample site is conditioned on this value,
  and the function returns the observed value. If \texttt{None}, the
  site is treated as a latent (unobserved) random variable. Defaults to
  \texttt{None}.
\item
  \emph{name} (str, optional): The name of the sample site in a BI
  model. This is used to uniquely identify the random variable. Defaults
  to `x'.
\end{itemize}

\paragraph{Returns:}\label{returns-27}

\begin{itemize}
\item
  When \texttt{sample=False}: A BI GammaPoisson distribution object (for
  model building).
\item
  When \texttt{sample=True}: A JAX array of samples drawn from the
  GammaPoisson distribution (for direct sampling).
\item
  When \texttt{create\_obj=True}: The raw BI distribution object (for
  advanced use cases).
\end{itemize}

\paragraph{Example Usage:}\label{example-usage-25}

\begin{Shaded}
\begin{Highlighting}[]
\ImportTok{from}\NormalTok{ BI }\ImportTok{import}\NormalTok{ bi}
\NormalTok{m }\OperatorTok{=}\NormalTok{ bi(}\StringTok{\textquotesingle{}cpu\textquotesingle{}}\NormalTok{)}
\NormalTok{m.dist.gamma\_poisson(concentration}\OperatorTok{=}\FloatTok{1.0}\NormalTok{, rate}\OperatorTok{=}\FloatTok{2.0}\NormalTok{, sample}\OperatorTok{=}\VariableTok{True}\NormalTok{)}
\end{Highlighting}
\end{Shaded}

\paragraph{Wrapper of:}\label{wrapper-of-19}

https://num.pyro.ai/en/stable/distributions.html\#gammapoisson

\begin{center}\rule{0.5\linewidth}{0.5pt}\end{center}

\subsubsection{Gaussian Copula}\label{gaussian-copula}

A distribution that links the \texttt{batch\_shape{[}:-1{]}} of a
marginal distribution with a multivariate Gaussian copula, modelling the
correlation between the axes. A copula is a multivariate distribution
over the uniform distribution on {[}0, 1{]}. The Gaussian copula links
the marginal distributions through a multivariate normal distribution.

\[
f(x_1, ..., x_d) = \prod_{i=1}^{d} f_i(x_i) \cdot \phi(F_1(x_1), ..., F_d(x_d); \mu, \Sigma)
\]

where: - \(f_i\) is the probability density function of the i-th
marginal distribution. - \(F_i\) is the cumulative distribution function
of the i-th marginal distribution. - \(\phi\) is the standard normal
PDF. - \(\mu\) is the mean vector of the multivariate normal
distribution. - \(\Sigma\) is the covariance matrix of the multivariate
normal distribution.

\paragraph{Args:}\label{args-28}

\begin{Shaded}
\begin{Highlighting}[]
\NormalTok{bi.dist.gaussian\_copula(}
\NormalTok{marginal\_dist,}
\NormalTok{correlation\_matrix}\OperatorTok{=}\VariableTok{None}\NormalTok{,}
\NormalTok{correlation\_cholesky}\OperatorTok{=}\VariableTok{None}\NormalTok{,}
\NormalTok{validate\_args}\OperatorTok{=}\VariableTok{None}\NormalTok{,}
\NormalTok{name}\OperatorTok{=}\StringTok{\textquotesingle{}x\textquotesingle{}}\NormalTok{,}
\NormalTok{obs}\OperatorTok{=}\VariableTok{None}\NormalTok{,}
\NormalTok{mask}\OperatorTok{=}\VariableTok{None}\NormalTok{,}
\NormalTok{sample}\OperatorTok{=}\VariableTok{False}\NormalTok{,}
\NormalTok{seed}\OperatorTok{=}\DecValTok{0}\NormalTok{,}
\NormalTok{shape}\OperatorTok{=}\NormalTok{(),}
\NormalTok{event}\OperatorTok{=}\DecValTok{0}\NormalTok{,}
\NormalTok{create\_obj}\OperatorTok{=}\VariableTok{False}\NormalTok{,}
\NormalTok{)}
\end{Highlighting}
\end{Shaded}

marginal\_dist (Distribution): Distribution whose last batch axis is to
be coupled.

correlation\_matrix (array\_like, optional): Correlation matrix of the
coupling multivariate normal distribution. Defaults to None.

correlation\_cholesky (array\_like, optional): Correlation Cholesky
factor of the coupling multivariate normal distribution. Defaults to
None.

\begin{itemize}
\item
  \emph{shape} (tuple): A multi-purpose argument for shaping. When
  \texttt{sample=False} (model building), this is used with
  \texttt{.expand(shape)} to set the distribution's batch shape. When
  \texttt{sample=True} (direct sampling), this is used as
  \texttt{sample\_shape} to draw a raw JAX array of the given shape.
\item
  \emph{event} (int): The number of batch dimensions to reinterpret as
  event dimensions (used in model building).
\item
  \emph{mask} (jnp.ndarray, bool, optional): Optional boolean array to
  mask observations. Defaults to None.
\item
  \emph{create\_obj} (bool, optional): If True, returns the raw BI
  distribution object instead of creating a sample site. This is
  essential for building complex distributions like
  \texttt{Mi\ \ xtureSameFamily}. Defaults to False.
\item
  \emph{sample} (bool, optional): A control-flow argument. If
  \texttt{True}, the function will directly sample a raw JAX array from
  the distribution, bypassing the BI model context. If \texttt{False},
  it will create a \texttt{BI.sample} site within a model. Defaults to
  \texttt{False}.
\item
  \emph{obs} (jnp.ndarray, optional): The observed value for this random
  variable. If provided, the sample site is conditioned on this value,
  and the function returns the observed value. If \texttt{None}, the
  site is treated as a latent (unobserved) random variable. Defaults to
  \texttt{None}.
\item
  \emph{name} (str, optional): The name of the sample site in a BI
  model. This is used to uniquely identify the random variable. Defaults
  to `x'.
\end{itemize}

\paragraph{Returns:}\label{returns-28}

BI GaussianCopula distribution object: When \texttt{sample=False} (for
model building). JAX array: When \texttt{sample=True} (for direct
sampling). BI distribution object: When \texttt{create\_obj=True} (for
advanced use cases).

\paragraph{Example Usage:}\label{example-usage-26}

\begin{Shaded}
\begin{Highlighting}[]
\ImportTok{from}\NormalTok{ BI }\ImportTok{import}\NormalTok{ bi}
\NormalTok{m }\OperatorTok{=}\NormalTok{ bi(}\StringTok{\textquotesingle{}cpu\textquotesingle{}}\NormalTok{)}
\NormalTok{m.dist.gaussian\_copula(}
\NormalTok{marginal\_dist }\OperatorTok{=}\NormalTok{ m.dist.beta(}\FloatTok{2.0}\NormalTok{, }\FloatTok{5.0}\NormalTok{, create\_obj }\OperatorTok{=} \VariableTok{True}\NormalTok{), }
\NormalTok{correlation\_matrix }\OperatorTok{=}\NormalTok{ jnp.array([[}\FloatTok{1.0}\NormalTok{, }\FloatTok{0.7}\NormalTok{],[}\FloatTok{0.7}\NormalTok{, }\FloatTok{1.0}\NormalTok{]]), }
\NormalTok{sample }\OperatorTok{=} \VariableTok{True}
\NormalTok{)}
\end{Highlighting}
\end{Shaded}

\paragraph{Wrapper of:}\label{wrapper-of-20}

https://num.pyro.ai/en/stable/distributions.html\#gaussiancopula

\begin{center}\rule{0.5\linewidth}{0.5pt}\end{center}

\subsubsection{Gaussian Copula Beta}\label{gaussian-copula-beta}

This distribution combines a Gaussian copula with a Beta distribution.
The Gaussian copula models the dependence structure between random
variables, while the Beta distribution defines the marginal
distributions of each variable.

\[
f(x) = \int_{-\infty}^{\infty} g(x|u) h(u) du
\]

Where: - g(x\textbar u) is the Gaussian copula density. - h(u) is the
Beta density.

\paragraph{Args:}\label{args-29}

\begin{Shaded}
\begin{Highlighting}[]
\NormalTok{bi.dist.gaussian\_copula\_beta(}
\NormalTok{concentration1,}
\NormalTok{concentration0,}
\NormalTok{correlation\_matrix}\OperatorTok{=}\VariableTok{None}\NormalTok{,}
\NormalTok{correlation\_cholesky}\OperatorTok{=}\VariableTok{None}\NormalTok{,}
\NormalTok{validate\_args}\OperatorTok{=}\VariableTok{False}\NormalTok{,}
\NormalTok{name}\OperatorTok{=}\StringTok{\textquotesingle{}x\textquotesingle{}}\NormalTok{,}
\NormalTok{obs}\OperatorTok{=}\VariableTok{None}\NormalTok{,}
\NormalTok{mask}\OperatorTok{=}\VariableTok{None}\NormalTok{,}
\NormalTok{sample}\OperatorTok{=}\VariableTok{False}\NormalTok{,}
\NormalTok{seed}\OperatorTok{=}\DecValTok{0}\NormalTok{,}
\NormalTok{shape}\OperatorTok{=}\NormalTok{(),}
\NormalTok{event}\OperatorTok{=}\DecValTok{0}\NormalTok{,}
\NormalTok{create\_obj}\OperatorTok{=}\VariableTok{False}\NormalTok{,}
\NormalTok{)}
\end{Highlighting}
\end{Shaded}

concentration1 (jnp.ndarray): The first shape parameter of the Beta
distribution.

concentration0 (jnp.ndarray): The second shape parameter of the Beta
distribution.

correlation\_matrix (array\_like, optional): Correlation matrix of the
coupling multivariate normal distribution. Defaults to None.

correlation\_cholesky (jnp.ndarray): The Cholesky decomposition of the
correlation matrix. - \emph{shape} (tuple): A multi-purpose argument for
shaping. When \texttt{sample=False} (model building), this is used with
\texttt{.expand(shape)} to set the distribution's batch shape. When
\texttt{sample=True} (direct sampling), this is used as
\texttt{sample\_shape} to draw a raw JAX array of the given shape.

\begin{itemize}
\item
  \emph{event} (int): The number of batch dimensions to reinterpret as
  event dimensions (used in model building).
\item
  \emph{mask} (jnp.ndarray, bool): Optional boolean array to mask
  observations.
\item
  \emph{create\_obj} (bool): If True, returns the raw BI distribution
  object instead of creating a sample site. This is essential for
  building complex distributions like \texttt{MixtureSameFamily}.
\item
  \emph{sample} (bool, optional): A control-flow argument. If
  \texttt{True}, the function will directly sample a raw JAX array from
  the distribution, bypassing the BI model context. If \texttt{False},
  it will create a \texttt{BI.sample} site within a model. Defaults to
  \texttt{False}.
\item
  \emph{obs} (jnp.ndarray, optional): The observed value for this random
  variable. If provided, the sample site is conditioned on this value,
  and the function returns the observed value. If \texttt{None}, the
  site is treated as a latent (unobserved) random variable. Defaults to
  \texttt{None}.
\item
  \emph{name} (str, optional): The name of the sample site in a BI
  model. This is used to uniquely identify the random variable. Defaults
  to `x'.
\end{itemize}

\paragraph{Returns:}\label{returns-29}

GaussianCopulaBeta: A BI GaussianCopulaBeta distribution object (for
model building). jnp.ndarray: A JAX array of samples drawn from the
GaussianCopulaBeta distribution (for direct sampling). Distribution: The
raw BI distribution object (for advanced use cases).

\paragraph{Example Usage:}\label{example-usage-27}

\begin{Shaded}
\begin{Highlighting}[]
\ImportTok{from}\NormalTok{ BI }\ImportTok{import}\NormalTok{ bi}
\NormalTok{m }\OperatorTok{=}\NormalTok{ bi(}\StringTok{\textquotesingle{}cpu\textquotesingle{}}\NormalTok{)}
\NormalTok{m.dist.gaussian\_copula\_beta(}
\NormalTok{concentration1 }\OperatorTok{=}\NormalTok{ jnp.array([}\FloatTok{2.0}\NormalTok{, }\FloatTok{3.0}\NormalTok{]), }
\NormalTok{concentration0 }\OperatorTok{=}\NormalTok{ jnp.array([}\FloatTok{5.0}\NormalTok{, }\FloatTok{3.0}\NormalTok{]),}
\NormalTok{correlation\_cholesky }\OperatorTok{=}\NormalTok{ jnp.linalg.cholesky(jnp.array([[}\FloatTok{1.0}\NormalTok{, }\FloatTok{0.7}\NormalTok{],[}\FloatTok{0.7}\NormalTok{, }\FloatTok{1.0}\NormalTok{]])), }
\NormalTok{sample }\OperatorTok{=} \VariableTok{True}
\NormalTok{)}
\end{Highlighting}
\end{Shaded}

\paragraph{Wrapper of:}\label{wrapper-of-21}

https://num.pyro.ai/en/stable/distributions.html\#gaussiancopulabetadistribution

\begin{center}\rule{0.5\linewidth}{0.5pt}\end{center}

\subsubsection{Gaussian Random Walk}\label{gaussian-random-walk}

Creates a distribution over a Gaussian random walk of a specified number
of steps. This is a discrete-time stochastic process where the value at
each step is the previous value plus a Gaussian-distributed increment.
The distribution is over the entire path.

\[
X_t = X_{t-1} + \epsilon_t, \quad \text{where} \quad \epsilon_t \sim \mathcal{N}(0, \sigma^2)
\]with the initial state \(X_0 = 0\). The resulting sample is a vector
of length \texttt{num\_steps}, representing the path
\((X_1, X_2, \dots, X_{\text{num\_steps}})\).

\paragraph{Args:}\label{args-30}

\begin{Shaded}
\begin{Highlighting}[]
\NormalTok{bi.dist.gaussian\_random\_walk(}
\NormalTok{scale}\OperatorTok{=}\FloatTok{1.0}\NormalTok{,}
\NormalTok{num\_steps}\OperatorTok{=}\DecValTok{1}\NormalTok{,}
\NormalTok{validate\_args}\OperatorTok{=}\VariableTok{None}\NormalTok{,}
\NormalTok{name}\OperatorTok{=}\StringTok{\textquotesingle{}x\textquotesingle{}}\NormalTok{,}
\NormalTok{obs}\OperatorTok{=}\VariableTok{None}\NormalTok{,}
\NormalTok{mask}\OperatorTok{=}\VariableTok{None}\NormalTok{,}
\NormalTok{sample}\OperatorTok{=}\VariableTok{False}\NormalTok{,}
\NormalTok{seed}\OperatorTok{=}\DecValTok{0}\NormalTok{,}
\NormalTok{shape}\OperatorTok{=}\NormalTok{(),}
\NormalTok{event}\OperatorTok{=}\DecValTok{0}\NormalTok{,}
\NormalTok{create\_obj}\OperatorTok{=}\VariableTok{False}\NormalTok{,}
\NormalTok{)}
\end{Highlighting}
\end{Shaded}

\begin{itemize}
\tightlist
\item
  \emph{sample} (float, jnp.ndarray, optional): The standard deviation
  (\(\sigma\)) of the Gaussian increments. Must be positive. Defaults to
  1.0. num\_steps (int, optional): The number of steps in the random
  walk, which determines the event shape of the distribution. Must be
  positive. Defaults to 1. validate\_args (bool, optional): Whether to
  enable validation of distribution parameters. Defaults to
  \texttt{None}.
\item
  \emph{name} (str, optional): The name of the sample site in a BI
  model. This is used to uniquely identify the random variable. Defaults
  to `x'.
\item
  \emph{obs} (jnp.ndarray, optional): The observed value for this random
  variable. If provided, the sample site is conditioned on this value,
  and the function returns the observed value. If \texttt{None}, the
  site is treated as a latent (unobserved) random variable. Defaults to
  \texttt{None}.
\item
  \emph{mask} (jnp.ndarray, bool, optional): Optional boolean array to
  mask observations. If provided, events with a \texttt{True} mask will
  be conditioned on \texttt{obs}, while the remaining events will be
  treated as latent variables. Defaults to \texttt{None}.
\item
  \emph{sample} (bool, optional): A control-flow argument. If
  \texttt{True}, the function will directly sample a raw JAX array from
  the distribution, bypassing the BI model context. If \texttt{False},
  it will create a \texttt{BI.sample} site within a model. Defaults to
  \texttt{False}.
\item
  \emph{seed} (int, optional): An integer used to generate a JAX PRNGKey
  for reproducible sampling when \texttt{sample=True}. This argument has
  no effect when \texttt{sample=False}. Defaults to 0.
\item
  \emph{shape} (tuple, optional): A multi-purpose argument for shaping.
  When \texttt{sample=False} (model building), this is used with
  \texttt{.expand(shape)} to set the distribution's batch shape. When
  \texttt{sample=True} (direct sampling), this is used as
  \texttt{sample\_shape} to draw a raw JAX array of the given shape.
\item
  \emph{event} (int, optional): The number of batch dimensions to
  reinterpret as event dimensions (used in model building).
\item
  \emph{create\_obj} (bool, optional): If True, returns the raw BI
  distribution object instead of creating a sample site. This is
  essential for building complex distributions like
  \texttt{MixtureSameFamily}. Defaults to \texttt{False}.
\end{itemize}

\paragraph{Returns:}\label{returns-30}

BI.primitives.Messenger: A BI sample site object when used in a model
context (\texttt{sample=False}). jnp.ndarray: A JAX array of samples
drawn from the GaussianRandomWalk distribution (for direct sampling,
\texttt{sample=True}). numpyro.distributions.Distribution: The raw BI
distribution object (if \texttt{create\_obj=True}).

\paragraph{Example Usage:}\label{example-usage-28}

\begin{Shaded}
\begin{Highlighting}[]
\ImportTok{from}\NormalTok{ BI }\ImportTok{import}\NormalTok{ bi}
\ImportTok{import}\NormalTok{ jax.numpy }\ImportTok{as}\NormalTok{ jnp}
\NormalTok{m }\OperatorTok{=}\NormalTok{ bi(}\StringTok{\textquotesingle{}cpu\textquotesingle{}}\NormalTok{)}

\CommentTok{\# Direct sampling of a random walk with 100 steps}
\NormalTok{path }\OperatorTok{=}\NormalTok{ m.dist.gaussian\_random\_walk(scale}\OperatorTok{=}\FloatTok{0.5}\NormalTok{, num\_steps}\OperatorTok{=}\DecValTok{100}\NormalTok{, sample}\OperatorTok{=}\VariableTok{True}\NormalTok{)}

\CommentTok{\# Usage within a model for a latent time series}
\KeywordTok{def}\NormalTok{ my\_model(data}\OperatorTok{=}\VariableTok{None}\NormalTok{):}
\CommentTok{\# Prior on the volatility of the random walk}
\NormalTok{volatility }\OperatorTok{=}\NormalTok{ m.dist.half\_cauchy(scale}\OperatorTok{=}\FloatTok{1.0}\NormalTok{, name}\OperatorTok{=}\StringTok{\textquotesingle{}volatility\textquotesingle{}}\NormalTok{)}

\CommentTok{\# The latent random walk}
\NormalTok{latent\_process }\OperatorTok{=}\NormalTok{ m.dist.gaussian\_random\_walk(}
\NormalTok{scale}\OperatorTok{=}\NormalTok{volatility,}
\NormalTok{num\_steps}\OperatorTok{=}\BuiltInTok{len}\NormalTok{(data) }\ControlFlowTok{if}\NormalTok{ data }\KeywordTok{is} \KeywordTok{not} \VariableTok{None} \ControlFlowTok{else} \DecValTok{10}\NormalTok{,}
\NormalTok{name}\OperatorTok{=}\StringTok{\textquotesingle{}latent\_process\textquotesingle{}}
\NormalTok{)}

\CommentTok{\# Observation model}
\CommentTok{\# Assumes the observed data is the latent process plus some noise}
\NormalTok{obs\_noise }\OperatorTok{=}\NormalTok{ m.dist.half\_cauchy(scale}\OperatorTok{=}\FloatTok{1.0}\NormalTok{, name}\OperatorTok{=}\StringTok{\textquotesingle{}obs\_noise\textquotesingle{}}\NormalTok{)}
\ControlFlowTok{with}\NormalTok{ m.plate(}\StringTok{\textquotesingle{}time\textquotesingle{}}\NormalTok{, }\BuiltInTok{len}\NormalTok{(data) }\ControlFlowTok{if}\NormalTok{ data }\KeywordTok{is} \KeywordTok{not} \VariableTok{None} \ControlFlowTok{else} \DecValTok{10}\NormalTok{):}
\ControlFlowTok{return}\NormalTok{ m.dist.normal(loc}\OperatorTok{=}\NormalTok{latent\_process, scale}\OperatorTok{=}\NormalTok{obs\_noise, obs}\OperatorTok{=}\NormalTok{data, name}\OperatorTok{=}\StringTok{\textquotesingle{}obs\textquotesingle{}}\NormalTok{)}
\end{Highlighting}
\end{Shaded}

\paragraph{Wrapper of:}\label{wrapper-of-22}

https://num.pyro.ai/en/stable/distributions.html\#gaussianrandomwalk

\begin{center}\rule{0.5\linewidth}{0.5pt}\end{center}

\subsubsection{Gaussian State Space}\label{gaussian-state-space}

Samples from a Gaussian state space model.

\[           
\mathbf{z}_{t} = \mathbf{A} \mathbf{z}_{t - 1} + \boldsymbol{\epsilon}_t\\
=\sum_{k=1} \mathbf{A}^{t-k} \boldsymbol{\epsilon}_t,
\]

where \(\mathbf{z}_t\) is the state vector at step \(t\), \(\mathbf{A}\)
is the transition matrix, and \(\boldsymbol\epsilon\) is the innovation
noise.

\paragraph{Args:}\label{args-31}

\begin{Shaded}
\begin{Highlighting}[]
\NormalTok{bi.dist.gaussian\_state\_space(}
\NormalTok{num\_steps,}
\NormalTok{transition\_matrix,}
\NormalTok{covariance\_matrix}\OperatorTok{=}\VariableTok{None}\NormalTok{,}
\NormalTok{precision\_matrix}\OperatorTok{=}\VariableTok{None}\NormalTok{,}
\NormalTok{scale\_tril}\OperatorTok{=}\VariableTok{None}\NormalTok{,}
\NormalTok{validate\_args}\OperatorTok{=}\VariableTok{None}\NormalTok{,}
\NormalTok{name}\OperatorTok{=}\StringTok{\textquotesingle{}x\textquotesingle{}}\NormalTok{,}
\NormalTok{obs}\OperatorTok{=}\VariableTok{None}\NormalTok{,}
\NormalTok{mask}\OperatorTok{=}\VariableTok{None}\NormalTok{,}
\NormalTok{sample}\OperatorTok{=}\VariableTok{False}\NormalTok{,}
\NormalTok{seed}\OperatorTok{=}\DecValTok{0}\NormalTok{,}
\NormalTok{shape}\OperatorTok{=}\NormalTok{(),}
\NormalTok{event}\OperatorTok{=}\DecValTok{0}\NormalTok{,}
\NormalTok{create\_obj}\OperatorTok{=}\VariableTok{False}\NormalTok{,}
\NormalTok{)}
\end{Highlighting}
\end{Shaded}

num\_steps (int): Number of steps.

transition\_matrix (jnp.ndarray): State space transition matrix
\(\mathbf{A}\).

covariance\_matrix (jnp.ndarray, optional): Covariance of the innovation
noise \(\boldsymbol{\epsilon}\). Defaults to None.

precision\_matrix (jnp.ndarray, optional): Precision matrix of the
innovation noise \(\boldsymbol{\epsilon}\). Defaults to None.

scale\_tril (jnp.ndarray, optional): Scale matrix of the innovation
noise \(\boldsymbol{\epsilon}\). Defaults to None.

\begin{itemize}
\item
  \emph{shape} (tuple): A multi-purpose argument for shaping. When
  \texttt{sample=False} (model building), this is used with
  \texttt{.expand(shape)} to set the distribution's batch shape. When
  \texttt{sample=True} (direct sampling), this is used as
  \texttt{sample\_shape} to draw a raw JAX array of the given shape.
\item
  \emph{event} (int): The number of batch dimensions to reinterpret as
  event dimensions (used in model building).
\item
  \emph{mask} (jnp.ndarray, bool, optional): Optional boolean array to
  mask observations. Defaults to None.
\item
  \emph{create\_obj} (bool, optional): If True, returns the raw BI
  distribution object instead of creating a sample site. Defaults to
  False.
\item
  \emph{sample} (bool, optional): A control-flow argument. If
  \texttt{True}, the function will directly sample a raw JAX array from
  the distribution, bypassing the BI model context. If \texttt{False},
  it will create a \texttt{BI.sample} site within a model. Defaults to
  \texttt{False}.
\item
  \emph{obs} (jnp.ndarray, optional): The observed value for this random
  variable. If provided, the sample site is conditioned on this value,
  and the function returns the observed value. If \texttt{None}, the
  site is treated as a latent (unobserved) random variable. Defaults to
  \texttt{None}.
\item
  \emph{name} (str, optional): The name of the sample site in a BI
  model. This is used to uniquely identify the random variable. Defaults
  to `x'.
\end{itemize}

\paragraph{Returns:}\label{returns-31}

\begin{itemize}
\tightlist
\item
  When \texttt{sample=False}: A BI GaussianStateSpace distribution
  object (for model building).
\item
  When \texttt{sample=True}: A JAX array of samples drawn from the
  GaussianStateSpace distribution (for direct sampling).
\item
  When \texttt{create\_obj=True}: The raw BI distribution object (for
  advanced use cases).
\end{itemize}

\paragraph{Example Usage:}\label{example-usage-29}

\begin{Shaded}
\begin{Highlighting}[]
\ImportTok{from}\NormalTok{ BI }\ImportTok{import}\NormalTok{ bi}
\NormalTok{m }\OperatorTok{=}\NormalTok{ bi(}\StringTok{\textquotesingle{}cpu\textquotesingle{}}\NormalTok{)}
\NormalTok{m.dist.gaussian\_state\_space(num\_steps}\OperatorTok{=}\DecValTok{5}\NormalTok{, transition\_matrix}\OperatorTok{=}\NormalTok{jnp.array([[}\FloatTok{0.5}\NormalTok{]]), covariance\_matrix }\OperatorTok{=}\NormalTok{  jnp.array([[}\FloatTok{1.0}\NormalTok{, }\FloatTok{0.6}\NormalTok{],[}\FloatTok{0.6}\NormalTok{, }\FloatTok{1.0}\NormalTok{]]), sample}\OperatorTok{=}\VariableTok{True}\NormalTok{)}
\end{Highlighting}
\end{Shaded}

\paragraph{Wrapper of:}\label{wrapper-of-23}

https://num.pyro.ai/en/stable/distributions.html\#gaussianstate

\begin{center}\rule{0.5\linewidth}{0.5pt}\end{center}

\subsubsection{Geometric distribution.}\label{geometric-distribution.}

The Geometric distribution models the number of failures before the
first success in a sequence of Bernoulli trials. It is characterized by
a single parameter, the probability of success on each trial.

\[   P(X = k) = (1 - p)^k p
\]

\paragraph{Args:}\label{args-32}

\begin{Shaded}
\begin{Highlighting}[]
\NormalTok{bi.dist.geometric(}
\NormalTok{probs}\OperatorTok{=}\VariableTok{None}\NormalTok{,}
\NormalTok{logits}\OperatorTok{=}\VariableTok{None}\NormalTok{,}
\NormalTok{validate\_args}\OperatorTok{=}\VariableTok{None}\NormalTok{,}
\NormalTok{name}\OperatorTok{=}\StringTok{\textquotesingle{}x\textquotesingle{}}\NormalTok{,}
\NormalTok{obs}\OperatorTok{=}\VariableTok{None}\NormalTok{,}
\NormalTok{mask}\OperatorTok{=}\VariableTok{None}\NormalTok{,}
\NormalTok{sample}\OperatorTok{=}\VariableTok{False}\NormalTok{,}
\NormalTok{seed}\OperatorTok{=}\DecValTok{0}\NormalTok{,}
\NormalTok{shape}\OperatorTok{=}\NormalTok{(),}
\NormalTok{event}\OperatorTok{=}\DecValTok{0}\NormalTok{,}
\NormalTok{create\_obj}\OperatorTok{=}\VariableTok{False}\NormalTok{,}
\NormalTok{)}
\end{Highlighting}
\end{Shaded}

\begin{itemize}
\item
  \emph{probs} (jnp.ndarray, optional): Probability of success on each
  trial. Must be between 0 and 1. logits (jnp.ndarray, optional):
  Log-odds of success on each trial.
  \texttt{probs\ =\ jax.nn.sigmoid(logits)}.
\item
  \emph{shape} (tuple): A multi-purpose argument for shaping. When
  \texttt{sample=False} (model building), this is used with
  \texttt{.expand(shape)} to set the distribution's batch shape. When
  \texttt{sample=True} (direct sampling), this is used as
  \texttt{sample\_shape} to draw a raw JAX array of the given shape.
\item
  \emph{event} (int): The number of batch dimensions to reinterpret as
  event dimensions (used in model building).
\item
  \emph{mask} (jnp.ndarray, bool, optional): Optional boolean array to
  mask observations.
\item
  \emph{create\_obj} (bool, optional): If True, returns the raw BI
  distribution object instead of creating a sample site. This is
  essential for building complex distributions like
  \texttt{MixtureSameFamily}.
\item
  \emph{sample} (bool, optional): A control-flow argument. If
  \texttt{True}, the function will directly sample a raw JAX array from
  the distribution, bypassing the BI model context. If \texttt{False},
  it will create a \texttt{BI.sample} site within a model. Defaults to
  \texttt{False}.
\item
  \emph{seed} (int, optional): An integer used to generate a JAX PRNGKey
  for reproducible sampling when \texttt{sample=True}. {[}7{]} This
  argument has no effect when \texttt{sample=False}, as randomness is
  handled by BI's inference engine. Defaults to 0.
\item
  \emph{obs} (jnp.ndarray, optional): The observed value for this random
  variable. If provided, the sample site is conditioned on this value,
  and the function returns the observed value. If \texttt{None}, the
  site is treated as a latent (unobserved) random variable. Defaults to
  \texttt{None}.
\item
  \emph{name} (str, optional): The name of the sample site in a BI
  model. This is used to uniquely identify the random variable. Defaults
  to `x'.
\end{itemize}

\paragraph{Returns:}\label{returns-32}

\begin{itemize}
\tightlist
\item
  When \texttt{sample=False}: A BI Geometric distribution object (for
  model building).
\item
  When \texttt{sample=True}: A JAX array of samples drawn from the
  Geometric distribution (for direct sampling).
\item
  When \texttt{create\_obj=True}: The raw BI distribution object (for
  advanced use cases).
\end{itemize}

\paragraph{Example Usage:}\label{example-usage-30}

\begin{Shaded}
\begin{Highlighting}[]
\ImportTok{from}\NormalTok{ BI }\ImportTok{import}\NormalTok{ bi}
\NormalTok{m }\OperatorTok{=}\NormalTok{ bi(}\StringTok{\textquotesingle{}cpu\textquotesingle{}}\NormalTok{)}
\NormalTok{m.dist.geometric(probs}\OperatorTok{=}\FloatTok{0.5}\NormalTok{, sample}\OperatorTok{=}\VariableTok{True}\NormalTok{)}
\end{Highlighting}
\end{Shaded}

\paragraph{Wrapper of:
https://num.pyro.ai/en/stable/distributions.html\#geometric}\label{wrapper-of-httpsnum.pyro.aienstabledistributions.htmlgeometric}

\begin{center}\rule{0.5\linewidth}{0.5pt}\end{center}

\subsubsection{GeometricLogits}\label{geometriclogits}

Samples from a GeometricLogits distribution, which models the number of
failures before the first success in a sequence of independent Bernoulli
trials. It is parameterized by logits, which are transformed into
probabilities using the sigmoid function.

\[
P(X = k) = (1 - p)^k p
\]

where:

\begin{itemize}
\tightlist
\item
  X is the number of failures before the first success.
\item
  k is the number of failures.
\item
  p is the probability of success on each trial (derived from the
  logits).
\end{itemize}

\paragraph{Args:}\label{args-33}

\begin{Shaded}
\begin{Highlighting}[]
\NormalTok{bi.dist.geometric\_logits(}
\NormalTok{logits,}
\NormalTok{validate\_args}\OperatorTok{=}\VariableTok{None}\NormalTok{,}
\NormalTok{name}\OperatorTok{=}\StringTok{\textquotesingle{}x\textquotesingle{}}\NormalTok{,}
\NormalTok{obs}\OperatorTok{=}\VariableTok{None}\NormalTok{,}
\NormalTok{mask}\OperatorTok{=}\VariableTok{None}\NormalTok{,}
\NormalTok{sample}\OperatorTok{=}\VariableTok{False}\NormalTok{,}
\NormalTok{seed}\OperatorTok{=}\DecValTok{0}\NormalTok{,}
\NormalTok{shape}\OperatorTok{=}\NormalTok{(),}
\NormalTok{event}\OperatorTok{=}\DecValTok{0}\NormalTok{,}
\NormalTok{create\_obj}\OperatorTok{=}\VariableTok{False}\NormalTok{,}
\NormalTok{)}
\end{Highlighting}
\end{Shaded}

logits (jnp.ndarray): Log-odds parameterization of the probability of
success.

\begin{itemize}
\item
  \emph{shape} (tuple): A multi-purpose argument for shaping. When
  \texttt{sample=False} (model building), this is used with
  \texttt{.expand(shape)} to set the distribution's batch shape. When
  \texttt{sample=True} (direct sampling), this is used as
  \texttt{sample\_shape} to draw a raw JAX array of the given shape.
\item
  \emph{event} (int): The number of batch dimensions to reinterpret as
  event dimensions (used in model building).
\item
  \emph{mask} (jnp.ndarray, bool): Optional boolean array to mask
  observations.
\item
  \emph{create\_obj} (bool): If True, returns the raw BI distribution
  object instead of creating a sample site. This is essential for
  building complex distributions like \texttt{MixtureSameFamily}.
\item
  \emph{sample} (bool, optional): A control-flow argument. If
  \texttt{True}, the function will directly sample a raw JAX array from
  the distribution, bypassing the BI model context. If \texttt{False},
  it will create a \texttt{BI.sample} site within a model. Defaults to
  \texttt{False}.
\item
  \emph{obs} (jnp.ndarray, optional): The observed value for this random
  variable. If provided, the sample site is conditioned on this value,
  and the function returns the observed value. If \texttt{None}, the
  site is treated as a latent (unobserved) random variable. Defaults to
  \texttt{None}.
\item
  \emph{name} (str, optional): The name of the sample site in a BI
  model. This is used to uniquely identify the random variable. Defaults
  to `x'.
\end{itemize}

\paragraph{Returns:}\label{returns-33}

BI GeometricLogits distribution object (for model building) when
\texttt{sample=False}. JAX array of samples drawn from the
GeometricLogits distribution (for direct sampling) when
\texttt{sample=True}. The raw BI distribution object (for advanced use
cases) when \texttt{create\_obj=True}.

\paragraph{Example Usage:}\label{example-usage-31}

\begin{Shaded}
\begin{Highlighting}[]
\ImportTok{from}\NormalTok{ BI }\ImportTok{import}\NormalTok{ bi}
\NormalTok{m }\OperatorTok{=}\NormalTok{ bi(}\StringTok{\textquotesingle{}cpu\textquotesingle{}}\NormalTok{)}
\NormalTok{m.dist.geometric\_logits(logits}\OperatorTok{=}\NormalTok{jnp.zeros(}\DecValTok{10}\NormalTok{), sample}\OperatorTok{=}\VariableTok{True}\NormalTok{)}
\end{Highlighting}
\end{Shaded}

\paragraph{Wrapper of:}\label{wrapper-of-24}

https://num.pyro.ai/en/stable/distributions.html\#geometriclogits

\begin{center}\rule{0.5\linewidth}{0.5pt}\end{center}

\subsubsection{GeometricProbs}\label{geometricprobs}

Samples from a Geometric

The Geometric distribution models the number of trials until the first
success in a sequence of independent Bernoulli trials, where each trial
has the same probability of success.

\[
P(X = k) = (1 - p)^k p
\]

\paragraph{Args:}\label{args-34}

\begin{Shaded}
\begin{Highlighting}[]
\NormalTok{bi.dist.geometric\_probs(}
\NormalTok{probs,}
\NormalTok{validate\_args}\OperatorTok{=}\VariableTok{None}\NormalTok{,}
\NormalTok{name}\OperatorTok{=}\StringTok{\textquotesingle{}x\textquotesingle{}}\NormalTok{,}
\NormalTok{obs}\OperatorTok{=}\VariableTok{None}\NormalTok{,}
\NormalTok{mask}\OperatorTok{=}\VariableTok{None}\NormalTok{,}
\NormalTok{sample}\OperatorTok{=}\VariableTok{False}\NormalTok{,}
\NormalTok{seed}\OperatorTok{=}\DecValTok{0}\NormalTok{,}
\NormalTok{shape}\OperatorTok{=}\NormalTok{(),}
\NormalTok{event}\OperatorTok{=}\DecValTok{0}\NormalTok{,}
\NormalTok{create\_obj}\OperatorTok{=}\VariableTok{False}\NormalTok{,}
\NormalTok{)}
\end{Highlighting}
\end{Shaded}

\begin{itemize}
\item
  \emph{probs} (jnp.ndarray): Probability of success on each trial. Must
  be between 0 and 1.
\item
  \emph{shape} (tuple): A multi-purpose argument for shaping. When
  \texttt{sample=False} (model building), this is used with
  \texttt{.expand(shape)} to set the distribution's batch shape. When
  \texttt{sample=True} (direct sampling), this is used as
  \texttt{sample\_shape} to draw a raw JAX array of the given shape.
\item
  \emph{event} (int): The number of batch dimensions to reinterpret as
  event dimensions (used in model building).
\item
  \emph{mask} (jnp.ndarray, bool): Optional boolean array to mask
  observations.
\item
  \emph{create\_obj} (bool): If True, returns the raw BI distribution
  object instead of creating a sample site. This is essential for
  building complex distributions like \texttt{MixtureSameFamily}.
\item
  \emph{sample} (bool, optional): A control-flow argument. If
  \texttt{True}, the function will directly sample a raw JAX array from
  the distribution, bypassing the BI model context. If \texttt{False},
  it will create a \texttt{BI.sample} site within a model. Defaults to
  \texttt{False}.
\item
  \emph{obs} (jnp.ndarray, optional): The observed value for this random
  variable. If provided, the sample site is conditioned on this value,
  and the function returns the observed value. If \texttt{None}, the
  site is treated as a latent (unobserved) random variable. Defaults to
  \texttt{None}.
\item
  \emph{name} (str, optional): The name of the sample site in a BI
  model. This is used to uniquely identify the random variable. Defaults
  to `x'.
\end{itemize}

\paragraph{Returns:}\label{returns-34}

BI GeometricProbs distribution object (for model building). JAX array of
samples drawn from the GeometricProbs distribution (for direct
sampling). The raw BI distribution object (for advanced use cases).

\paragraph{Example Usage:}\label{example-usage-32}

\begin{Shaded}
\begin{Highlighting}[]
\ImportTok{from}\NormalTok{ BI }\ImportTok{import}\NormalTok{ bi}
\NormalTok{m }\OperatorTok{=}\NormalTok{ bi(}\StringTok{\textquotesingle{}cpu\textquotesingle{}}\NormalTok{)}
\NormalTok{m.dist.geometric\_probs(probs}\OperatorTok{=}\FloatTok{0.5}\NormalTok{, sample}\OperatorTok{=}\VariableTok{True}\NormalTok{)}
\end{Highlighting}
\end{Shaded}

\paragraph{Wrapper of:}\label{wrapper-of-25}

https://num.pyro.ai/en/stable/distributions.html\#geometricprobs

\begin{center}\rule{0.5\linewidth}{0.5pt}\end{center}

\subsubsection{Gompertz}\label{gompertz}

The Gompertz distribution is a distribution with support on the positive
real line that is closely related to the Gumbel distribution. This
implementation follows the notation used in the Wikipedia entry for the
Gompertz distribution. See
https://en.wikipedia.org/wiki/Gompertz\_distribution.

The probability density function (PDF) is:

\[
f(x) = \frac{con}{rate} \exp \left\{ - \frac{con}{rate} \left [ \exp\{x * rate \} - 1 \right ] \right\} \exp(-x * rate)
\]

\paragraph{Args:}\label{args-35}

\begin{Shaded}
\begin{Highlighting}[]
\NormalTok{bi.dist.gompertz(}
\NormalTok{concentration,}
\NormalTok{rate}\OperatorTok{=}\FloatTok{1.0}\NormalTok{,}
\NormalTok{validate\_args}\OperatorTok{=}\VariableTok{None}\NormalTok{,}
\NormalTok{name}\OperatorTok{=}\StringTok{\textquotesingle{}x\textquotesingle{}}\NormalTok{,}
\NormalTok{obs}\OperatorTok{=}\VariableTok{None}\NormalTok{,}
\NormalTok{mask}\OperatorTok{=}\VariableTok{None}\NormalTok{,}
\NormalTok{sample}\OperatorTok{=}\VariableTok{False}\NormalTok{,}
\NormalTok{seed}\OperatorTok{=}\DecValTok{0}\NormalTok{,}
\NormalTok{shape}\OperatorTok{=}\NormalTok{(),}
\NormalTok{event}\OperatorTok{=}\DecValTok{0}\NormalTok{,}
\NormalTok{create\_obj}\OperatorTok{=}\VariableTok{False}\NormalTok{,}
\NormalTok{)}
\end{Highlighting}
\end{Shaded}

concentration (jnp.ndarray): The concentration parameter. Must be
positive. rate (jnp.ndarray): The rate parameter. Must be positive.

\begin{itemize}
\item
  \emph{shape} (tuple): A multi-purpose argument for shaping. When
  \texttt{sample=False} (model building), this is used with
  \texttt{.expand(shape)} to set the distribution's batch shape. When
  \texttt{sample=True} (direct sampling), this is used as
  \texttt{sample\_shape} to draw a raw JAX array of the given shape.
\item
  \emph{event} (int): The number of batch dimensions to reinterpret as
  event dimensions (used in model building).
\item
  \emph{mask} (jnp.ndarray, bool): Optional boolean array to mask
  observations.
\item
  \emph{create\_obj} (bool): If True, returns the raw BI distribution
  object instead of creating a sample site. This is essential for
  building complex distributions like \texttt{MixtureSameFamily}.
\item
  \emph{sample} (bool, optional): A control-flow argument. If
  \texttt{True}, the function will directly sample a raw JAX array from
  the distribution, bypassing the BI model context. If \texttt{False},
  it will create a \texttt{BI.sample} site within a model. Defaults to
  \texttt{False}.
\item
  \emph{obs} (jnp.ndarray, optional): The observed value for this random
  variable. If provided, the sample site is conditioned on this value,
  and the function returns the observed value. If \texttt{None}, the
  site is treated as a latent (unobserved) random variable. Defaults to
  \texttt{None}.
\item
  \emph{name} (str, optional): The name of the sample site in a BI
  model. This is used to uniquely identify the random variable. Defaults
  to `x'.
\end{itemize}

\paragraph{Returns:}\label{returns-35}

BI Gompertz distribution object: When \texttt{sample=False} (for model
building). JAX array: When \texttt{sample=True} (for direct sampling).
BI distribution object: When \texttt{create\_obj=True} (for advanced use
cases).

\paragraph{Example Usage:}\label{example-usage-33}

\begin{Shaded}
\begin{Highlighting}[]
\ImportTok{from}\NormalTok{ BI }\ImportTok{import}\NormalTok{ bi}
\NormalTok{m }\OperatorTok{=}\NormalTok{ bi(}\StringTok{\textquotesingle{}cpu\textquotesingle{}}\NormalTok{)}
\NormalTok{m.dist.gompertz(concentration}\OperatorTok{=}\FloatTok{1.0}\NormalTok{, rate}\OperatorTok{=}\FloatTok{1.0}\NormalTok{, sample}\OperatorTok{=}\VariableTok{True}\NormalTok{)}
\end{Highlighting}
\end{Shaded}

\paragraph{Wrapper of:}\label{wrapper-of-26}

https://num.pyro.ai/en/stable/distributions.html\#gompertz

\begin{center}\rule{0.5\linewidth}{0.5pt}\end{center}

\subsubsection{Gumbel}\label{gumbel}

Samples from a Gumbel (or Extreme Value) distribution.

The Gumbel distribution is a continuous probability distribution named
after German mathematician Carl Gumbel. It is often used to model the
distribution of maximum values in a sequence of independent random
variables.

\[   f(x) = \frac{1}{s} e^{-(x - \mu) / s} e^{-e^{- (x - \mu) / s}}
\]

\paragraph{Args:}\label{args-36}

\begin{Shaded}
\begin{Highlighting}[]
\NormalTok{bi.dist.gumbel(}
\NormalTok{loc}\OperatorTok{=}\FloatTok{0.0}\NormalTok{,}
\NormalTok{scale}\OperatorTok{=}\FloatTok{1.0}\NormalTok{,}
\NormalTok{validate\_args}\OperatorTok{=}\VariableTok{None}\NormalTok{,}
\NormalTok{name}\OperatorTok{=}\StringTok{\textquotesingle{}x\textquotesingle{}}\NormalTok{,}
\NormalTok{obs}\OperatorTok{=}\VariableTok{None}\NormalTok{,}
\NormalTok{mask}\OperatorTok{=}\VariableTok{None}\NormalTok{,}
\NormalTok{sample}\OperatorTok{=}\VariableTok{False}\NormalTok{,}
\NormalTok{seed}\OperatorTok{=}\DecValTok{0}\NormalTok{,}
\NormalTok{shape}\OperatorTok{=}\NormalTok{(),}
\NormalTok{event}\OperatorTok{=}\DecValTok{0}\NormalTok{,}
\NormalTok{create\_obj}\OperatorTok{=}\VariableTok{False}\NormalTok{,}
\NormalTok{)}
\end{Highlighting}
\end{Shaded}

\begin{itemize}
\item
  \emph{loc} (jnp.ndarray or float, optional): Location parameter.
  Defaults to 0.0.
\item
  \emph{sample} (jnp.ndarray or float, optional): Scale parameter. Must
  be positive. Defaults to 1.0.
\item
  \emph{shape} (tuple): A multi-purpose argument for shaping. When
  \texttt{sample=False} (model building), this is used with
  \texttt{.expand(shape)} to set the distribution's batch shape. When
  \texttt{sample=True} (direct sampling), this is used as
  \texttt{sample\_shape} to draw a raw JAX array of the given shape.
\item
  \emph{event} (int, optional): The number of batch dimensions to
  reinterpret as event dimensions (used in model building). Defaults to
  1.
\item
  \emph{mask} (jnp.ndarray, bool, optional): Optional boolean array to
  mask observations. Defaults to None.
\item
  \emph{create\_obj} (bool, optional): If True, returns the raw BI
  distribution object instead of creating a sample site. This is
  essential for building complex distributions like
  \texttt{MixtureSameFamily}. Defaults to False.
\item
  \emph{sample} (bool, optional): A control-flow argument. If
  \texttt{True}, the function will directly sample a raw JAX array from
  the distribution, bypassing the BI model context. If \texttt{False},
  it will create a \texttt{BI.sample} site within a model. Defaults to
  \texttt{False}.
\item
  \emph{obs} (jnp.ndarray, optional): The observed value for this random
  variable. If provided, the sample site is conditioned on this value,
  and the function returns the observed value. If \texttt{None}, the
  site is treated as a latent (unobserved) random variable. Defaults to
  \texttt{None}.
\item
  \emph{name} (str, optional): The name of the sample site in a BI
  model. This is used to uniquely identify the random variable. Defaults
  to `x'.
\end{itemize}

\paragraph{Returns:}\label{returns-36}

\begin{itemize}
\tightlist
\item
  When \texttt{sample=False}: A BI Gumbel distribution object (for model
  building).
\item
  When \texttt{sample=True}: A JAX array of samples drawn from the
  Gumbel distribution (for direct sampling).
\item
  When \texttt{create\_obj=True}: The raw BI distribution object (for
  advanced use cases).
\end{itemize}

\paragraph{Example Usage:}\label{example-usage-34}

\begin{Shaded}
\begin{Highlighting}[]
\ImportTok{from}\NormalTok{ BI }\ImportTok{import}\NormalTok{ bi}
\NormalTok{m }\OperatorTok{=}\NormalTok{ bi(}\StringTok{\textquotesingle{}cpu\textquotesingle{}}\NormalTok{)}
\NormalTok{m.dist.gumbel(loc}\OperatorTok{=}\FloatTok{0.0}\NormalTok{, scale}\OperatorTok{=}\FloatTok{1.0}\NormalTok{, sample}\OperatorTok{=}\VariableTok{True}\NormalTok{)}
\end{Highlighting}
\end{Shaded}

\paragraph{Wrapper of:}\label{wrapper-of-27}

https://num.pyro.ai/en/stable/distributions.html\#gumbel

\begin{center}\rule{0.5\linewidth}{0.5pt}\end{center}

\subsubsection{HalfCauchy}\label{halfcauchy}

The HalfCauchy distribution is a probability distribution that is half
of the Cauchy distribution. It is defined on the positive real numbers
and is often used in situations where only positive values are relevant.

\[   
f(x) = \frac{1}{2} \cdot \frac{1}{\pi \cdot \frac{1}{scale} \cdot (x^2 + \frac{1}{scale^2})}
\]

\paragraph{Args:}\label{args-37}

\begin{Shaded}
\begin{Highlighting}[]
\NormalTok{bi.dist.half\_cauchy(}
\NormalTok{scale}\OperatorTok{=}\FloatTok{1.0}\NormalTok{,}
\NormalTok{validate\_args}\OperatorTok{=}\VariableTok{None}\NormalTok{,}
\NormalTok{name}\OperatorTok{=}\StringTok{\textquotesingle{}x\textquotesingle{}}\NormalTok{,}
\NormalTok{obs}\OperatorTok{=}\VariableTok{None}\NormalTok{,}
\NormalTok{mask}\OperatorTok{=}\VariableTok{None}\NormalTok{,}
\NormalTok{sample}\OperatorTok{=}\VariableTok{False}\NormalTok{,}
\NormalTok{seed}\OperatorTok{=}\DecValTok{0}\NormalTok{,}
\NormalTok{shape}\OperatorTok{=}\NormalTok{(),}
\NormalTok{event}\OperatorTok{=}\DecValTok{0}\NormalTok{,}
\NormalTok{create\_obj}\OperatorTok{=}\VariableTok{False}\NormalTok{,}
\NormalTok{)}
\end{Highlighting}
\end{Shaded}

\begin{itemize}
\item
  \emph{sample} (jnp.ndarray): The scale parameter of the Cauchy
  distribution. Must be positive.
\item
  \emph{shape} (tuple): A multi-purpose argument for shaping. When
  \texttt{sample=False} (model building), this is used with
  \texttt{.expand(shape)} to set the distribution's batch shape. When
  \texttt{sample=True} (direct sampling), this is used as
  \texttt{sample\_shape} to draw a raw JAX array of the given shape.
\item
  \emph{event} (int): The number of batch dimensions to reinterpret as
  event dimensions (used in model building).
\item
  \emph{mask} (jnp.ndarray, bool): Optional boolean array to mask
  observations.
\item
  \emph{create\_obj} (bool): If True, returns the raw BI distribution
  object instead of creating a sample site. This is essential for
  building complex distributions like \texttt{MixtureSameFamily}.
\item
  \emph{sample} (bool, optional): A control-flow argument. If
  \texttt{True}, the function will directly sample a raw JAX array from
  the distribution, bypassing the BI model context. If \texttt{False},
  it will create a \texttt{BI.sample} site within a model. Defaults to
  \texttt{False}.
\item
  \emph{obs} (jnp.ndarray, optional): The observed value for this random
  variable. If provided, the sample site is conditioned on this value,
  and the function returns the observed value. If \texttt{None}, the
  site is treated as a latent (unobserved) random variable. Defaults to
  \texttt{None}.
\item
  \emph{name} (str, optional): The name of the sample site in a BI
  model. This is used to uniquely identify the random variable. Defaults
  to `x'.
\end{itemize}

\paragraph{Returns:}\label{returns-37}

\begin{itemize}
\tightlist
\item
  When \texttt{sample=False}: A BI HalfCauchy distribution object (for
  model building).
\item
  When \texttt{sample=True}: A JAX array of samples drawn from the
  HalfCauchy distribution (for direct sampling).
\item
  When \texttt{create\_obj=True}: The raw BI distribution object (for
  advanced use cases).
\end{itemize}

\paragraph{Example Usage:}\label{example-usage-35}

\begin{Shaded}
\begin{Highlighting}[]
\ImportTok{from}\NormalTok{ BI }\ImportTok{import}\NormalTok{ bi}
\NormalTok{m }\OperatorTok{=}\NormalTok{ bi(}\StringTok{\textquotesingle{}cpu\textquotesingle{}}\NormalTok{)}
\NormalTok{m.dist.half\_cauchy(scale}\OperatorTok{=}\FloatTok{1.0}\NormalTok{, sample}\OperatorTok{=}\VariableTok{True}\NormalTok{)}
\end{Highlighting}
\end{Shaded}

\paragraph{Wrapper of:
https://num.pyro.ai/en/stable/distributions.html\#halfcauchy}\label{wrapper-of-httpsnum.pyro.aienstabledistributions.htmlhalfcauchy}

\begin{center}\rule{0.5\linewidth}{0.5pt}\end{center}

\subsubsection{HalfNormal}\label{halfnormal}

Samples from a HalfNormal distribution.

The HalfNormal distribution is a distribution of the absolute value of a
normal random variable. It is defined by a location parameter
(implicitly 0) and a scale parameter.

\[   f(x) = \frac{1}{\sqrt{2\pi}\sigma} e^{-\frac{x^2}{2\sigma^2}} \text{ for } x > 0
\]

\paragraph{Args:}\label{args-38}

\begin{Shaded}
\begin{Highlighting}[]
\NormalTok{bi.dist.half\_normal(}
\NormalTok{scale}\OperatorTok{=}\FloatTok{1.0}\NormalTok{,}
\NormalTok{validate\_args}\OperatorTok{=}\VariableTok{None}\NormalTok{,}
\NormalTok{name}\OperatorTok{=}\StringTok{\textquotesingle{}x\textquotesingle{}}\NormalTok{,}
\NormalTok{obs}\OperatorTok{=}\VariableTok{None}\NormalTok{,}
\NormalTok{mask}\OperatorTok{=}\VariableTok{None}\NormalTok{,}
\NormalTok{sample}\OperatorTok{=}\VariableTok{False}\NormalTok{,}
\NormalTok{seed}\OperatorTok{=}\DecValTok{0}\NormalTok{,}
\NormalTok{shape}\OperatorTok{=}\NormalTok{(),}
\NormalTok{event}\OperatorTok{=}\DecValTok{0}\NormalTok{,}
\NormalTok{create\_obj}\OperatorTok{=}\VariableTok{False}\NormalTok{,}
\NormalTok{)}
\end{Highlighting}
\end{Shaded}

\begin{itemize}
\item
  \emph{sample} (float, array): The scale parameter of the distribution.
  Must be positive.
\item
  \emph{shape} (tuple): A multi-purpose argument for shaping. When
  \texttt{sample=False} (model building), this is used with
  \texttt{.expand(shape)} to set the distribution's batch shape. When
  \texttt{sample=True} (direct sampling), this is used as
  \texttt{sample\_shape} to draw a raw JAX array of the given shape.
\item
  \emph{event} (int): The number of batch dimensions to reinterpret as
  event dimensions (used in model building).
\item
  \emph{mask} (jnp.ndarray, bool): Optional boolean array to mask
  observations.
\item
  \emph{create\_obj} (bool): If True, returns the raw BI distribution
  object instead of creating a sample site. This is essential for
  building complex distributions like \texttt{MixtureSameFamily}.
\item
  \emph{sample} (bool, optional): A control-flow argument. If
  \texttt{True}, the function will directly sample a raw JAX array from
  the distribution, bypassing the BI model context. If \texttt{False},
  it will create a \texttt{BI.sample} site within a model. Defaults to
  \texttt{False}.
\item
  \emph{obs} (jnp.ndarray, optional): The observed value for this random
  variable. If provided, the sample site is conditioned on this value,
  and the function returns the observed value. If \texttt{None}, the
  site is treated as a latent (unobserved) random variable. Defaults to
  \texttt{None}.
\item
  \emph{name} (str, optional): The name of the sample site in a BI
  model. This is used to uniquely identify the random variable. Defaults
  to `x'.
\end{itemize}

\paragraph{Returns:}\label{returns-38}

\begin{itemize}
\tightlist
\item
  When \texttt{sample=False}: A BI HalfNormal distribution object (for
  model building).
\item
  When \texttt{sample=True}: A JAX array of samples drawn from the
  HalfNormal distribution (for direct sampling).
\item
  When \texttt{create\_obj=True}: The raw BI distribution object (for
  advanced use cases).
\end{itemize}

\paragraph{Example Usage:}\label{example-usage-36}

\begin{Shaded}
\begin{Highlighting}[]
\ImportTok{from}\NormalTok{ BI }\ImportTok{import}\NormalTok{ bi}
\NormalTok{m }\OperatorTok{=}\NormalTok{ bi(}\StringTok{\textquotesingle{}cpu\textquotesingle{}}\NormalTok{)}
\NormalTok{m.dist.half\_normal(scale}\OperatorTok{=}\FloatTok{1.0}\NormalTok{, sample}\OperatorTok{=}\VariableTok{True}\NormalTok{)}
\end{Highlighting}
\end{Shaded}

\paragraph{Wrapper of:}\label{wrapper-of-28}

https://num.pyro.ai/en/stable/distributions.html\#halfnormal

\begin{center}\rule{0.5\linewidth}{0.5pt}\end{center}

\subsubsection{Improper Uniform}\label{improper-uniform}

A helper distribution with zero :meth:\texttt{log\_prob} over the
\texttt{support} domain.

\[   p(x) = 0
\]

\paragraph{Args:}\label{args-39}

\begin{Shaded}
\begin{Highlighting}[]
\NormalTok{bi.dist.improper\_uniform(}
\NormalTok{support,}
\NormalTok{batch\_shape,}
\NormalTok{event\_shape,}
\NormalTok{validate\_args}\OperatorTok{=}\VariableTok{None}\NormalTok{,}
\NormalTok{name}\OperatorTok{=}\StringTok{\textquotesingle{}x\textquotesingle{}}\NormalTok{,}
\NormalTok{obs}\OperatorTok{=}\VariableTok{None}\NormalTok{,}
\NormalTok{mask}\OperatorTok{=}\VariableTok{None}\NormalTok{,}
\NormalTok{sample}\OperatorTok{=}\VariableTok{False}\NormalTok{,}
\NormalTok{seed}\OperatorTok{=}\DecValTok{0}\NormalTok{,}
\NormalTok{shape}\OperatorTok{=}\NormalTok{(),}
\NormalTok{event}\OperatorTok{=}\DecValTok{0}\NormalTok{,}
\NormalTok{create\_obj}\OperatorTok{=}\VariableTok{False}\NormalTok{,}
\NormalTok{)}
\end{Highlighting}
\end{Shaded}

support (numpyro.distributions.constraints.Constraint): The support of
this distribution.

batch\_shape (tuple): Batch shape of this distribution. It is usually
safe to set \texttt{batch\_shape=()}.

event\_shape (tuple): Event shape of this distribution.

\begin{itemize}
\item
  \emph{shape} (tuple): A multi-purpose argument for shaping. When
  \texttt{sample=False} (model building), this is used with
  \texttt{.expand(shape)} to set the distribution's batch shape. When
  \texttt{sample=True} (direct sampling), this is used as
  \texttt{sample\_shape} to draw a raw JAX array of the given shape.
\item
  \emph{event} (int): The number of batch dimensions to reinterpret as
  event dimensions (used in model building).
\item
  \emph{mask} (jnp.ndarray, bool): Optional boolean array to mask
  observations.
\item
  \emph{create\_obj} (bool): If True, returns the raw BI distribution
  object instead of creating a sample site. This is essential for
  building complex distributions like \texttt{MixtureSameFamily}.
\item
  \emph{sample} (bool, optional): A control-flow argument. If
  \texttt{True}, the function will directly sample a raw JAX array from
  the distribution, bypassing the BI model context. If \texttt{False},
  it will create a \texttt{BI.sample} site within a model. Defaults to
  \texttt{False}.
\item
  \emph{obs} (jnp.ndarray, optional): The observed value for this random
  variable. If provided, the sample site is conditioned on this value,
  and the function returns the observed value. If \texttt{None}, the
  site is treated as a latent (unobserved) random variable. Defaults to
  \texttt{None}.
\item
  \emph{name} (str, optional): The name of the sample site in a BI
  model. This is used to uniquely identify the random variable. Defaults
  to `x'.
\end{itemize}

\paragraph{Returns:}\label{returns-39}

\begin{itemize}
\tightlist
\item
  When \texttt{sample=False}: A BI ImproperUniform distribution object
  (for model building).
\item
  When \texttt{sample=True}: A JAX array of samples drawn from the
  ImproperUniform distribution (for direct sampling).
\item
  When \texttt{create\_obj=True}: The raw BI distribution object (for
  advanced use cases).
\end{itemize}

\paragraph{Example Usage:}\label{example-usage-37}

\begin{Shaded}
\begin{Highlighting}[]
\ImportTok{from}\NormalTok{ BI }\ImportTok{import}\NormalTok{ sample}
\ImportTok{from}\NormalTok{ numpyro.distributions }\ImportTok{import}\NormalTok{ ImproperUniform, Normal, constraints}

\KeywordTok{def}\NormalTok{ model():}
\NormalTok{x }\OperatorTok{=}\NormalTok{ sample(}\StringTok{\textquotesingle{}x\textquotesingle{}}\NormalTok{, ImproperUniform(constraints.ordered\_vector, (), event\_shape}\OperatorTok{=}\NormalTok{(}\DecValTok{10}\NormalTok{,)))}
\end{Highlighting}
\end{Shaded}

\paragraph{Wrapper of:}\label{wrapper-of-29}

https://num.pyro.ai/en/stable/distributions.html\#improperuniform

\begin{center}\rule{0.5\linewidth}{0.5pt}\end{center}

\subsubsection{Inverse Gamma}\label{inverse-gamma}

The InverseGamma distribution is a two-parameter family of continuous
probability distributions. It is defined by its shape and rate
parameters. It is often used as a prior distribution for precision
parameters (inverse variance) in Bayesian statistics.

\[
p(x) = \frac{1}{Gamma(\alpha)} \left( \frac{\beta}{\Gamma(\alpha)} \right)^{\alpha} x^{\alpha - 1} e^{-\beta x}
\text{ for } x > 0
\]

\paragraph{Args:}\label{args-40}

\begin{Shaded}
\begin{Highlighting}[]
\NormalTok{bi.dist.inverse\_gamma(}
\NormalTok{concentration,}
\NormalTok{rate}\OperatorTok{=}\FloatTok{1.0}\NormalTok{,}
\NormalTok{validate\_args}\OperatorTok{=}\VariableTok{None}\NormalTok{,}
\NormalTok{name}\OperatorTok{=}\StringTok{\textquotesingle{}x\textquotesingle{}}\NormalTok{,}
\NormalTok{obs}\OperatorTok{=}\VariableTok{None}\NormalTok{,}
\NormalTok{mask}\OperatorTok{=}\VariableTok{None}\NormalTok{,}
\NormalTok{sample}\OperatorTok{=}\VariableTok{False}\NormalTok{,}
\NormalTok{seed}\OperatorTok{=}\DecValTok{0}\NormalTok{,}
\NormalTok{shape}\OperatorTok{=}\NormalTok{(),}
\NormalTok{event}\OperatorTok{=}\DecValTok{0}\NormalTok{,}
\NormalTok{create\_obj}\OperatorTok{=}\VariableTok{False}\NormalTok{,}
\NormalTok{)}
\end{Highlighting}
\end{Shaded}

concentration (jnp.ndarray): The shape parameter (\textbackslash alpha)
of the InverseGamma distribution. Must be positive. rate (jnp.ndarray):
The rate parameter (\textbackslash beta) of the InverseGamma
distribution. Must be positive.

\begin{itemize}
\item
  \emph{shape} (tuple): A multi-purpose argument for shaping. When
  \texttt{sample=False} (model building), this is used with
  \texttt{.expand(shape)} to set the distribution's batch shape. When
  \texttt{sample=True} (direct sampling), this is used as
  \texttt{sample\_shape} to draw a raw JAX array of the given shape.
\item
  \emph{event} (int): The number of batch dimensions to reinterpret as
  event dimensions (used in model building).
\item
  \emph{mask} (jnp.ndarray, bool): Optional boolean array to mask
  observations.
\item
  \emph{create\_obj} (bool): If True, returns the raw BI distribution
  object instead of creating a sample site. This is essential for
  building complex distributions like \texttt{MixtureSameFamily}.
\item
  \emph{sample} (bool, optional): A control-flow argument. If
  \texttt{True}, the function will directly sample a raw JAX array from
  the distribution, bypassing the BI model context. If \texttt{False},
  it will create a \texttt{BI.sample} site within a model. Defaults to
  \texttt{False}.
\item
  \emph{obs} (jnp.ndarray, optional): The observed value for this random
  variable. If provided, the sample site is conditioned on this value,
  and the function returns the observed value. If \texttt{None}, the
  site is treated as a latent (unobserved) random variable. Defaults to
  \texttt{None}.
\item
  \emph{name} (str, optional): The name of the sample site in a BI
  model. This is used to uniquely identify the random variable. Defaults
  to `x'.
\end{itemize}

\paragraph{Returns:}\label{returns-40}

\begin{itemize}
\tightlist
\item
  When \texttt{sample=False}: A BI InverseGamma distribution object (for
  model building).
\item
  When \texttt{sample=True}: A JAX array of samples drawn from the
  InverseGamma distribution (for direct sampling).
\item
  When \texttt{create\_obj=True}: The raw BI distribution object (for
  advanced use cases).
\end{itemize}

\paragraph{Example Usage:}\label{example-usage-38}

\begin{Shaded}
\begin{Highlighting}[]
\ImportTok{from}\NormalTok{ BI }\ImportTok{import}\NormalTok{ bi}
\NormalTok{m }\OperatorTok{=}\NormalTok{ bi(}\StringTok{\textquotesingle{}cpu\textquotesingle{}}\NormalTok{)}
\NormalTok{m.dist.inverse\_gamma(concentration}\OperatorTok{=}\FloatTok{2.0}\NormalTok{, rate}\OperatorTok{=}\FloatTok{1.0}\NormalTok{, sample}\OperatorTok{=}\VariableTok{True}\NormalTok{)}
\end{Highlighting}
\end{Shaded}

\paragraph{Wrapper of:}\label{wrapper-of-30}

https://num.pyro.ai/en/stable/distributions.html\#inversegamma

\begin{center}\rule{0.5\linewidth}{0.5pt}\end{center}

\subsubsection{Kumaraswamy}\label{kumaraswamy}

The Kumaraswamy distribution is a continuous probability distribution
defined on the interval {[}0, 1{]}. It is a flexible distribution that
can take on various shapes depending on its parameters.

\[
f(x; a, b) = a b x^{a b - 1} (1 - x)^{b - 1}
\]

\paragraph{Args:}\label{args-41}

\begin{Shaded}
\begin{Highlighting}[]
\NormalTok{bi.dist.kumaraswamy(}
\NormalTok{concentration1,}
\NormalTok{concentration0,}
\NormalTok{validate\_args}\OperatorTok{=}\VariableTok{None}\NormalTok{,}
\NormalTok{name}\OperatorTok{=}\StringTok{\textquotesingle{}x\textquotesingle{}}\NormalTok{,}
\NormalTok{obs}\OperatorTok{=}\VariableTok{None}\NormalTok{,}
\NormalTok{mask}\OperatorTok{=}\VariableTok{None}\NormalTok{,}
\NormalTok{sample}\OperatorTok{=}\VariableTok{False}\NormalTok{,}
\NormalTok{seed}\OperatorTok{=}\DecValTok{0}\NormalTok{,}
\NormalTok{shape}\OperatorTok{=}\NormalTok{(),}
\NormalTok{event}\OperatorTok{=}\DecValTok{0}\NormalTok{,}
\NormalTok{create\_obj}\OperatorTok{=}\VariableTok{False}\NormalTok{,}
\NormalTok{)}
\end{Highlighting}
\end{Shaded}

concentration1 (jnp.ndarray): The first shape parameter. Must be
positive. concentration0 (jnp.ndarray): The second shape parameter. Must
be positive.

\begin{itemize}
\item
  \emph{shape} (tuple): A multi-purpose argument for shaping. When
  \texttt{sample=False} (model building), this is used with
  \texttt{.expand(shape)} to set the distribution's batch shape. When
  \texttt{sample=True} (direct sampling), this is used as
  \texttt{sample\_shape} to draw a raw JAX array of the given shape.
\item
  \emph{event} (int): The number of batch dimensions to reinterpret as
  event dimensions (used in model building).
\item
  \emph{mask} (jnp.ndarray, bool): Optional boolean array to mask
  observations.
\item
  \emph{create\_obj} (bool): If True, returns the raw BI distribution
  object instead of creating a sample site. This is essential for
  building complex distributions like \texttt{MixtureSameFamily}.
\item
  \emph{sample} (bool, optional): A control-flow argument. If
  \texttt{True}, the function will directly sample a raw JAX array from
  the distribution, bypassing the BI model context. If \texttt{False},
  it will create a \texttt{BI.sample} site within a model. Defaults to
  \texttt{False}.
\item
  \emph{obs} (jnp.ndarray, optional): The observed value for this random
  variable. If provided, the sample site is conditioned on this value,
  and the function returns the observed value. If \texttt{None}, the
  site is treated as a latent (unobserved) random variable. Defaults to
  \texttt{None}.
\item
  \emph{name} (str, optional): The name of the sample site in a BI
  model. This is used to uniquely identify the random variable. Defaults
  to `x'.
\end{itemize}

\paragraph{Returns:}\label{returns-41}

\begin{itemize}
\tightlist
\item
  When \texttt{sample=False}: A BI Kumaraswamy distribution object (for
  model building).
\item
  When \texttt{sample=True}: A JAX array of samples drawn from the
  Kumaraswamy distribution (for direct sampling).
\item
  When \texttt{create\_obj=True}: The raw BI distribution object (for
  advanced use cases).
\end{itemize}

\paragraph{Example Usage:}\label{example-usage-39}

\begin{Shaded}
\begin{Highlighting}[]
\ImportTok{from}\NormalTok{ BI }\ImportTok{import}\NormalTok{ bi}
\NormalTok{m }\OperatorTok{=}\NormalTok{ bi(}\StringTok{\textquotesingle{}cpu\textquotesingle{}}\NormalTok{)}
\NormalTok{m.dist.kumaraswamy(concentration1}\OperatorTok{=}\FloatTok{2.0}\NormalTok{, concentration0}\OperatorTok{=}\FloatTok{3.0}\NormalTok{, sample}\OperatorTok{=}\VariableTok{True}\NormalTok{)}
\end{Highlighting}
\end{Shaded}

\paragraph{Wrapper of:
https://num.pyro.ai/en/stable/distributions.html\#kumaraswamy}\label{wrapper-of-httpsnum.pyro.aienstabledistributions.htmlkumaraswamy}

\begin{center}\rule{0.5\linewidth}{0.5pt}\end{center}

\subsubsection{Laplace}\label{laplace}

Samples from a Laplace distribution, also known as the double
exponential distribution. The Laplace distribution is defined by its
location parameter (loc) and scale parameter (scale).

\[   f(x) = \frac{1}{2s} \exp\left(-\frac{|x - \mu|}{s}\right)
\]

\paragraph{Args:}\label{args-42}

\begin{Shaded}
\begin{Highlighting}[]
\NormalTok{bi.dist.laplace(}
\NormalTok{loc}\OperatorTok{=}\FloatTok{0.0}\NormalTok{,}
\NormalTok{scale}\OperatorTok{=}\FloatTok{1.0}\NormalTok{,}
\NormalTok{validate\_args}\OperatorTok{=}\VariableTok{None}\NormalTok{,}
\NormalTok{name}\OperatorTok{=}\StringTok{\textquotesingle{}x\textquotesingle{}}\NormalTok{,}
\NormalTok{obs}\OperatorTok{=}\VariableTok{None}\NormalTok{,}
\NormalTok{mask}\OperatorTok{=}\VariableTok{None}\NormalTok{,}
\NormalTok{sample}\OperatorTok{=}\VariableTok{False}\NormalTok{,}
\NormalTok{seed}\OperatorTok{=}\DecValTok{0}\NormalTok{,}
\NormalTok{shape}\OperatorTok{=}\NormalTok{(),}
\NormalTok{event}\OperatorTok{=}\DecValTok{0}\NormalTok{,}
\NormalTok{create\_obj}\OperatorTok{=}\VariableTok{False}\NormalTok{,}
\NormalTok{)}
\end{Highlighting}
\end{Shaded}

\begin{itemize}
\item
  \emph{loc} (jnp.ndarray): Location parameter of the Laplace
  distribution.
\item
  \emph{sample} (jnp.ndarray): Scale parameter of the Laplace
  distribution. Must be positive.
\item
  \emph{shape} (tuple): A multi-purpose argument for shaping. When
  \texttt{sample=False} (model building), this is used with
  \texttt{.expand(shape)} to set the distribution's batch shape. When
  \texttt{sample=True} (direct sampling), this is used as
  \texttt{sample\_shape} to draw a raw JAX array of the given shape.
\item
  \emph{event} (int): The number of batch dimensions to reinterpret as
  event dimensions (used in model building).
\item
  \emph{mask} (jnp.ndarray, bool): Optional boolean array to mask
  observations.
\item
  \emph{create\_obj} (bool): If True, returns the raw BI distribution
  object instead of creating a sample site. This is essential for
  building complex distributions like \texttt{MixtureSameFamily}.
\item
  \emph{sample} (bool, optional): A control-flow argument. If
  \texttt{True}, the function will directly sample a raw JAX array from
  the distribution, bypassing the BI model context. If \texttt{False},
  it will create a \texttt{BI.sample} site within a model. Defaults to
  \texttt{False}.
\item
  \emph{seed} (int, optional): An integer used to generate a JAX PRNGKey
  for reproducible sampling when \texttt{sample=True}. {[}7{]} This
  argument has no effect when \texttt{sample=False}, as randomness is
  handled by BI's inference engine. Defaults to 0.
\item
  \emph{obs} (jnp.ndarray, optional): The observed value for this random
  variable. If provided, the sample site is conditioned on this value,
  and the function returns the observed value. If \texttt{None}, the
  site is treated as a latent (unobserved) random variable. Defaults to
  \texttt{None}.
\item
  \emph{name} (str, optional): The name of the sample site in a BI
  model. This is used to uniquely identify the random variable. Defaults
  to `x'.
\end{itemize}

\paragraph{Returns:}\label{returns-42}

\begin{itemize}
\tightlist
\item
  When \texttt{sample=False}: A BI Laplace distribution object (for
  model building).
\item
  When \texttt{sample=True}: A JAX array of samples drawn from the
  Laplace distribution (for direct sampling).
\item
  When \texttt{create\_obj=True}: The raw BI distribution object (for
  advanced use cases).
\end{itemize}

\paragraph{Example Usage:}\label{example-usage-40}

\begin{Shaded}
\begin{Highlighting}[]
\ImportTok{from}\NormalTok{ BI }\ImportTok{import}\NormalTok{ bi}
\NormalTok{m }\OperatorTok{=}\NormalTok{ bi(}\StringTok{\textquotesingle{}cpu\textquotesingle{}}\NormalTok{)}
\NormalTok{m.dist.laplace(loc}\OperatorTok{=}\FloatTok{0.0}\NormalTok{, scale}\OperatorTok{=}\FloatTok{1.0}\NormalTok{, sample}\OperatorTok{=}\VariableTok{True}\NormalTok{)}
\end{Highlighting}
\end{Shaded}

\paragraph{Wrapper of:}\label{wrapper-of-31}

https://num.pyro.ai/en/stable/distributions.html\#laplace

\begin{center}\rule{0.5\linewidth}{0.5pt}\end{center}

\subsubsection{Left Truncated}\label{left-truncated}

Samples from a left-truncated distribution.

A left-truncated distribution is a probability distribution obtained by
restricting the support of another distribution to values greater than a
specified lower bound. This is useful when dealing with data that is
known to be greater than a certain value.

\[   
f(x) = 
\begin{cases}
\dfrac{p(x)}{P(X > \text{low})}, & x > \text{low}, \\[6pt]
0, & \text{otherwise},
\end{cases}
\] \#\#\#\# Args:

\begin{Shaded}
\begin{Highlighting}[]
\NormalTok{bi.dist.left\_truncated\_distribution(}
\NormalTok{base\_dist,}
\NormalTok{low}\OperatorTok{=}\FloatTok{0.0}\NormalTok{,}
\NormalTok{validate\_args}\OperatorTok{=}\VariableTok{None}\NormalTok{,}
\NormalTok{name}\OperatorTok{=}\StringTok{\textquotesingle{}x\textquotesingle{}}\NormalTok{,}
\NormalTok{obs}\OperatorTok{=}\VariableTok{None}\NormalTok{,}
\NormalTok{mask}\OperatorTok{=}\VariableTok{None}\NormalTok{,}
\NormalTok{sample}\OperatorTok{=}\VariableTok{False}\NormalTok{,}
\NormalTok{seed}\OperatorTok{=}\DecValTok{0}\NormalTok{,}
\NormalTok{shape}\OperatorTok{=}\NormalTok{(),}
\NormalTok{event}\OperatorTok{=}\DecValTok{0}\NormalTok{,}
\NormalTok{create\_obj}\OperatorTok{=}\VariableTok{False}\NormalTok{,}
\NormalTok{)}
\end{Highlighting}
\end{Shaded}

base\_dist: The base distribution to truncate. Must be univariate and
have real support. low: The lower truncation bound. Values less than
this are excluded from the distribution.

\begin{itemize}
\item
  \emph{shape} (tuple): A multi-purpose argument for shaping. When
  \texttt{sample=False} (model building), this is used with
  \texttt{.expand(shape)} to set the distribution's batch shape. When
  \texttt{sample=True} (direct sampling), this is used as
  \texttt{sample\_shape} to draw a raw JAX array of the given shape.
\item
  \emph{event} (int): The number of batch dimensions to reinterpret as
  event dimensions (used in model building).
\item
  \emph{mask} (jnp.ndarray, bool): Optional boolean array to mask
  observations.
\item
  \emph{create\_obj} (bool): If True, returns the raw BI distribution
  object instead of creating a sample site. This is essential for
  building complex distributions like \texttt{MixtureSameFamily}.
\item
  \emph{sample} (bool, optional): A control-flow argument. If
  \texttt{True}, the function will directly sample a raw JAX array from
  the distribution, bypassing the BI model context. If \texttt{False},
  it will create a \texttt{BI.sample} site within a model. Defaults to
  \texttt{False}.
\item
  \emph{seed} (int, optional): An integer used to generate a JAX PRNGKey
  for reproducible sampling when \texttt{sample=True}. {[}7{]} This
  argument has no effect when \texttt{sample=False}, as randomness is
  handled by BI's inference engine. Defaults to 0.
\item
  \emph{obs} (jnp.ndarray, optional): The observed value for this random
  variable. If provided, the sample site is conditioned on this value,
  and the function returns the observed value. If \texttt{None}, the
  site is treated as a latent (unobserved) random variable. Defaults to
  \texttt{None}.
\item
  \emph{name} (str, optional): The name of the sample site in a BI
  model. This is used to uniquely identify the random variable. Defaults
  to `x'.
\end{itemize}

\paragraph{Returns:}\label{returns-43}

\begin{itemize}
\item
  When \texttt{sample=False}: A BI LeftTruncatedDistribution
  distribution object (for model building).
\item
  When \texttt{sample=True}: A JAX array of samples drawn from the
  LeftTruncatedDistribution distribution (for direct sampling).
\item
  When \texttt{create\_obj=True}: The raw BI distribution object (for
  advanced use cases).
\end{itemize}

\paragraph{Wrapper of:
https://num.pyro.ai/en/stable/distributions.html\#lefttruncateddistribution}\label{wrapper-of-httpsnum.pyro.aienstabledistributions.htmllefttruncateddistribution}

\begin{center}\rule{0.5\linewidth}{0.5pt}\end{center}

\subsubsection{Levy}\label{levy}

Samples from a Levy distribution.

The probability density function is given by,

\[
f(x\mid \mu, c) = \sqrt{\frac{c}{2\pi(x-\mu)^{3}}} \exp\left(-\frac{c}{2(x-\mu)}\right), \qquad x > \mu
\]

where \(\mu\) is the location parameter and \(c\) is the scale
parameter.

\paragraph{Args:}\label{args-43}

\begin{Shaded}
\begin{Highlighting}[]
\NormalTok{bi.dist.levy(}
\NormalTok{loc,}
\NormalTok{scale,}
\NormalTok{validate\_args}\OperatorTok{=}\VariableTok{None}\NormalTok{,}
\NormalTok{name}\OperatorTok{=}\StringTok{\textquotesingle{}x\textquotesingle{}}\NormalTok{,}
\NormalTok{obs}\OperatorTok{=}\VariableTok{None}\NormalTok{,}
\NormalTok{mask}\OperatorTok{=}\VariableTok{None}\NormalTok{,}
\NormalTok{sample}\OperatorTok{=}\VariableTok{False}\NormalTok{,}
\NormalTok{seed}\OperatorTok{=}\DecValTok{0}\NormalTok{,}
\NormalTok{shape}\OperatorTok{=}\NormalTok{(),}
\NormalTok{event}\OperatorTok{=}\DecValTok{0}\NormalTok{,}
\NormalTok{create\_obj}\OperatorTok{=}\VariableTok{False}\NormalTok{,}
\NormalTok{)}
\end{Highlighting}
\end{Shaded}

\begin{itemize}
\item
  \emph{loc} (jnp.ndarray): Location parameter.
\item
  \emph{sample} (jnp.ndarray): Scale parameter.
\item
  \emph{shape} (tuple): A multi-purpose argument for shaping. When
  \texttt{sample=False} (model building), this is used with
  \texttt{.expand(shape)} to set the distribution's batch shape. When
  \texttt{sample=True} (direct sampling), this is used as
  \texttt{sample\_shape} to draw a raw JAX array of the given shape.
\item
  \emph{event} (int): The number of batch dimensions to reinterpret as
  event dimensions (used in model building).
\item
  \emph{mask} (jnp.ndarray, bool): Optional boolean array to mask
  observations.
\item
  \emph{create\_obj} (bool): If True, returns the raw BI distribution
  object instead of creating a sample site.
\item
  \emph{sample} (bool, optional): A control-flow argument. If
  \texttt{True}, the function will directly sample a raw JAX array from
  the distribution, bypassing the BI model context. If \texttt{False},
  it will create a \texttt{BI.sample} site within a model. Defaults to
  \texttt{False}.
\item
  \emph{seed} (int, optional): An integer used to generate a JAX PRNGKey
  for reproducible sampling when \texttt{sample=True}. {[}7{]} This
  argument has no effect when \texttt{sample=False}, as randomness is
  handled by BI's inference engine. Defaults to 0.
\item
  \emph{obs} (jnp.ndarray, optional): The observed value for this random
  variable. If provided, the sample site is conditioned on this value,
  and the function returns the observed value. If \texttt{None}, the
  site is treated as a latent (unobserved) random variable. Defaults to
  \texttt{None}.
\item
  \emph{name} (str, optional): The name of the sample site in a BI
  model. This is used to uniquely identify the random variable. Defaults
  to `x'.
\end{itemize}

\paragraph{Returns:}\label{returns-44}

BI Levy distribution object: When \texttt{sample=False} (for model
building). JAX array: When \texttt{sample=True} (for direct sampling).
BI distribution object: When \texttt{create\_obj=True} (for advanced use
cases).

\paragraph{Example Usage:}\label{example-usage-41}

\begin{Shaded}
\begin{Highlighting}[]
\ImportTok{from}\NormalTok{ BI }\ImportTok{import}\NormalTok{ bi}
\NormalTok{m }\OperatorTok{=}\NormalTok{ bi(}\StringTok{\textquotesingle{}cpu\textquotesingle{}}\NormalTok{)}
\NormalTok{m.dist.levy(loc}\OperatorTok{=}\FloatTok{0.0}\NormalTok{, scale}\OperatorTok{=}\FloatTok{1.0}\NormalTok{, sample}\OperatorTok{=}\VariableTok{True}\NormalTok{)}
\end{Highlighting}
\end{Shaded}

\paragraph{Wrapper of:}\label{wrapper-of-32}

https://num.pyro.ai/en/stable/distributions.html\#levy

\begin{center}\rule{0.5\linewidth}{0.5pt}\end{center}

\subsubsection{Lewandowski Kurowicka Joe
(LKJ)}\label{lewandowski-kurowicka-joe-lkj}

The LKJ distribution is controlled by the concentration parameter
\(\eta\) to make the probability of the correlation matrix \(M\)
proportional to \(\det(M)^{\eta - 1}\). When \(\eta = 1\), the
distribution is uniform over correlation matrices. When \(\eta > 1\),
the distribution favors samples with large determinants. When
\(\eta < 1\), the distribution favors samples with small determinants.

\[
P(M) \propto |\det(M)|^{\eta - 1}
\]

\paragraph{Args:}\label{args-44}

\begin{Shaded}
\begin{Highlighting}[]
\NormalTok{bi.dist.lkj(}
\NormalTok{dimension,}
\NormalTok{concentration}\OperatorTok{=}\FloatTok{1.0}\NormalTok{,}
\NormalTok{sample\_method}\OperatorTok{=}\StringTok{\textquotesingle{}onion\textquotesingle{}}\NormalTok{,}
\NormalTok{validate\_args}\OperatorTok{=}\VariableTok{None}\NormalTok{,}
\NormalTok{name}\OperatorTok{=}\StringTok{\textquotesingle{}x\textquotesingle{}}\NormalTok{,}
\NormalTok{obs}\OperatorTok{=}\VariableTok{None}\NormalTok{,}
\NormalTok{mask}\OperatorTok{=}\VariableTok{None}\NormalTok{,}
\NormalTok{sample}\OperatorTok{=}\VariableTok{False}\NormalTok{,}
\NormalTok{seed}\OperatorTok{=}\DecValTok{0}\NormalTok{,}
\NormalTok{shape}\OperatorTok{=}\NormalTok{(),}
\NormalTok{event}\OperatorTok{=}\DecValTok{0}\NormalTok{,}
\NormalTok{create\_obj}\OperatorTok{=}\VariableTok{False}\NormalTok{,}
\NormalTok{)}
\end{Highlighting}
\end{Shaded}

dimension (int): The dimension of the correlation matrices.

concentration (ndarray): The concentration/shape parameter of the
distribution (often referred to as eta). Must be positive.

sample\_method (str): Either ``cvine'' or ``onion''. Both methods are
proposed in {[}1{]} and offer the same distribution over correlation
matrices. But they are different in how to generate samples. Defaults to
``onion''.

\begin{itemize}
\item
  \emph{shape} (tuple): A multi-purpose argument for shaping. When
  \texttt{sample=False} (model building), this is used with
  \texttt{.expand(shape)} to set the distribution's batch shape. When
  \texttt{sample=True} (direct sampling), this is used as
  \texttt{sample\_shape} to draw a raw JAX array of the given shape.
\item
  \emph{event} (int): The number of batch dimensions to reinterpret as
  event dimensions (used in model building).
\item
  \emph{mask} (jnp.ndarray, bool): Optional boolean array to mask
  observations.
\item
  \emph{create\_obj} (bool): If True, returns the raw BI distribution
  object instead of creating a sample site. This is essential for
  building complex distributions like \texttt{MixtureSameFamily}.
\item
  \emph{sample} (bool, optional): A control-flow argument. If
  \texttt{True}, the function will directly sample a raw JAX array from
  the distribution, bypassing the BI model context. If \texttt{False},
  it will create a \texttt{BI.sample} site within a model. Defaults to
  \texttt{False}.
\item
  \emph{obs} (jnp.ndarray, optional): The observed value for this random
  variable. If provided, the sample site is conditioned on this value,
  and the function returns the observed value. If \texttt{None}, the
  site is treated as a latent (unobserved) random variable. Defaults to
  \texttt{None}.
\item
  \emph{name} (str, optional): The name of the sample site in a BI
  model. This is used to uniquely identify the random variable. Defaults
  to `x'.
\end{itemize}

\paragraph{Returns:}\label{returns-45}

\begin{itemize}
\item
  When \texttt{sample=False}: A BI LKJ distribution object (for model
  building).
\item
  When \texttt{sample=True}: A JAX array of samples drawn from the LKJ
  distribution (for direct sampling).
\item
  When \texttt{create\_obj=True}: The raw BI distribution object (for
  advanced use cases).
\end{itemize}

\paragraph{Example Usage:}\label{example-usage-42}

\begin{Shaded}
\begin{Highlighting}[]
\ImportTok{from}\NormalTok{ BI }\ImportTok{import}\NormalTok{ bi}
\NormalTok{m }\OperatorTok{=}\NormalTok{ bi(}\StringTok{\textquotesingle{}cpu\textquotesingle{}}\NormalTok{)}
\NormalTok{m.dist.lkj(dimension}\OperatorTok{=}\DecValTok{2}\NormalTok{, concentration}\OperatorTok{=}\FloatTok{1.0}\NormalTok{, sample}\OperatorTok{=}\VariableTok{True}\NormalTok{)}
\end{Highlighting}
\end{Shaded}

\paragraph{Wrapper of:}\label{wrapper-of-33}

https://num.pyro.ai/en/stable/distributions.html\#lkj

\begin{center}\rule{0.5\linewidth}{0.5pt}\end{center}

\subsubsection{LKJ Cholesky}\label{lkj-cholesky}

The LKJ (Leonard-Kjærgaard-Jørgensen) Cholesky distribution is a family
of distributions on symmetric matrices, often used as a prior for the
Cholesky decomposition of a symmetric matrix. It is particularly useful
in Bayesian inference for models with covariance structure. \$\$

\paragraph{Args:}\label{args-45}

\begin{Shaded}
\begin{Highlighting}[]
\NormalTok{bi.dist.lkj\_cholesky(}
\NormalTok{dimension,}
\NormalTok{concentration}\OperatorTok{=}\FloatTok{1.0}\NormalTok{,}
\NormalTok{sample\_method}\OperatorTok{=}\StringTok{\textquotesingle{}onion\textquotesingle{}}\NormalTok{,}
\NormalTok{validate\_args}\OperatorTok{=}\VariableTok{None}\NormalTok{,}
\NormalTok{name}\OperatorTok{=}\StringTok{\textquotesingle{}x\textquotesingle{}}\NormalTok{,}
\NormalTok{obs}\OperatorTok{=}\VariableTok{None}\NormalTok{,}
\NormalTok{mask}\OperatorTok{=}\VariableTok{None}\NormalTok{,}
\NormalTok{sample}\OperatorTok{=}\VariableTok{False}\NormalTok{,}
\NormalTok{seed}\OperatorTok{=}\DecValTok{0}\NormalTok{,}
\NormalTok{shape}\OperatorTok{=}\NormalTok{(),}
\NormalTok{event}\OperatorTok{=}\DecValTok{0}\NormalTok{,}
\NormalTok{create\_obj}\OperatorTok{=}\VariableTok{False}\NormalTok{,}
\NormalTok{)}
\end{Highlighting}
\end{Shaded}

dimension (int): The dimension of the correlation matrices.

concentration (float): A parameter controlling the concentration of the
distribution around the identity matrix. Higher values indicate greater
concentration. Must be greater than 1.

\begin{itemize}
\item
  \emph{shape} (tuple): A multi-purpose argument for shaping. When
  \texttt{sample=False} (model building), this is used with
  \texttt{.expand(shape)} to set the distribution's batch shape. When
  \texttt{sample=True} (direct sampling), this is used as
  \texttt{sample\_shape} to draw a raw JAX array of the given shape.
\item
  \emph{event} (int): The number of batch dimensions to reinterpret as
  event dimensions (used in model building).
\item
  \emph{mask} (jnp.ndarray, bool): Optional boolean array to mask
  observations.
\item
  \emph{create\_obj} (bool): If True, returns the raw BI distribution
  object instead of creating a sample site. This is essential for
  building complex distributions like \texttt{MixtureSameFamily}.
\item
  \emph{sample} (bool, optional): A control-flow argument. If
  \texttt{True}, the function will directly sample a raw JAX array from
  the distribution, bypassing the BI model context. If \texttt{False},
  it will create a \texttt{BI.sample} site within a model. Defaults to
  \texttt{False}.
\item
  \emph{seed} (int, optional): An integer used to generate a JAX PRNGKey
  for reproducible sampling when \texttt{sample=True}. {[}7{]} This
  argument has no effect when \texttt{sample=False}, as randomness is
  handled by BI's inference engine. Defaults to 0.
\item
  \emph{obs} (jnp.ndarray, optional): The observed value for this random
  variable. If provided, the sample site is conditioned on this value,
  and the function returns the observed value. If \texttt{None}, the
  site is treated as a latent (unobserved) random variable. Defaults to
  \texttt{None}.
\item
  \emph{name} (str, optional): The name of the sample site in a BI
  model. This is used to uniquely identify the random variable. Defaults
  to `x'.
\end{itemize}

Attributes: concentration (float): The concentration parameter.

\begin{center}\rule{0.5\linewidth}{0.5pt}\end{center}

\subsubsection{Log-Normal}\label{log-normal}

The LogNormal distribution is a probability distribution defined for
positive real-valued random variables, parameterized by a location
parameter (loc) and a scale parameter (scale). It arises when the
logarithm of a random variable is normally distributed.

\[   f(x) = \frac{1}{x \sigma \sqrt{2\pi}} e^{-\frac{(log(x) - \mu)^2}{2\sigma^2}}
\]

\paragraph{Args:}\label{args-46}

\begin{Shaded}
\begin{Highlighting}[]
\NormalTok{bi.dist.log\_normal(}
\NormalTok{loc}\OperatorTok{=}\FloatTok{0.0}\NormalTok{,}
\NormalTok{scale}\OperatorTok{=}\FloatTok{1.0}\NormalTok{,}
\NormalTok{validate\_args}\OperatorTok{=}\VariableTok{None}\NormalTok{,}
\NormalTok{name}\OperatorTok{=}\StringTok{\textquotesingle{}x\textquotesingle{}}\NormalTok{,}
\NormalTok{obs}\OperatorTok{=}\VariableTok{None}\NormalTok{,}
\NormalTok{mask}\OperatorTok{=}\VariableTok{None}\NormalTok{,}
\NormalTok{sample}\OperatorTok{=}\VariableTok{False}\NormalTok{,}
\NormalTok{seed}\OperatorTok{=}\DecValTok{0}\NormalTok{,}
\NormalTok{shape}\OperatorTok{=}\NormalTok{(),}
\NormalTok{event}\OperatorTok{=}\DecValTok{0}\NormalTok{,}
\NormalTok{create\_obj}\OperatorTok{=}\VariableTok{False}\NormalTok{,}
\NormalTok{)}
\end{Highlighting}
\end{Shaded}

\begin{itemize}
\item
  \emph{loc} (float): Location parameter.
\item
  \emph{sample} (float): Scale parameter.
\item
  \emph{shape} (tuple): A multi-purpose argument for shaping. When
  \texttt{sample=False} (model building), this is used with
  \texttt{.expand(shape)} to set the distribution's batch shape. When
  \texttt{sample=True} (direct sampling), this is used as
  \texttt{sample\_shape} to draw a raw JAX array of the given shape.
\item
  \emph{event} (int): The number of batch dimensions to reinterpret as
  event dimensions (used in model building).
\item
  \emph{mask} (jnp.ndarray, bool): Optional boolean array to mask
  observations.
\item
  \emph{create\_obj} (bool): If True, returns the raw BI distribution
  object instead of creating a sample site. This is essential for
  building complex distributions like \texttt{MixtureSameFamily}.
\item
  \emph{sample} (bool, optional): A control-flow argument. If
  \texttt{True}, the function will directly sample a raw JAX array from
  the distribution, bypassing the BI model context. If \texttt{False},
  it will create a \texttt{BI.sample} site within a model. Defaults to
  \texttt{False}.
\item
  \emph{seed} (int, optional): An integer used to generate a JAX PRNGKey
  for reproducible sampling when \texttt{sample=True}. {[}7{]} This
  argument has no effect when \texttt{sample=False}, as randomness is
  handled by BI's inference engine. Defaults to 0.
\item
  \emph{obs} (jnp.ndarray, optional): The observed value for this random
  variable. If provided, the sample site is conditioned on this value,
  and the function returns the observed value. If \texttt{None}, the
  site is treated as a latent (unobserved) random variable. Defaults to
  \texttt{None}.
\item
  \emph{name} (str, optional): The name of the sample site in a BI
  model. This is used to uniquely identify the random variable. Defaults
  to `x'.
\end{itemize}

\paragraph{Returns:}\label{returns-46}

BI LogNormal distribution object (for model building). JAX array of
samples drawn from the LogNormal distribution (for direct sampling). The
raw BI distribution object (for advanced use cases).

\paragraph{Example Usage:}\label{example-usage-43}

\begin{Shaded}
\begin{Highlighting}[]
\ImportTok{from}\NormalTok{ BI }\ImportTok{import}\NormalTok{ bi}
\NormalTok{m }\OperatorTok{=}\NormalTok{ bi(}\StringTok{\textquotesingle{}cpu\textquotesingle{}}\NormalTok{)}
\NormalTok{m.dist.log\_normal(loc}\OperatorTok{=}\FloatTok{0.0}\NormalTok{, scale}\OperatorTok{=}\FloatTok{1.0}\NormalTok{, sample}\OperatorTok{=}\VariableTok{True}\NormalTok{)}
\end{Highlighting}
\end{Shaded}

\paragraph{Wrapper of:
https://num.pyro.ai/en/stable/distributions.html\#lognormal}\label{wrapper-of-httpsnum.pyro.aienstabledistributions.htmllognormal}

\begin{center}\rule{0.5\linewidth}{0.5pt}\end{center}

\subsubsection{Log-Uniform}\label{log-uniform}

Samples from a LogUniform distribution.

The LogUniform distribution is defined over the positive real numbers
and is the result of applying an exponential transformation to a uniform
distribution over the interval {[}low, high{]}. It is often used when
modeling parameters that must be positive.

\[   f(x) = \frac{1}{(high - low) \log(high / low)}
\text{ for } low \le x \le high
\]

\paragraph{Args:}\label{args-47}

\begin{Shaded}
\begin{Highlighting}[]
\NormalTok{bi.dist.log\_uniform(}
\NormalTok{low,}
\NormalTok{high,}
\NormalTok{validate\_args}\OperatorTok{=}\VariableTok{None}\NormalTok{,}
\NormalTok{name}\OperatorTok{=}\StringTok{\textquotesingle{}x\textquotesingle{}}\NormalTok{,}
\NormalTok{obs}\OperatorTok{=}\VariableTok{None}\NormalTok{,}
\NormalTok{mask}\OperatorTok{=}\VariableTok{None}\NormalTok{,}
\NormalTok{sample}\OperatorTok{=}\VariableTok{False}\NormalTok{,}
\NormalTok{seed}\OperatorTok{=}\DecValTok{0}\NormalTok{,}
\NormalTok{shape}\OperatorTok{=}\NormalTok{(),}
\NormalTok{event}\OperatorTok{=}\DecValTok{0}\NormalTok{,}
\NormalTok{create\_obj}\OperatorTok{=}\VariableTok{False}\NormalTok{,}
\NormalTok{)}
\end{Highlighting}
\end{Shaded}

low (jnp.ndarray): The lower bound of the uniform distribution's
log-space. Must be positive.

high (jnp.ndarray): The upper bound of the uniform distribution's
log-space. Must be positive.

\begin{itemize}
\item
  \emph{shape} (tuple): A multi-purpose argument for shaping. When
  \texttt{sample=False} (model building), this is used with
  \texttt{.expand(shape)} to set the distribution's batch shape. When
  \texttt{sample=True} (direct sampling), this is used as
  \texttt{sample\_shape} to draw a raw JAX array of the given shape.
\item
  \emph{event} (int): The number of batch dimensions to reinterpret as
  event dimensions (used in model building).
\item
  \emph{mask} (jnp.ndarray, bool): Optional boolean array to mask
  observations.
\item
  \emph{create\_obj} (bool): If True, returns the raw BI distribution
  object instead of creating a sample site. This is essential for
  building complex distributions like \texttt{MixtureSameFamily}.
\item
  \emph{sample} (bool, optional): A control-flow argument. If
  \texttt{True}, the function will directly sample a raw JAX array from
  the distribution, bypassing the BI model context. If \texttt{False},
  it will create a \texttt{BI.sample} site within a model. Defaults to
  \texttt{False}.
\item
  \emph{seed} (int, optional): An integer used to generate a JAX PRNGKey
  for reproducible sampling when \texttt{sample=True}. {[}7{]} This
  argument has no effect when \texttt{sample=False}, as randomness is
  handled by BI's inference engine. Defaults to 0.
\item
  \emph{obs} (jnp.ndarray, optional): The observed value for this random
  variable. If provided, the sample site is conditioned on this value,
  and the function returns the observed value. If \texttt{None}, the
  site is treated as a latent (unobserved) random variable. Defaults to
  \texttt{None}.
\item
  \emph{name} (str, optional): The name of the sample site in a BI
  model. This is used to uniquely identify the random variable. Defaults
  to `x'.
\end{itemize}

\paragraph{Returns:}\label{returns-47}

BI LogUniform distribution object (for model building) when
\texttt{sample=False}.

JAX array of samples drawn from the LogUniform distribution (for direct
sampling) when \texttt{sample=True}.

The raw BI distribution object (for advanced use cases) when
\texttt{create\_obj=True}.

\paragraph{Example Usage:}\label{example-usage-44}

\begin{Shaded}
\begin{Highlighting}[]
\ImportTok{from}\NormalTok{ BI }\ImportTok{import}\NormalTok{ bi}
\NormalTok{m }\OperatorTok{=}\NormalTok{ bi(}\StringTok{\textquotesingle{}cpu\textquotesingle{}}\NormalTok{)}
\NormalTok{m.dist.log\_uniform(low}\OperatorTok{=}\FloatTok{0.1}\NormalTok{, high}\OperatorTok{=}\FloatTok{10.0}\NormalTok{, sample}\OperatorTok{=}\VariableTok{True}\NormalTok{)}
\end{Highlighting}
\end{Shaded}

\paragraph{Wrapper of:}\label{wrapper-of-34}

https://num.pyro.ai/en/stable/distributions.html\#loguniform

\begin{center}\rule{0.5\linewidth}{0.5pt}\end{center}

\subsubsection{Logistic}\label{logistic}

Samples from a Logistic distribution.

The Logistic distribution is a continuous probability distribution
defined by two parameters: location and scale. It is often used to model
growth processes and is closely related to the normal distribution.

\[   f(x) = \frac{1}{s} \exp\left(-\frac{(x - \mu)}{s}\right)
\]

\paragraph{Args:}\label{args-48}

\begin{Shaded}
\begin{Highlighting}[]
\NormalTok{bi.dist.logistic(}
\NormalTok{loc}\OperatorTok{=}\FloatTok{0.0}\NormalTok{,}
\NormalTok{scale}\OperatorTok{=}\FloatTok{1.0}\NormalTok{,}
\NormalTok{validate\_args}\OperatorTok{=}\VariableTok{None}\NormalTok{,}
\NormalTok{name}\OperatorTok{=}\StringTok{\textquotesingle{}x\textquotesingle{}}\NormalTok{,}
\NormalTok{obs}\OperatorTok{=}\VariableTok{None}\NormalTok{,}
\NormalTok{mask}\OperatorTok{=}\VariableTok{None}\NormalTok{,}
\NormalTok{sample}\OperatorTok{=}\VariableTok{False}\NormalTok{,}
\NormalTok{seed}\OperatorTok{=}\DecValTok{0}\NormalTok{,}
\NormalTok{shape}\OperatorTok{=}\NormalTok{(),}
\NormalTok{event}\OperatorTok{=}\DecValTok{0}\NormalTok{,}
\NormalTok{create\_obj}\OperatorTok{=}\VariableTok{False}\NormalTok{,}
\NormalTok{)}
\end{Highlighting}
\end{Shaded}

\begin{itemize}
\item
  \emph{loc} (jnp.ndarray or float): The location parameter, specifying
  the median of the distribution. Defaults to 0.0.
\item
  \emph{sample} (jnp.ndarray or float): The scale parameter, which
  determines the spread of the distribution. Must be positive. Defaults
  to 1.0.
\item
  \emph{shape} (tuple): A multi-purpose argument for shaping. When
  \texttt{sample=False} (model building), this is used with
  \texttt{.expand(shape)} to set the distribution's batch shape. When
  \texttt{sample=True} (direct sampling), this is used as
  \texttt{sample\_shape} to draw a raw JAX array of the given shape.
\item
  \emph{event} (int): The number of batch dimensions to reinterpret as
  event dimensions (used in model building).
\item
  \emph{mask} (jnp.ndarray, bool): Optional boolean array to mask
  observations.
\item
  \emph{create\_obj} (bool): If True, returns the raw BI distribution
  object instead of creating a sample site. This is essential for
  building complex distributions like \texttt{MixtureSameFamily}.
\item
  \emph{sample} (bool, optional): A control-flow argument. If
  \texttt{True}, the function will directly sample a raw JAX array from
  the distribution, bypassing the BI model context. If \texttt{False},
  it will create a \texttt{BI.sample} site within a model. Defaults to
  \texttt{False}.
\item
  \emph{seed} (int, optional): An integer used to generate a JAX PRNGKey
  for reproducible sampling when \texttt{sample=True}. {[}7{]} This
  argument has no effect when \texttt{sample=False}, as randomness is
  handled by BI's inference engine. Defaults to 0.
\item
  \emph{obs} (jnp.ndarray, optional): The observed value for this random
  variable. If provided, the sample site is conditioned on this value,
  and the function returns the observed value. If \texttt{None}, the
  site is treated as a latent (unobserved) random variable. Defaults to
  \texttt{None}.
\item
  \emph{name} (str, optional): The name of the sample site in a BI
  model. This is used to uniquely identify the random variable. Defaults
  to `x'.
\end{itemize}

\paragraph{Returns:}\label{returns-48}

BI Logistic distribution object (for model building) when
\texttt{sample=False}. JAX array of samples drawn from the Logistic
distribution (for direct sampling) when \texttt{sample=True}. The raw BI
distribution object (for advanced use cases) when
\texttt{create\_obj=True}.

\paragraph{Example Usage:}\label{example-usage-45}

\begin{Shaded}
\begin{Highlighting}[]
\ImportTok{from}\NormalTok{ BI }\ImportTok{import}\NormalTok{ bi}
\NormalTok{m }\OperatorTok{=}\NormalTok{ bi(}\StringTok{\textquotesingle{}cpu\textquotesingle{}}\NormalTok{)}
\NormalTok{m.dist.logistic(loc}\OperatorTok{=}\FloatTok{0.0}\NormalTok{, scale}\OperatorTok{=}\FloatTok{1.0}\NormalTok{, sample}\OperatorTok{=}\VariableTok{True}\NormalTok{)}
\end{Highlighting}
\end{Shaded}

\paragraph{Wrapper of:}\label{wrapper-of-35}

https://num.pyro.ai/en/stable/distributions.html\#logistic

\begin{center}\rule{0.5\linewidth}{0.5pt}\end{center}

\subsubsection{Low Rank Multivariate
Normal}\label{low-rank-multivariate-normal}

Represents a multivariate normal distribution with a low-rank covariance
structure.

\[
p(x) = \frac{1}{\sqrt{(2\pi)^K |\Sigma|}} 
\exp\left(-\tfrac{1}{2} (x - \mu)^T \Sigma^{-1} (x - \mu)\right)
\]

where:

\begin{itemize}
\tightlist
\item
  \(x\) is a vector of observations.
\item
  \(\mu\) is the mean vector.
\item
  \(\Sigma\) is the covariance matrix, represented in a low-rank form.
\end{itemize}

Parameters: - \emph{loc} (jnp.ndarray): Mean vector.

cov\_factor (jnp.ndarray): Matrix used to construct the covariance
matrix.

cov\_diag (jnp.ndarray): Diagonal elements of the covariance matrix.

\begin{itemize}
\item
  \emph{sample} (bool, optional): A control-flow argument. If
  \texttt{True}, the function will directly sample a raw JAX array from
  the distribution, bypassing the BI model context. If \texttt{False},
  it will create a \texttt{BI.sample} site within a model. Defaults to
  \texttt{False}.
\item
  \emph{seed} (int, optional): An integer used to generate a JAX PRNGKey
  for reproducible sampling when \texttt{sample=True}. {[}7{]} This
  argument has no effect when \texttt{sample=False}, as randomness is
  handled by BI's inference engine. Defaults to 0.
\item
  \emph{obs} (jnp.ndarray, optional): The observed value for this random
  variable. If provided, the sample site is conditioned on this value,
  and the function returns the observed value. If \texttt{None}, the
  site is treated as a latent (unobserved) random variable. Defaults to
  \texttt{None}.
\item
  \emph{name} (str, optional): The name of the sample site in a BI
  model. This is used to uniquely identify the random variable. Defaults
  to `x'.
\end{itemize}

\paragraph{Example Usage:}\label{example-usage-46}

from BI import bi m = bi(`cpu') event\_size = 100 \# Our distribution
has 100 dimensions rank = 5\\
m.dist.low\_rank\_multivariate\_normal( - \emph{loc}=m.dist.normal(0,1,
shape = (event\_size,), sample=True)\emph{2,
cov\_factor=m.dist.normal(0,1, shape = (event\_size, rank),
sample=True), cov\_diag=jnp.exp(m.dist.normal(0,1, shape =
(event\_size,), sample=True)) } 0.1, sample=True )

\paragraph{Wrapper of:}\label{wrapper-of-36}

https://num.pyro.ai/en/stable/distributions.html\#lowrankmultivariatenormal

\begin{Shaded}
\begin{Highlighting}[]
\NormalTok{bi.dist.low\_rank\_multivariate\_normal(}
\NormalTok{loc,}
\NormalTok{cov\_factor,}
\NormalTok{cov\_diag,}
\NormalTok{validate\_args}\OperatorTok{=}\VariableTok{None}\NormalTok{,}
\NormalTok{name}\OperatorTok{=}\StringTok{\textquotesingle{}x\textquotesingle{}}\NormalTok{,}
\NormalTok{obs}\OperatorTok{=}\VariableTok{None}\NormalTok{,}
\NormalTok{mask}\OperatorTok{=}\VariableTok{None}\NormalTok{,}
\NormalTok{sample}\OperatorTok{=}\VariableTok{False}\NormalTok{,}
\NormalTok{seed}\OperatorTok{=}\DecValTok{0}\NormalTok{,}
\NormalTok{shape}\OperatorTok{=}\NormalTok{(),}
\NormalTok{event}\OperatorTok{=}\DecValTok{0}\NormalTok{,}
\NormalTok{create\_obj}\OperatorTok{=}\VariableTok{False}\NormalTok{,}
\NormalTok{)}
\end{Highlighting}
\end{Shaded}

\begin{center}\rule{0.5\linewidth}{0.5pt}\end{center}

\subsubsection{Lower Truncated Power
Law}\label{lower-truncated-power-law}

Lower truncated power law distribution with \(\alpha\) index.

The probability density function (PDF) is given by:

\[
f(x; \alpha, a) = (-\alpha-1)a^{-\alpha - 1}x^{-\alpha},
\qquad x \geq a, \qquad \alpha < -1,
\]

where \(a\) is the lower bound.

\paragraph{Args:}\label{args-49}

\begin{Shaded}
\begin{Highlighting}[]
\NormalTok{bi.dist.lower\_truncated\_power\_law(}
\NormalTok{alpha,}
\NormalTok{low,}
\NormalTok{validate\_args}\OperatorTok{=}\VariableTok{None}\NormalTok{,}
\NormalTok{name}\OperatorTok{=}\StringTok{\textquotesingle{}x\textquotesingle{}}\NormalTok{,}
\NormalTok{obs}\OperatorTok{=}\VariableTok{None}\NormalTok{,}
\NormalTok{mask}\OperatorTok{=}\VariableTok{None}\NormalTok{,}
\NormalTok{sample}\OperatorTok{=}\VariableTok{False}\NormalTok{,}
\NormalTok{seed}\OperatorTok{=}\DecValTok{0}\NormalTok{,}
\NormalTok{shape}\OperatorTok{=}\NormalTok{(),}
\NormalTok{event}\OperatorTok{=}\DecValTok{0}\NormalTok{,}
\NormalTok{create\_obj}\OperatorTok{=}\VariableTok{False}\NormalTok{,}
\NormalTok{)}
\end{Highlighting}
\end{Shaded}

alpha (jnp.ndarray): index of the power law distribution. Must be less
than -1. low (jnp.ndarray): lower bound of the distribution. Must be
greater than 0.

\begin{itemize}
\item
  \emph{shape} (tuple): A multi-purpose argument for shaping. When
  \texttt{sample=False} (model building), this is used with
  \texttt{.expand(shape)} to set the distribution's batch shape. When
  \texttt{sample=True} (direct sampling), this is used as
  \texttt{sample\_shape} to draw a raw JAX array of the given shape.
\item
  \emph{event} (int): The number of batch dimensions to reinterpret as
  event dimensions (used in model building).
\item
  \emph{mask} (jnp.ndarray, bool): Optional boolean array to mask
  observations.
\item
  \emph{create\_obj} (bool): If True, returns the raw BI distribution
  object instead of creating a sample site. This is essential for
  building complex distributions like \texttt{MixtureSameFamily}.
\item
  \emph{sample} (bool, optional): A control-flow argument. If
  \texttt{True}, the function will directly sample a raw JAX array from
  the distribution, bypassing the BI model context. If \texttt{False},
  it will create a \texttt{BI.sample} site within a model. Defaults to
  \texttt{False}.
\item
  \emph{seed} (int, optional): An integer used to generate a JAX PRNGKey
  for reproducible sampling when \texttt{sample=True}. {[}7{]} This
  argument has no effect when \texttt{sample=False}, as randomness is
  handled by BI's inference engine. Defaults to 0.
\item
  \emph{obs} (jnp.ndarray, optional): The observed value for this random
  variable. If provided, the sample site is conditioned on this value,
  and the function returns the observed value. If \texttt{None}, the
  site is treated as a latent (unobserved) random variable. Defaults to
  \texttt{None}.
\item
  \emph{name} (str, optional): The name of the sample site in a BI
  model. This is used to uniquely identify the random variable. Defaults
  to `x'.
\end{itemize}

\paragraph{Returns:}\label{returns-49}

\begin{itemize}
\tightlist
\item
  When \texttt{sample=False}: A BI LowerTruncatedPowerLaw distribution
  object (for model building).
\item
  When \texttt{sample=True}: A JAX array of samples drawn from the
  LowerTruncatedPowerLaw distribution (for direct sampling).
\item
  When \texttt{create\_obj=True}: The raw BI distribution object (for
  advanced use cases).
\end{itemize}

\paragraph{Example Usage:}\label{example-usage-47}

\begin{Shaded}
\begin{Highlighting}[]
\ImportTok{from}\NormalTok{ BI }\ImportTok{import}\NormalTok{ bi}
\NormalTok{m }\OperatorTok{=}\NormalTok{ bi(}\StringTok{\textquotesingle{}cpu\textquotesingle{}}\NormalTok{)}
\NormalTok{m.dist.lower\_truncated\_power\_law(alpha}\OperatorTok{={-}}\FloatTok{2.0}\NormalTok{, low}\OperatorTok{=}\FloatTok{1.0}\NormalTok{, sample}\OperatorTok{=}\VariableTok{True}\NormalTok{)}
\end{Highlighting}
\end{Shaded}

\paragraph{Wrapper of:}\label{wrapper-of-37}

https://num.pyro.ai/en/stable/distributions.html\#lowertruncatedpowerlaw

\begin{center}\rule{0.5\linewidth}{0.5pt}\end{center}

\subsubsection{Matrix Normal}\label{matrix-normal}

Samples from a Matrix Normal distribution, which is a multivariate
normal distribution over matrices. The distribution is characterized by
a location matrix and two lower triangular matrices that define the
correlation structure. The distribution is related to the multivariate
normal distribution in the following way. If \(X ~ MN(loc,U,V)\) then
\(vec(X) ~ MVN(vec(loc), kron(V,U) )\).

\[
p(x) = \frac{1}{2\pi^{p/2} |\Sigma|^{1/2}} \exp\left(-\frac{1}{2} (x - \mu)^T \Sigma^{-1} (x - \mu)\right)
\]

\paragraph{Args:}\label{args-50}

\begin{Shaded}
\begin{Highlighting}[]
\NormalTok{bi.dist.matrix\_normal(}
\NormalTok{loc,}
\NormalTok{scale\_tril\_row,}
\NormalTok{scale\_tril\_column,}
\NormalTok{validate\_args}\OperatorTok{=}\VariableTok{None}\NormalTok{,}
\NormalTok{name}\OperatorTok{=}\StringTok{\textquotesingle{}x\textquotesingle{}}\NormalTok{,}
\NormalTok{obs}\OperatorTok{=}\VariableTok{None}\NormalTok{,}
\NormalTok{mask}\OperatorTok{=}\VariableTok{None}\NormalTok{,}
\NormalTok{sample}\OperatorTok{=}\VariableTok{False}\NormalTok{,}
\NormalTok{seed}\OperatorTok{=}\DecValTok{0}\NormalTok{,}
\NormalTok{shape}\OperatorTok{=}\NormalTok{(),}
\NormalTok{event}\OperatorTok{=}\DecValTok{0}\NormalTok{,}
\NormalTok{create\_obj}\OperatorTok{=}\VariableTok{False}\NormalTok{,}
\NormalTok{)}
\end{Highlighting}
\end{Shaded}

\begin{itemize}
\tightlist
\item
  \emph{loc} (array\_like): Location of the distribution.
\end{itemize}

scale\_tril\_row (array\_like): Lower cholesky of rows correlation
matrix.

scale\_tril\_column (array\_like): Lower cholesky of columns correlation
matrix.

\begin{itemize}
\item
  \emph{shape} (tuple): A multi-purpose argument for shaping. When
  \texttt{sample=False} (model building), this is used with
  \texttt{.expand(shape)} to set the distribution's batch shape. When
  \texttt{sample=True} (direct sampling), this is used as
  \texttt{sample\_shape} to draw a raw JAX array of the given shape.
\item
  \emph{event} (int): The number of batch dimensions to reinterpret as
  event dimensions (used in model building).
\item
  \emph{mask} (jnp.ndarray, bool): Optional boolean array to mask
  observations.
\item
  \emph{create\_obj} (bool): If True, returns the raw BI distribution
  object instead of creating a sample site. This is essential for
  building complex distributions like \texttt{MixtureSameFamily}.
\item
  \emph{sample} (bool, optional): A control-flow argument. If
  \texttt{True}, the function will directly sample a raw JAX array from
  the distribution, bypassing the BI model context. If \texttt{False},
  it will create a \texttt{BI.sample} site within a model. Defaults to
  \texttt{False}.
\item
  \emph{seed} (int, optional): An integer used to generate a JAX PRNGKey
  for reproducible sampling when \texttt{sample=True}. {[}7{]} This
  argument has no effect when \texttt{sample=False}, as randomness is
  handled by BI's inference engine. Defaults to 0.
\item
  \emph{obs} (jnp.ndarray, optional): The observed value for this random
  variable. If provided, the sample site is conditioned on this value,
  and the function returns the observed value. If \texttt{None}, the
  site is treated as a latent (unobserved) random variable. Defaults to
  \texttt{None}.
\item
  \emph{name} (str, optional): The name of the sample site in a BI
  model. This is used to uniquely identify the random variable. Defaults
  to `x'.
\end{itemize}

\paragraph{Returns:}\label{returns-50}

\begin{itemize}
\item
  When \texttt{sample=False}: A BI MatrixNormal distribution object (for
  model building).
\item
  When \texttt{sample=True}: A JAX array of samples drawn from the
  MatrixNormal distribution (for direct sampling).
\item
  When \texttt{create\_obj=True}: The raw BI distribution object (for
  advanced use cases).
\end{itemize}

\paragraph{Example Usage:}\label{example-usage-48}

\begin{Shaded}
\begin{Highlighting}[]
\ImportTok{from}\NormalTok{ BI }\ImportTok{import}\NormalTok{ bi}
\ImportTok{import}\NormalTok{ jax.numpy }\ImportTok{as}\NormalTok{ jnp}
\NormalTok{m }\OperatorTok{=}\NormalTok{ bi(}\StringTok{\textquotesingle{}cpu\textquotesingle{}}\NormalTok{)            }
\NormalTok{n\_rows, n\_cols }\OperatorTok{=} \DecValTok{3}\NormalTok{, }\DecValTok{4}

\OperatorTok{{-}} \OperatorTok{*}\NormalTok{loc}\OperatorTok{*} \OperatorTok{=}\NormalTok{ jnp.zeros((n\_rows, n\_cols))}
\NormalTok{U\_row\_cov }\OperatorTok{=}\NormalTok{ jnp.array([[}\FloatTok{1.0}\NormalTok{, }\FloatTok{0.5}\NormalTok{, }\FloatTok{0.2}\NormalTok{],}
\NormalTok{[}\FloatTok{0.5}\NormalTok{, }\FloatTok{1.0}\NormalTok{, }\FloatTok{0.3}\NormalTok{],}
\NormalTok{[}\FloatTok{0.2}\NormalTok{, }\FloatTok{0.3}\NormalTok{, }\FloatTok{1.0}\NormalTok{]])}
\NormalTok{scale\_tril\_row }\OperatorTok{=}\NormalTok{ jnp.linalg.cholesky(U\_row\_cov)}

\NormalTok{V\_col\_cov }\OperatorTok{=}\NormalTok{ jnp.array([[}\FloatTok{2.0}\NormalTok{, }\OperatorTok{{-}}\FloatTok{0.8}\NormalTok{, }\FloatTok{0.1}\NormalTok{, }\FloatTok{0.4}\NormalTok{],}
\NormalTok{[}\OperatorTok{{-}}\FloatTok{0.8}\NormalTok{, }\FloatTok{2.0}\NormalTok{, }\FloatTok{0.2}\NormalTok{, }\OperatorTok{{-}}\FloatTok{0.2}\NormalTok{],}
\NormalTok{[}\FloatTok{0.1}\NormalTok{, }\FloatTok{0.2}\NormalTok{, }\FloatTok{2.0}\NormalTok{, }\FloatTok{0.0}\NormalTok{],}
\NormalTok{[}\FloatTok{0.4}\NormalTok{, }\OperatorTok{{-}}\FloatTok{0.2}\NormalTok{, }\FloatTok{0.0}\NormalTok{, }\FloatTok{2.0}\NormalTok{]])}

\CommentTok{\# The argument passed to the distribution is its Cholesky factor}
\NormalTok{scale\_tril\_column }\OperatorTok{=}\NormalTok{ jnp.linalg.cholesky(V\_col\_cov)}

\NormalTok{m.dist.matrix\_normal(}
\NormalTok{oc}\OperatorTok{=}\NormalTok{loc, }
\NormalTok{scale\_tril\_row}\OperatorTok{=}\NormalTok{scale\_tril\_row, }
\NormalTok{scale\_tril\_column}\OperatorTok{=}\NormalTok{scale\_tril\_column, }
\NormalTok{sample}\OperatorTok{=}\VariableTok{True}
\NormalTok{)}
\end{Highlighting}
\end{Shaded}

\paragraph{Wrapper of:
https://num.pyro.ai/en/stable/distributions.html\#matrixnormal\_lowercase}\label{wrapper-of-httpsnum.pyro.aienstabledistributions.htmlmatrixnormal_lowercase}

\begin{center}\rule{0.5\linewidth}{0.5pt}\end{center}

\subsubsection{A marginalized finite mixture of component
distributions.}\label{a-marginalized-finite-mixture-of-component-distributions.}

This distribution represents a mixture of component distributions, where
the mixing weights are determined by a Categorical distribution. The
resulting distribution can be either a MixtureGeneral (when component
distributions are a list) or a MixtureSameFamily (when component
distributions are a single distribution).

\[   p(x) = \sum_{i=1}^{K} w_i p_i(x)
\]

\paragraph{Args:}\label{args-51}

\begin{Shaded}
\begin{Highlighting}[]
\NormalTok{bi.dist.mixture(}
\NormalTok{mixing\_distribution,}
\NormalTok{component\_distributions,}
\NormalTok{validate\_args}\OperatorTok{=}\VariableTok{None}\NormalTok{,}
\NormalTok{name}\OperatorTok{=}\StringTok{\textquotesingle{}x\textquotesingle{}}\NormalTok{,}
\NormalTok{obs}\OperatorTok{=}\VariableTok{None}\NormalTok{,}
\NormalTok{mask}\OperatorTok{=}\VariableTok{None}\NormalTok{,}
\NormalTok{sample}\OperatorTok{=}\VariableTok{False}\NormalTok{,}
\NormalTok{seed}\OperatorTok{=}\DecValTok{0}\NormalTok{,}
\NormalTok{shape}\OperatorTok{=}\NormalTok{(),}
\NormalTok{event}\OperatorTok{=}\DecValTok{0}\NormalTok{,}
\NormalTok{create\_obj}\OperatorTok{=}\VariableTok{False}\NormalTok{,}
\NormalTok{)}
\end{Highlighting}
\end{Shaded}

\begin{itemize}
\item
  \emph{shape} (tuple): A multi-purpose argument for shaping. When
  \texttt{sample=False} (model building), this is used with
  \texttt{.expand(shape)} to set the distribution's batch shape. When
  \texttt{sample=True} (direct sampling), this is used as
  \texttt{sample\_shape} to draw a raw JAX array of the given shape.
\item
  \emph{event} (int): The number of batch dimensions to reinterpret as
  event dimensions (used in model building).
\item
  \emph{mask} (jnp.ndarray, bool): Optional boolean array to mask
  observations.
\item
  \emph{create\_obj} (bool): If True, returns the raw BI distribution
  object instead of creating a sample site. This is essential for
  building complex distributions like \texttt{MixtureSameFamily}.
\item
  \emph{sample} (bool, optional): A control-flow argument. If
  \texttt{True}, the function will directly sample a raw JAX array from
  the distribution, bypassing the BI model context. If \texttt{False},
  it will create a \texttt{BI.sample} site within a model. Defaults to
  \texttt{False}.
\item
  \emph{seed} (int, optional): An integer used to generate a JAX PRNGKey
  for reproducible sampling when \texttt{sample=True}. {[}7{]} This
  argument has no effect when \texttt{sample=False}, as randomness is
  handled by BI's inference engine. Defaults to 0.
\item
  \emph{obs} (jnp.ndarray, optional): The observed value for this random
  variable. If provided, the sample site is conditioned on this value,
  and the function returns the observed value. If \texttt{None}, the
  site is treated as a latent (unobserved) random variable. Defaults to
  \texttt{None}.
\item
  \emph{name} (str, optional): The name of the sample site in a BI
  model. This is used to uniquely identify the random variable. Defaults
  to `x'.
\end{itemize}

\paragraph{Returns:}\label{returns-51}

\begin{itemize}
\item
  When \texttt{sample=False}: A BI Mixture distribution object (for
  model building).
\item
  When \texttt{sample=True}: A JAX array of samples drawn from the
  Mixture distribution (for direct sampling).
\item
  When \texttt{create\_obj=True}: The raw BI distribution object (for
  advanced use cases).
\end{itemize}

\paragraph{Example Usage:}\label{example-usage-49}

\begin{Shaded}
\begin{Highlighting}[]
\ImportTok{from}\NormalTok{ jax }\ImportTok{import}\NormalTok{ random}
\ImportTok{import}\NormalTok{ BI }\ImportTok{as}\NormalTok{ pyro}
\NormalTok{m }\OperatorTok{=}\NormalTok{ pyro.distributions.Mixture(}
\NormalTok{pyro.distributions.Categorical(torch.ones(}\DecValTok{2}\NormalTok{)),}
\NormalTok{[pyro.distributions.Normal(}\DecValTok{0}\NormalTok{, }\DecValTok{1}\NormalTok{), pyro.distributions.Normal(}\DecValTok{2}\NormalTok{, }\DecValTok{1}\NormalTok{)]}
\NormalTok{)}
\NormalTok{samples }\OperatorTok{=}\NormalTok{ m.sample(sample\_shape}\OperatorTok{=}\NormalTok{(}\DecValTok{10}\NormalTok{,))}
\end{Highlighting}
\end{Shaded}

\paragraph{Wrapper of:
https://num.pyro.ai/en/stable/distributions.html\#mixture}\label{wrapper-of-httpsnum.pyro.aienstabledistributions.htmlmixture}

\begin{center}\rule{0.5\linewidth}{0.5pt}\end{center}

\subsubsection{Mixture General}\label{mixture-general}

A finite mixture of component distributions from different families.

\begin{itemize}
\tightlist
\item
  mixing\_distribution: A
  :class:\texttt{\textasciitilde{}numpyro.distributions.Categorical}
  specifying the weights for each mixture component. The size of this
  distribution specifies the number of components in the mixture.
\item
  component\_distributions: A list of \texttt{mixture\_size}
  :class:\texttt{\textasciitilde{}numpyro.distributions.Distribution}
  objects.
\item
  support: A
  :class:\texttt{\textasciitilde{}numpyro.distributions.constraints.Constraint}
  object specifying the support of the mixture distribution. If not
  provided, the support will be inferred from the component
  distributions.
\end{itemize}

The probability density function (PDF) of a MixtureGeneral distribution
is given by:

\[p(x) = \sum_{i=1}^{K} \pi_i p_i(x)
\] where:

\begin{itemize}
\tightlist
\item
  \(K\) is the number of components in the mixture.
\item
  \(\pi_i\) is the mixing weight for the \(i\)-th component, such that
  \(\sum_{i=1}^{K} \pi_i = 1\).
\item
  \(p_i(x)\) is the probability density function of the \(i\)-th
  component distribution.
\end{itemize}

\textbf{Parameters:}

\begin{itemize}
\item
  \textbf{mixing\_distribution}: A \texttt{Categorical} distribution
  representing the mixing weights.
\item
  \textbf{component\_distributions}: A list of distributions
  representing the components of the mixture.
\item
  **sample (bool, optional): A control-flow argument. If \texttt{True},
  the function will directly sample a raw JAX array from the
  distribution, bypassing the BI model context. If \texttt{False}, it
  will create a \texttt{BI.sample} site within a model. Defaults to
  \texttt{False}.
\item
  \textbf{seed (int, optional)}: An integer used to generate a JAX
  PRNGKey for reproducible sampling when \texttt{sample=True}. {[}7{]}
  This argument has no effect when \texttt{sample=False}, as randomness
  is handled by BI's inference engine. Defaults to 0.
\item
  \textbf{obs (jnp.ndarray, optional)}: The observed value for this
  random variable. If provided, the sample site is conditioned on this
  value, and the function returns the observed value. If \texttt{None},
  the site is treated as a latent (unobserved) random variable. Defaults
  to \texttt{None}.
\item
  \textbf{name (str, optional)}: The name of the sample site in a BI
  model. This is used to uniquely identify the random variable. Defaults
  to `x'.
\end{itemize}

\textbf{\#\#\#\# Returns:}

\begin{itemize}
\tightlist
\item
  \textbf{When \texttt{sample=False}}: A BI MixtureGeneral distribution
  object (for model building).
\item
  \textbf{When \texttt{sample=True}}: A JAX array of samples drawn from
  the MixtureGeneral distribution (for direct sampling).
\item
  \textbf{When \texttt{create\_obj=True}}: The raw BI distribution
  object (for advanced use cases).
\end{itemize}

\textbf{\#\#\#\# Example Usage:}

from BI import bi m = bi(`cpu') m.dist.mixture\_general(
mixing\_distribution=m.dist.categorical(probs=jnp.array({[}0.3, 0.7{]}),
create\_obj = True),\\
component\_distributions={[}m.dist.normal(loc=0.0, scale=1.0,
create\_obj=True),m.dist.normal (loc=0.0, scale=1.0,
create\_obj=True){]}, sample = True )

\paragraph{\texorpdfstring{Wrapper of:
\url{https://num.pyro.ai/en/stable/distributions.html\#mixturegeneral}}{Wrapper of: https://num.pyro.ai/en/stable/distributions.html\#mixturegeneral}}\label{wrapper-of-httpsnum.pyro.aienstabledistributions.htmlmixturegeneral}

\begin{Shaded}
\begin{Highlighting}[]
\NormalTok{bi.dist.mixture\_general(}
\NormalTok{mixing\_distribution,}
\NormalTok{component\_distributions,}
\NormalTok{support}\OperatorTok{=}\VariableTok{None}\NormalTok{,}
\NormalTok{validate\_args}\OperatorTok{=}\VariableTok{None}\NormalTok{,}
\NormalTok{name}\OperatorTok{=}\StringTok{\textquotesingle{}x\textquotesingle{}}\NormalTok{,}
\NormalTok{obs}\OperatorTok{=}\VariableTok{None}\NormalTok{,}
\NormalTok{mask}\OperatorTok{=}\VariableTok{None}\NormalTok{,}
\NormalTok{sample}\OperatorTok{=}\VariableTok{False}\NormalTok{,}
\NormalTok{seed}\OperatorTok{=}\DecValTok{0}\NormalTok{,}
\NormalTok{shape}\OperatorTok{=}\NormalTok{(),}
\NormalTok{event}\OperatorTok{=}\DecValTok{0}\NormalTok{,}
\NormalTok{create\_obj}\OperatorTok{=}\VariableTok{False}\NormalTok{,}
\NormalTok{)}
\end{Highlighting}
\end{Shaded}

\begin{center}\rule{0.5\linewidth}{0.5pt}\end{center}

\subsubsection{Finite mixture of component distributions from the same
family.}\label{finite-mixture-of-component-distributions-from-the-same-family.}

This mixture only supports a mixture of component distributions that are
all of the same family. The different components are specified along the
last batch dimension of the input \texttt{component\_distribution}. If
you need a mixture of distributions from different families, use the
more general implementation in
:class:\texttt{\textasciitilde{}numpyro.distributions.MixtureGeneral}.

\[   p(x) = \sum_{k=1}^{K} w_k p_k(x)
\]

where:

\begin{itemize}
\tightlist
\item
  \(K\) is the number of mixture components.
\item
  \(w_k\) is the mixing weight for component \(k\).
\item
  \(p_k(x)\) is the probability density function (PDF) of the \(k\)-th
  component distribution.
\end{itemize}

\paragraph{Args:}\label{args-52}

\begin{Shaded}
\begin{Highlighting}[]
\NormalTok{bi.dist.mixture\_same\_family(}
\NormalTok{mixing\_distribution,}
\NormalTok{component\_distribution,}
\NormalTok{validate\_args}\OperatorTok{=}\VariableTok{None}\NormalTok{,}
\NormalTok{name}\OperatorTok{=}\StringTok{\textquotesingle{}x\textquotesingle{}}\NormalTok{,}
\NormalTok{obs}\OperatorTok{=}\VariableTok{None}\NormalTok{,}
\NormalTok{mask}\OperatorTok{=}\VariableTok{None}\NormalTok{,}
\NormalTok{sample}\OperatorTok{=}\VariableTok{False}\NormalTok{,}
\NormalTok{seed}\OperatorTok{=}\DecValTok{0}\NormalTok{,}
\NormalTok{shape}\OperatorTok{=}\NormalTok{(),}
\NormalTok{event}\OperatorTok{=}\DecValTok{0}\NormalTok{,}
\NormalTok{create\_obj}\OperatorTok{=}\VariableTok{False}\NormalTok{,}
\NormalTok{)}
\end{Highlighting}
\end{Shaded}

\begin{itemize}
\item
  \textbf{Distribution Args}:
\item
  \textbf{mixing\_distribution}: A \texttt{Categorical} distribution
  representing the mixing weights.
\item
  \textbf{component\_distributions}: A list of distributions
  representing the components of the mixture.
\item
  \textbf{Sampling / Modeling Args}:
\item
  \texttt{shape} (tuple): A multi-purpose argument for shaping. When
  \texttt{sample=False} (model building), this is used with
  \texttt{.expand(shape)} to set the distribution's batch shape. When
  \texttt{sample=True} (direct sampling), this is used as
  \texttt{sample\_shape} to draw a raw JAX array of the given shape.
\item
  \texttt{event} (int): The number of batch dimensions to reinterpret as
  event dimensions (used in model building).
\item
  \texttt{mask} (jnp.ndarray, bool): Optional boolean array to mask
  observations.
\item
  \texttt{create\_obj} (bool): If True, returns the raw BI distribution
  object instead of creating a sample site. This is essential for
  building complex distributions like \texttt{MixtureSameFamily}.
\item
  **sample (bool, optional): A control-flow argument. If \texttt{True},
  the function will directly sample a raw JAX array from the
  distribution, bypassing the BI model context. If \texttt{False}, it
  will create a \texttt{BI.sample} site within a model. Defaults to
  \texttt{False}.
\item
  \textbf{seed (int, optional)}: An integer used to generate a JAX
  PRNGKey for reproducible sampling when \texttt{sample=True}. {[}7{]}
  This argument has no effect when \texttt{sample=False}, as randomness
  is handled by BI's inference engine. Defaults to 0.
\item
  \textbf{obs (jnp.ndarray, optional)}: The observed value for this
  random variable. If provided, the sample site is conditioned on this
  value, and the function returns the observed value. If \texttt{None},
  the site is treated as a latent (unobserved) random variable. Defaults
  to \texttt{None}.
\item
  \textbf{name (str, optional)}: The name of the sample site in a BI
  model. This is used to uniquely identify the random variable. Defaults
  to `x'.
\end{itemize}

\paragraph{Returns:}\label{returns-52}

\begin{itemize}
\tightlist
\item
  \textbf{When \texttt{sample=False}}: A BI MixtureSameFamily
  distribution object (for model building).
\item
  \textbf{When \texttt{sample=True}}: A JAX array of samples drawn from
  the MixtureSameFamily distribution (for direct sampling).
\item
  \textbf{When \texttt{create\_obj=True}}: The raw BI distribution
  object (for advanced use cases).
\end{itemize}

\paragraph{Example Usage:}\label{example-usage-50}

\begin{Shaded}
\begin{Highlighting}[]
\ImportTok{from}\NormalTok{ BI }\ImportTok{import}\NormalTok{ bi}
\NormalTok{m }\OperatorTok{=}\NormalTok{ bi(}\StringTok{\textquotesingle{}cpu\textquotesingle{}}\NormalTok{)}
\NormalTok{m.dist.mixture\_same\_family(}
\NormalTok{mixing\_distribution}\OperatorTok{=}\NormalTok{m.dist.categorical(probs}\OperatorTok{=}\NormalTok{jnp.array([}\FloatTok{0.3}\NormalTok{, }\FloatTok{0.7}\NormalTok{]), create\_obj }\OperatorTok{=} \VariableTok{True}\NormalTok{), }
\NormalTok{component\_distribution}\OperatorTok{=}\NormalTok{m.dist.normal(loc}\OperatorTok{=}\FloatTok{0.0}\NormalTok{, scale}\OperatorTok{=}\FloatTok{1.0}\NormalTok{, shape }\OperatorTok{=}\NormalTok{ (}\DecValTok{2}\NormalTok{,), create\_obj}\OperatorTok{=}\VariableTok{True}\NormalTok{),}
\NormalTok{sample }\OperatorTok{=} \VariableTok{True}
\NormalTok{)}
\end{Highlighting}
\end{Shaded}

\paragraph{Wrapper of:
https://num.pyro.ai/en/stable/distributions.html\#mixture-same-family}\label{wrapper-of-httpsnum.pyro.aienstabledistributions.htmlmixture-same-family}

\begin{center}\rule{0.5\linewidth}{0.5pt}\end{center}

\subsubsection{Multinomial}\label{multinomial}

Samples from a Multinomial distribution, which models the probability of
different outcomes in a sequence of independent trials, each with a
fixed number of trials and a fixed set of possible outcomes. It
generalizes the binomial distribution to multiple categories.

\[   P(X = x) = \frac{n!}{x_1! x_2! \cdots x_k!} p_1^{x_1} p_2^{x_2} \cdots p_k^{x_k}
\]

\paragraph{Args:}\label{args-53}

\begin{Shaded}
\begin{Highlighting}[]
\NormalTok{bi.dist.multinomial(}
\NormalTok{total\_count}\OperatorTok{=}\DecValTok{1}\NormalTok{,}
\NormalTok{probs}\OperatorTok{=}\VariableTok{None}\NormalTok{,}
\NormalTok{logits}\OperatorTok{=}\VariableTok{None}\NormalTok{,}
\NormalTok{total\_count\_max}\OperatorTok{=}\VariableTok{None}\NormalTok{,}
\NormalTok{validate\_args}\OperatorTok{=}\VariableTok{None}\NormalTok{,}
\NormalTok{name}\OperatorTok{=}\StringTok{\textquotesingle{}x\textquotesingle{}}\NormalTok{,}
\NormalTok{obs}\OperatorTok{=}\VariableTok{None}\NormalTok{,}
\NormalTok{mask}\OperatorTok{=}\VariableTok{None}\NormalTok{,}
\NormalTok{sample}\OperatorTok{=}\VariableTok{False}\NormalTok{,}
\NormalTok{seed}\OperatorTok{=}\DecValTok{0}\NormalTok{,}
\NormalTok{shape}\OperatorTok{=}\NormalTok{(),}
\NormalTok{event}\OperatorTok{=}\DecValTok{0}\NormalTok{,}
\NormalTok{create\_obj}\OperatorTok{=}\VariableTok{False}\NormalTok{,}
\NormalTok{)}
\end{Highlighting}
\end{Shaded}

total\_count (int or jnp.ndarray): The number of trials.

\begin{itemize}
\tightlist
\item
  \emph{probs} (jnp.ndarray, optional): Event probabilities. Must sum to
  1.
\end{itemize}

logits (jnp.ndarray, optional): Event log probabilities.

total\_count\_max (int, optional): An optional integer providing an
upper bound on \texttt{total\_count}. This is used for performance
optimization with \texttt{lax.scan} when \texttt{total\_count} is a
dynamic JAX tracer, helping to avoid recompilation.

\begin{itemize}
\item
  \emph{shape} (tuple): A multi-purpose argument for shaping. When
  \texttt{sample=False} (model building), this is used with
  \texttt{.expand(shape)} to set the distribution's batch shape. When
  \texttt{sample=True} (direct sampling), this is used as
  \texttt{sample\_shape} to draw a raw JAX array of the given shape.
\item
  \emph{event} (int): The number of batch dimensions to reinterpret as
  event dimensions (used in model building).
\item
  \emph{mask} (jnp.ndarray, bool, optional): Optional boolean array to
  mask observations.
\item
  \emph{create\_obj} (bool, optional): If True, returns the raw BI
  distribution object instead of creating a sample site. This is
  essential for building complex distributions like
  \texttt{MixtureSameFamily}.
\item
  \emph{sample} (bool, optional): A control-flow argument. If
  \texttt{True}, the function will directly sample a raw JAX array from
  the distribution, bypassing the BI model context. If \texttt{False},
  it will create a \texttt{BI.sample} site within a model. Defaults to
  \texttt{False}.
\item
  \emph{seed} (int, optional): An integer used to generate a JAX PRNGKey
  for reproducible sampling when \texttt{sample=True}. {[}7{]} This
  argument has no effect when \texttt{sample=False}, as randomness is
  handled by BI's inference engine. Defaults to 0.
\item
  \emph{obs} (jnp.ndarray, optional): The observed value for this random
  variable. If provided, the sample site is conditioned on this value,
  and the function returns the observed value. If \texttt{None}, the
  site is treated as a latent (unobserved) random variable. Defaults to
  \texttt{None}.
\item
  \emph{name} (str, optional): The name of the sample site in a BI
  model. This is used to uniquely identify the random variable. Defaults
  to `x'.
\end{itemize}

\paragraph{Returns:}\label{returns-53}

\begin{itemize}
\tightlist
\item
  When \texttt{sample=False}: A BI Multinomial distribution object (for
  model building).
\item
  When \texttt{sample=True}: A JAX array of samples drawn from the
  Multinomial distribution (for direct sampling).
\item
  When \texttt{create\_obj=True}: The raw BI distribution object (for
  advanced use cases).
\end{itemize}

\paragraph{Example Usage:}\label{example-usage-51}

\begin{Shaded}
\begin{Highlighting}[]
\ImportTok{from}\NormalTok{ BI }\ImportTok{import}\NormalTok{ bi}
\NormalTok{m }\OperatorTok{=}\NormalTok{ bi(}\StringTok{\textquotesingle{}cpu\textquotesingle{}}\NormalTok{)}
\NormalTok{m.dist.multinomial(total\_count}\OperatorTok{=}\DecValTok{10}\NormalTok{, probs}\OperatorTok{=}\NormalTok{jnp.array([}\FloatTok{0.2}\NormalTok{, }\FloatTok{0.3}\NormalTok{, }\FloatTok{0.5}\NormalTok{]), sample}\OperatorTok{=}\VariableTok{True}\NormalTok{)}
\end{Highlighting}
\end{Shaded}

\paragraph{Wrapper of:
https://num.pyro.ai/en/stable/distributions.html\#multinomial}\label{wrapper-of-httpsnum.pyro.aienstabledistributions.htmlmultinomial}

\begin{center}\rule{0.5\linewidth}{0.5pt}\end{center}

\subsubsection{Multinomial Logits}\label{multinomial-logits}

Samples from a MultinomialLogits distribution.

This distribution represents the probability of observing a specific
outcome from a multinomial experiment, given the logits for each
outcome. The logits are the natural logarithm of the odds of each
outcome.

\[   P(k | \mathbf{\pi}) = \frac{n!}{k! (n-k)!} \prod_{i=1}^k \pi_i
\]

\paragraph{Args:}\label{args-54}

\begin{Shaded}
\begin{Highlighting}[]
\NormalTok{bi.dist.multinomial\_logits(}
\NormalTok{logits,}
\NormalTok{total\_count}\OperatorTok{=}\DecValTok{1}\NormalTok{,}
\NormalTok{total\_count\_max}\OperatorTok{=}\VariableTok{None}\NormalTok{,}
\NormalTok{validate\_args}\OperatorTok{=}\VariableTok{None}\NormalTok{,}
\NormalTok{name}\OperatorTok{=}\StringTok{\textquotesingle{}x\textquotesingle{}}\NormalTok{,}
\NormalTok{obs}\OperatorTok{=}\VariableTok{None}\NormalTok{,}
\NormalTok{mask}\OperatorTok{=}\VariableTok{None}\NormalTok{,}
\NormalTok{sample}\OperatorTok{=}\VariableTok{False}\NormalTok{,}
\NormalTok{seed}\OperatorTok{=}\DecValTok{0}\NormalTok{,}
\NormalTok{shape}\OperatorTok{=}\NormalTok{(),}
\NormalTok{event}\OperatorTok{=}\DecValTok{0}\NormalTok{,}
\NormalTok{create\_obj}\OperatorTok{=}\VariableTok{False}\NormalTok{,}
\NormalTok{)}
\end{Highlighting}
\end{Shaded}

logits (jnp.ndarray): Logits for each outcome. Must be at least
one-dimensional.

total\_count (jnp.ndarray): The total number of trials.

\begin{itemize}
\item
  \emph{shape} (tuple): A multi-purpose argument for shaping. When
  \texttt{sample=False} (model building), this is used with
  \texttt{.expand(shape)} to set the distribution's batch shape. When
  \texttt{sample=True} (direct sampling), this is used as
  \texttt{sample\_shape} to draw a raw JAX array of the given shape.
\item
  \emph{event} (int): The number of batch dimensions to reinterpret as
  event dimensions (used in model building).
\item
  \emph{mask} (jnp.ndarray, bool): Optional boolean array to mask
  observations.
\item
  \emph{create\_obj} (bool): If True, returns the raw BI distribution
  object instead of creating a sample site. This is essential for
  building complex distributions like \texttt{MixtureSameFamily}.
\item
  \emph{sample} (bool, optional): A control-flow argument. If
  \texttt{True}, the function will directly sample a raw JAX array from
  the distribution, bypassing the BI model context. If \texttt{False},
  it will create a \texttt{BI.sample} site within a model. Defaults to
  \texttt{False}.
\item
  \emph{seed} (int, optional): An integer used to generate a JAX PRNGKey
  for reproducible sampling when \texttt{sample=True}. {[}7{]} This
  argument has no effect when \texttt{sample=False}, as randomness is
  handled by BI's inference engine. Defaults to 0.
\item
  \emph{obs} (jnp.ndarray, optional): The observed value for this random
  variable. If provided, the sample site is conditioned on this value,
  and the function returns the observed value. If \texttt{None}, the
  site is treated as a latent (unobserved) random variable. Defaults to
  \texttt{None}.
\item
  \emph{name} (str, optional): The name of the sample site in a BI
  model. This is used to uniquely identify the random variable. Defaults
  to `x'.
\end{itemize}

\paragraph{Returns:}\label{returns-54}

\begin{itemize}
\item
  When \texttt{sample=False}: A BI MultinomialLogits distribution object
  (for model building).
\item
  When \texttt{sample=True}: A JAX array of samples drawn from the
  MultinomialLogits distribution (for direct sampling).
\item
  When \texttt{create\_obj=True}: The raw BI distribution object (for
  advanced use cases).
\end{itemize}

\paragraph{Example Usage:}\label{example-usage-52}

\begin{Shaded}
\begin{Highlighting}[]
\ImportTok{from}\NormalTok{ BI }\ImportTok{import}\NormalTok{ bi}
\NormalTok{m }\OperatorTok{=}\NormalTok{ bi(}\StringTok{\textquotesingle{}cpu\textquotesingle{}}\NormalTok{)}
\NormalTok{m.dist.multinomial\_logits(logits}\OperatorTok{=}\NormalTok{jnp.array([}\FloatTok{1.0}\NormalTok{, }\FloatTok{0.5}\NormalTok{], dtype}\OperatorTok{=}\NormalTok{jnp.float32), total\_count}\OperatorTok{=}\NormalTok{jnp.array(}\DecValTok{5}\NormalTok{, dtype}\OperatorTok{=}\NormalTok{jnp.int32), sample}\OperatorTok{=}\VariableTok{True}\NormalTok{)}
\end{Highlighting}
\end{Shaded}

\paragraph{Wrapper of:}\label{wrapper-of-38}

https://num.pyro.ai/en/stable/distributions.html\#multinomiallogits

\begin{center}\rule{0.5\linewidth}{0.5pt}\end{center}

\subsubsection{Multinomial Probs}\label{multinomial-probs}

Samples from a Multinomial distribution.

The Multinomial distribution models the number of times each of several
discrete outcomes occurs in a fixed number of trials. Each trial
independently results in one of several outcomes, and each outcome has a
probability of occurring.

\[   P(X = x) = \frac{n!}{x_1! x_2! \cdots x_k!} p_1^{x_1} p_2^{x_2} \cdots p_k^{x_k}
\]

where:

\begin{itemize}
\tightlist
\item
  n is the total number of trials.
\item
  x is a vector of counts for each outcome.
\item
  p is a vector of probabilities for each outcome. \$\$
\end{itemize}

\paragraph{Args:}\label{args-55}

\begin{Shaded}
\begin{Highlighting}[]
\NormalTok{bi.dist.multinomial\_probs(}
\NormalTok{probs,}
\NormalTok{total\_count}\OperatorTok{=}\DecValTok{1}\NormalTok{,}
\NormalTok{total\_count\_max}\OperatorTok{=}\VariableTok{None}\NormalTok{,}
\NormalTok{validate\_args}\OperatorTok{=}\VariableTok{None}\NormalTok{,}
\NormalTok{name}\OperatorTok{=}\StringTok{\textquotesingle{}x\textquotesingle{}}\NormalTok{,}
\NormalTok{obs}\OperatorTok{=}\VariableTok{None}\NormalTok{,}
\NormalTok{mask}\OperatorTok{=}\VariableTok{None}\NormalTok{,}
\NormalTok{sample}\OperatorTok{=}\VariableTok{False}\NormalTok{,}
\NormalTok{seed}\OperatorTok{=}\DecValTok{0}\NormalTok{,}
\NormalTok{shape}\OperatorTok{=}\NormalTok{(),}
\NormalTok{event}\OperatorTok{=}\DecValTok{0}\NormalTok{,}
\NormalTok{create\_obj}\OperatorTok{=}\VariableTok{False}\NormalTok{,}
\NormalTok{)}
\end{Highlighting}
\end{Shaded}

\begin{itemize}
\item
  \emph{probs} (jnp.ndarray): Vector of probabilities for each outcome.
  Must sum to 1. total\_count (jnp.ndarray): The number of trials.
\item
  \emph{shape} (tuple): A multi-purpose argument for shaping. When
  \texttt{sample=False} (model building), this is used with
  \texttt{.expand(shape)} to set the distribution's batch shape. When
  \texttt{sample=True} (direct sampling), this is used as
  \texttt{sample\_shape} to draw a raw JAX array of the given shape.
\item
  \emph{event} (int): The number of batch dimensions to reinterpret as
  event dimensions (used in model building).
\item
  \emph{mask} (jnp.ndarray, bool): Optional boolean array to mask
  observations.
\item
  \emph{create\_obj} (bool): If True, returns the raw BI distribution
  object instead of creating a sample site. This is essential for
  building complex distributions like \texttt{MixtureSameFamily}.
\item
  \emph{sample} (bool, optional): A control-flow argument. If
  \texttt{True}, the function will directly sample a raw JAX array from
  the distribution, bypassing the BI model context. If \texttt{False},
  it will create a \texttt{BI.sample} site within a model. Defaults to
  \texttt{False}.
\item
  \emph{seed} (int, optional): An integer used to generate a JAX PRNGKey
  for reproducible sampling when \texttt{sample=True}. {[}7{]} This
  argument has no effect when \texttt{sample=False}, as randomness is
  handled by BI's inference engine. Defaults to 0.
\item
  \emph{obs} (jnp.ndarray, optional): The observed value for this random
  variable. If provided, the sample site is conditioned on this value,
  and the function returns the observed value. If \texttt{None}, the
  site is treated as a latent (unobserved) random variable. Defaults to
  \texttt{None}.
\item
  \emph{name} (str, optional): The name of the sample site in a BI
  model. This is used to uniquely identify the random variable. Defaults
  to `x'.
\end{itemize}

\paragraph{Returns:}\label{returns-55}

BI MultinomialProbs distribution object (for model building). JAX array
of samples drawn from the MultinomialProbs distribution (for direct
sampling). The raw BI distribution object (for advanced use cases).

\paragraph{Example Usage:}\label{example-usage-53}

\begin{Shaded}
\begin{Highlighting}[]
\ImportTok{from}\NormalTok{ BI }\ImportTok{import}\NormalTok{ bi}
\NormalTok{m }\OperatorTok{=}\NormalTok{ bi(}\StringTok{\textquotesingle{}cpu\textquotesingle{}}\NormalTok{)}
\NormalTok{m.dist.multinomial\_probs(probs}\OperatorTok{=}\NormalTok{jnp.array([}\FloatTok{0.2}\NormalTok{, }\FloatTok{0.3}\NormalTok{, }\FloatTok{0.5}\NormalTok{]), total\_count}\OperatorTok{=}\DecValTok{10}\NormalTok{, sample}\OperatorTok{=}\VariableTok{True}\NormalTok{)}
\end{Highlighting}
\end{Shaded}

\paragraph{Wrapper of:}\label{wrapper-of-39}

https://num.pyro.ai/en/stable/distributions.html\#multinomialprobs

\begin{center}\rule{0.5\linewidth}{0.5pt}\end{center}

\subsubsection{Multivariate Normal}\label{multivariate-normal}

The Multivariate Normal distribution, also known as the Gaussian
distribution in multiple dimensions, is a probability distribution that
arises frequently in statistics and machine learning. It is defined by
its mean vector and covariance matrix, which describe the central
tendency and spread of the distribution, respectively.

\[
p(x) = \frac{1}{\sqrt{(2\pi)^n |\Sigma|}} \exp\left(-\frac{1}{2}(x - \mu)^T \Sigma^{-1} (x - \mu)\right)
\]

where: - \(x\) is a \(n\)-dimensional vector of random variables. -
\(\mu\) is the mean vector. - \(\Sigma\) is the covariance matrix.

\paragraph{Args:}\label{args-56}

\begin{Shaded}
\begin{Highlighting}[]
\NormalTok{bi.dist.multivariate\_normal(}
\NormalTok{loc}\OperatorTok{=}\FloatTok{0.0}\NormalTok{,}
\NormalTok{covariance\_matrix}\OperatorTok{=}\VariableTok{None}\NormalTok{,}
\NormalTok{precision\_matrix}\OperatorTok{=}\VariableTok{None}\NormalTok{,}
\NormalTok{scale\_tril}\OperatorTok{=}\VariableTok{None}\NormalTok{,}
\NormalTok{validate\_args}\OperatorTok{=}\VariableTok{None}\NormalTok{,}
\NormalTok{name}\OperatorTok{=}\StringTok{\textquotesingle{}x\textquotesingle{}}\NormalTok{,}
\NormalTok{obs}\OperatorTok{=}\VariableTok{None}\NormalTok{,}
\NormalTok{mask}\OperatorTok{=}\VariableTok{None}\NormalTok{,}
\NormalTok{sample}\OperatorTok{=}\VariableTok{False}\NormalTok{,}
\NormalTok{seed}\OperatorTok{=}\DecValTok{0}\NormalTok{,}
\NormalTok{shape}\OperatorTok{=}\NormalTok{(),}
\NormalTok{event}\OperatorTok{=}\DecValTok{0}\NormalTok{,}
\NormalTok{create\_obj}\OperatorTok{=}\VariableTok{False}\NormalTok{,}
\NormalTok{)}
\end{Highlighting}
\end{Shaded}

\begin{itemize}
\tightlist
\item
  \emph{loc} (tuple): The mean vector of the distribution.
\end{itemize}

covariance\_matrix (jnp.ndarray, optional): The covariance matrix of the
distribution. Must be positive definite.

precision\_matrix (jnp.ndarray, optional): The precision matrix (inverse
of the covariance matrix) of the distribution. Must be positive
definite.

scale\_tril (jnp.ndarray, optional): The lower triangular Cholesky
decomposition of the covariance matrix.

\begin{itemize}
\item
  \emph{shape} (tuple): A multi-purpose argument for shaping. When
  \texttt{sample=False} (model building), this is used with
  \texttt{.expand(shape)} to set the distribution's batch shape. When
  \texttt{sample=True} (direct sampling), this is used as
  \texttt{sample\_shape} to draw a raw JAX array of the given shape.
\item
  \emph{event} (int): The number of batch dimensions to reinterpret as
  event dimensions (used in model building).
\item
  \emph{mask} (jnp.ndarray, bool): Optional boolean array to mask
  observations.
\item
  \emph{create\_obj} (bool): If True, returns the raw BI distribution
  object instead of creating a sample site. This is essential for
  building complex distributions like \texttt{MixtureSameFamily}.
\item
  \emph{sample} (bool, optional): A control-flow argument. If
  \texttt{True}, the function will directly sample a raw JAX array from
  the distribution, bypassing the BI model context. If \texttt{False},
  it will create a \texttt{BI.sample} site within a model. Defaults to
  \texttt{False}.
\item
  \emph{seed} (int, optional): An integer used to generate a JAX PRNGKey
  for reproducible sampling when \texttt{sample=True}. {[}7{]} This
  argument has no effect when \texttt{sample=False}, as randomness is
  handled by BI's inference engine. Defaults to 0.
\item
  \emph{obs} (jnp.ndarray, optional): The observed value for this random
  variable. If provided, the sample site is conditioned on this value,
  and the function returns the observed value. If \texttt{None}, the
  site is treated as a latent (unobserved) random variable. Defaults to
  \texttt{None}.
\item
  \emph{name} (str, optional): The name of the sample site in a BI
  model. This is used to uniquely identify the random variable. Defaults
  to `x'.
\end{itemize}

\paragraph{Returns:}\label{returns-56}

\begin{itemize}
\item
  When \texttt{sample=False}: A BI MultivariateNormal distribution
  object (for model building).
\item
  When \texttt{sample=True}: A JAX array of samples drawn from the
  MultivariateNormal distribution (for direct sampling).
\item
  When \texttt{create\_obj=True}: The raw BI distribution object (for
  advanced use cases).
\end{itemize}

\paragraph{Example Usage:}\label{example-usage-54}

\begin{Shaded}
\begin{Highlighting}[]
\ImportTok{from}\NormalTok{ BI }\ImportTok{import}\NormalTok{ bi}
\NormalTok{m }\OperatorTok{=}\NormalTok{ bi(}\StringTok{\textquotesingle{}cpu\textquotesingle{}}\NormalTok{)}
\NormalTok{m.dist.multivariate\_normal(}
\OperatorTok{{-}} \OperatorTok{*}\NormalTok{loc}\OperatorTok{*=}\NormalTok{jnp.array([}\FloatTok{1.0}\NormalTok{, }\FloatTok{0.0}\NormalTok{, }\OperatorTok{{-}}\FloatTok{2.0}\NormalTok{]), }
\NormalTok{covariance\_matrix}\OperatorTok{=}\NormalTok{jnp.array([[ }\FloatTok{2.0}\NormalTok{,  }\FloatTok{0.7}\NormalTok{, }\OperatorTok{{-}}\FloatTok{0.3}\NormalTok{],}
\NormalTok{[ }\FloatTok{0.7}\NormalTok{,  }\FloatTok{1.0}\NormalTok{,  }\FloatTok{0.5}\NormalTok{],}
\NormalTok{[}\OperatorTok{{-}}\FloatTok{0.3}\NormalTok{,  }\FloatTok{0.5}\NormalTok{,  }\FloatTok{1.5}\NormalTok{]]), }
\NormalTok{sample}\OperatorTok{=}\VariableTok{True}
\NormalTok{)}
\end{Highlighting}
\end{Shaded}

\paragraph{Wrapper of:
https://num.pyro.ai/en/stable/distributions.html\#multivariate-normal}\label{wrapper-of-httpsnum.pyro.aienstabledistributions.htmlmultivariate-normal}

\begin{center}\rule{0.5\linewidth}{0.5pt}\end{center}

\subsubsection{Multivariate Student's t}\label{multivariate-students-t}

The Multivariate Student's t distribution is a generalization of the
Student's t distribution to multiple dimensions. It is a heavy-tailed
distribution that is often used to model data that is not normally
distributed.

\[
p(x) = \frac{1}{B(df/2, n/2)} \frac{\Gamma(df/2 + n/2)}{\Gamma(df/2)}
\left(1 + \frac{(x - \mu)^T \Sigma^{-1} (x - \mu)}{df}\right)^{-(df + n)/2}
\]

\paragraph{Args:}\label{args-57}

\begin{Shaded}
\begin{Highlighting}[]
\NormalTok{bi.dist.multivariate\_student\_t(}
\NormalTok{df,}
\NormalTok{loc}\OperatorTok{=}\FloatTok{0.0}\NormalTok{,}
\NormalTok{scale\_tril}\OperatorTok{=}\VariableTok{None}\NormalTok{,}
\NormalTok{validate\_args}\OperatorTok{=}\VariableTok{None}\NormalTok{,}
\NormalTok{name}\OperatorTok{=}\StringTok{\textquotesingle{}x\textquotesingle{}}\NormalTok{,}
\NormalTok{obs}\OperatorTok{=}\VariableTok{None}\NormalTok{,}
\NormalTok{mask}\OperatorTok{=}\VariableTok{None}\NormalTok{,}
\NormalTok{sample}\OperatorTok{=}\VariableTok{False}\NormalTok{,}
\NormalTok{seed}\OperatorTok{=}\DecValTok{0}\NormalTok{,}
\NormalTok{shape}\OperatorTok{=}\NormalTok{(),}
\NormalTok{event}\OperatorTok{=}\DecValTok{0}\NormalTok{,}
\NormalTok{create\_obj}\OperatorTok{=}\VariableTok{False}\NormalTok{,}
\NormalTok{)}
\end{Highlighting}
\end{Shaded}

\begin{itemize}
\item
  \emph{df} (jnp.ndarray): Degrees of freedom, must be positive.
\item
  \emph{loc} (jnp.ndarray): Location vector, representing the mean of
  the distribution. scale\_tril (jnp.ndarray): Lower triangular matrix
  defining the scale.
\item
  \emph{shape} (tuple): A multi-purpose argument for shaping. When
  \texttt{sample=False} (model building), this is used with
  \texttt{.expand(shape)} to set the distribution's batch shape. When
  \texttt{sample=True} (direct sampling), this is used as
  \texttt{sample\_shape} to draw a raw JAX array of the given shape.
\item
  \emph{event} (int): The number of batch dimensions to reinterpret as
  event dimensions (used in model building).
\item
  \emph{mask} (jnp.ndarray, bool): Optional boolean array to mask
  observations.
\item
  \emph{create\_obj} (bool): If True, returns the raw BI distribution
  object instead of creating a sample site. This is essential for
  building complex distributions like \texttt{MixtureSameFamily}.
\item
  \emph{sample} (bool, optional): A control-flow argument. If
  \texttt{True}, the function will directly sample a raw JAX array from
  the distribution, bypassing the BI model context. If \texttt{False},
  it will create a \texttt{BI.sample} site within a model. Defaults to
  \texttt{False}.
\item
  \emph{seed} (int, optional): An integer used to generate a JAX PRNGKey
  for reproducible sampling when \texttt{sample=True}. {[}7{]} This
  argument has no effect when \texttt{sample=False}, as randomness is
  handled by BI's inference engine. Defaults to 0.
\item
  \emph{obs} (jnp.ndarray, optional): The observed value for this random
  variable. If provided, the sample site is conditioned on this value,
  and the function returns the observed value. If \texttt{None}, the
  site is treated as a latent (unobserved) random variable. Defaults to
  \texttt{None}.
\item
  \emph{name} (str, optional): The name of the sample site in a BI
  model. This is used to uniquely identify the random variable. Defaults
  to `x'.
\end{itemize}

\paragraph{Returns:}\label{returns-57}

\begin{itemize}
\tightlist
\item
  When \texttt{sample=False}: A BI MultivariateStudentT distribution
  object (for model building).
\item
  When \texttt{sample=True}: A JAX array of samples drawn from the
  MultivariateStudentT distribution (for direct sampling).
\item
  When \texttt{create\_obj=True}: The raw BI distribution object (for
  advanced use cases).
\end{itemize}

\paragraph{Example Usage:}\label{example-usage-55}

\begin{Shaded}
\begin{Highlighting}[]
\ImportTok{from}\NormalTok{ BI }\ImportTok{import}\NormalTok{ bi}
\ImportTok{import}\NormalTok{ jax.numpy }\ImportTok{as}\NormalTok{ jnp}
\NormalTok{m }\OperatorTok{=}\NormalTok{ bi(}\StringTok{\textquotesingle{}cpu\textquotesingle{}}\NormalTok{)}
\NormalTok{m.dist.multivariate\_student\_t(}
\NormalTok{df }\OperatorTok{=} \DecValTok{2}\NormalTok{,}
\OperatorTok{{-}} \OperatorTok{*}\NormalTok{loc}\OperatorTok{*=}\NormalTok{jnp.array([}\FloatTok{1.0}\NormalTok{, }\FloatTok{0.0}\NormalTok{, }\OperatorTok{{-}}\FloatTok{2.0}\NormalTok{]), }
\NormalTok{scale\_tril}\OperatorTok{=}\NormalTok{jnp.linalg.cholesky(}
\NormalTok{jnp.array([[ }\FloatTok{2.0}\NormalTok{,  }\FloatTok{0.7}\NormalTok{, }\OperatorTok{{-}}\FloatTok{0.3}\NormalTok{],}
\NormalTok{[ }\FloatTok{0.7}\NormalTok{,  }\FloatTok{1.0}\NormalTok{,  }\FloatTok{0.5}\NormalTok{],}
\NormalTok{[}\OperatorTok{{-}}\FloatTok{0.3}\NormalTok{,  }\FloatTok{0.5}\NormalTok{,  }\FloatTok{1.5}\NormalTok{]])), }
\NormalTok{sample}\OperatorTok{=}\VariableTok{True}
\NormalTok{)}
\end{Highlighting}
\end{Shaded}

\paragraph{Wrapper of:}\label{wrapper-of-40}

https://num.pyro.ai/en/stable/distributions.html\#multivariatestudentt

\begin{center}\rule{0.5\linewidth}{0.5pt}\end{center}

\subsubsection{Negative Binomial}\label{negative-binomial}

The NegativeBinomial distribution models the number of failures before
the first success in a sequence of independent Bernoulli trials. It is
characterized by two parameters: `total\_count' (r) and `probs' or
`logits' (p).

\[P(k) = \binom{k+r-1}{r-1} p^r (1-p)^k
\]

\paragraph{Args:}\label{args-58}

\begin{Shaded}
\begin{Highlighting}[]
\NormalTok{bi.dist.negative\_binomial(}
\NormalTok{total\_count,}
\NormalTok{probs}\OperatorTok{=}\VariableTok{None}\NormalTok{,}
\NormalTok{logits}\OperatorTok{=}\VariableTok{None}\NormalTok{,}
\NormalTok{validate\_args}\OperatorTok{=}\VariableTok{None}\NormalTok{,}
\NormalTok{name}\OperatorTok{=}\StringTok{\textquotesingle{}x\textquotesingle{}}\NormalTok{,}
\NormalTok{obs}\OperatorTok{=}\VariableTok{None}\NormalTok{,}
\NormalTok{mask}\OperatorTok{=}\VariableTok{None}\NormalTok{,}
\NormalTok{sample}\OperatorTok{=}\VariableTok{False}\NormalTok{,}
\NormalTok{seed}\OperatorTok{=}\DecValTok{0}\NormalTok{,}
\NormalTok{shape}\OperatorTok{=}\NormalTok{(),}
\NormalTok{event}\OperatorTok{=}\DecValTok{0}\NormalTok{,}
\NormalTok{create\_obj}\OperatorTok{=}\VariableTok{False}\NormalTok{,}
\NormalTok{)}
\end{Highlighting}
\end{Shaded}

total\_count (jnp.ndarray): The total number of events.

\begin{itemize}
\item
  \emph{shape} (tuple): A multi-purpose argument for shaping. When
  \texttt{sample=False} (model building), this is used with
  \texttt{.expand(shape)} to set the distribution's batch shape. When
  \texttt{sample=True} (direct sampling), this is used as
  \texttt{sample\_shape} to draw a raw JAX array of the given shape.
\item
  \emph{event} (int): The number of batch dimensions to reinterpret as
  event dimensions (used in model building).
\item
  \emph{mask} (jnp.ndarray, bool): Optional boolean array to mask
  observations.
\item
  \emph{create\_obj} (bool): If True, returns the raw BI distribution
  object instead of creating a sample site. This is essential for
  building complex distributions like \texttt{MixtureSameFamily}.
\item
  \emph{sample} (bool, optional): A control-flow argument. If
  \texttt{True}, the function will directly sample a raw JAX array from
  the distribution, bypassing the BI model context. If \texttt{False},
  it will create a \texttt{BI.sample} site within a model. Defaults to
  \texttt{False}.
\item
  \emph{seed} (int, optional): An integer used to generate a JAX PRNGKey
  for reproducible sampling when \texttt{sample=True}. {[}7{]} This
  argument has no effect when \texttt{sample=False}, as randomness is
  handled by BI's inference engine. Defaults to 0.
\item
  \emph{obs} (jnp.ndarray, optional): The observed value for this random
  variable. If provided, the sample site is conditioned on this value,
  and the function returns the observed value. If \texttt{None}, the
  site is treated as a latent (unobserved) random variable. Defaults to
  \texttt{None}.
\item
  \emph{name} (str, optional): The name of the sample site in a BI
  model. This is used to uniquely identify the random variable. Defaults
  to `x'.
\end{itemize}

\paragraph{Returns:}\label{returns-58}

BI NegativeBinomial distribution object (for model building). JAX array
of samples drawn from the NegativeBinomial distribution (for direct
sampling). The raw BI distribution object (for advanced use cases).

\paragraph{Example Usage:}\label{example-usage-56}

\begin{Shaded}
\begin{Highlighting}[]
\ImportTok{from}\NormalTok{ BI }\ImportTok{import}\NormalTok{ bi}
\NormalTok{m }\OperatorTok{=}\NormalTok{ bi(}\StringTok{\textquotesingle{}cpu\textquotesingle{}}\NormalTok{)}
\NormalTok{m.dist.negative\_binomial(total\_count}\OperatorTok{=}\FloatTok{5.0}\NormalTok{,probs }\OperatorTok{=}\NormalTok{ jnp.array([}\FloatTok{0.2}\NormalTok{, }\FloatTok{0.3}\NormalTok{, }\FloatTok{0.5}\NormalTok{]), sample}\OperatorTok{=}\VariableTok{True}\NormalTok{)}
\end{Highlighting}
\end{Shaded}

\paragraph{Wrapper of:
https://num.pyro.ai/en/stable/distributions.html\#negativebinomial}\label{wrapper-of-httpsnum.pyro.aienstabledistributions.htmlnegativebinomial}

\begin{center}\rule{0.5\linewidth}{0.5pt}\end{center}

\subsubsection{Negative Binomial Logits}\label{negative-binomial-logits}

Samples from a Negative Binomial Logits distribution.

The Negative Binomial Logits distribution is a generalization of the
Negative Binomial distribution where the parameter `r' (number of
successes) is expressed as a function of a logit parameter. This allows
for more flexible modeling of count data.

\[
P(k) = \frac{e^{-n \cdot \text{softplus}(x)} \cdot \text{softplus}(-x)^k}{k!}
\]

\paragraph{Args:}\label{args-59}

\begin{Shaded}
\begin{Highlighting}[]
\NormalTok{bi.dist.negative\_binomial\_logits(}
\NormalTok{total\_count,}
\NormalTok{logits,}
\NormalTok{validate\_args}\OperatorTok{=}\VariableTok{None}\NormalTok{,}
\NormalTok{name}\OperatorTok{=}\StringTok{\textquotesingle{}x\textquotesingle{}}\NormalTok{,}
\NormalTok{obs}\OperatorTok{=}\VariableTok{None}\NormalTok{,}
\NormalTok{mask}\OperatorTok{=}\VariableTok{None}\NormalTok{,}
\NormalTok{sample}\OperatorTok{=}\VariableTok{False}\NormalTok{,}
\NormalTok{seed}\OperatorTok{=}\DecValTok{0}\NormalTok{,}
\NormalTok{shape}\OperatorTok{=}\NormalTok{(),}
\NormalTok{event}\OperatorTok{=}\DecValTok{0}\NormalTok{,}
\NormalTok{create\_obj}\OperatorTok{=}\VariableTok{False}\NormalTok{,}
\NormalTok{)}
\end{Highlighting}
\end{Shaded}

total\_count (jnp.ndarray): The parameter controlling the shape of the
distribution. Represents the total number of trials.

logits (jnp.ndarray): The log-odds parameter. Related to the probability
of success.

\begin{itemize}
\item
  \emph{shape} (tuple): A multi-purpose argument for shaping. When
  \texttt{sample=False} (model building), this is used with
  \texttt{.expand(shape)} to set the distribution's batch shape. When
  \texttt{sample=True} (direct sampling), this is used as
  \texttt{sample\_shape} to draw a raw JAX array of the given shape.
\item
  \emph{event} (int): The number of batch dimensions to reinterpret as
  event dimensions (used in model building).
\item
  \emph{mask} (jnp.ndarray, bool): Optional boolean array to mask
  observations.
\item
  \emph{create\_obj} (bool): If True, returns the raw BI distribution
  object instead of creating a sample site. This is essential for
  building complex distributions like \texttt{MixtureSameFamily}.
\item
  \emph{sample} (bool, optional): A control-flow argument. If
  \texttt{True}, the function will directly sample a raw JAX array from
  the distribution, bypassing the BI model context. If \texttt{False},
  it will create a \texttt{BI.sample} site within a model. Defaults to
  \texttt{False}.
\item
  \emph{seed} (int, optional): An integer used to generate a JAX PRNGKey
  for reproducible sampling when \texttt{sample=True}. {[}7{]} This
  argument has no effect when \texttt{sample=False}, as randomness is
  handled by BI's inference engine. Defaults to 0.
\item
  \emph{obs} (jnp.ndarray, optional): The observed value for this random
  variable. If provided, the sample site is conditioned on this value,
  and the function returns the observed value. If \texttt{None}, the
  site is treated as a latent (unobserved) random variable. Defaults to
  \texttt{None}.
\item
  \emph{name} (str, optional): The name of the sample site in a BI
  model. This is used to uniquely identify the random variable. Defaults
  to `x'.
\end{itemize}

\paragraph{Returns:}\label{returns-59}

Negative Binomial Logits: A BI Negative Binomial Logits distribution
object (for model building).

jnp.ndarray: A JAX array of samples drawn from the Negative Binomial
Logits distribution (for direct sampling).

Negative Binomial Logits: The raw BI distribution object (for advanced
use cases).

\paragraph{Example Usage:}\label{example-usage-57}

\begin{Shaded}
\begin{Highlighting}[]
\ImportTok{from}\NormalTok{ BI }\ImportTok{import}\NormalTok{ bi}
\NormalTok{m }\OperatorTok{=}\NormalTok{ bi(}\StringTok{\textquotesingle{}cpu\textquotesingle{}}\NormalTok{)}
\NormalTok{m.dist.negative\_binomial\_logits(total\_count}\OperatorTok{=}\FloatTok{5.0}\NormalTok{, logits}\OperatorTok{=}\FloatTok{0.0}\NormalTok{, sample}\OperatorTok{=}\VariableTok{True}\NormalTok{)}
\end{Highlighting}
\end{Shaded}

\paragraph{Wrapper of:}\label{wrapper-of-41}

https://num.pyro.ai/en/stable/distributions.html\#Negative Binomial
Logits

\begin{center}\rule{0.5\linewidth}{0.5pt}\end{center}

\subsubsection{Negative Binomial with
probabilities.}\label{negative-binomial-with-probabilities.}

The Negative Binomial distribution models the number of failures before
the first success in a sequence of independent Bernoulli trials. It is
characterized by two parameters: `concentration' (r) and `rate' (p). In
this implementation, the `concentration' parameter is derived from
`total\_count' and the `rate' parameter is derived from `probs'.

\[
P(k) = \binom{k+r-1}{r-1} p^r (1-p)^k
\]

\paragraph{Args:}\label{args-60}

\begin{Shaded}
\begin{Highlighting}[]
\NormalTok{bi.dist.negative\_binomial\_probs(}
\NormalTok{total\_count,}
\NormalTok{probs,}
\NormalTok{validate\_args}\OperatorTok{=}\VariableTok{None}\NormalTok{,}
\NormalTok{name}\OperatorTok{=}\StringTok{\textquotesingle{}x\textquotesingle{}}\NormalTok{,}
\NormalTok{obs}\OperatorTok{=}\VariableTok{None}\NormalTok{,}
\NormalTok{mask}\OperatorTok{=}\VariableTok{None}\NormalTok{,}
\NormalTok{sample}\OperatorTok{=}\VariableTok{False}\NormalTok{,}
\NormalTok{seed}\OperatorTok{=}\DecValTok{0}\NormalTok{,}
\NormalTok{shape}\OperatorTok{=}\NormalTok{(),}
\NormalTok{event}\OperatorTok{=}\DecValTok{0}\NormalTok{,}
\NormalTok{create\_obj}\OperatorTok{=}\VariableTok{False}\NormalTok{,}
\NormalTok{)}
\end{Highlighting}
\end{Shaded}

total\_count (jnp.ndarray): A numeric vector, matrix, or array
representing the parameter.

\begin{itemize}
\item
  \emph{probs} (jnp.ndarray): A numeric vector representing event
  probabilities. Must sum to 1.
\item
  \emph{shape} (tuple): A multi-purpose argument for shaping. When
  \texttt{sample=False} (model building), this is used with
  \texttt{.expand(shape)} to set the distribution's batch shape. When
  \texttt{sample=True} (direct sampling), this is used as
  \texttt{sample\_shape} to draw a raw JAX array of the given shape.
\item
  \emph{event} (int): The number of batch dimensions to reinterpret as
  event dimensions (used in model building).
\item
  \emph{mask} (jnp.ndarray, bool): Optional boolean array to mask
  observations.
\item
  \emph{create\_obj} (bool): If True, returns the raw BI distribution
  object instead of creating a sample site. This is essential for
  building complex distributions like \texttt{MixtureSameFamily}.
\item
  \emph{sample} (bool, optional): A control-flow argument. If
  \texttt{True}, the function will directly sample a raw JAX array from
  the distribution, bypassing the BI model context. If \texttt{False},
  it will create a \texttt{BI.sample} site within a model. Defaults to
  \texttt{False}.
\item
  \emph{seed} (int, optional): An integer used to generate a JAX PRNGKey
  for reproducible sampling when \texttt{sample=True}. {[}7{]} This
  argument has no effect when \texttt{sample=False}, as randomness is
  handled by BI's inference engine. Defaults to 0.
\item
  \emph{obs} (jnp.ndarray, optional): The observed value for this random
  variable. If provided, the sample site is conditioned on this value,
  and the function returns the observed value. If \texttt{None}, the
  site is treated as a latent (unobserved) random variable. Defaults to
  \texttt{None}.
\item
  \emph{name} (str, optional): The name of the sample site in a BI
  model. This is used to uniquely identify the random variable. Defaults
  to `x'.
\end{itemize}

\paragraph{Returns:}\label{returns-60}

BI NegativeBinomialProbs distribution object (for model building). JAX
array of samples drawn from the NegativeBinomialProbs distribution (for
direct sampling). The raw BI distribution object (for advanced use
cases).

\paragraph{Example Usage:}\label{example-usage-58}

\begin{Shaded}
\begin{Highlighting}[]
\ImportTok{from}\NormalTok{ BI }\ImportTok{import}\NormalTok{ bi}
\NormalTok{m }\OperatorTok{=}\NormalTok{ bi(}\StringTok{\textquotesingle{}cpu\textquotesingle{}}\NormalTok{)}
\NormalTok{m.dist.negative\_binomial\_probs(total\_count}\OperatorTok{=}\FloatTok{10.0}\NormalTok{, probs }\OperatorTok{=}\NormalTok{ jnp.array([}\FloatTok{0.2}\NormalTok{, }\FloatTok{0.3}\NormalTok{, }\FloatTok{0.5}\NormalTok{]), sample}\OperatorTok{=}\VariableTok{True}\NormalTok{)}
\end{Highlighting}
\end{Shaded}

\paragraph{Wrapper of:
https://num.pyro.ai/en/stable/distributions.html\#negativebinomialprobs}\label{wrapper-of-httpsnum.pyro.aienstabledistributions.htmlnegativebinomialprobs}

\begin{center}\rule{0.5\linewidth}{0.5pt}\end{center}

\subsubsection{Normal}\label{normal}

Samples from a Normal (Gaussian) distribution.

The Normal distribution is characterized by its mean (loc) and standard
deviation (scale). It's a continuous probability distribution that
arises frequently in statistics and probability theory.

\[   p(x) = \frac{1}{\sqrt{2\pi\sigma^2}} \exp\left(-\frac{(x-\mu)^2}{2\sigma^2}\right)
\]

\paragraph{Args:}\label{args-61}

\begin{Shaded}
\begin{Highlighting}[]
\NormalTok{bi.dist.normal(}
\NormalTok{loc}\OperatorTok{=}\FloatTok{0.0}\NormalTok{,}
\NormalTok{scale}\OperatorTok{=}\FloatTok{1.0}\NormalTok{,}
\NormalTok{validate\_args}\OperatorTok{=}\VariableTok{None}\NormalTok{,}
\NormalTok{name}\OperatorTok{=}\StringTok{\textquotesingle{}x\textquotesingle{}}\NormalTok{,}
\NormalTok{obs}\OperatorTok{=}\VariableTok{None}\NormalTok{,}
\NormalTok{mask}\OperatorTok{=}\VariableTok{None}\NormalTok{,}
\NormalTok{sample}\OperatorTok{=}\VariableTok{False}\NormalTok{,}
\NormalTok{seed}\OperatorTok{=}\DecValTok{0}\NormalTok{,}
\NormalTok{shape}\OperatorTok{=}\NormalTok{(),}
\NormalTok{event}\OperatorTok{=}\DecValTok{0}\NormalTok{,}
\NormalTok{create\_obj}\OperatorTok{=}\VariableTok{False}\NormalTok{,}
\NormalTok{)}
\end{Highlighting}
\end{Shaded}

\begin{itemize}
\item
  \emph{loc} (jnp.ndarray): The mean of the distribution.
\item
  \emph{sample} (jnp.ndarray): The standard deviation of the
  distribution.
\item
  \emph{shape} (tuple): A multi-purpose argument for shaping. When
  \texttt{sample=False} (model building), this is used with
  \texttt{.expand(shape)} to set the distribution's batch shape. When
  \texttt{sample=True} (direct sampling), this is used as
  \texttt{sample\_shape} to draw a raw JAX array of the given shape.
\item
  \emph{event} (int): The number of batch dimensions to reinterpret as
  event dimensions (used in model building).
\item
  \emph{mask} (jnp.ndarray, bool): Optional boolean array to mask
  observations.
\item
  \emph{create\_obj} (bool): If True, returns the raw BI distribution
  object instead of creating a sample site. This is essential for
  building complex distributions like \texttt{MixtureSameFamily}.
\item
  \emph{sample} (bool, optional): A control-flow argument. If
  \texttt{True}, the function will directly sample a raw JAX array from
  the distribution, bypassing the BI model context. If \texttt{False},
  it will create a \texttt{BI.sample} site within a model. Defaults to
  \texttt{False}.
\item
  \emph{seed} (int, optional): An integer used to generate a JAX PRNGKey
  for reproducible sampling when \texttt{sample=True}. {[}7{]} This
  argument has no effect when \texttt{sample=False}, as randomness is
  handled by BI's inference engine. Defaults to 0.
\item
  \emph{obs} (jnp.ndarray, optional): The observed value for this random
  variable. If provided, the sample site is conditioned on this value,
  and the function returns the observed value. If \texttt{None}, the
  site is treated as a latent (unobserved) random variable. Defaults to
  \texttt{None}.
\item
  \emph{name} (str, optional): The name of the sample site in a BI
  model. This is used to uniquely identify the random variable. Defaults
  to `x'.
\end{itemize}

\paragraph{Returns:}\label{returns-61}

\begin{itemize}
\item
  When \texttt{sample=False}: A BI Normal distribution object (for model
  building).
\item
  When \texttt{sample=True}: A JAX array of samples drawn from the
  Normal distribution (for direct sampling).
\item
  When \texttt{create\_obj=True}: The raw BI distribution object (for
  advanced use cases).
\end{itemize}

\paragraph{Example Usage:}\label{example-usage-59}

\begin{Shaded}
\begin{Highlighting}[]
\ImportTok{from}\NormalTok{ BI }\ImportTok{import}\NormalTok{ bi}
\NormalTok{m }\OperatorTok{=}\NormalTok{ bi(}\StringTok{\textquotesingle{}cpu\textquotesingle{}}\NormalTok{)}
\NormalTok{m.dist.normal(loc}\OperatorTok{=}\FloatTok{0.0}\NormalTok{, scale}\OperatorTok{=}\FloatTok{1.0}\NormalTok{, sample}\OperatorTok{=}\VariableTok{True}\NormalTok{)}
\end{Highlighting}
\end{Shaded}

\paragraph{Wrapper of:}\label{wrapper-of-42}

https://num.pyro.ai/en/stable/distributions.html\#normal

\begin{center}\rule{0.5\linewidth}{0.5pt}\end{center}

\subsubsection{Ordered Logistic}\label{ordered-logistic}

A categorical distribution with ordered outcomes. This distribution
represents the probability of an event falling into one of several
ordered categories, based on a predictor variable and a set of
cutpoints. The probability of an event falling into a particular
category is determined by the number of categories above it.

\[   P(Y = k) = \begin{cases}
1 & \text{if } k = 0 \\
\frac{1}{k} & \text{if } k > 0
\end{cases}
\]

\paragraph{Args:}\label{args-62}

\begin{Shaded}
\begin{Highlighting}[]
\NormalTok{bi.dist.ordered\_logistic(}
\NormalTok{predictor,}
\NormalTok{cutpoints,}
\NormalTok{validate\_args}\OperatorTok{=}\VariableTok{None}\NormalTok{,}
\NormalTok{name}\OperatorTok{=}\StringTok{\textquotesingle{}x\textquotesingle{}}\NormalTok{,}
\NormalTok{obs}\OperatorTok{=}\VariableTok{None}\NormalTok{,}
\NormalTok{mask}\OperatorTok{=}\VariableTok{None}\NormalTok{,}
\NormalTok{sample}\OperatorTok{=}\VariableTok{False}\NormalTok{,}
\NormalTok{seed}\OperatorTok{=}\DecValTok{0}\NormalTok{,}
\NormalTok{shape}\OperatorTok{=}\NormalTok{(),}
\NormalTok{event}\OperatorTok{=}\DecValTok{0}\NormalTok{,}
\NormalTok{create\_obj}\OperatorTok{=}\VariableTok{False}\NormalTok{,}
\NormalTok{)}
\end{Highlighting}
\end{Shaded}

\begin{itemize}
\item
  predictor (jnp.ndarray): Prediction in real domain; typically this is
  output of a linear model.
\item
  cutpoints (jnp.ndarray): Positions in real domain to separate
  categories.
\item
  \emph{shape} (tuple): A multi-purpose argument for shaping. When
  \texttt{sample=False} (model building), this is used with
  \texttt{.expand(shape)} to set the distribution's batch shape. When
  \texttt{sample=True} (direct sampling), this is used as
  \texttt{sample\_shape} to draw a raw JAX array of the given shape.
\item
  \emph{event} (int): The number of batch dimensions to reinterpret as
  event dimensions (used in model building).
\item
  \emph{mask} (jnp.ndarray, bool): Optional boolean array to mask
  observations.
\item
  \emph{create\_obj} (bool): If True, returns the raw BI distribution
  object instead of creating a sample site. This is essential for
  building complex distributions like \texttt{MixtureSameFamily}.
\item
  \emph{sample} (bool, optional): A control-flow argument. If
  \texttt{True}, the function will directly sample a raw JAX array from
  the distribution, bypassing the BI model context. If \texttt{False},
  it will create a \texttt{BI.sample} site within a model. Defaults to
  \texttt{False}.
\item
  \emph{seed} (int, optional): An integer used to generate a JAX PRNGKey
  for reproducible sampling when \texttt{sample=True}. {[}7{]} This
  argument has no effect when \texttt{sample=False}, as randomness is
  handled by BI's inference engine. Defaults to 0.
\item
  \emph{obs} (jnp.ndarray, optional): The observed value for this random
  variable. If provided, the sample site is conditioned on this value,
  and the function returns the observed value. If \texttt{None}, the
  site is treated as a latent (unobserved) random variable. Defaults to
  \texttt{None}.
\item
  \emph{name} (str, optional): The name of the sample site in a BI
  model. This is used to uniquely identify the random variable. Defaults
  to `x'.
\end{itemize}

\paragraph{Returns:}\label{returns-62}

\begin{itemize}
\tightlist
\item
  When \texttt{sample=False}: A BI OrderedLogistic distribution object
  (for model building).
\item
  When \texttt{sample=True}: A JAX array of samples drawn from the
  OrderedLogistic distribution (for direct sampling).
\item
  When \texttt{create\_obj=True}: The raw BI distribution object (for
  advanced use cases).
\end{itemize}

\paragraph{Example Usage:}\label{example-usage-60}

\begin{Shaded}
\begin{Highlighting}[]
\ImportTok{from}\NormalTok{ BI }\ImportTok{import}\NormalTok{ bi}
\NormalTok{m }\OperatorTok{=}\NormalTok{ bi(}\StringTok{\textquotesingle{}cpu\textquotesingle{}}\NormalTok{)}
\NormalTok{m.dist.ordered\_logistic(predictor}\OperatorTok{=}\NormalTok{jnp.array([}\FloatTok{0.2}\NormalTok{, }\FloatTok{0.5}\NormalTok{, }\FloatTok{0.8}\NormalTok{]), cutpoints}\OperatorTok{=}\NormalTok{jnp.array([}\OperatorTok{{-}}\FloatTok{1.0}\NormalTok{, }\FloatTok{0.0}\NormalTok{, }\FloatTok{1.0}\NormalTok{]), sample}\OperatorTok{=}\VariableTok{True}\NormalTok{)}
\end{Highlighting}
\end{Shaded}

\paragraph{Wrapper of:}\label{wrapper-of-43}

https://num.pyro.ai/en/stable/distributions.html\#orderedlogistic

\begin{center}\rule{0.5\linewidth}{0.5pt}\end{center}

\subsubsection{Pareto}\label{pareto}

Samples from a Pareto distribution.

The Pareto distribution is a power-law probability distribution that is
often used to model income, wealth, and the size of cities. It is
defined by two parameters: alpha (shape) and scale.

\[
f(x) = \frac{\alpha \cdot \text{scale}^{\alpha}}{x^{\alpha + 1}}
\text{ for } x \geq \text{scale}
\]

\paragraph{Args:}\label{args-63}

\begin{Shaded}
\begin{Highlighting}[]
\NormalTok{bi.dist.pareto(}
\NormalTok{scale,}
\NormalTok{alpha,}
\NormalTok{validate\_args}\OperatorTok{=}\VariableTok{None}\NormalTok{,}
\NormalTok{name}\OperatorTok{=}\StringTok{\textquotesingle{}x\textquotesingle{}}\NormalTok{,}
\NormalTok{obs}\OperatorTok{=}\VariableTok{None}\NormalTok{,}
\NormalTok{mask}\OperatorTok{=}\VariableTok{None}\NormalTok{,}
\NormalTok{sample}\OperatorTok{=}\VariableTok{False}\NormalTok{,}
\NormalTok{seed}\OperatorTok{=}\DecValTok{0}\NormalTok{,}
\NormalTok{shape}\OperatorTok{=}\NormalTok{(),}
\NormalTok{event}\OperatorTok{=}\DecValTok{0}\NormalTok{,}
\NormalTok{create\_obj}\OperatorTok{=}\VariableTok{False}\NormalTok{,}
\NormalTok{)}
\end{Highlighting}
\end{Shaded}

\begin{itemize}
\tightlist
\item
  \emph{sample} (jnp.ndarray or float): Scale parameter of the Pareto
  distribution. Must be positive.
\end{itemize}

alpha (jnp.ndarray or float): Shape parameter of the Pareto
distribution. Must be positive.

\begin{itemize}
\item
  \emph{shape} (tuple): A multi-purpose argument for shaping. When
  \texttt{sample=False} (model building), this is used with
  \texttt{.expand(shape)} to set the distribution's batch shape. When
  \texttt{sample=True} (direct sampling), this is used as
  \texttt{sample\_shape} to draw a raw JAX array of the given shape.
\item
  \emph{event} (int): The number of batch dimensions to reinterpret as
  event dimensions (used in model building).
\item
  \emph{mask} (jnp.ndarray, bool): Optional boolean array to mask
  observations.
\item
  \emph{create\_obj} (bool): If True, returns the raw BI distribution
  object instead of creating a sample site. This is essential for
  building complex distributions like \texttt{MixtureSameFamily}.
\item
  \emph{sample} (bool, optional): A control-flow argument. If
  \texttt{True}, the function will directly sample a raw JAX array from
  the distribution, bypassing the BI model context. If \texttt{False},
  it will create a \texttt{BI.sample} site within a model. Defaults to
  \texttt{False}.
\item
  \emph{seed} (int, optional): An integer used to generate a JAX PRNGKey
  for reproducible sampling when \texttt{sample=True}. {[}7{]} This
  argument has no effect when \texttt{sample=False}, as randomness is
  handled by BI's inference engine. Defaults to 0.
\item
  \emph{obs} (jnp.ndarray, optional): The observed value for this random
  variable. If provided, the sample site is conditioned on this value,
  and the function returns the observed value. If \texttt{None}, the
  site is treated as a latent (unobserved) random variable. Defaults to
  \texttt{None}.
\item
  \emph{name} (str, optional): The name of the sample site in a BI
  model. This is used to uniquely identify the random variable. Defaults
  to `x'.
\end{itemize}

\paragraph{Returns:}\label{returns-63}

\begin{itemize}
\item
  When \texttt{sample=False}: A BI Pareto distribution object (for model
  building).
\item
  When \texttt{sample=True}: A JAX array of samples drawn from the
  Pareto distribution (for direct sampling).
\item
  When \texttt{create\_obj=True}: The raw BI distribution object (for
  advanced use cases).
\end{itemize}

\paragraph{Example Usage:}\label{example-usage-61}

\begin{Shaded}
\begin{Highlighting}[]
\ImportTok{from}\NormalTok{ BI }\ImportTok{import}\NormalTok{ bi}
\NormalTok{m }\OperatorTok{=}\NormalTok{ bi(}\StringTok{\textquotesingle{}cpu\textquotesingle{}}\NormalTok{)}
\NormalTok{m.dist.pareto(scale}\OperatorTok{=}\FloatTok{2.0}\NormalTok{, alpha}\OperatorTok{=}\FloatTok{3.0}\NormalTok{, sample}\OperatorTok{=}\VariableTok{True}\NormalTok{)}
\end{Highlighting}
\end{Shaded}

\paragraph{Wrapper of:}\label{wrapper-of-44}

https://num.pyro.ai/en/stable/distributions.html\#pareto

\begin{center}\rule{0.5\linewidth}{0.5pt}\end{center}

\subsubsection{Poisson}\label{poisson}

Creates a Poisson distribution, a discrete probability distribution that
models the number of events occurring in a fixed interval of time or
space if these events occur with a known average rate and independently
of the time since the last event.

\[  \mathrm{rate}^k \frac{e^{-\mathrm{rate}}}{k!}
\]

\paragraph{Args:}\label{args-64}

\begin{Shaded}
\begin{Highlighting}[]
\NormalTok{bi.dist.poisson(}
\NormalTok{rate,}
\NormalTok{is\_sparse}\OperatorTok{=}\VariableTok{False}\NormalTok{,}
\NormalTok{validate\_args}\OperatorTok{=}\VariableTok{None}\NormalTok{,}
\NormalTok{name}\OperatorTok{=}\StringTok{\textquotesingle{}x\textquotesingle{}}\NormalTok{,}
\NormalTok{obs}\OperatorTok{=}\VariableTok{None}\NormalTok{,}
\NormalTok{mask}\OperatorTok{=}\VariableTok{None}\NormalTok{,}
\NormalTok{sample}\OperatorTok{=}\VariableTok{False}\NormalTok{,}
\NormalTok{seed}\OperatorTok{=}\DecValTok{0}\NormalTok{,}
\NormalTok{shape}\OperatorTok{=}\NormalTok{(),}
\NormalTok{event}\OperatorTok{=}\DecValTok{0}\NormalTok{,}
\NormalTok{create\_obj}\OperatorTok{=}\VariableTok{False}\NormalTok{,}
\NormalTok{)}
\end{Highlighting}
\end{Shaded}

\begin{itemize}
\item
  \emph{rate} (jnp.ndarray): The rate parameter, representing the
  average number of events. is\_sparse (bool, optional): Indicates
  whether the \texttt{rate} parameter is sparse. If \texttt{True}, a
  specialized sparse sampling implementation is used, which can be more
  efficient for models with many zero-rate components (e.g.,
  zero-inflated models). Defaults to \texttt{False}.
\item
  \emph{shape} (tuple): A multi-purpose argument for shaping. When
  \texttt{sample=False} (model building), this is used with
  \texttt{.expand(shape)} to set the distribution's batch shape. When
  \texttt{sample=True} (direct sampling), this is used as
  \texttt{sample\_shape} to draw a raw JAX array of the given shape.
\item
  \emph{event} (int): The number of batch dimensions to reinterpret as
  event dimensions (used in model building).
\item
  \emph{mask} (jnp.ndarray, bool): Optional boolean array to mask
  observations.
\item
  \emph{create\_obj} (bool): If True, returns the raw BI distribution
  object instead of creating a sample site. This is essential for
  building complex distributions like \texttt{MixtureSameFamily}.
\item
  \emph{sample} (bool, optional): A control-flow argument. If
  \texttt{True}, the function will directly sample a raw JAX array from
  the distribution, bypassing the BI model context. If \texttt{False},
  it will create a \texttt{BI.sample} site within a model. Defaults to
  \texttt{False}.
\item
  \emph{seed} (int, optional): An integer used to generate a JAX PRNGKey
  for reproducible sampling when \texttt{sample=True}. {[}7{]} This
  argument has no effect when \texttt{sample=False}, as randomness is
  handled by BI's inference engine. Defaults to 0.
\item
  \emph{obs} (jnp.ndarray, optional): The observed value for this random
  variable. If provided, the sample site is conditioned on this value,
  and the function returns the observed value. If \texttt{None}, the
  site is treated as a latent (unobserved) random variable. Defaults to
  \texttt{None}.
\item
  \emph{name} (str, optional): The name of the sample site in a BI
  model. This is used to uniquely identify the random variable. Defaults
  to `x'.
\end{itemize}

\paragraph{Returns:}\label{returns-64}

\begin{itemize}
\tightlist
\item
  When \texttt{sample=False}: A BI Poisson distribution object (for
  model building).
\item
  When \texttt{sample=True}: A JAX array of samples drawn from the
  Poisson distribution (for direct sampling).
\item
  When \texttt{create\_obj=True}: The raw BI distribution object (for
  advanced use cases).
\end{itemize}

\paragraph{Example Usage:}\label{example-usage-62}

\begin{Shaded}
\begin{Highlighting}[]
\ImportTok{from}\NormalTok{ BI }\ImportTok{import}\NormalTok{ bi}
\NormalTok{m }\OperatorTok{=}\NormalTok{ bi(}\StringTok{\textquotesingle{}cpu\textquotesingle{}}\NormalTok{)}
\NormalTok{m.dist.poisson(rate}\OperatorTok{=}\FloatTok{2.0}\NormalTok{, sample}\OperatorTok{=}\VariableTok{True}\NormalTok{)}
\end{Highlighting}
\end{Shaded}

\paragraph{Wrapper of:
https://num.pyro.ai/en/stable/distributions.html\#poisson}\label{wrapper-of-httpsnum.pyro.aienstabledistributions.htmlpoisson}

\begin{center}\rule{0.5\linewidth}{0.5pt}\end{center}

\subsubsection{Projected Normal}\label{projected-normal}

This distribution over directional data is qualitatively similar to the
von Mises and von Mises-Fisher distributions, but permits tractable
variational inference via reparametrized gradients.

\[   p(x) = \frac{1}{Z} \exp\left(-\frac{1}{2\sigma^2} ||x - \mu||^2\right)
\]

\paragraph{Args:}\label{args-65}

\begin{Shaded}
\begin{Highlighting}[]
\NormalTok{bi.dist.projected\_normal(}
\NormalTok{concentration,}
\NormalTok{validate\_args}\OperatorTok{=}\VariableTok{None}\NormalTok{,}
\NormalTok{name}\OperatorTok{=}\StringTok{\textquotesingle{}x\textquotesingle{}}\NormalTok{,}
\NormalTok{obs}\OperatorTok{=}\VariableTok{None}\NormalTok{,}
\NormalTok{mask}\OperatorTok{=}\VariableTok{None}\NormalTok{,}
\NormalTok{sample}\OperatorTok{=}\VariableTok{False}\NormalTok{,}
\NormalTok{seed}\OperatorTok{=}\DecValTok{0}\NormalTok{,}
\NormalTok{shape}\OperatorTok{=}\NormalTok{(),}
\NormalTok{event}\OperatorTok{=}\DecValTok{0}\NormalTok{,}
\NormalTok{create\_obj}\OperatorTok{=}\VariableTok{False}\NormalTok{,}
\NormalTok{)}
\end{Highlighting}
\end{Shaded}

concentration (jnp.ndarray): The concentration parameter, representing
the direction towards which the samples are concentrated. Must be a JAX
array with at least one dimension.

\begin{itemize}
\item
  \emph{shape} (tuple): A multi-purpose argument for shaping. When
  \texttt{sample=False} (model building), this is used with
  \texttt{.expand(shape)} to set the distribution's batch shape. When
  \texttt{sample=True} (direct sampling), this is used as
  \texttt{sample\_shape} to draw a raw JAX array of the given shape.
\item
  \emph{event} (int): The number of batch dimensions to reinterpret as
  event dimensions (used in model building).
\item
  \emph{mask} (jnp.ndarray, bool): Optional boolean array to mask
  observations.
\item
  \emph{create\_obj} (bool): If True, returns the raw BI distribution
  object instead of creating a sample site. This is essential for
  building complex distributions like \texttt{MixtureSameFamily}.
\item
  \emph{sample} (bool, optional): A control-flow argument. If
  \texttt{True}, the function will directly sample a raw JAX array from
  the distribution, bypassing the BI model context. If \texttt{False},
  it will create a \texttt{BI.sample} site within a model. Defaults to
  \texttt{False}.
\item
  \emph{seed} (int, optional): An integer used to generate a JAX PRNGKey
  for reproducible sampling when \texttt{sample=True}. {[}7{]} This
  argument has no effect when \texttt{sample=False}, as randomness is
  handled by BI's inference engine. Defaults to 0.
\item
  \emph{obs} (jnp.ndarray, optional): The observed value for this random
  variable. If provided, the sample site is conditioned on this value,
  and the function returns the observed value. If \texttt{None}, the
  site is treated as a latent (unobserved) random variable. Defaults to
  \texttt{None}.
\item
  \emph{name} (str, optional): The name of the sample site in a BI
  model. This is used to uniquely identify the random variable. Defaults
  to `x'.
\end{itemize}

\paragraph{Returns:}\label{returns-65}

\begin{itemize}
\item
  When \texttt{sample=False}: A BI ProjectedNormal distribution object
  (for model building).
\item
  When \texttt{sample=True}: A JAX array of samples drawn from the
  ProjectedNormal distribution (for direct sampling).
\item
  When \texttt{create\_obj=True}: The raw BI distribution object (for
  advanced use cases).
\end{itemize}

\paragraph{Example Usage:}\label{example-usage-63}

\begin{Shaded}
\begin{Highlighting}[]
\ImportTok{from}\NormalTok{ BI }\ImportTok{import}\NormalTok{ bi}
\NormalTok{m }\OperatorTok{=}\NormalTok{ bi(}\StringTok{\textquotesingle{}cpu\textquotesingle{}}\NormalTok{)}
\NormalTok{m.dist.projected\_normal(concentration}\OperatorTok{=}\NormalTok{jnp.array([}\FloatTok{1.0}\NormalTok{, }\FloatTok{3.0}\NormalTok{, }\FloatTok{2.0}\NormalTok{]), sample}\OperatorTok{=}\VariableTok{True}\NormalTok{)}
\end{Highlighting}
\end{Shaded}

\paragraph{Wrapper of:}\label{wrapper-of-45}

https://num.pyro.ai/en/stable/distributions.html\#projectednormal

\begin{center}\rule{0.5\linewidth}{0.5pt}\end{center}

\subsubsection{Relaxed Bernoulli}\label{relaxed-bernoulli}

The Relaxed Bernoulli distribution is a continuous relaxation of the
discrete Bernoulli distribution. It's useful for variational inference
and other applications where a differentiable approximation of the
Bernoulli is needed. The probability density function (PDF) is defined
as:

\[
p(x) = \frac{1}{2} \left( 1 + \tanh\left(\frac{x - \beta \log(\frac{p}{1-p})}{1}\right) \right)
\]

\paragraph{Args:}\label{args-66}

\begin{Shaded}
\begin{Highlighting}[]
\NormalTok{bi.dist.relaxed\_bernoulli(}
\NormalTok{temperature,}
\NormalTok{probs}\OperatorTok{=}\VariableTok{None}\NormalTok{,}
\NormalTok{logits}\OperatorTok{=}\VariableTok{None}\NormalTok{,}
\NormalTok{validate\_args}\OperatorTok{=}\VariableTok{None}\NormalTok{,}
\NormalTok{name}\OperatorTok{=}\StringTok{\textquotesingle{}x\textquotesingle{}}\NormalTok{,}
\NormalTok{obs}\OperatorTok{=}\VariableTok{None}\NormalTok{,}
\NormalTok{mask}\OperatorTok{=}\VariableTok{None}\NormalTok{,}
\NormalTok{sample}\OperatorTok{=}\VariableTok{False}\NormalTok{,}
\NormalTok{seed}\OperatorTok{=}\DecValTok{0}\NormalTok{,}
\NormalTok{shape}\OperatorTok{=}\NormalTok{(),}
\NormalTok{event}\OperatorTok{=}\DecValTok{0}\NormalTok{,}
\NormalTok{create\_obj}\OperatorTok{=}\VariableTok{False}\NormalTok{,}
\NormalTok{)}
\end{Highlighting}
\end{Shaded}

temperature (float): The temperature parameter.

\begin{itemize}
\tightlist
\item
  \emph{probs} (jnp.ndarray, optional): The probability of success. Must
  be in the interval \texttt{{[}0,\ 1{]}}. Only one of \texttt{probs} or
  \texttt{logits} can be specified.
\end{itemize}

logits (jnp.ndarray, optional): The log-odds of success. Only one of
\texttt{probs} or \texttt{logits} can be specified.

\begin{itemize}
\item
  \emph{shape} (tuple): A multi-purpose argument for shaping. When
  \texttt{sample=False} (model building), this is used with
  \texttt{.expand(shape)} to set the distribution's batch shape. When
  \texttt{sample=True} (direct sampling), this is used as
  \texttt{sample\_shape} to draw a raw JAX array of the given shape.
\item
  \emph{event} (int): The number of batch dimensions to reinterpret as
  event dimensions (used in model building).
\item
  \emph{mask} (jnp.ndarray, bool): Optional boolean array to mask
  observations.
\item
  \emph{create\_obj} (bool): If True, returns the raw BI distribution
  object instead of creating a sample site. This is essential for
  building complex distributions like \texttt{MixtureSameFamily}.
\item
  \emph{sample} (bool, optional): A control-flow argument. If
  \texttt{True}, the function will directly sample a raw JAX array from
  the distribution, bypassing the BI model context. If \texttt{False},
  it will create a \texttt{BI.sample} site within a model. Defaults to
  \texttt{False}.
\item
  \emph{seed} (int, optional): An integer used to generate a JAX PRNGKey
  for reproducible sampling when \texttt{sample=True}. {[}7{]} This
  argument has no effect when \texttt{sample=False}, as randomness is
  handled by BI's inference engine. Defaults to 0.
\item
  \emph{obs} (jnp.ndarray, optional): The observed value for this random
  variable. If provided, the sample site is conditioned on this value,
  and the function returns the observed value. If \texttt{None}, the
  site is treated as a latent (unobserved) random variable. Defaults to
  \texttt{None}.
\item
  \emph{name} (str, optional): The name of the sample site in a BI
  model. This is used to uniquely identify the random variable. Defaults
  to `x'.
\end{itemize}

\paragraph{Returns:}\label{returns-66}

BI RelaxedBernoulli distribution object (for model building) when
\texttt{sample=False}. A JAX array of samples drawn from the
RelaxedBernoulli distribution (for direct sampling) when
\texttt{sample=True}. The raw BI distribution object (for advanced use
cases) when \texttt{create\_obj=True}.

\paragraph{Example Usage:}\label{example-usage-64}

\begin{Shaded}
\begin{Highlighting}[]
\ImportTok{from}\NormalTok{ BI }\ImportTok{import}\NormalTok{ bi}
\NormalTok{m }\OperatorTok{=}\NormalTok{ bi(}\StringTok{\textquotesingle{}cpu\textquotesingle{}}\NormalTok{)}
\NormalTok{m.dist.relaxed\_bernoulli(temperature}\OperatorTok{=}\FloatTok{1.0}\NormalTok{, probs }\OperatorTok{=}\NormalTok{ jnp.array([}\FloatTok{0.2}\NormalTok{, }\FloatTok{0.3}\NormalTok{, }\FloatTok{0.5}\NormalTok{]), sample}\OperatorTok{=}\VariableTok{True}\NormalTok{)}
\end{Highlighting}
\end{Shaded}

\paragraph{Wrapper of:
https://num.pyro.ai/en/stable/distributions.html\#relaxedbernoulli}\label{wrapper-of-httpsnum.pyro.aienstabledistributions.htmlrelaxedbernoulli}

\begin{center}\rule{0.5\linewidth}{0.5pt}\end{center}

\subsubsection{Relaxed Bernoulli Logits}\label{relaxed-bernoulli-logits}

Represents a relaxed version of the Bernoulli distribution,
parameterized by logits and a temperature. The temperature parameter
controls the sharpness of the distribution. The distribution is defined
by transforming the output of a Logistic distribution through a sigmoid
function.

\[
P(x) = \sigma\left(\frac{x}{\text{temperature}}\right)
\]

\paragraph{Args:}\label{args-67}

\begin{Shaded}
\begin{Highlighting}[]
\NormalTok{bi.dist.relaxed\_bernoulli\_logits(}
\NormalTok{temperature,}
\NormalTok{logits,}
\NormalTok{validate\_args}\OperatorTok{=}\VariableTok{None}\NormalTok{,}
\NormalTok{name}\OperatorTok{=}\StringTok{\textquotesingle{}x\textquotesingle{}}\NormalTok{,}
\NormalTok{obs}\OperatorTok{=}\VariableTok{None}\NormalTok{,}
\NormalTok{mask}\OperatorTok{=}\VariableTok{None}\NormalTok{,}
\NormalTok{sample}\OperatorTok{=}\VariableTok{False}\NormalTok{,}
\NormalTok{seed}\OperatorTok{=}\DecValTok{0}\NormalTok{,}
\NormalTok{shape}\OperatorTok{=}\NormalTok{(),}
\NormalTok{event}\OperatorTok{=}\DecValTok{0}\NormalTok{,}
\NormalTok{create\_obj}\OperatorTok{=}\VariableTok{False}\NormalTok{,}
\NormalTok{)}
\end{Highlighting}
\end{Shaded}

temperature (jnp.ndarray): The temperature parameter, must be positive.
logits (jnp.ndarray): The logits parameter.

\begin{itemize}
\item
  \emph{shape} (tuple): A multi-purpose argument for shaping. When
  \texttt{sample=False} (model building), this is used with
  \texttt{.expand(shape)} to set the distribution's batch shape. When
  \texttt{sample=True} (direct sampling), this is used as
  \texttt{sample\_shape} to draw a raw JAX array of the given shape.
\item
  \emph{event} (int): The number of batch dimensions to reinterpret as
  event dimensions (used in model building).
\item
  \emph{mask} (jnp.ndarray, bool): Optional boolean array to mask
  observations.
\item
  \emph{create\_obj} (bool): If True, returns the raw BI distribution
  object instead of creating a sample site. This is essential for
  building complex distributions like \texttt{MixtureSameFamily}.
\item
  \emph{sample} (bool, optional): A control-flow argument. If
  \texttt{True}, the function will directly sample a raw JAX array from
  the distribution, bypassing the BI model context. If \texttt{False},
  it will create a \texttt{BI.sample} site within a model. Defaults to
  \texttt{False}.
\item
  \emph{seed} (int, optional): An integer used to generate a JAX PRNGKey
  for reproducible sampling when \texttt{sample=True}. {[}7{]} This
  argument has no effect when \texttt{sample=False}, as randomness is
  handled by BI's inference engine. Defaults to 0.
\item
  \emph{obs} (jnp.ndarray, optional): The observed value for this random
  variable. If provided, the sample site is conditioned on this value,
  and the function returns the observed value. If \texttt{None}, the
  site is treated as a latent (unobserved) random variable. Defaults to
  \texttt{None}.
\item
  \emph{name} (str, optional): The name of the sample site in a BI
  model. This is used to uniquely identify the random variable. Defaults
  to `x'.
\end{itemize}

\paragraph{Returns:}\label{returns-67}

RelaxedBernoulliLogits: A BI RelaxedBernoulliLogits distribution object
(for model building). jnp.ndarray: A JAX array of samples drawn from the
RelaxedBernoulliLogits distribution (for direct sampling).
RelaxedBernoulliLogits: The raw BI distribution object (for advanced use
cases).

\paragraph{Example Usage:}\label{example-usage-65}

\begin{Shaded}
\begin{Highlighting}[]
\ImportTok{from}\NormalTok{ BI }\ImportTok{import}\NormalTok{ bi}
\NormalTok{m }\OperatorTok{=}\NormalTok{ bi(}\StringTok{\textquotesingle{}cpu\textquotesingle{}}\NormalTok{)}
\NormalTok{m.dist.relaxed\_bernoulli\_logits(temperature}\OperatorTok{=}\FloatTok{1.0}\NormalTok{, logits}\OperatorTok{=}\FloatTok{0.0}\NormalTok{, sample}\OperatorTok{=}\VariableTok{True}\NormalTok{)}
\end{Highlighting}
\end{Shaded}

\paragraph{Wrapper of:}\label{wrapper-of-46}

https://num.pyro.ai/en/stable/distributions.html\#relaxed-bernoulli-logits

\begin{center}\rule{0.5\linewidth}{0.5pt}\end{center}

\subsubsection{Right Truncated}\label{right-truncated}

Samples from a right-truncated distribution.

This distribution truncates the base distribution at a specified high
value. Values greater than \texttt{high} are discarded, effectively
creating a distribution that is only supported up to that point. This is
useful for modeling data where observations are only possible within a
certain range.

The probability density function (PDF) of the truncated distribution is:

\[
f_{\text{trunc}}(x) = \frac{f_{\text{base}}(x)}{F_{\text{base}}(\text{high})} \quad \text{for } x \le \text{high}
\]

where \(f_{\text{base}}(x)\) is the PDF of the base distribution and
\(F_{\text{base}}(\text{high})\) is the cumulative distribution function
(CDF) of the base distribution evaluated at \texttt{high}.

where \(f(x)\) is the probability density function (PDF) of the base
distribution and \(P(X \le high)\) is the cumulative distribution
function (CDF) of the base distribution evaluated at \texttt{high}. \$\$

\paragraph{Args:}\label{args-68}

\begin{Shaded}
\begin{Highlighting}[]
\NormalTok{bi.dist.right\_truncated\_distribution(}
\NormalTok{base\_dist,}
\NormalTok{high}\OperatorTok{=}\FloatTok{0.0}\NormalTok{,}
\NormalTok{validate\_args}\OperatorTok{=}\VariableTok{None}\NormalTok{,}
\NormalTok{name}\OperatorTok{=}\StringTok{\textquotesingle{}x\textquotesingle{}}\NormalTok{,}
\NormalTok{obs}\OperatorTok{=}\VariableTok{None}\NormalTok{,}
\NormalTok{mask}\OperatorTok{=}\VariableTok{None}\NormalTok{,}
\NormalTok{sample}\OperatorTok{=}\VariableTok{False}\NormalTok{,}
\NormalTok{seed}\OperatorTok{=}\DecValTok{0}\NormalTok{,}
\NormalTok{shape}\OperatorTok{=}\NormalTok{(),}
\NormalTok{event}\OperatorTok{=}\DecValTok{0}\NormalTok{,}
\NormalTok{create\_obj}\OperatorTok{=}\VariableTok{False}\NormalTok{,}
\NormalTok{)}
\end{Highlighting}
\end{Shaded}

base\_dist: The base distribution to truncate. Must be a univariate
distribution with real support.

high (float, jnp.ndarray, optional): The upper truncation point. The
support of the new distribution is \((-\infty, \text{high}]\). Defaults
to 0.0.

\begin{itemize}
\item
  \emph{shape} (tuple): A multi-purpose argument for shaping. When
  \texttt{sample=False} (model building), this is used with
  \texttt{.expand(shape)} to set the distribution's batch shape. When
  \texttt{sample=True} (direct sampling), this is used as
  \texttt{sample\_shape} to draw a raw JAX array of the given shape.
\item
  \emph{event} (int): The number of batch dimensions to reinterpret as
  event dimensions (used in model building).
\item
  \emph{mask} (jnp.ndarray, bool): Optional boolean array to mask
  observations.
\item
  \emph{create\_obj} (bool): If True, returns the raw BI distribution
  object instead of creating a sample site. This is essential for
  building complex distributions like \texttt{MixtureSameFamily}.
\item
  \emph{sample} (bool, optional): A control-flow argument. If
  \texttt{True}, the function will directly sample a raw JAX array from
  the distribution, bypassing the BI model context. If \texttt{False},
  it will create a \texttt{BI.sample} site within a model. Defaults to
  \texttt{False}.
\item
  \emph{seed} (int, optional): An integer used to generate a JAX PRNGKey
  for reproducible sampling when \texttt{sample=True}. {[}7{]} This
  argument has no effect when \texttt{sample=False}, as randomness is
  handled by BI's inference engine. Defaults to 0.
\item
  \emph{obs} (jnp.ndarray, optional): The observed value for this random
  variable. If provided, the sample site is conditioned on this value,
  and the function returns the observed value. If \texttt{None}, the
  site is treated as a latent (unobserved) random variable. Defaults to
  \texttt{None}.
\item
  \emph{name} (str, optional): The name of the sample site in a BI
  model. This is used to uniquely identify the random variable. Defaults
  to `x'.
\end{itemize}

\paragraph{Returns:}\label{returns-68}

\begin{itemize}
\item
  When \texttt{sample=False}: A BI RightTruncatedDistribution
  distribution object (for model building).
\item
  When \texttt{sample=True}: A JAX array of samples drawn from the
  RightTruncatedDistribution distribution (for direct sampling).
\item
  When \texttt{create\_obj=True}: The raw BI distribution object (for
  advanced use cases).
\end{itemize}

\paragraph{Example Usage:}\label{example-usage-66}

\begin{Shaded}
\begin{Highlighting}[]
\ImportTok{from}\NormalTok{ BI }\ImportTok{import}\NormalTok{ bi}
\NormalTok{m }\OperatorTok{=}\NormalTok{ bi(}\StringTok{\textquotesingle{}cpu\textquotesingle{}}\NormalTok{)}
\NormalTok{m.dist.right\_truncated\_distribution(base\_dist }\OperatorTok{=}\NormalTok{ m.dist.normal(}\DecValTok{0}\NormalTok{,}\DecValTok{1}\NormalTok{, create\_obj }\OperatorTok{=} \VariableTok{True}\NormalTok{), high}\OperatorTok{=}\DecValTok{0}\NormalTok{, sample}\OperatorTok{=}\VariableTok{True}\NormalTok{)}
\end{Highlighting}
\end{Shaded}

\paragraph{Wrapper of:}\label{wrapper-of-47}

https://num.pyro.ai/en/stable/distributions.html\#righttruncateddistribution

\begin{center}\rule{0.5\linewidth}{0.5pt}\end{center}

\subsubsection{Sine Bivariate Von Mises}\label{sine-bivariate-von-mises}

A unimodal distribution for two dependent angles on the 2-torus
(\(S^1 \otimes S^1\)), which is useful for modeling coupled angles like
torsion angles in peptide chains. {[}1{]}

The probability density function is given by:

\[
C^{-1}\exp(\kappa_1\cos(x_1-\mu_1) + \kappa_2\cos(x_2 -\mu_2) + \rho\sin(x_1 - \mu_1)\sin(x_2 - \mu_2))
\]

where the normalization constant \(C\) is:

\[
C = (2\pi)^2 \sum_{i=0}^{\infty} \binom{2i}{i} \left(\frac{\rho^2}{4\kappa_1\kappa_2}\right)^i I_i(\kappa_1)I_i(\kappa_2)
\]

Here, \(I_i(\cdot)\) is the modified Bessel function of the first kind,
\(\mu\)'s are the locations, \(\kappa\)'s are the concentrations, and
\(\rho\) represents the correlation between the angles \(x_1\) and
\(x_2\).

\paragraph{Args:}\label{args-69}

\begin{Shaded}
\begin{Highlighting}[]
\NormalTok{bi.dist.sine\_bivariate\_vonmises(}
\NormalTok{phi\_loc,}
\NormalTok{psi\_loc,}
\NormalTok{phi\_concentration,}
\NormalTok{psi\_concentration,}
\NormalTok{correlation}\OperatorTok{=}\VariableTok{None}\NormalTok{,}
\NormalTok{weighted\_correlation}\OperatorTok{=}\VariableTok{None}\NormalTok{,}
\NormalTok{validate\_args}\OperatorTok{=}\VariableTok{None}\NormalTok{,}
\NormalTok{name}\OperatorTok{=}\StringTok{\textquotesingle{}x\textquotesingle{}}\NormalTok{,}
\NormalTok{obs}\OperatorTok{=}\VariableTok{None}\NormalTok{,}
\NormalTok{mask}\OperatorTok{=}\VariableTok{None}\NormalTok{,}
\NormalTok{sample}\OperatorTok{=}\VariableTok{False}\NormalTok{,}
\NormalTok{seed}\OperatorTok{=}\DecValTok{0}\NormalTok{,}
\NormalTok{shape}\OperatorTok{=}\NormalTok{(),}
\NormalTok{event}\OperatorTok{=}\DecValTok{0}\NormalTok{,}
\NormalTok{create\_obj}\OperatorTok{=}\VariableTok{False}\NormalTok{,}
\NormalTok{)}
\end{Highlighting}
\end{Shaded}

\begin{itemize}
\item
  \emph{phi\_loc} (jnp.ndarray): The location parameter for the first
  angle (phi).
\item
  \emph{psi\_loc} (jnp.ndarray): The location parameter for the second
  angle (psi).
\item
  \emph{phi\_concentration} (jnp.ndarray): The concentration parameter
  for the first angle (phi). Must be positive.
\item
  \emph{psi\_concentration} (jnp.ndarray): The concentration parameter
  for the second angle (psi). Must be positive.
\item
  \emph{correlation} (jnp.ndarray, optional): The correlation parameter
  between the two angles. One of \texttt{correlation} or
  \texttt{weighted\_correlation} must be specified.
\item
  \emph{weighted\_correlation} (jnp.ndarray, optional): An alternative
  correlation parameter. One of \texttt{correlation} or
  \texttt{weighted\_correlation} must be specified.
\item
  \emph{validate\_args} (bool, optional): Whether to enable validation
  of distribution parameters. Defaults to \texttt{None}.
\item
  \emph{name} (str, optional): The name of the sample site in a BI
  model. This is used to uniquely identify the random variable. Defaults
  to `x'.
\item
  \emph{obs} (jnp.ndarray, optional): The observed value for this random
  variable. If provided, the sample site is conditioned on this value,
  and the function returns the observed value. If \texttt{None}, the
  site is treated as a latent (unobserved) random variable. Defaults to
  \texttt{None}.
\item
  \emph{mask} (jnp.ndarray, bool, optional): Optional boolean array to
  mask observations. If provided, events with a \texttt{True} mask will
  be conditioned on \texttt{obs}, while the remaining events will be
  treated as latent variables. Defaults to \texttt{None}.
\item
  \emph{sample} (bool, optional): A control-flow argument. If
  \texttt{True}, the function will directly sample a raw JAX array from
  the distribution, bypassing the BI model context. If \texttt{False},
  it will create a \texttt{BI.sample} site within a model. Defaults to
  \texttt{False}.
\item
  \emph{seed} (int, optional): An integer used to generate a JAX PRNGKey
  for reproducible sampling when \texttt{sample=True}. This argument has
  no effect when \texttt{sample=False}, as randomness is handled by BI's
  inference engine. Defaults to 0.
\item
  \emph{shape} (tuple, optional): A multi-purpose argument for shaping.
  When \texttt{sample=False} (model building), this is used with
  \texttt{.expand(shape)} to set the distribution's batch shape. When
  \texttt{sample=True} (direct sampling), this is used as
  \texttt{sample\_shape} to draw a raw JAX array of the given shape.
\item
  \emph{event} (int, optional): The number of batch dimensions to
  reinterpret as event dimensions (used in model building).
\item
  \emph{create\_obj} (bool, optional): If True, returns the raw BI
  distribution object instead of creating a sample site. This is
  essential for building complex distributions like
  \texttt{MixtureSameFamily}. Defaults to \texttt{False}.
\end{itemize}

\paragraph{Returns:}\label{returns-69}

BI.primitives.Messenger: A BI sample site object when used in a model
context (\texttt{sample=False}). jnp.ndarray: A JAX array of samples
drawn from the SineBivariateVonMises distribution (for direct sampling,
\texttt{sample=True}). numpyro.distributions.Distribution: The raw BI
distribution object (if \texttt{create\_obj=True}).

\paragraph{Example Usage:}\label{example-usage-67}

\begin{Shaded}
\begin{Highlighting}[]
\ImportTok{from}\NormalTok{ BI }\ImportTok{import}\NormalTok{ bi}
\ImportTok{import}\NormalTok{ jax.numpy }\ImportTok{as}\NormalTok{ jnp}
\NormalTok{m }\OperatorTok{=}\NormalTok{ bi(}\StringTok{\textquotesingle{}cpu\textquotesingle{}}\NormalTok{)}

\CommentTok{\# Direct sampling}
\NormalTok{samples }\OperatorTok{=}\NormalTok{ m.dist.sine\_bivariate\_vonmises(}
\NormalTok{phi\_loc}\OperatorTok{=}\FloatTok{0.0}\NormalTok{,}
\NormalTok{psi\_loc}\OperatorTok{=}\NormalTok{jnp.pi,}
\NormalTok{phi\_concentration}\OperatorTok{=}\FloatTok{1.0}\NormalTok{,}
\NormalTok{psi\_concentration}\OperatorTok{=}\FloatTok{1.0}\NormalTok{,}
\NormalTok{correlation}\OperatorTok{=}\FloatTok{0.5}\NormalTok{,}
\NormalTok{sample}\OperatorTok{=}\VariableTok{True}\NormalTok{,}
\OperatorTok{{-}} \OperatorTok{*}\NormalTok{shape}\OperatorTok{*=}\NormalTok{(}\DecValTok{10}\NormalTok{,)}
\NormalTok{)}

\CommentTok{\# Usage within a model}
\KeywordTok{def}\NormalTok{ my\_model():}
\NormalTok{angles }\OperatorTok{=}\NormalTok{ m.dist.sine\_bivariate\_vonmises(}
\NormalTok{phi\_loc}\OperatorTok{=}\FloatTok{0.0}\NormalTok{,}
\NormalTok{psi\_loc}\OperatorTok{=}\FloatTok{0.0}\NormalTok{,}
\NormalTok{phi\_concentration}\OperatorTok{=}\FloatTok{2.0}\NormalTok{,}
\NormalTok{psi\_concentration}\OperatorTok{=}\FloatTok{2.0}\NormalTok{,}
\NormalTok{weighted\_correlation}\OperatorTok{=}\FloatTok{0.9}\NormalTok{,}
\NormalTok{name}\OperatorTok{=}\StringTok{\textquotesingle{}angles\textquotesingle{}}
\NormalTok{)}
\CommentTok{\# ... rest of the model}
\end{Highlighting}
\end{Shaded}

\paragraph{Wrapper of:}\label{wrapper-of-48}

https://num.pyro.ai/en/stable/distributions.html\#sinebivariatevonmises

\begin{center}\rule{0.5\linewidth}{0.5pt}\end{center}

\subsubsection{Sine-skewing}\label{sine-skewing}

Sine-skewing {[}1{]} is a procedure for producing a distribution that
breaks pointwise symmetry on a torus distribution. The new distribution
is called the Sine Skewed X distribution, where X is the name of the
(symmetric) base distribution. Torus distributions are distributions
with support on products of circles (i.e., \(\otimes S^1\) where
\(S^1 = [-pi,pi)\)). So, a 0-torus is a point, the 1-torus is a circle,
and the 2-torus is commonly associated with the donut shape.

.. note: This distribution is available in BI:
\url{https://num.pyro.ai/en/stable/distributions.html\#sineskewed}

\textbf{Parameters:}

\begin{itemize}
\item
  \textbf{base\_dist:} Base density on a d-dimensional torus. Supported
  base distributions include: 1D
  :class:\texttt{\textasciitilde{}numpyro.distributions.VonMises},
  :class:\texttt{\textasciitilde{}numnumpyro.distributions.SineBivariateVonMises},
  1D
  :class:\texttt{\textasciitilde{}numpyro.distributions.ProjectedNormal},
  and :class:\texttt{\textasciitilde{}numpyro.distributions.Uniform}
  (-pi, pi).
\item
  \textbf{skewness:} Skewness of the distribution.
\item
  \textbf{sample (bool, optional):} A control-flow argument. If
  \texttt{True}, the function will directly sample a raw JAX array from
  the distribution, bypassing the BI model context. If \texttt{False},
  it will create a \texttt{BI.sample} site within a model. Defaults to
  \texttt{False}.
\item
  \textbf{seed (int, optional):} An integer used to generate a JAX
  PRNGKey for reproducible sampling when \texttt{sample=True}. {[}7{]}
  This argument has no effect when \texttt{sample=False}, as randomness
  is handled by BI's inference engine. Defaults to 0.
\item
  \textbf{obs (jnp.ndarray, optional):} The observed value for this
  random variable. If provided, the sample site is conditioned on this
  value, and the function returns the observed value. If \texttt{None},
  the site is treated as a latent (unobserved) random variable. Defaults
  to \texttt{None}.
\item
  \textbf{name (str, optional):} The name of the sample site in a BI
  model. This is used to uniquely identify the random variable. Defaults
  to `x'.
\end{itemize}

\textbf{PDF:}

The probability density function (PDF) of the Sine Skewed X distribution
is not explicitly defined here, but it is derived from the base
distribution and the skewness parameter.

\paragraph{Example Usage:}\label{example-usage-68}

from num.pyro import distributions as dist import num.pyro as pyro
import num.numpy as np

m = pyro.distributions.Normal(loc=0.0, scale=1.0) skewness =
np.array({[}0.5, 0.5{]}) sine\_skewed = dist.SineSkewed(base\_dist=m,
skewness=skewness) samples = sine\_skewed.sample((1000,))

\begin{Shaded}
\begin{Highlighting}[]
\NormalTok{bi.dist.sine\_skewed(}
\NormalTok{base\_dist: numpyro.distributions.distribution.Distribution,}
\NormalTok{skewness,}
\NormalTok{validate\_args}\OperatorTok{=}\VariableTok{None}\NormalTok{,}
\NormalTok{name}\OperatorTok{=}\StringTok{\textquotesingle{}x\textquotesingle{}}\NormalTok{,}
\NormalTok{obs}\OperatorTok{=}\VariableTok{None}\NormalTok{,}
\NormalTok{mask}\OperatorTok{=}\VariableTok{None}\NormalTok{,}
\NormalTok{sample}\OperatorTok{=}\VariableTok{False}\NormalTok{,}
\NormalTok{seed}\OperatorTok{=}\DecValTok{0}\NormalTok{,}
\NormalTok{shape}\OperatorTok{=}\NormalTok{(),}
\NormalTok{event}\OperatorTok{=}\DecValTok{0}\NormalTok{,}
\NormalTok{create\_obj}\OperatorTok{=}\VariableTok{False}\NormalTok{,}
\NormalTok{)}
\end{Highlighting}
\end{Shaded}

\begin{center}\rule{0.5\linewidth}{0.5pt}\end{center}

\subsubsection{SoftLaplace}\label{softlaplace}

Samples from a SoftLaplace distribution.

This distribution is a smooth approximation of a Laplace distribution,
characterized by its log-convex density. It offers Laplace-like tails
while being infinitely differentiable, making it suitable for HMC and
Laplace approximation.

\[
f(x) = \log\!\left(\tfrac{2}{\pi}\right) - \log(\text{scale})
- \log\!\left( e^{\tfrac{x - \text{loc}}{\text{scale}}} + e^{-\tfrac{x - \text{loc}}{\text{scale}}} \right)
\]

\paragraph{Args:}\label{args-70}

\begin{Shaded}
\begin{Highlighting}[]
\NormalTok{bi.dist.soft\_laplace(}
\NormalTok{loc,}
\NormalTok{scale,}
\NormalTok{validate\_args}\OperatorTok{=}\VariableTok{None}\NormalTok{,}
\NormalTok{name}\OperatorTok{=}\StringTok{\textquotesingle{}x\textquotesingle{}}\NormalTok{,}
\NormalTok{obs}\OperatorTok{=}\VariableTok{None}\NormalTok{,}
\NormalTok{mask}\OperatorTok{=}\VariableTok{None}\NormalTok{,}
\NormalTok{sample}\OperatorTok{=}\VariableTok{False}\NormalTok{,}
\NormalTok{seed}\OperatorTok{=}\DecValTok{0}\NormalTok{,}
\NormalTok{shape}\OperatorTok{=}\NormalTok{(),}
\NormalTok{event}\OperatorTok{=}\DecValTok{0}\NormalTok{,}
\NormalTok{create\_obj}\OperatorTok{=}\VariableTok{False}\NormalTok{,}
\NormalTok{)}
\end{Highlighting}
\end{Shaded}

\begin{itemize}
\tightlist
\item
  \emph{loc}: Location parameter. scale: Scale parameter. \$\$
\end{itemize}

\paragraph{Args:}\label{args-71}

\begin{itemize}
\item
  \emph{shape} (tuple): A multi-purpose argument for shaping. When
  \texttt{sample=False} (model building), this is used with
  \texttt{.expand(shape)} to set the distribution's batch shape. When
  \texttt{sample=True} (direct sampling), this is used as
  \texttt{sample\_shape} to draw a raw JAX array of the given shape.
\item
  \emph{event} (int): The number of batch dimensions to reinterpret as
  event dimensions (used in model building).
\item
  \emph{mask} (jnp.ndarray, bool): Optional boolean array to mask
  observations.
\item
  \emph{create\_obj} (bool): If True, returns the raw BI distribution
  object instead of creating a sample site. This is essential for
  building complex distributions like \texttt{MixtureSameFamily}.
\item
  \emph{sample} (bool, optional): A control-flow argument. If
  \texttt{True}, the function will directly sample a raw JAX array from
  the distribution, bypassing the BI model context. If \texttt{False},
  it will create a \texttt{BI.sample} site within a model. Defaults to
  \texttt{False}.
\item
  \emph{seed} (int, optional): An integer used to generate a JAX PRNGKey
  for reproducible sampling when \texttt{sample=True}. {[}7{]} This
  argument has no effect when \texttt{sample=False}, as randomness is
  handled by BI's inference engine. Defaults to 0.
\item
  \emph{obs} (jnp.ndarray, optional): The observed value for this random
  variable. If provided, the sample site is conditioned on this value,
  and the function returns the observed value. If \texttt{None}, the
  site is treated as a latent (unobserved) random variable. Defaults to
  \texttt{None}.
\item
  \emph{name} (str, optional): The name of the sample site in a BI
  model. This is used to uniquely identify the random variable. Defaults
  to `x'.
\end{itemize}

\paragraph{Returns:}\label{returns-70}

\begin{itemize}
\tightlist
\item
  When \texttt{sample=False}: A BI SoftLaplace distribution object (for
  model building).
\item
  When \texttt{sample=True}: A JAX array of samples drawn from the
  SoftLaplace distribution (for direct sampling).
\item
  When \texttt{create\_obj=True}: The raw BI distribution object (for
  advanced use cases).
\end{itemize}

\paragraph{Example Usage:}\label{example-usage-69}

\begin{Shaded}
\begin{Highlighting}[]
\ImportTok{from}\NormalTok{ BI }\ImportTok{import}\NormalTok{ bi}
\NormalTok{m }\OperatorTok{=}\NormalTok{ bi(}\StringTok{\textquotesingle{}cpu\textquotesingle{}}\NormalTok{)}
\NormalTok{m.dist.soft\_laplace(loc}\OperatorTok{=}\FloatTok{0.0}\NormalTok{, scale}\OperatorTok{=}\FloatTok{1.0}\NormalTok{, sample}\OperatorTok{=}\VariableTok{True}\NormalTok{)}
\end{Highlighting}
\end{Shaded}

\paragraph{Wrapper of:}\label{wrapper-of-49}

https://num.pyro.ai/en/stable/distributions.html\#softlaplace

\begin{center}\rule{0.5\linewidth}{0.5pt}\end{center}

\subsubsection{Student's t}\label{students-t}

The Student's t-distribution is a probability distribution that arises
in hypothesis testing involving the mean of a normally distributed
population when the population standard deviation is unknown. It is
similar to the normal distribution, but has heavier tails, making it
more robust to outliers.

\[
f(x) = \frac{1}{\Gamma(\nu/2) \sqrt{\nu \pi}} \left(1 + \frac{x^2}{\nu}\right)^{-(\nu+1)/2}
\]

\paragraph{Args:}\label{args-72}

\begin{Shaded}
\begin{Highlighting}[]
\NormalTok{bi.dist.student\_t(}
\NormalTok{df,}
\NormalTok{loc}\OperatorTok{=}\FloatTok{0.0}\NormalTok{,}
\NormalTok{scale}\OperatorTok{=}\FloatTok{1.0}\NormalTok{,}
\NormalTok{validate\_args}\OperatorTok{=}\VariableTok{None}\NormalTok{,}
\NormalTok{name}\OperatorTok{=}\StringTok{\textquotesingle{}x\textquotesingle{}}\NormalTok{,}
\NormalTok{obs}\OperatorTok{=}\VariableTok{None}\NormalTok{,}
\NormalTok{mask}\OperatorTok{=}\VariableTok{None}\NormalTok{,}
\NormalTok{sample}\OperatorTok{=}\VariableTok{False}\NormalTok{,}
\NormalTok{seed}\OperatorTok{=}\DecValTok{0}\NormalTok{,}
\NormalTok{shape}\OperatorTok{=}\NormalTok{(),}
\NormalTok{event}\OperatorTok{=}\DecValTok{0}\NormalTok{,}
\NormalTok{create\_obj}\OperatorTok{=}\VariableTok{False}\NormalTok{,}
\NormalTok{)}
\end{Highlighting}
\end{Shaded}

df (jnp.ndarray): Degrees of freedom, must be positive. - \emph{loc}
(jnp.ndarray): Location parameter, defaults to 0.0. - \emph{sample}
(jnp.ndarray): Scale parameter, defaults to 1.0.

\begin{itemize}
\item
  \emph{shape} (tuple): A multi-purpose argument for shaping. When
  \texttt{sample=False} (model building), this is used with
  \texttt{.expand(shape)} to set the distribution's batch shape. When
  \texttt{sample=True} (direct sampling), this is used as
  \texttt{sample\_shape} to draw a raw JAX array of the given shape.
\item
  \emph{event} (int): The number of batch dimensions to reinterpret as
  event dimensions (used in model building).
\item
  \emph{mask} (jnp.ndarray, bool): Optional boolean array to mask
  observations.
\item
  \emph{create\_obj} (bool): If True, returns the raw BI distribution
  object instead of creating a sample site. This is essential for
  building complex distributions like \texttt{MixtureSameFamily}.
\item
  \emph{sample} (bool, optional): A control-flow argument. If
  \texttt{True}, the function will directly sample a raw JAX array from
  the distribution, bypassing the BI model context. If \texttt{False},
  it will create a \texttt{BI.sample} site within a model. Defaults to
  \texttt{False}.
\item
  \emph{seed} (int, optional): An integer used to generate a JAX PRNGKey
  for reproducible sampling when \texttt{sample=True}. {[}7{]} This
  argument has no effect when \texttt{sample=False}, as randomness is
  handled by BI's inference engine. Defaults to 0.
\item
  \emph{obs} (jnp.ndarray, optional): The observed value for this random
  variable. If provided, the sample site is conditioned on this value,
  and the function returns the observed value. If \texttt{None}, the
  site is treated as a latent (unobserved) random variable. Defaults to
  \texttt{None}.
\item
  \emph{name} (str, optional): The name of the sample site in a BI
  model. This is used to uniquely identify the random variable. Defaults
  to `x'.
\end{itemize}

\paragraph{Returns:}\label{returns-71}

\begin{itemize}
\tightlist
\item
  When \texttt{sample=False}: A BI StudentT distribution object (for
  model building).
\item
  When \texttt{sample=True}: A JAX array of samples drawn from the
  StudentT distribution (for direct sampling).
\item
  When \texttt{create\_obj=True}: The raw BI distribution object (for
  advanced use cases).
\end{itemize}

\paragraph{Example Usage:}\label{example-usage-70}

\begin{Shaded}
\begin{Highlighting}[]
\ImportTok{from}\NormalTok{ BI }\ImportTok{import}\NormalTok{ bi}
\NormalTok{m }\OperatorTok{=}\NormalTok{ bi(}\StringTok{\textquotesingle{}cpu\textquotesingle{}}\NormalTok{)}
\NormalTok{m.dist.student\_t(df }\OperatorTok{=} \DecValTok{2}\NormalTok{, loc}\OperatorTok{=}\FloatTok{0.0}\NormalTok{, scale}\OperatorTok{=}\FloatTok{1.0}\NormalTok{, sample}\OperatorTok{=}\VariableTok{True}\NormalTok{)}
\end{Highlighting}
\end{Shaded}

\paragraph{Wrapper of:}\label{wrapper-of-50}

https://num.pyro.ai/en/stable/distributions.html\#studentt

\begin{center}\rule{0.5\linewidth}{0.5pt}\end{center}

\subsubsection{Truncated Cauchy}\label{truncated-cauchy}

The Cauchy distribution, also known as the Lorentz distribution, is a
continuous probability distribution that appears frequently in various
areas of mathematics and physics. It is characterized by its heavy
tails, which extend to infinity. The truncated version limits the
support of the Cauchy distribution to a specified interval.

\[
f(x) = \frac{1}{\pi \cdot c \cdot (1 + ((x - b) / c)^2)}  \text{ for } a < x < b
\]

\paragraph{Args:}\label{args-73}

\begin{Shaded}
\begin{Highlighting}[]
\NormalTok{bi.dist.truncated\_cauchy(}
\NormalTok{loc}\OperatorTok{=}\FloatTok{0.0}\NormalTok{,}
\NormalTok{scale}\OperatorTok{=}\FloatTok{1.0}\NormalTok{,}
\NormalTok{low}\OperatorTok{=}\VariableTok{None}\NormalTok{,}
\NormalTok{high}\OperatorTok{=}\VariableTok{None}\NormalTok{,}
\NormalTok{validate\_args}\OperatorTok{=}\VariableTok{None}\NormalTok{,}
\NormalTok{name}\OperatorTok{=}\StringTok{\textquotesingle{}x\textquotesingle{}}\NormalTok{,}
\NormalTok{obs}\OperatorTok{=}\VariableTok{None}\NormalTok{,}
\NormalTok{mask}\OperatorTok{=}\VariableTok{None}\NormalTok{,}
\NormalTok{sample}\OperatorTok{=}\VariableTok{False}\NormalTok{,}
\NormalTok{seed}\OperatorTok{=}\DecValTok{0}\NormalTok{,}
\NormalTok{shape}\OperatorTok{=}\NormalTok{(),}
\NormalTok{event}\OperatorTok{=}\DecValTok{0}\NormalTok{,}
\NormalTok{create\_obj}\OperatorTok{=}\VariableTok{False}\NormalTok{,}
\NormalTok{)}
\end{Highlighting}
\end{Shaded}

\begin{itemize}
\item
  \emph{loc} (float): Location parameter of the Cauchy distribution.
\item
  \emph{sample} (float): Scale parameter of the Cauchy distribution.
\item
  \emph{shape} (tuple): A multi-purpose argument for shaping. When
  \texttt{sample=False} (model building), this is used with
  \texttt{.expand(shape)} to set the distribution's batch shape. When
  \texttt{sample=True} (direct sampling), this is used as
  \texttt{sample\_shape} to draw a raw JAX array of the given shape.
\item
  \emph{event} (int): The number of batch dimensions to reinterpret as
  event dimensions (used in model building).
\item
  \emph{mask} (jnp.ndarray, bool): Optional boolean array to mask
  observations.
\item
  \emph{create\_obj} (bool): If True, returns the raw BI distribution
  object instead of creating a sample site. This is essential for
  building complex distributions like \texttt{MixtureSameFamily}.
\item
  \emph{sample} (bool, optional): A control-flow argument. If
  \texttt{True}, the function will directly sample a raw JAX array from
  the distribution, bypassing the BI model context. If \texttt{False},
  it will create a \texttt{BI.sample} site within a model. Defaults to
  \texttt{False}.
\item
  \emph{seed} (int, optional): An integer used to generate a JAX PRNGKey
  for reproducible sampling when \texttt{sample=True}. {[}7{]} This
  argument has no effect when \texttt{sample=False}, as randomness is
  handled by BI's inference engine. Defaults to 0.
\item
  \emph{obs} (jnp.ndarray, optional): The observed value for this random
  variable. If provided, the sample site is conditioned on this value,
  and the function returns the observed value. If \texttt{None}, the
  site is treated as a latent (unobserved) random variable. Defaults to
  \texttt{None}.
\item
  \emph{name} (str, optional): The name of the sample site in a BI
  model. This is used to uniquely identify the random variable. Defaults
  to `x'.
\end{itemize}

\paragraph{Returns:}\label{returns-72}

BI TruncatedCauchy distribution object (for model building) when
\texttt{sample=False}.

JAX array of samples drawn from the TruncatedCauchy distribution (for
direct sampling) when \texttt{sample=True}.

The raw BI distribution object (for advanced use cases) when
\texttt{create\_obj=True}.

\paragraph{Example Usage:}\label{example-usage-71}

\begin{Shaded}
\begin{Highlighting}[]
\ImportTok{from}\NormalTok{ BI }\ImportTok{import}\NormalTok{ bi}
\NormalTok{m }\OperatorTok{=}\NormalTok{ bi(}\StringTok{\textquotesingle{}cpu\textquotesingle{}}\NormalTok{)}
\NormalTok{m.dist.truncated\_cauchy(loc}\OperatorTok{=}\FloatTok{0.0}\NormalTok{, scale}\OperatorTok{=}\FloatTok{1.0}\NormalTok{, sample}\OperatorTok{=}\VariableTok{True}\NormalTok{)}
\end{Highlighting}
\end{Shaded}

\paragraph{Wrapper of:
https://num.pyro.ai/en/stable/distributions.html\#truncatedcauchy}\label{wrapper-of-httpsnum.pyro.aienstabledistributions.htmltruncatedcauchy}

\begin{center}\rule{0.5\linewidth}{0.5pt}\end{center}

\subsubsection{Truncated}\label{truncated}

Samples from a Truncated Distribution.

This distribution represents a base distribution truncated between
specified lower and upper bounds. The truncation modifies the
probability density function (PDF) of the base distribution, effectively
removing observations outside the defined interval.

\[   p(x) = \frac{p(x)}{P(\text{lower} \le x \le \text{upper})}
\]

\paragraph{Args:}\label{args-74}

\begin{Shaded}
\begin{Highlighting}[]
\NormalTok{bi.dist.truncated\_distribution(}
\NormalTok{base\_dist,}
\NormalTok{low}\OperatorTok{=}\VariableTok{None}\NormalTok{,}
\NormalTok{high}\OperatorTok{=}\VariableTok{None}\NormalTok{,}
\NormalTok{validate\_args}\OperatorTok{=}\VariableTok{None}\NormalTok{,}
\NormalTok{name}\OperatorTok{=}\StringTok{\textquotesingle{}x\textquotesingle{}}\NormalTok{,}
\NormalTok{obs}\OperatorTok{=}\VariableTok{None}\NormalTok{,}
\NormalTok{mask}\OperatorTok{=}\VariableTok{None}\NormalTok{,}
\NormalTok{sample}\OperatorTok{=}\VariableTok{False}\NormalTok{,}
\NormalTok{seed}\OperatorTok{=}\DecValTok{0}\NormalTok{,}
\NormalTok{shape}\OperatorTok{=}\NormalTok{(),}
\NormalTok{event}\OperatorTok{=}\DecValTok{0}\NormalTok{,}
\NormalTok{create\_obj}\OperatorTok{=}\VariableTok{False}\NormalTok{,}
\NormalTok{)}
\end{Highlighting}
\end{Shaded}

base\_dist: The base distribution to be truncated. This should be a
univariate distribution. Currently, only the following distributions are
supported: Cauchy, Laplace, Logistic, Normal, and StudentT.

\begin{itemize}
\item
  \emph{shape} (tuple): A multi-purpose argument for shaping. When
  \texttt{sample=False} (model building), this is used with
  \texttt{.expand(shape)} to set the distribution's batch shape. When
  \texttt{sample=True} (direct sampling), this is used as
  \texttt{sample\_shape} to draw a raw JAX array of the given shape.
\item
  \emph{event} (int): The number of batch dimensions to reinterpret as
  event dimensions (used in model building).
\item
  \emph{mask} (jnp.ndarray, bool): Optional boolean array to mask
  observations.
\item
  \emph{create\_obj} (bool): If True, returns the raw BI distribution
  object instead of creating a sample site. This is essential for
  building complex distributions like \texttt{MixtureSameFamily}.
\item
  \emph{sample} (bool, optional): A control-flow argument. If
  \texttt{True}, the function will directly sample a raw JAX array from
  the distribution, bypassing the BI model context. If \texttt{False},
  it will create a \texttt{BI.sample} site within a model. Defaults to
  \texttt{False}.
\item
  \emph{seed} (int, optional): An integer used to generate a JAX PRNGKey
  for reproducible sampling when \texttt{sample=True}. {[}7{]} This
  argument has no effect when \texttt{sample=False}, as randomness is
  handled by BI's inference engine. Defaults to 0.
\item
  \emph{obs} (jnp.ndarray, optional): The observed value for this random
  variable. If provided, the sample site is conditioned on this value,
  and the function returns the observed value. If \texttt{None}, the
  site is treated as a latent (unobserved) random variable. Defaults to
  \texttt{None}.
\item
  \emph{name} (str, optional): The name of the sample site in a BI
  model. This is used to uniquely identify the random variable. Defaults
  to `x'.
\end{itemize}

\paragraph{Returns:}\label{returns-73}

\begin{itemize}
\tightlist
\item
  When \texttt{sample=False}: A BI TruncatedDistribution distribution
  object (for model building).
\item
  When \texttt{sample=True}: A JAX array of samples drawn from the
  TruncatedDistribution distribution (for direct sampling).
\item
  When \texttt{create\_obj=True}: The raw BI distribution object (for
  advanced use cases).
\end{itemize}

\paragraph{Example Usage:}\label{example-usage-72}

\begin{Shaded}
\begin{Highlighting}[]
\ImportTok{from}\NormalTok{ BI }\ImportTok{import}\NormalTok{ bi}
\NormalTok{m }\OperatorTok{=}\NormalTok{ bi(}\StringTok{\textquotesingle{}cpu\textquotesingle{}}\NormalTok{)}
\NormalTok{m.dist.truncated\_distribution(base\_dist }\OperatorTok{=}\NormalTok{ m.dist.normal(}\DecValTok{0}\NormalTok{,}\DecValTok{1}\NormalTok{, create\_obj }\OperatorTok{=} \VariableTok{True}\NormalTok{), high}\OperatorTok{=}\DecValTok{1}\NormalTok{, low }\OperatorTok{=} \DecValTok{0}\NormalTok{, sample}\OperatorTok{=}\VariableTok{True}\NormalTok{)}
\end{Highlighting}
\end{Shaded}

\paragraph{Wrapper of:
https://num.pyro.ai/en/stable/distributions.html\#truncateddistribution}\label{wrapper-of-httpsnum.pyro.aienstabledistributions.htmltruncateddistribution}

\begin{center}\rule{0.5\linewidth}{0.5pt}\end{center}

\subsubsection{Truncated Normal}\label{truncated-normal}

The Truncated Normal distribution is a normal distribution truncated to
a specified interval. It is defined by its location (\texttt{loc}),
scale (\texttt{scale}), lower bound (\texttt{low}), and upper bound
(\texttt{high}).

\[
f(x) = \frac{p(x)}{\alpha}, \quad x \in [\text{low}, \text{high}]
\]

where

\[
p(x) = \frac{1}{\text{scale}\,\sqrt{2\pi}}
\exp\!\left(-\tfrac{1}{2}\left(\tfrac{x - \text{loc}}{\text{scale}}\right)^2\right),
\]

and

\[
\alpha = \int_{\text{low}}^{\text{high}} p(x)\,dx.
\]

\paragraph{Args:}\label{args-75}

\begin{Shaded}
\begin{Highlighting}[]
\NormalTok{bi.dist.truncated\_normal(}
\NormalTok{loc}\OperatorTok{=}\FloatTok{0.0}\NormalTok{,}
\NormalTok{scale}\OperatorTok{=}\FloatTok{1.0}\NormalTok{,}
\NormalTok{low}\OperatorTok{=}\VariableTok{None}\NormalTok{,}
\NormalTok{high}\OperatorTok{=}\VariableTok{None}\NormalTok{,}
\NormalTok{validate\_args}\OperatorTok{=}\VariableTok{None}\NormalTok{,}
\NormalTok{name}\OperatorTok{=}\StringTok{\textquotesingle{}x\textquotesingle{}}\NormalTok{,}
\NormalTok{obs}\OperatorTok{=}\VariableTok{None}\NormalTok{,}
\NormalTok{mask}\OperatorTok{=}\VariableTok{None}\NormalTok{,}
\NormalTok{sample}\OperatorTok{=}\VariableTok{False}\NormalTok{,}
\NormalTok{seed}\OperatorTok{=}\DecValTok{0}\NormalTok{,}
\NormalTok{shape}\OperatorTok{=}\NormalTok{(),}
\NormalTok{event}\OperatorTok{=}\DecValTok{0}\NormalTok{,}
\NormalTok{create\_obj}\OperatorTok{=}\VariableTok{False}\NormalTok{,}
\NormalTok{)}
\end{Highlighting}
\end{Shaded}

\begin{itemize}
\item
  \emph{loc} (float): The location parameter of the normal distribution.
\item
  \emph{sample} (float): The scale parameter of the normal distribution.
\item
  \emph{shape} (tuple): A multi-purpose argument for shaping. When
  \texttt{sample=False} (model building), this is used with
  \texttt{.expand(shape)} to set the distribution's batch shape. When
  \texttt{sample=True} (direct sampling), this is used as
  \texttt{sample\_shape} to draw a raw JAX array of the given shape.
\item
  \emph{event} (int): The number of batch dimensions to reinterpret as
  event dimensions (used in model building).
\item
  \emph{mask} (jnp.ndarray, bool): Optional boolean array to mask
  observations.
\item
  \emph{create\_obj} (bool): If True, returns the raw BI distribution
  object instead of creating a sample site. This is essential for
  building complex distributions like \texttt{MixtureSameFamily}.
\item
  \emph{sample} (bool, optional): A control-flow argument. If
  \texttt{True}, the function will directly sample a raw JAX array from
  the distribution, bypassing the BI model context. If \texttt{False},
  it will create a \texttt{BI.sample} site within a model. Defaults to
  \texttt{False}.
\item
  \emph{seed} (int, optional): An integer used to generate a JAX PRNGKey
  for reproducible sampling when \texttt{sample=True}. {[}7{]} This
  argument has no effect when \texttt{sample=False}, as randomness is
  handled by BI's inference engine. Defaults to 0.
\item
  \emph{obs} (jnp.ndarray, optional): The observed value for this random
  variable. If provided, the sample site is conditioned on this value,
  and the function returns the observed value. If \texttt{None}, the
  site is treated as a latent (unobserved) random variable. Defaults to
  \texttt{None}.
\item
  \emph{name} (str, optional): The name of the sample site in a BI
  model. This is used to uniquely identify the random variable. Defaults
  to `x'.
\end{itemize}

\paragraph{Returns:}\label{returns-74}

BI TruncatedNormal distribution object (for model building). JAX array
of samples drawn from the TruncatedNormal distribution (for direct
sampling). The raw BI distribution object (for advanced use cases).

\paragraph{Example Usage:}\label{example-usage-73}

\begin{Shaded}
\begin{Highlighting}[]
\ImportTok{from}\NormalTok{ BI }\ImportTok{import}\NormalTok{ bi}
\NormalTok{m }\OperatorTok{=}\NormalTok{ bi(}\StringTok{\textquotesingle{}cpu\textquotesingle{}}\NormalTok{)}
\NormalTok{m.dist.truncated\_normal(loc}\OperatorTok{=}\FloatTok{0.0}\NormalTok{, scale}\OperatorTok{=}\FloatTok{1.0}\NormalTok{, sample}\OperatorTok{=}\VariableTok{True}\NormalTok{)}
\end{Highlighting}
\end{Shaded}

\paragraph{Wrapper of:
https://num.pyro.ai/en/stable/distributions.html\#truncatednormal\_lowercase}\label{wrapper-of-httpsnum.pyro.aienstabledistributions.htmltruncatednormal_lowercase}

\begin{center}\rule{0.5\linewidth}{0.5pt}\end{center}

\subsubsection{Truncated PolyaGamma}\label{truncated-polyagamma}

Samples from a Truncated PolyaGamma distribution.

This distribution is a truncated version of the PolyaGamma distribution,
defined over the interval {[}0, truncation\_point{]}. It is often used
in Bayesian non-parametric models.

\[   p(x) = \frac{1}{Z} \exp\left( \sum_{n=0}^{N} \left( \log(2n+1) - 1.5 \log(x) - \frac{(2n+1)^2}{4x} \right) \right)
\]

\paragraph{Args:}\label{args-76}

\begin{Shaded}
\begin{Highlighting}[]
\NormalTok{bi.dist.truncated\_polya\_gamma(}
\NormalTok{batch\_shape}\OperatorTok{=}\NormalTok{(),}
\NormalTok{validate\_args}\OperatorTok{=}\VariableTok{None}\NormalTok{,}
\NormalTok{name}\OperatorTok{=}\StringTok{\textquotesingle{}x\textquotesingle{}}\NormalTok{,}
\NormalTok{obs}\OperatorTok{=}\VariableTok{None}\NormalTok{,}
\NormalTok{mask}\OperatorTok{=}\VariableTok{None}\NormalTok{,}
\NormalTok{sample}\OperatorTok{=}\VariableTok{False}\NormalTok{,}
\NormalTok{seed}\OperatorTok{=}\DecValTok{0}\NormalTok{,}
\NormalTok{shape}\OperatorTok{=}\NormalTok{(),}
\NormalTok{event}\OperatorTok{=}\DecValTok{0}\NormalTok{,}
\NormalTok{create\_obj}\OperatorTok{=}\VariableTok{False}\NormalTok{,}
\NormalTok{)}
\end{Highlighting}
\end{Shaded}

batch\_shape (tuple): The shape of the batch dimension.

\begin{itemize}
\item
  \emph{event} (int): The number of batch dimensions to reinterpret as
  event dimensions.
\item
  \emph{mask} (jnp.ndarray, bool): Optional boolean array to mask
  observations.
\item
  \emph{create\_obj} (bool): If True, returns the raw BI distribution
  object instead of creating a sample site.
\item
  \emph{sample} (bool, optional): A control-flow argument. If
  \texttt{True}, the function will directly sample a raw JAX array from
  the distribution, bypassing the BI model context. If \texttt{False},
  it will create a \texttt{BI.sample} site within a model. Defaults to
  \texttt{False}.
\item
  \emph{seed} (int, optional): An integer used to generate a JAX PRNGKey
  for reproducible sampling when \texttt{sample=True}. {[}7{]} This
  argument has no effect when \texttt{sample=False}, as randomness is
  handled by BI's inference engine. Defaults to 0.
\item
  \emph{obs} (jnp.ndarray, optional): The observed value for this random
  variable. If provided, the sample site is conditioned on this value,
  and the function returns the observed value. If \texttt{None}, the
  site is treated as a latent (unobserved) random variable. Defaults to
  \texttt{None}.
\item
  \emph{name} (str, optional): The name of the sample site in a BI
  model. This is used to uniquely identify the random variable. Defaults
  to `x'.
\end{itemize}

\paragraph{Returns:}\label{returns-75}

\begin{itemize}
\tightlist
\item
  When \texttt{sample=False}: A BI Truncated PolyaGamma distribution
  object (for model building).
\item
  When \texttt{sample=True}: A JAX array of samples drawn from the
  Truncated PolyaGamma distribution (for direct sampling).
\item
  When \texttt{create\_obj=True}: The raw BI distribution object (for
  advanced use cases).
\end{itemize}

\paragraph{Example Usage:}\label{example-usage-74}

\begin{Shaded}
\begin{Highlighting}[]
\ImportTok{from}\NormalTok{ BI }\ImportTok{import}\NormalTok{ bi}
\NormalTok{m }\OperatorTok{=}\NormalTok{ bi(}\StringTok{\textquotesingle{}cpu\textquotesingle{}}\NormalTok{)}
\NormalTok{m.dist.truncated\_polya\_gamma(batch\_shape}\OperatorTok{=}\NormalTok{(), sample}\OperatorTok{=}\VariableTok{True}\NormalTok{)}
\end{Highlighting}
\end{Shaded}

\paragraph{Wrapper of:}\label{wrapper-of-51}

https://num.pyro.ai/en/stable/distributions.html\#truncatedpolygammadistribution

\begin{center}\rule{0.5\linewidth}{0.5pt}\end{center}

\subsubsection{Two Sided Truncated}\label{two-sided-truncated}

This distribution truncates a base distribution between two specified
lower and upper bounds.

\[
f(x) =
\begin{cases}
\dfrac{p(x)}{P(\text{low} \le X \le \text{high})}, & \text{if } \text{low} \le x \le \text{high}, \\[6pt]
0, & \text{otherwise}.
\end{cases}
\]

where \(p(x)\) is the probability density function of the base
distribution.

\paragraph{Args:}\label{args-77}

\begin{Shaded}
\begin{Highlighting}[]
\NormalTok{bi.dist.two\_sided\_truncated\_distribution(}
\NormalTok{base\_dist,}
\NormalTok{low}\OperatorTok{=}\FloatTok{0.0}\NormalTok{,}
\NormalTok{high}\OperatorTok{=}\FloatTok{1.0}\NormalTok{,}
\NormalTok{validate\_args}\OperatorTok{=}\VariableTok{None}\NormalTok{,}
\NormalTok{name}\OperatorTok{=}\StringTok{\textquotesingle{}x\textquotesingle{}}\NormalTok{,}
\NormalTok{obs}\OperatorTok{=}\VariableTok{None}\NormalTok{,}
\NormalTok{mask}\OperatorTok{=}\VariableTok{None}\NormalTok{,}
\NormalTok{sample}\OperatorTok{=}\VariableTok{False}\NormalTok{,}
\NormalTok{seed}\OperatorTok{=}\DecValTok{0}\NormalTok{,}
\NormalTok{shape}\OperatorTok{=}\NormalTok{(),}
\NormalTok{event}\OperatorTok{=}\DecValTok{0}\NormalTok{,}
\NormalTok{create\_obj}\OperatorTok{=}\VariableTok{False}\NormalTok{,}
\NormalTok{)}
\end{Highlighting}
\end{Shaded}

base\_dist: The base distribution to truncate.

low: The lower bound for truncation.

high: The upper bound for truncation.

\begin{itemize}
\item
  \emph{sample} (bool, optional): A control-flow argument. If
  \texttt{True}, the function will directly sample a raw JAX array from
  the distribution, bypassing the BI model context. If \texttt{False},
  it will create a \texttt{BI.sample} site within a model. Defaults to
  \texttt{False}.
\item
  \emph{seed} (int, optional): An integer used to generate a JAX PRNGKey
  for reproducible sampling when \texttt{sample=True}. {[}7{]} This
  argument has no effect when \texttt{sample=False}, as randomness is
  handled by BI's inference engine. Defaults to 0.
\item
  \emph{obs} (jnp.ndarray, optional): The observed value for this random
  variable. If provided, the sample site is conditioned on this value,
  and the function returns the observed value. If \texttt{None}, the
  site is treated as a latent (unobserved) random variable. Defaults to
  \texttt{None}.
\item
  \emph{name} (str, optional): The name of the sample site in a BI
  model. This is used to uniquely identify the random variable. Defaults
  to `x'.
\end{itemize}

\paragraph{Returns:}\label{returns-76}

\begin{itemize}
\tightlist
\item
  When \texttt{sample=False}: A BI TwoSidedTruncatedDistribution
  distribution object (for model building).
\item
  When \texttt{sample=True}: A JAX array of samples drawn from the
  TwoSidedTruncatedDistribution distribution (for direct sampling).
\item
  When \texttt{create\_obj=True}: The raw BI distribution object (for
  advanced use cases).
\end{itemize}

\paragraph{Wrapper of:
https://num.pyro.ai/en/stable/distributions.html\#twosidedtruncateddistribution}\label{wrapper-of-httpsnum.pyro.aienstabledistributions.htmltwosidedtruncateddistribution}

\begin{center}\rule{0.5\linewidth}{0.5pt}\end{center}

\subsubsection{Uniform}\label{uniform}

Samples from a Uniform distribution, which is a continuous probability
distribution where all values within a given interval are equally
likely.

\[   f(x) = \frac{1}{b - a}, \text{ for } a \le x \le b
\]

\paragraph{Args:}\label{args-78}

\begin{Shaded}
\begin{Highlighting}[]
\NormalTok{bi.dist.uniform(}
\NormalTok{low}\OperatorTok{=}\FloatTok{0.0}\NormalTok{,}
\NormalTok{high}\OperatorTok{=}\FloatTok{1.0}\NormalTok{,}
\NormalTok{validate\_args}\OperatorTok{=}\VariableTok{None}\NormalTok{,}
\NormalTok{name}\OperatorTok{=}\StringTok{\textquotesingle{}x\textquotesingle{}}\NormalTok{,}
\NormalTok{obs}\OperatorTok{=}\VariableTok{None}\NormalTok{,}
\NormalTok{mask}\OperatorTok{=}\VariableTok{None}\NormalTok{,}
\NormalTok{sample}\OperatorTok{=}\VariableTok{False}\NormalTok{,}
\NormalTok{seed}\OperatorTok{=}\DecValTok{0}\NormalTok{,}
\NormalTok{shape}\OperatorTok{=}\NormalTok{(),}
\NormalTok{event}\OperatorTok{=}\DecValTok{0}\NormalTok{,}
\NormalTok{create\_obj}\OperatorTok{=}\VariableTok{False}\NormalTok{,}
\NormalTok{)}
\end{Highlighting}
\end{Shaded}

low (jnp.ndarray): The lower bound of the uniform interval.

high (jnp.ndarray): The upper bound of the uniform interval.

\begin{itemize}
\item
  \emph{shape} (tuple): A multi-purpose argument for shaping. When
  \texttt{sample=False} (model building), this is used with
  \texttt{.expand(shape)} to set the distribution's batch shape. When
  \texttt{sample=True} (direct sampling), this is used as
  \texttt{sample\_shape} to draw a raw JAX array of the given shape.
\item
  \emph{event} (int): The number of batch dimensions to reinterpret as
  event dimensions (used in model building).
\item
  \emph{mask} (jnp.ndarray, bool): Optional boolean array to mask
  observations.
\item
  \emph{create\_obj} (bool): If True, returns the raw BI distribution
  object instead of creating a sample site. This is essential for
  building complex distributions like \texttt{MixtureSameFamily}.
\item
  \emph{sample} (bool, optional): A control-flow argument. If
  \texttt{True}, the function will directly sample a raw JAX array from
  the distribution, bypassing the BI model context. If \texttt{False},
  it will create a \texttt{BI.sample} site within a model. Defaults to
  \texttt{False}.
\item
  \emph{seed} (int, optional): An integer used to generate a JAX PRNGKey
  for reproducible sampling when \texttt{sample=True}. {[}7{]} This
  argument has no effect when \texttt{sample=False}, as randomness is
  handled by BI's inference engine. Defaults to 0.
\item
  \emph{obs} (jnp.ndarray, optional): The observed value for this random
  variable. If provided, the sample site is conditioned on this value,
  and the function returns the observed value. If \texttt{None}, the
  site is treated as a latent (unobserved) random variable. Defaults to
  \texttt{None}.
\item
  \emph{name} (str, optional): The name of the sample site in a BI
  model. This is used to uniquely identify the random variable. Defaults
  to `x'.
\end{itemize}

\paragraph{Returns:}\label{returns-77}

BI Uniform distribution object (for model building) when
\texttt{sample=False}.

JAX array of samples drawn from the Uniform distribution (for direct
sampling) when \texttt{sample=True}.

The raw BI distribution object (for advanced use cases) when
\texttt{create\_obj=True}.

\paragraph{Example Usage:}\label{example-usage-75}

\begin{Shaded}
\begin{Highlighting}[]
\ImportTok{from}\NormalTok{ BI }\ImportTok{import}\NormalTok{ bi}
\NormalTok{m }\OperatorTok{=}\NormalTok{ bi(}\StringTok{\textquotesingle{}cpu\textquotesingle{}}\NormalTok{)}
\NormalTok{m.dist.uniform(low}\OperatorTok{=}\FloatTok{0.0}\NormalTok{, high}\OperatorTok{=}\FloatTok{1.0}\NormalTok{, sample}\OperatorTok{=}\VariableTok{True}\NormalTok{)}
\end{Highlighting}
\end{Shaded}

\paragraph{Wrapper of:}\label{wrapper-of-52}

https://num.pyro.ai/en/stable/distributions.html\#uniform

\begin{center}\rule{0.5\linewidth}{0.5pt}\end{center}

\subsubsection{Unit}\label{unit}

The Unit distribution is a trivial, non-normalized distribution
representing the unit type. It has a single value with no data,
effectively a placeholder often used in probabilistic programming for
situations where no actual data is involved.

\[
p(x) = 1
\]

\paragraph{Args:}\label{args-79}

\begin{Shaded}
\begin{Highlighting}[]
\NormalTok{bi.dist.unit(}
\NormalTok{log\_factor,}
\NormalTok{validate\_args}\OperatorTok{=}\VariableTok{None}\NormalTok{,}
\NormalTok{name}\OperatorTok{=}\StringTok{\textquotesingle{}x\textquotesingle{}}\NormalTok{,}
\NormalTok{obs}\OperatorTok{=}\VariableTok{None}\NormalTok{,}
\NormalTok{mask}\OperatorTok{=}\VariableTok{None}\NormalTok{,}
\NormalTok{sample}\OperatorTok{=}\VariableTok{False}\NormalTok{,}
\NormalTok{seed}\OperatorTok{=}\DecValTok{0}\NormalTok{,}
\NormalTok{shape}\OperatorTok{=}\NormalTok{(),}
\NormalTok{event}\OperatorTok{=}\DecValTok{0}\NormalTok{,}
\NormalTok{create\_obj}\OperatorTok{=}\VariableTok{False}\NormalTok{,}
\NormalTok{)}
\end{Highlighting}
\end{Shaded}

log\_factor (jnp.ndarray): Log factor for the unit distribution. This
parameter determines the - \emph{shape} and batch size of the
distribution.

\begin{itemize}
\item
  \emph{shape} (tuple): A multi-purpose argument for shaping. When
  \texttt{sample=False} (model building), this is used with
  \texttt{.expand(shape)} to set the distribution's batch shape. When
  \texttt{sample=True} (direct sampling), this is used as
  \texttt{sample\_shape} to draw a raw JAX array of the given shape.
\item
  \emph{event} (int): The number of batch dimensions to reinterpret as
  event dimensions (used in model building).
\item
  \emph{mask} (jnp.ndarray, bool): Optional boolean array to mask
  observations.
\item
  \emph{create\_obj} (bool): If True, returns the raw BI distribution
  object instead of creating a sample site. This is essential for
  building complex distributions like \texttt{MixtureSameFamily}.
\item
  \emph{sample} (bool, optional): A control-flow argument. If
  \texttt{True}, the function will directly sample a raw JAX array from
  the distribution, bypassing the BI model context. If \texttt{False},
  it will create a \texttt{BI.sample} site within a model. Defaults to
  \texttt{False}.
\item
  \emph{seed} (int, optional): An integer used to generate a JAX PRNGKey
  for reproducible sampling when \texttt{sample=True}. {[}7{]} This
  argument has no effect when \texttt{sample=False}, as randomness is
  handled by BI's inference engine. Defaults to 0.
\item
  \emph{obs} (jnp.ndarray, optional): The observed value for this random
  variable. If provided, the sample site is conditioned on this value,
  and the function returns the observed value. If \texttt{None}, the
  site is treated as a latent (unobserved) random variable. Defaults to
  \texttt{None}.
\item
  \emph{name} (str, optional): The name of the sample site in a BI
  model. This is used to uniquely identify the random variable. Defaults
  to `x'.
\end{itemize}

\paragraph{Returns:}\label{returns-78}

BI Unit distribution object: When \texttt{sample=False} (for model
building). jnp.ndarray: A JAX array of samples drawn from the Unit
distribution (for direct sampling). BI Unit distribution object: When
\texttt{create\_obj=True} (for advanced use cases).

\paragraph{Example Usage:}\label{example-usage-76}

\begin{Shaded}
\begin{Highlighting}[]
\ImportTok{from}\NormalTok{ BI }\ImportTok{import}\NormalTok{ bi}
\NormalTok{m }\OperatorTok{=}\NormalTok{ bi(}\StringTok{\textquotesingle{}cpu\textquotesingle{}}\NormalTok{)}
\NormalTok{m.dist.unit(log\_factor}\OperatorTok{=}\NormalTok{jnp.ones(}\DecValTok{5}\NormalTok{), sample}\OperatorTok{=}\VariableTok{True}\NormalTok{)}
\end{Highlighting}
\end{Shaded}

\paragraph{Wrapper of:}\label{wrapper-of-53}

https://num.pyro.ai/en/stable/distributions.html\#unit

\begin{center}\rule{0.5\linewidth}{0.5pt}\end{center}

\subsubsection{Weibull}\label{weibull}

Samples from a Weibull distribution.

The Weibull distribution is a versatile distribution often used to model
failure rates in engineering and reliability studies. It is
characterized by its shape and scale parameters.

\[
f(x) = \frac{\beta}{\alpha} \left(\frac{x}{\alpha}\right)^{\beta - 1} e^{-\left(\frac{x}{\alpha}\right)^{\beta}} \text{ for } x \ge 0
\]

where \(\alpha\) is the scale parameter and \(\beta\) is the shape
parameter.

\paragraph{Args:}\label{args-80}

\begin{Shaded}
\begin{Highlighting}[]
\NormalTok{bi.dist.weibull(}
\NormalTok{scale,}
\NormalTok{concentration,}
\NormalTok{validate\_args}\OperatorTok{=}\VariableTok{None}\NormalTok{,}
\NormalTok{name}\OperatorTok{=}\StringTok{\textquotesingle{}x\textquotesingle{}}\NormalTok{,}
\NormalTok{obs}\OperatorTok{=}\VariableTok{None}\NormalTok{,}
\NormalTok{mask}\OperatorTok{=}\VariableTok{None}\NormalTok{,}
\NormalTok{sample}\OperatorTok{=}\VariableTok{False}\NormalTok{,}
\NormalTok{seed}\OperatorTok{=}\DecValTok{0}\NormalTok{,}
\NormalTok{shape}\OperatorTok{=}\NormalTok{(),}
\NormalTok{event}\OperatorTok{=}\DecValTok{0}\NormalTok{,}
\NormalTok{create\_obj}\OperatorTok{=}\VariableTok{False}\NormalTok{,}
\NormalTok{)}
\end{Highlighting}
\end{Shaded}

\begin{itemize}
\item
  \emph{sample} (jnp.ndarray): The scale parameter of the Weibull
  distribution. Must be positive. concentration (jnp.ndarray): The shape
  parameter of the Weibull distribution. Must be positive.
\item
  \emph{shape} (tuple): A multi-purpose argument for shaping. When
  \texttt{sample=False} (model building), this is used with
  \texttt{.expand(shape)} to set the distribution's batch shape. When
  \texttt{sample=True} (direct sampling), this is used as
  \texttt{sample\_shape} to draw a raw JAX array of the given shape.
\item
  \emph{event} (int): The number of batch dimensions to reinterpret as
  event dimensions (used in model building).
\item
  \emph{mask} (jnp.ndarray, bool): Optional boolean array to mask
  observations.
\item
  \emph{create\_obj} (bool): If True, returns the raw BI distribution
  object instead of creating a sample site. This is essential for
  building complex distributions like \texttt{MixtureSameFamily}.
\item
  \emph{sample} (bool, optional): A control-flow argument. If
  \texttt{True}, the function will directly sample a raw JAX array from
  the distribution, bypassing the BI model context. If \texttt{False},
  it will create a \texttt{BI.sample} site within a model. Defaults to
  \texttt{False}.
\item
  \emph{seed} (int, optional): An integer used to generate a JAX PRNGKey
  for reproducible sampling when \texttt{sample=True}. {[}7{]} This
  argument has no effect when \texttt{sample=False}, as randomness is
  handled by BI's inference engine. Defaults to 0.
\item
  \emph{obs} (jnp.ndarray, optional): The observed value for this random
  variable. If provided, the sample site is conditioned on this value,
  and the function returns the observed value. If \texttt{None}, the
  site is treated as a latent (unobserved) random variable. Defaults to
  \texttt{None}.
\item
  \emph{name} (str, optional): The name of the sample site in a BI
  model. This is used to uniquely identify the random variable. Defaults
  to `x'.
\end{itemize}

\paragraph{Returns:}\label{returns-79}

BI Weibull distribution object (for model building) when
\texttt{sample=False}. JAX array of samples drawn from the Weibull
distribution (for direct sampling) when \texttt{sample=True}. The raw BI
distribution object (for advanced use cases) when
\texttt{create\_obj=True}.

\paragraph{Example Usage:}\label{example-usage-77}

\begin{Shaded}
\begin{Highlighting}[]
\ImportTok{from}\NormalTok{ BI }\ImportTok{import}\NormalTok{ bi}
\NormalTok{m }\OperatorTok{=}\NormalTok{ bi(}\StringTok{\textquotesingle{}cpu\textquotesingle{}}\NormalTok{)}
\NormalTok{m.dist.weibull(scale}\OperatorTok{=}\FloatTok{1.0}\NormalTok{, concentration}\OperatorTok{=}\FloatTok{2.0}\NormalTok{, sample}\OperatorTok{=}\VariableTok{True}\NormalTok{)}
\end{Highlighting}
\end{Shaded}

\paragraph{Wrapper of:}\label{wrapper-of-54}

https://num.pyro.ai/en/stable/distributions.html\#weibull

\begin{center}\rule{0.5\linewidth}{0.5pt}\end{center}

\subsubsection{Wishart}\label{wishart}

The Wishart distribution is a multivariate distribution used to model
positive definite matrices, often representing covariance matrices. It's
commonly used in Bayesian statistics and machine learning, particularly
in models involving covariance estimation.

\[   p(X) = \frac{1}{W^{p/2} \Gamma_p(concentration/2)} \left|X\right|^{-concentration/2} \exp\left(-\frac{1}{2} \text{tr}(X^{-1} X)\right)
\]

\paragraph{Args:}\label{args-81}

\begin{Shaded}
\begin{Highlighting}[]
\NormalTok{bi.dist.wishart(}
\NormalTok{concentration,}
\NormalTok{scale\_matrix}\OperatorTok{=}\VariableTok{None}\NormalTok{,}
\NormalTok{rate\_matrix}\OperatorTok{=}\VariableTok{None}\NormalTok{,}
\NormalTok{scale\_tril}\OperatorTok{=}\VariableTok{None}\NormalTok{,}
\NormalTok{validate\_args}\OperatorTok{=}\VariableTok{None}\NormalTok{,}
\NormalTok{name}\OperatorTok{=}\StringTok{\textquotesingle{}x\textquotesingle{}}\NormalTok{,}
\NormalTok{obs}\OperatorTok{=}\VariableTok{None}\NormalTok{,}
\NormalTok{mask}\OperatorTok{=}\VariableTok{None}\NormalTok{,}
\NormalTok{sample}\OperatorTok{=}\VariableTok{False}\NormalTok{,}
\NormalTok{seed}\OperatorTok{=}\DecValTok{0}\NormalTok{,}
\NormalTok{shape}\OperatorTok{=}\NormalTok{(),}
\NormalTok{event}\OperatorTok{=}\DecValTok{0}\NormalTok{,}
\NormalTok{create\_obj}\OperatorTok{=}\VariableTok{False}\NormalTok{,}
\NormalTok{)}
\end{Highlighting}
\end{Shaded}

concentration (jnp.ndarray): Positive concentration parameter analogous
to the concentration of a :class:\texttt{Gamma} distribution. The
concentration must be larger than the dimensionality of the scale
matrix.

scale\_matrix (jnp.ndarray, optional): Scale matrix analogous to the
inverse rate of a :class:\texttt{Gamma} distribution.

rate\_matrix (jnp.ndarray, optional): Rate matrix anaologous to the rate
of a :class:\texttt{Gamma} distribution.

scale\_tril (jnp.ndarray, optional): Cholesky decomposition of the
:code:\texttt{scale\_matrix}.

\begin{itemize}
\item
  \emph{shape} (tuple): A multi-purpose argument for shaping. When
  \texttt{sample=False} (model building), this is used with
  \texttt{.expand(shape)} to set the distribution's batch shape. When
  \texttt{sample=True} (direct sampling), this is used as
  \texttt{sample\_shape} to draw a raw JAX array of the given shape.
\item
  \emph{event} (int): The number of batch dimensions to reinterpret as
  event dimensions (used in model building).
\item
  \emph{mask} (jnp.ndarray, bool, optional): Optional boolean array to
  mask observations.
\item
  \emph{create\_obj} (bool, optional): If True, returns the raw BI
  distribution object instead of creating a sample site. This is
  essential for building complex distributions like
  \texttt{MixtureSameFamily}.
\item
  \emph{sample} (bool, optional): A control-flow argument. If
  \texttt{True}, the function will directly sample a raw JAX array from
  the distribution, bypassing the BI model context. If \texttt{False},
  it will create a \texttt{BI.sample} site within a model. Defaults to
  \texttt{False}.
\item
  \emph{seed} (int, optional): An integer used to generate a JAX PRNGKey
  for reproducible sampling when \texttt{sample=True}. {[}7{]} This
  argument has no effect when \texttt{sample=False}, as randomness is
  handled by BI's inference engine. Defaults to 0.
\item
  \emph{obs} (jnp.ndarray, optional): The observed value for this random
  variable. If provided, the sample site is conditioned on this value,
  and the function returns the observed value. If \texttt{None}, the
  site is treated as a latent (unobserved) random variable. Defaults to
  \texttt{None}.
\item
  \emph{name} (str, optional): The name of the sample site in a BI
  model. This is used to uniquely identify the random variable. Defaults
  to `x'.
\end{itemize}

\paragraph{Returns:}\label{returns-80}

\begin{itemize}
\tightlist
\item
  When \texttt{sample=False}: A BI Wishart distribution object (for
  model building).
\item
  When \texttt{sample=True}: A JAX array of samples drawn from the
  Wishart distribution (for direct sampling).
\item
  When \texttt{create\_obj=True}: The raw BI distribution object (for
  advanced use cases).
\end{itemize}

\paragraph{Example Usage:}\label{example-usage-78}

\begin{Shaded}
\begin{Highlighting}[]
\ImportTok{from}\NormalTok{ BI }\ImportTok{import}\NormalTok{ bi}
\NormalTok{m }\OperatorTok{=}\NormalTok{ bi(}\StringTok{\textquotesingle{}cpu\textquotesingle{}}\NormalTok{)}
\NormalTok{m.dist.wishart(concentration}\OperatorTok{=}\FloatTok{5.0}\NormalTok{, scale\_matrix}\OperatorTok{=}\NormalTok{jnp.eye(}\DecValTok{2}\NormalTok{), sample}\OperatorTok{=}\VariableTok{True}\NormalTok{)}
\end{Highlighting}
\end{Shaded}

\paragraph{Wrapper of:}\label{wrapper-of-55}

https://num.pyro.ai/en/stable/distributions.html\#wishart

\begin{center}\rule{0.5\linewidth}{0.5pt}\end{center}

\subsubsection{Wishart Cholesky}\label{wishart-cholesky}

The Wishart distribution is a multivariate distribution used as a prior
distribution for covariance matrices. This implementation represents the
distribution in terms of its Cholesky decomposition.

.. rubric:: Probability Density Function

The probability density function (PDF) is given by:

PDF = (1 / ((2 * pi)\^{}(k * (k - 1) / 2) * Gamma(k/2)) \emph{
(concentration\^{}(k/2) } det(scale\_matrix))\^{}(-1/2) \emph{ exp(-1/2
} trace(rate\_matrix @ scale\_matrix)))

where:

\begin{itemize}
\tightlist
\item
  k is the dimensionality of the covariance matrix.
\item
  concentration is a positive concentration parameter.
\item
  scale\_matrix is the scale matrix.
\item
  rate\_matrix is the rate matrix.
\item
  Gamma is the gamma function.
\end{itemize}

.. rubric:: Parameters

\begin{itemize}
\item
  concentration: (Tensor) Positive concentration parameter analogous to
  the concentration of a :class:\texttt{Gamma} distribution. The
  concentration must be larger than the dimensionality of the scale
  matrix.
\item
  scale\_matrix: (Tensor, optional) Scale matrix analogous to the
  inverse rate of a :class:\texttt{Gamma} distribution. If not provided,
  \texttt{rate\_matrix} or \texttt{scale\_tril} must be.
\item
  rate\_matrix: (Tensor, optional) Rate matrix anaologous to the rate of
  a :class:\texttt{Gamma} distribution. If not provided,
  \texttt{scale\_matrix} or \texttt{scale\_tril} must be.
\item
  scale\_tril: (Tensor, optional) Cholesky decomposition of the
  :code:\texttt{scale\_matrix}. If not provided, \texttt{scale\_matrix}
  or \texttt{rate\_matrix} must be.
\item
  sample (bool, optional): A control-flow argument. If \texttt{True},
  the function will directly sample a raw JAX array from the
  distribution, bypassing the BI model context. If \texttt{False}, it
  will create a \texttt{BI.sample} site within a model. Defaults to
  \texttt{False}.
\item
  seed (int, optional): An integer used to generate a JAX PRNGKey for
  reproducible sampling when \texttt{sample=True}. {[}7{]} This argument
  has no effect when \texttt{sample=False}, as randomness is handled by
  BI's inference engine. Defaults to 0.
\item
  obs (jnp.ndarray, optional): The observed value for this random
  variable. If provided, the sample site is conditioned on this value,
  and the function returns the observed value. If \texttt{None}, the
  site is treated as a latent (unobserved) random variable. Defaults to
  \texttt{None}.
\item
  name (str, optional): The name of the sample site in a BI model. This
  is used to uniquely identify the random variable. Defaults to `x'.
\end{itemize}

\begin{Shaded}
\begin{Highlighting}[]
\NormalTok{bi.dist.wishart\_cholesky(}
\NormalTok{concentration,}
\NormalTok{scale\_matrix}\OperatorTok{=}\VariableTok{None}\NormalTok{,}
\NormalTok{rate\_matrix}\OperatorTok{=}\VariableTok{None}\NormalTok{,}
\NormalTok{scale\_tril}\OperatorTok{=}\VariableTok{None}\NormalTok{,}
\NormalTok{validate\_args}\OperatorTok{=}\VariableTok{None}\NormalTok{,}
\NormalTok{name}\OperatorTok{=}\StringTok{\textquotesingle{}x\textquotesingle{}}\NormalTok{,}
\NormalTok{obs}\OperatorTok{=}\VariableTok{None}\NormalTok{,}
\NormalTok{mask}\OperatorTok{=}\VariableTok{None}\NormalTok{,}
\NormalTok{sample}\OperatorTok{=}\VariableTok{False}\NormalTok{,}
\NormalTok{seed}\OperatorTok{=}\DecValTok{0}\NormalTok{,}
\NormalTok{shape}\OperatorTok{=}\NormalTok{(),}
\NormalTok{event}\OperatorTok{=}\DecValTok{0}\NormalTok{,}
\NormalTok{create\_obj}\OperatorTok{=}\VariableTok{False}\NormalTok{,}
\NormalTok{)}
\end{Highlighting}
\end{Shaded}

\begin{center}\rule{0.5\linewidth}{0.5pt}\end{center}

\subsubsection{Generic Zero Inflated}\label{generic-zero-inflated}

A Zero-Inflated distribution combines a base distribution with a
Bernoulli distribution to model data with an excess of zero values. It
assumes that each observation is either drawn from the base distribution
or is a zero with probability determined by the Bernoulli distribution
(the ``gate''). This is useful for modeling data where zeros are more
frequent than expected under a single distribution, often due to a
different underlying process.

\[
P(x) = \pi \cdot I(x=0) + (1 - \pi) \cdot P_{base}(x)
\]where: - \(P_{base}(x)\) is the probability density function (PDF) or
probability mass function (PMF) of the base distribution. - \(\pi\) is
the probability of generating a zero, governed by the Bernoulli gate. -
\(I(x=0)\) is an indicator function that equals 1 if x=0 and 0
otherwise.

\paragraph{Args:}\label{args-82}

\begin{Shaded}
\begin{Highlighting}[]
\NormalTok{bi.dist.zero\_inflated\_distribution(}
\NormalTok{base\_dist,}
\NormalTok{gate}\OperatorTok{=}\VariableTok{None}\NormalTok{,}
\NormalTok{gate\_logits}\OperatorTok{=}\VariableTok{None}\NormalTok{,}
\NormalTok{validate\_args}\OperatorTok{=}\VariableTok{None}\NormalTok{,}
\NormalTok{name}\OperatorTok{=}\StringTok{\textquotesingle{}x\textquotesingle{}}\NormalTok{,}
\NormalTok{obs}\OperatorTok{=}\VariableTok{None}\NormalTok{,}
\NormalTok{mask}\OperatorTok{=}\VariableTok{None}\NormalTok{,}
\NormalTok{sample}\OperatorTok{=}\VariableTok{False}\NormalTok{,}
\NormalTok{seed}\OperatorTok{=}\DecValTok{0}\NormalTok{,}
\NormalTok{shape}\OperatorTok{=}\NormalTok{(),}
\NormalTok{event}\OperatorTok{=}\DecValTok{0}\NormalTok{,}
\NormalTok{create\_obj}\OperatorTok{=}\VariableTok{False}\NormalTok{,}
\NormalTok{)}
\end{Highlighting}
\end{Shaded}

base\_dist (Distribution): The base distribution to be zero-inflated
(e.g., Poisson, NegativeBinomial).

gate (jnp.ndarray, optional): Probability of extra zeros (between 0 and
1).

gate\_logits (jnp.ndarray, optional): Log-odds of extra zeros.

validate\_args (bool, optional): Whether to validate parameter values.
Defaults to None.

\begin{itemize}
\item
  \emph{shape} (tuple): A multi-purpose argument for shaping. When
  \texttt{sample=False} (model building), this is used with
  \texttt{.expand(shape)} to set the distribution's batch shape. When
  \texttt{sample=True} (direct sampling), this is used as
  \texttt{sample\_shape} to draw a raw JAX array of the given shape.
\item
  \emph{event} (int): The number of batch dimensions to reinterpret as
  event dimensions (used in model building).
\item
  \emph{mask} (jnp.ndarray, bool): Optional boolean array to mask
  observations.
\item
  \emph{create\_obj} (bool): If True, returns the raw BI distribution
  object instead of creating a sample site. This is essential for
  building complex distributions like \texttt{MixtureSameFamily}.
\item
  \emph{sample} (bool, optional): A control-flow argument. If
  \texttt{True}, the function will directly sample a raw JAX array from
  the distribution, bypassing the BI model context. If \texttt{False},
  it will create a \texttt{BI.sample} site within a model. Defaults to
  \texttt{False}.
\item
  \emph{seed} (int, optional): An integer used to generate a JAX PRNGKey
  for reproducible sampling when \texttt{sample=True}. {[}7{]} This
  argument has no effect when \texttt{sample=False}, as randomness is
  handled by BI's inference engine. Defaults to 0.
\item
  \emph{obs} (jnp.ndarray, optional): The observed value for this random
  variable. If provided, the sample site is conditioned on this value,
  and the function returns the observed value. If \texttt{None}, the
  site is treated as a latent (unobserved) random variable. Defaults to
  \texttt{None}.
\item
  \emph{name} (str, optional): The name of the sample site in a BI
  model. This is used to uniquely identify the random variable. Defaults
  to `x'.
\end{itemize}

\paragraph{Returns:}\label{returns-81}

\begin{itemize}
\tightlist
\item
  When \texttt{sample=False}: A BI ZeroInflatedDistribution distribution
  object (for model building).
\item
  When \texttt{sample=True}: A JAX array of samples drawn from the
  ZeroInflatedDistribution distribution (for direct sampling).
\item
  When \texttt{create\_obj=True}: The raw BI distribution object (for
  advanced use cases).
\end{itemize}

\paragraph{Example Usage:}\label{example-usage-79}

\begin{Shaded}
\begin{Highlighting}[]
\ImportTok{from}\NormalTok{ BI }\ImportTok{import}\NormalTok{ bi}
\NormalTok{m }\OperatorTok{=}\NormalTok{ bi(}\StringTok{\textquotesingle{}cpu\textquotesingle{}}\NormalTok{)}
\NormalTok{m.dist.zero\_inflated\_distribution(base\_dist}\OperatorTok{=}\NormalTok{m.dist.poisson(rate}\OperatorTok{=}\DecValTok{5}\NormalTok{, create\_obj }\OperatorTok{=} \VariableTok{True}\NormalTok{), gate }\OperatorTok{=} \FloatTok{0.3}\NormalTok{, sample}\OperatorTok{=}\VariableTok{True}\NormalTok{)}
\end{Highlighting}
\end{Shaded}

\paragraph{Wrapper of:
https://num.pyro.ai/en/stable/distributions.html\#zeroinflateddistribution}\label{wrapper-of-httpsnum.pyro.aienstabledistributions.htmlzeroinflateddistribution}

\begin{center}\rule{0.5\linewidth}{0.5pt}\end{center}

\subsubsection{Zero-Inflated Negative
Binomial}\label{zero-inflated-negative-binomial}

This distribution combines a Negative Binomial distribution with a
binary gate variable. Observations are either drawn from the Negative
Binomial distribution with probability (1 - gate) or are treated as zero
with probability `gate'. This models data with excess zeros compared to
what a standard Negative Binomial distribution would predict.

\[
P(X = x) = (1 - gate) \cdot \frac{\Gamma(x + \alpha)}{\Gamma(x + \alpha + \beta) \Gamma(\alpha)} \left(\frac{\beta}{\alpha + \beta}\right)^x + gate \cdot \delta_{x, 0}
\]

\paragraph{Args:}\label{args-83}

\begin{Shaded}
\begin{Highlighting}[]
\NormalTok{bi.dist.zero\_inflated\_negative\_binomial2(}
\NormalTok{mean,}
\NormalTok{concentration,}
\NormalTok{gate}\OperatorTok{=}\VariableTok{None}\NormalTok{,}
\NormalTok{gate\_logits}\OperatorTok{=}\VariableTok{None}\NormalTok{,}
\NormalTok{validate\_args}\OperatorTok{=}\VariableTok{None}\NormalTok{,}
\NormalTok{name}\OperatorTok{=}\StringTok{\textquotesingle{}x\textquotesingle{}}\NormalTok{,}
\NormalTok{obs}\OperatorTok{=}\VariableTok{None}\NormalTok{,}
\NormalTok{mask}\OperatorTok{=}\VariableTok{None}\NormalTok{,}
\NormalTok{sample}\OperatorTok{=}\VariableTok{False}\NormalTok{,}
\NormalTok{seed}\OperatorTok{=}\DecValTok{0}\NormalTok{,}
\NormalTok{shape}\OperatorTok{=}\NormalTok{(),}
\NormalTok{event}\OperatorTok{=}\DecValTok{0}\NormalTok{,}
\NormalTok{create\_obj}\OperatorTok{=}\VariableTok{False}\NormalTok{,}
\NormalTok{)}
\end{Highlighting}
\end{Shaded}

mean (jnp.ndarray or float): The mean of the Negative Binomial 2
distribution. concentration (jnp.ndarray or float): The concentration
parameter of the Negative Binomial 2 distribution.

\begin{itemize}
\item
  \emph{shape} (tuple): A multi-purpose argument for shaping. When
  \texttt{sample=False} (model building), this is used with
  \texttt{.expand(shape)} to set the distribution's batch shape. When
  \texttt{sample=True} (direct sampling), this is used as
  \texttt{sample\_shape} to draw a raw JAX array of the given shape.
\item
  \emph{event} (int): The number of batch dimensions to reinterpret as
  event dimensions (used in model building).
\item
  \emph{mask} (jnp.ndarray, bool): Optional boolean array to mask
  observations.
\item
  \emph{create\_obj} (bool): If True, returns the raw BI distribution
  object instead of creating a sample site. This is essential for
  building complex distributions like \texttt{MixtureSameFamily}.
\item
  \emph{sample} (bool, optional): A control-flow argument. If
  \texttt{True}, the function will directly sample a raw JAX array from
  the distribution, bypassing the BI model context. If \texttt{False},
  it will create a \texttt{BI.sample} site within a model. Defaults to
  \texttt{False}.
\item
  \emph{seed} (int, optional): An integer used to generate a JAX PRNGKey
  for reproducible sampling when \texttt{sample=True}. {[}7{]} This
  argument has no effect when \texttt{sample=False}, as randomness is
  handled by BI's inference engine. Defaults to 0.
\item
  \emph{obs} (jnp.ndarray, optional): The observed value for this random
  variable. If provided, the sample site is conditioned on this value,
  and the function returns the observed value. If \texttt{None}, the
  site is treated as a latent (unobserved) random variable. Defaults to
  \texttt{None}.
\item
  \emph{name} (str, optional): The name of the sample site in a BI
  model. This is used to uniquely identify the random variable. Defaults
  to `x'.
\end{itemize}

\paragraph{Returns:}\label{returns-82}

\begin{itemize}
\tightlist
\item
  When \texttt{sample=False}: A BI ZeroInflatedNegativeBinomial2
  distribution object (for model building).
\item
  When \texttt{sample=True}: A JAX array of samples drawn from the
  ZeroInflatedNegativeBinomial2 distribution (for direct sampling).
\item
  When \texttt{create\_obj=True}: The raw BI distribution object (for
  advanced use cases).
\end{itemize}

\paragraph{Example Usage:}\label{example-usage-80}

\begin{Shaded}
\begin{Highlighting}[]
\ImportTok{from}\NormalTok{ BI }\ImportTok{import}\NormalTok{ bi}
\NormalTok{m }\OperatorTok{=}\NormalTok{ bi(}\StringTok{\textquotesingle{}cpu\textquotesingle{}}\NormalTok{)}
\NormalTok{m.dist.zero\_inflated\_negative\_binomial2(mean}\OperatorTok{=}\FloatTok{2.0}\NormalTok{, concentration}\OperatorTok{=}\FloatTok{3.0}\NormalTok{, gate }\OperatorTok{=} \FloatTok{0.3}\NormalTok{, sample}\OperatorTok{=}\VariableTok{True}\NormalTok{)}
\end{Highlighting}
\end{Shaded}

\paragraph{Wrapper of:
https://num.pyro.ai/en/stable/distributions.html\#zeroinflatednegativebinomial2}\label{wrapper-of-httpsnum.pyro.aienstabledistributions.htmlzeroinflatednegativebinomial2}

\begin{center}\rule{0.5\linewidth}{0.5pt}\end{center}

\subsubsection{A Zero Inflated Poisson}\label{a-zero-inflated-poisson}

This distribution combines two Poisson processes: one with a rate
parameter and another that generates only zeros. The probability of
observing a zero is determined by the `gate' parameter, while the
probability of observing a non-zero value is governed by the `rate'
parameter of the underlying Poisson distribution.

\[
P(X = k) = (1 - gate) * \frac{e^{-rate} rate^k}{k!} + gate
\]

\paragraph{Args:}\label{args-84}

\begin{Shaded}
\begin{Highlighting}[]
\NormalTok{bi.dist.zero\_inflated\_poisson(}
\NormalTok{gate,}
\NormalTok{rate}\OperatorTok{=}\FloatTok{1.0}\NormalTok{,}
\NormalTok{validate\_args}\OperatorTok{=}\VariableTok{None}\NormalTok{,}
\NormalTok{name}\OperatorTok{=}\StringTok{\textquotesingle{}x\textquotesingle{}}\NormalTok{,}
\NormalTok{obs}\OperatorTok{=}\VariableTok{None}\NormalTok{,}
\NormalTok{mask}\OperatorTok{=}\VariableTok{None}\NormalTok{,}
\NormalTok{sample}\OperatorTok{=}\VariableTok{False}\NormalTok{,}
\NormalTok{seed}\OperatorTok{=}\DecValTok{0}\NormalTok{,}
\NormalTok{shape}\OperatorTok{=}\NormalTok{(),}
\NormalTok{event}\OperatorTok{=}\DecValTok{0}\NormalTok{,}
\NormalTok{create\_obj}\OperatorTok{=}\VariableTok{False}\NormalTok{,}
\NormalTok{)}
\end{Highlighting}
\end{Shaded}

rate (jnp.ndarray): The rate parameter of the underlying Poisson
distribution.

\begin{itemize}
\item
  \emph{shape} (tuple): A multi-purpose argument for shaping. When
  \texttt{sample=False} (model building), this is used with
  \texttt{.expand(shape)} to set the distribution's batch shape. When
  \texttt{sample=True} (direct sampling), this is used as
  \texttt{sample\_shape} to draw a raw JAX array of the given shape.
\item
  \emph{event} (int): The number of batch dimensions to reinterpret as
  event dimensions (used in model building).
\item
  \emph{mask} (jnp.ndarray, bool): Optional boolean array to mask
  observations.
\item
  \emph{create\_obj} (bool): If True, returns the raw BI distribution
  object instead of creating a sample site. This is essential for
  building complex distributions like \texttt{MixtureSameFamily}.
\item
  \emph{sample} (bool, optional): A control-flow argument. If
  \texttt{True}, the function will directly sample a raw JAX array from
  the distribution, bypassing the BI model context. If \texttt{False},
  it will create a \texttt{BI.sample} site within a model. Defaults to
  \texttt{False}.
\item
  \emph{seed} (int, optional): An integer used to generate a JAX PRNGKey
  for reproducible sampling when \texttt{sample=True}. {[}7{]} This
  argument has no effect when \texttt{sample=False}, as randomness is
  handled by BI's inference engine. Defaults to 0.
\item
  \emph{obs} (jnp.ndarray, optional): The observed value for this random
  variable. If provided, the sample site is conditioned on this value,
  and the function returns the observed value. If \texttt{None}, the
  site is treated as a latent (unobserved) random variable. Defaults to
  \texttt{None}.
\item
  \emph{name} (str, optional): The name of the sample site in a BI
  model. This is used to uniquely identify the random variable. Defaults
  to `x'.
\end{itemize}

\paragraph{Returns:}\label{returns-83}

BI ZeroInflatedPoisson distribution object (when \texttt{sample=False}).
JAX array of samples drawn from the ZeroInflatedPoisson distribution
(when \texttt{sample=True}). The raw BI distribution object (when
\texttt{create\_obj=True}).

\paragraph{Example Usage:}\label{example-usage-81}

\begin{Shaded}
\begin{Highlighting}[]
\ImportTok{from}\NormalTok{ BI }\ImportTok{import}\NormalTok{ bi}
\NormalTok{m }\OperatorTok{=}\NormalTok{ bi(}\StringTok{\textquotesingle{}cpu\textquotesingle{}}\NormalTok{)}
\NormalTok{m.dist.zero\_inflated\_poisson(gate }\OperatorTok{=} \FloatTok{0.3}\NormalTok{, rate}\OperatorTok{=}\FloatTok{2.0}\NormalTok{, sample}\OperatorTok{=}\VariableTok{True}\NormalTok{)}
\end{Highlighting}
\end{Shaded}

\paragraph{Wrapper of:
https://num.pyro.ai/en/stable/distributions.html\#zeroinflatedpoisson}\label{wrapper-of-httpsnum.pyro.aienstabledistributions.htmlzeroinflatedpoisson}

\begin{center}\rule{0.5\linewidth}{0.5pt}\end{center}

\subsubsection{Zero Sum Normal}\label{zero-sum-normal}

Samples from a ZeroSumNormal distribution, which is a Normal
distribution where one or more axes are constrained to sum to zero.

\[
ZSN(\sigma) = N(0, \sigma^2 (I - \tfrac{1}{n}J)) \\
\text{where} \ ~ J_{ij} = 1 \ ~ \text{and} \\
n = \text{number of zero-sum axes}
\]

\paragraph{Args:}\label{args-85}

\begin{Shaded}
\begin{Highlighting}[]
\NormalTok{bi.dist.zero\_sum\_normal(}
\NormalTok{scale,}
\NormalTok{event\_shape,}
\NormalTok{validate\_args}\OperatorTok{=}\VariableTok{None}\NormalTok{,}
\NormalTok{name}\OperatorTok{=}\StringTok{\textquotesingle{}x\textquotesingle{}}\NormalTok{,}
\NormalTok{obs}\OperatorTok{=}\VariableTok{None}\NormalTok{,}
\NormalTok{mask}\OperatorTok{=}\VariableTok{None}\NormalTok{,}
\NormalTok{sample}\OperatorTok{=}\VariableTok{False}\NormalTok{,}
\NormalTok{seed}\OperatorTok{=}\DecValTok{0}\NormalTok{,}
\NormalTok{shape}\OperatorTok{=}\NormalTok{(),}
\NormalTok{event}\OperatorTok{=}\DecValTok{0}\NormalTok{,}
\NormalTok{create\_obj}\OperatorTok{=}\VariableTok{False}\NormalTok{,}
\NormalTok{)}
\end{Highlighting}
\end{Shaded}

\begin{itemize}
\item
  \emph{sample} (array\_like): Standard deviation of the underlying
  normal distribution before the zerosum constraint is enforced.
\item
  \emph{shape} (tuple): A multi-purpose argument for shaping. When
  \texttt{sample=False} (model building), this is used with
  \texttt{.expand(shape)} to set the distribution's batch shape. When
  \texttt{sample=True} (direct sampling), this is used as
  \texttt{sample\_shape} to draw a raw JAX array of the given shape.
\item
  \emph{event} (int): The number of batch dimensions to reinterpret as
  event dimensions (used in model building).
\item
  \emph{mask} (jnp.ndarray, bool): Optional boolean array to mask
  observations.
\item
  \emph{create\_obj} (bool): If True, returns the raw BI distribution
  object instead of creating a sample site. This is essential for
  building complex distributions like \texttt{MixtureSameFamily}.
\item
  \emph{sample} (bool, optional): A control-flow argument. If
  \texttt{True}, the function will directly sample a raw JAX array from
  the distribution, bypassing the BI model context. If \texttt{False},
  it will create a \texttt{BI.sample} site within a model. Defaults to
  \texttt{False}.
\item
  \emph{seed} (int, optional): An integer used to generate a JAX PRNGKey
  for reproducible sampling when \texttt{sample=True}. {[}7{]} This
  argument has no effect when \texttt{sample=False}, as randomness is
  handled by BI's inference engine. Defaults to 0.
\item
  \emph{obs} (jnp.ndarray, optional): The observed value for this random
  variable. If provided, the sample site is conditioned on this value,
  and the function returns the observed value. If \texttt{None}, the
  site is treated as a latent (unobserved) random variable. Defaults to
  \texttt{None}.
\item
  \emph{name} (str, optional): The name of the sample site in a BI
  model. This is used to uniquely identify the random variable. Defaults
  to `x'.
\end{itemize}

\paragraph{Returns:}\label{returns-84}

\begin{itemize}
\item
  When \texttt{sample=False}: A BI ZeroSumNormal distribution object
  (for model building).
\item
  When \texttt{sample=True}: A JAX array of samples drawn from the
  ZeroSumNormal distribution (for direct sampling).
\item
  When \texttt{create\_obj=True}: The raw BI distribution object (for
  advanced use cases).
\end{itemize}

\paragraph{Example Usage:}\label{example-usage-82}

\begin{Shaded}
\begin{Highlighting}[]
\ImportTok{from}\NormalTok{ BI }\ImportTok{import}\NormalTok{ bi}
\NormalTok{m }\OperatorTok{=}\NormalTok{ bi(}\StringTok{\textquotesingle{}cpu\textquotesingle{}}\NormalTok{)}
\NormalTok{m.dist.zero\_sum\_normal(scale}\OperatorTok{=}\FloatTok{1.0}\NormalTok{, event\_shape }\OperatorTok{=}\NormalTok{ (}\DecValTok{2}\NormalTok{,), sample }\OperatorTok{=} \VariableTok{True}\NormalTok{)}
\end{Highlighting}
\end{Shaded}

\paragraph{Wrapper of:}\label{wrapper-of-56}

https://num.pyro.ai/en/stable/distributions.html\#zerosumnormal

\begin{center}\rule{0.5\linewidth}{0.5pt}\end{center}




\end{document}
