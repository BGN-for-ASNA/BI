% Options for packages loaded elsewhere
% Options for packages loaded elsewhere
\PassOptionsToPackage{unicode}{hyperref}
\PassOptionsToPackage{hyphens}{url}
\PassOptionsToPackage{dvipsnames,svgnames,x11names}{xcolor}
%
\documentclass[
  letterpaper,
  DIV=11,
  numbers=noendperiod]{scrartcl}
\usepackage{xcolor}
\usepackage{amsmath,amssymb}
\setcounter{secnumdepth}{-\maxdimen} % remove section numbering
\usepackage{iftex}
\ifPDFTeX
  \usepackage[T1]{fontenc}
  \usepackage[utf8]{inputenc}
  \usepackage{textcomp} % provide euro and other symbols
\else % if luatex or xetex
  \usepackage{unicode-math} % this also loads fontspec
  \defaultfontfeatures{Scale=MatchLowercase}
  \defaultfontfeatures[\rmfamily]{Ligatures=TeX,Scale=1}
\fi
\usepackage{lmodern}
\ifPDFTeX\else
  % xetex/luatex font selection
\fi
% Use upquote if available, for straight quotes in verbatim environments
\IfFileExists{upquote.sty}{\usepackage{upquote}}{}
\IfFileExists{microtype.sty}{% use microtype if available
  \usepackage[]{microtype}
  \UseMicrotypeSet[protrusion]{basicmath} % disable protrusion for tt fonts
}{}
\makeatletter
\@ifundefined{KOMAClassName}{% if non-KOMA class
  \IfFileExists{parskip.sty}{%
    \usepackage{parskip}
  }{% else
    \setlength{\parindent}{0pt}
    \setlength{\parskip}{6pt plus 2pt minus 1pt}}
}{% if KOMA class
  \KOMAoptions{parskip=half}}
\makeatother
% Make \paragraph and \subparagraph free-standing
\makeatletter
\ifx\paragraph\undefined\else
  \let\oldparagraph\paragraph
  \renewcommand{\paragraph}{
    \@ifstar
      \xxxParagraphStar
      \xxxParagraphNoStar
  }
  \newcommand{\xxxParagraphStar}[1]{\oldparagraph*{#1}\mbox{}}
  \newcommand{\xxxParagraphNoStar}[1]{\oldparagraph{#1}\mbox{}}
\fi
\ifx\subparagraph\undefined\else
  \let\oldsubparagraph\subparagraph
  \renewcommand{\subparagraph}{
    \@ifstar
      \xxxSubParagraphStar
      \xxxSubParagraphNoStar
  }
  \newcommand{\xxxSubParagraphStar}[1]{\oldsubparagraph*{#1}\mbox{}}
  \newcommand{\xxxSubParagraphNoStar}[1]{\oldsubparagraph{#1}\mbox{}}
\fi
\makeatother

\usepackage{color}
\usepackage{fancyvrb}
\newcommand{\VerbBar}{|}
\newcommand{\VERB}{\Verb[commandchars=\\\{\}]}
\DefineVerbatimEnvironment{Highlighting}{Verbatim}{commandchars=\\\{\}}
% Add ',fontsize=\small' for more characters per line
\usepackage{framed}
\definecolor{shadecolor}{RGB}{241,243,245}
\newenvironment{Shaded}{\begin{snugshade}}{\end{snugshade}}
\newcommand{\AlertTok}[1]{\textcolor[rgb]{0.68,0.00,0.00}{#1}}
\newcommand{\AnnotationTok}[1]{\textcolor[rgb]{0.37,0.37,0.37}{#1}}
\newcommand{\AttributeTok}[1]{\textcolor[rgb]{0.40,0.45,0.13}{#1}}
\newcommand{\BaseNTok}[1]{\textcolor[rgb]{0.68,0.00,0.00}{#1}}
\newcommand{\BuiltInTok}[1]{\textcolor[rgb]{0.00,0.23,0.31}{#1}}
\newcommand{\CharTok}[1]{\textcolor[rgb]{0.13,0.47,0.30}{#1}}
\newcommand{\CommentTok}[1]{\textcolor[rgb]{0.37,0.37,0.37}{#1}}
\newcommand{\CommentVarTok}[1]{\textcolor[rgb]{0.37,0.37,0.37}{\textit{#1}}}
\newcommand{\ConstantTok}[1]{\textcolor[rgb]{0.56,0.35,0.01}{#1}}
\newcommand{\ControlFlowTok}[1]{\textcolor[rgb]{0.00,0.23,0.31}{\textbf{#1}}}
\newcommand{\DataTypeTok}[1]{\textcolor[rgb]{0.68,0.00,0.00}{#1}}
\newcommand{\DecValTok}[1]{\textcolor[rgb]{0.68,0.00,0.00}{#1}}
\newcommand{\DocumentationTok}[1]{\textcolor[rgb]{0.37,0.37,0.37}{\textit{#1}}}
\newcommand{\ErrorTok}[1]{\textcolor[rgb]{0.68,0.00,0.00}{#1}}
\newcommand{\ExtensionTok}[1]{\textcolor[rgb]{0.00,0.23,0.31}{#1}}
\newcommand{\FloatTok}[1]{\textcolor[rgb]{0.68,0.00,0.00}{#1}}
\newcommand{\FunctionTok}[1]{\textcolor[rgb]{0.28,0.35,0.67}{#1}}
\newcommand{\ImportTok}[1]{\textcolor[rgb]{0.00,0.46,0.62}{#1}}
\newcommand{\InformationTok}[1]{\textcolor[rgb]{0.37,0.37,0.37}{#1}}
\newcommand{\KeywordTok}[1]{\textcolor[rgb]{0.00,0.23,0.31}{\textbf{#1}}}
\newcommand{\NormalTok}[1]{\textcolor[rgb]{0.00,0.23,0.31}{#1}}
\newcommand{\OperatorTok}[1]{\textcolor[rgb]{0.37,0.37,0.37}{#1}}
\newcommand{\OtherTok}[1]{\textcolor[rgb]{0.00,0.23,0.31}{#1}}
\newcommand{\PreprocessorTok}[1]{\textcolor[rgb]{0.68,0.00,0.00}{#1}}
\newcommand{\RegionMarkerTok}[1]{\textcolor[rgb]{0.00,0.23,0.31}{#1}}
\newcommand{\SpecialCharTok}[1]{\textcolor[rgb]{0.37,0.37,0.37}{#1}}
\newcommand{\SpecialStringTok}[1]{\textcolor[rgb]{0.13,0.47,0.30}{#1}}
\newcommand{\StringTok}[1]{\textcolor[rgb]{0.13,0.47,0.30}{#1}}
\newcommand{\VariableTok}[1]{\textcolor[rgb]{0.07,0.07,0.07}{#1}}
\newcommand{\VerbatimStringTok}[1]{\textcolor[rgb]{0.13,0.47,0.30}{#1}}
\newcommand{\WarningTok}[1]{\textcolor[rgb]{0.37,0.37,0.37}{\textit{#1}}}

\usepackage{longtable,booktabs,array}
\usepackage{calc} % for calculating minipage widths
% Correct order of tables after \paragraph or \subparagraph
\usepackage{etoolbox}
\makeatletter
\patchcmd\longtable{\par}{\if@noskipsec\mbox{}\fi\par}{}{}
\makeatother
% Allow footnotes in longtable head/foot
\IfFileExists{footnotehyper.sty}{\usepackage{footnotehyper}}{\usepackage{footnote}}
\makesavenoteenv{longtable}
\usepackage{graphicx}
\makeatletter
\newsavebox\pandoc@box
\newcommand*\pandocbounded[1]{% scales image to fit in text height/width
  \sbox\pandoc@box{#1}%
  \Gscale@div\@tempa{\textheight}{\dimexpr\ht\pandoc@box+\dp\pandoc@box\relax}%
  \Gscale@div\@tempb{\linewidth}{\wd\pandoc@box}%
  \ifdim\@tempb\p@<\@tempa\p@\let\@tempa\@tempb\fi% select the smaller of both
  \ifdim\@tempa\p@<\p@\scalebox{\@tempa}{\usebox\pandoc@box}%
  \else\usebox{\pandoc@box}%
  \fi%
}
% Set default figure placement to htbp
\def\fps@figure{htbp}
\makeatother





\setlength{\emergencystretch}{3em} % prevent overfull lines

\providecommand{\tightlist}{%
  \setlength{\itemsep}{0pt}\setlength{\parskip}{0pt}}



 


\KOMAoption{captions}{tableheading}
\usepackage{amsmath}
\usepackage{amssymb}
\makeatletter
\@ifpackageloaded{tcolorbox}{}{\usepackage[skins,breakable]{tcolorbox}}
\@ifpackageloaded{fontawesome5}{}{\usepackage{fontawesome5}}
\definecolor{quarto-callout-color}{HTML}{909090}
\definecolor{quarto-callout-note-color}{HTML}{0758E5}
\definecolor{quarto-callout-important-color}{HTML}{CC1914}
\definecolor{quarto-callout-warning-color}{HTML}{EB9113}
\definecolor{quarto-callout-tip-color}{HTML}{00A047}
\definecolor{quarto-callout-caution-color}{HTML}{FC5300}
\definecolor{quarto-callout-color-frame}{HTML}{acacac}
\definecolor{quarto-callout-note-color-frame}{HTML}{4582ec}
\definecolor{quarto-callout-important-color-frame}{HTML}{d9534f}
\definecolor{quarto-callout-warning-color-frame}{HTML}{f0ad4e}
\definecolor{quarto-callout-tip-color-frame}{HTML}{02b875}
\definecolor{quarto-callout-caution-color-frame}{HTML}{fd7e14}
\makeatother
\makeatletter
\@ifpackageloaded{caption}{}{\usepackage{caption}}
\AtBeginDocument{%
\ifdefined\contentsname
  \renewcommand*\contentsname{Table of contents}
\else
  \newcommand\contentsname{Table of contents}
\fi
\ifdefined\listfigurename
  \renewcommand*\listfigurename{List of Figures}
\else
  \newcommand\listfigurename{List of Figures}
\fi
\ifdefined\listtablename
  \renewcommand*\listtablename{List of Tables}
\else
  \newcommand\listtablename{List of Tables}
\fi
\ifdefined\figurename
  \renewcommand*\figurename{Figure}
\else
  \newcommand\figurename{Figure}
\fi
\ifdefined\tablename
  \renewcommand*\tablename{Table}
\else
  \newcommand\tablename{Table}
\fi
}
\@ifpackageloaded{float}{}{\usepackage{float}}
\floatstyle{ruled}
\@ifundefined{c@chapter}{\newfloat{codelisting}{h}{lop}}{\newfloat{codelisting}{h}{lop}[chapter]}
\floatname{codelisting}{Listing}
\newcommand*\listoflistings{\listof{codelisting}{List of Listings}}
\makeatother
\makeatletter
\makeatother
\makeatletter
\@ifpackageloaded{caption}{}{\usepackage{caption}}
\@ifpackageloaded{subcaption}{}{\usepackage{subcaption}}
\makeatother
\usepackage{bookmark}
\IfFileExists{xurl.sty}{\usepackage{xurl}}{} % add URL line breaks if available
\urlstyle{same}
\hypersetup{
  pdftitle={Gaussian Mixture Models},
  colorlinks=true,
  linkcolor={blue},
  filecolor={Maroon},
  citecolor={Blue},
  urlcolor={Blue},
  pdfcreator={LaTeX via pandoc}}


\title{Gaussian Mixture Models}
\author{}
\date{}
\begin{document}
\maketitle


\subsection{General Principles}\label{general-principles}

To discover group structures or clusters in data, we can use a
\textbf{Gaussian Mixture Model (GMM)}. This is a {parametric 🛈}
clustering method. A GMM assumes that the data is generated from a
mixture of a \textbf{pre-specified number (\texttt{K})} of different
Gaussian distributions. The model's goal is to figure out:

\begin{enumerate}
\def\labelenumi{\arabic{enumi}.}
\tightlist
\item
  \textbf{The properties of each of the \texttt{K} clusters}: For each
  of the \texttt{K} clusters, it estimates its center (mean \(\mu\)) and
  its shape/spread (covariance \(\Sigma\)).
\item
  \textbf{The mixture weights}: It estimates the proportion of the data
  that belongs to each cluster.
\item
  \textbf{The assignment of each data point}: It determines the
  probability of each data point belonging to each of the \texttt{K}
  clusters.
\end{enumerate}

\subsection{Considerations}\label{considerations}

\begin{tcolorbox}[enhanced jigsaw, title=\textcolor{quarto-callout-caution-color}{\faFire}\hspace{0.5em}{Caution}, toprule=.15mm, breakable, bottomtitle=1mm, opacitybacktitle=0.6, leftrule=.75mm, colframe=quarto-callout-caution-color-frame, bottomrule=.15mm, coltitle=black, left=2mm, rightrule=.15mm, toptitle=1mm, titlerule=0mm, arc=.35mm, colbacktitle=quarto-callout-caution-color!10!white, opacityback=0, colback=white]

\begin{itemize}
\item
  A GMM is a Bayesian model 🛈 that considers uncertainty in all its
  parameters, \emph{except for the number of clusters, \texttt{K}},
  which must be fixed in advance.
\item
  The key parameters and their priors are:

  \begin{itemize}
  \tightlist
  \item
    \textbf{Number of Clusters \texttt{K}}: This is a \textbf{fixed
    hyperparameter} that you must choose before running the model.
    Choosing the right \texttt{K} often involves running the model
    multiple times and using model comparison criteria (like
    cross-validation, AIC, or BIC).
  \item
    \textbf{Cluster Weights \texttt{w}}: These are the probabilities of
    drawing a data point from any given cluster. Since there are a fixed
    number \texttt{K} of them and they must sum to 1, they are typically
    given a \texttt{Dirichlet} prior. A symmetric \texttt{Dirichlet}
    prior (e.g., \texttt{Dirichlet(1,\ 1,\ ...,\ 1)}) represents an
    initial belief that all clusters are equally likely.
  \item
    \textbf{Cluster Parameters (}\(\boldsymbol{\mu}\), \(\Sigma\)): Each
    of the \texttt{K} clusters has a mean \(\boldsymbol{\mu}\) and a
    covariance matrix \(\Sigma\). We place priors on these to define our
    beliefs about their plausible values.
  \end{itemize}
\item
  Like the DPMM, the model is often implemented in its {marginalized
  form 🛈}. Instead of explicitly assigning each data point to a cluster,
  we integrate out this choice. This creates a smoother probability
  surface for the inference algorithm to explore, leading to much more
  efficient computation.
\item
  To increase accuracy we run a k-means algorithm to initialize the
  cluster mean priors.
\end{itemize}

\end{tcolorbox}

\subsection{Example}\label{example}

Below is an example of a GMM implemented in BI. The goal is to cluster a
synthetic dataset into \textbf{a pre-specified \texttt{K=4} groups}.

\subsection{Python}

\begin{Shaded}
\begin{Highlighting}[]
\ImportTok{from}\NormalTok{ BI }\ImportTok{import}\NormalTok{ bi}
\ImportTok{from}\NormalTok{ sklearn.datasets }\ImportTok{import}\NormalTok{ make\_blobs}

\CommentTok{\# Generate synthetic data}
\NormalTok{data, true\_labels }\OperatorTok{=}\NormalTok{ make\_blobs(}
\NormalTok{    n\_samples}\OperatorTok{=}\DecValTok{500}\NormalTok{, centers}\OperatorTok{=}\DecValTok{8}\NormalTok{, cluster\_std}\OperatorTok{=}\FloatTok{0.8}\NormalTok{,}
\NormalTok{    center\_box}\OperatorTok{=}\NormalTok{(}\OperatorTok{{-}}\DecValTok{10}\NormalTok{,}\DecValTok{10}\NormalTok{), random\_state}\OperatorTok{=}\DecValTok{101}
\NormalTok{)}


\CommentTok{\#  The model}
\KeywordTok{def}\NormalTok{ gmm(data, K, initial\_means): }\CommentTok{\# Here K is the *exact* number of clusters}
\NormalTok{    D }\OperatorTok{=}\NormalTok{ data.shape[}\DecValTok{1}\NormalTok{]  }\CommentTok{\# Number of features}
\NormalTok{    alpha\_prior }\OperatorTok{=} \FloatTok{0.5} \OperatorTok{*}\NormalTok{ jnp.ones(K)}
\NormalTok{    w }\OperatorTok{=}\NormalTok{ dist.dirichlet(concentration}\OperatorTok{=}\NormalTok{alpha\_prior, name}\OperatorTok{=}\StringTok{\textquotesingle{}weights\textquotesingle{}}\NormalTok{) }

    \ControlFlowTok{with}\NormalTok{ dist.plate(}\StringTok{"components"}\NormalTok{, K): }\CommentTok{\# Use fixed K}
\NormalTok{        mu }\OperatorTok{=}\NormalTok{ dist.multivariatenormal(loc}\OperatorTok{=}\NormalTok{initial\_means, covariance\_matrix}\OperatorTok{=}\FloatTok{0.1}\OperatorTok{*}\NormalTok{jnp.eye(D), name}\OperatorTok{=}\StringTok{\textquotesingle{}mu\textquotesingle{}}\NormalTok{)        }
\NormalTok{        sigma }\OperatorTok{=}\NormalTok{ dist.halfcauchy(}\DecValTok{1}\NormalTok{, shape}\OperatorTok{=}\NormalTok{(D,), event}\OperatorTok{=}\DecValTok{1}\NormalTok{, name}\OperatorTok{=}\StringTok{\textquotesingle{}sigma\textquotesingle{}}\NormalTok{)}
\NormalTok{        Lcorr }\OperatorTok{=}\NormalTok{ dist.lkjcholesky(dimension}\OperatorTok{=}\NormalTok{D, concentration}\OperatorTok{=}\FloatTok{1.0}\NormalTok{, name}\OperatorTok{=}\StringTok{\textquotesingle{}Lcorr\textquotesingle{}}\NormalTok{)}

\NormalTok{        scale\_tril }\OperatorTok{=}\NormalTok{ sigma[..., }\VariableTok{None}\NormalTok{] }\OperatorTok{*}\NormalTok{ Lcorr}

\NormalTok{    dist.mixturesamefamily(}
\NormalTok{        mixing\_distribution}\OperatorTok{=}\NormalTok{dist.categorical(probs}\OperatorTok{=}\NormalTok{w, create\_obj}\OperatorTok{=}\VariableTok{True}\NormalTok{),}
\NormalTok{        component\_distribution}\OperatorTok{=}\NormalTok{dist.multivariatenormal(loc}\OperatorTok{=}\NormalTok{mu, scale\_tril}\OperatorTok{=}\NormalTok{scale\_tril, create\_obj}\OperatorTok{=}\VariableTok{True}\NormalTok{),}
\NormalTok{        name}\OperatorTok{=}\StringTok{"obs"}\NormalTok{,}
\NormalTok{        obs}\OperatorTok{=}\NormalTok{data}
\NormalTok{    )}

\NormalTok{m.data\_on\_model }\OperatorTok{=}\NormalTok{ \{}\StringTok{"data"}\NormalTok{: data,}\StringTok{"K"}\NormalTok{: }\DecValTok{4}\NormalTok{ \}}
\NormalTok{m.fit(gmm) }\CommentTok{\# Optimize model parameters through MCMC sampling}
\NormalTok{m.plot(X}\OperatorTok{=}\NormalTok{data,sampler}\OperatorTok{=}\NormalTok{m.sampler) }\CommentTok{\# Prebuild plot function for GMM}
\end{Highlighting}
\end{Shaded}

\subsection{R}

\begin{Shaded}
\begin{Highlighting}[]

\end{Highlighting}
\end{Shaded}

\subsection{Mathematical Details}\label{mathematical-details}

This section describes the generative process for a GMM. \$\$
\pi\_\{{[}1:K{]}\} \sim \text{Dirichlet}(0.5, \dots, 0.5)
\textbackslash{}

\begin{pmatrix}
\mu_{k,1} \\
\vdots \\
\mu_{k,D}
\end{pmatrix}

\sim  \text{MultivariateNormal}!\left(

\begin{pmatrix}
\alpha_{k,1} \\
\vdots \\
\alpha_{k,D}
\end{pmatrix}

, , \Sigma\_k \right) \textbackslash{}

\Sigma\_k = diag()

\%

\$\$

\$\$

\begin{aligned}
&\textbf{Mixture Model:} \\[4pt]
%
&\pi_{1:K} \sim \text{Dirichlet}(0.5, \dots, 0.5) \\[6pt]
%
&\begin{pmatrix}
\mu_{k,1} \\
\vdots \\
\mu_{k,D}
\end{pmatrix}
\sim 
\text{MultivariateNormal}\!\left(
\begin{pmatrix}
\text{initial\_means}_{k,1} \\
\vdots \\
\text{initial\_means}_{k,D}
\end{pmatrix},
\, 0.1\,\mathbf{I}_D
\right) \\
%
&\sigma_{k,1:D} \sim \text{HalfCauchy}(1) \\
%
&L_k \sim \text{LKJCholesky}(D,\, 1.0)\\
%
&\Sigma_k = 
\text{diag}(\sigma_{k,1:D}) \,
L_k L_k^\top \,
\text{diag}(\sigma_{k,1:D}) \\
%
&z_i \sim \text{Categorical}(\pi_{1:K}) \\
%
&\begin{pmatrix}
X_{i,1} \\
\vdots \\
X_{i,D}
\end{pmatrix}
\Bigg| z_i = k
\sim 
\text{MultivariateNormal}\!\left(
\begin{pmatrix}
\mu_{k,1} \\
\vdots \\
\mu_{k,D}
\end{pmatrix},
\, 
\text{diag}(\sigma_{k,1:D}) \,
L_k L_k^\top \,
\text{diag}(\sigma_{k,1:D})
\right) \\
%
&
\begin{pmatrix}
X_{i,1} \\
\vdots \\
X_{i,D}
\end{pmatrix}
\sim 
\sum_{k=1}^{K} \pi_k \,
\mathcal{N}_D\!\left(
\begin{pmatrix}
\mu_{k,1} \\
\vdots \\
\mu_{k,D}
\end{pmatrix},
\,
\Sigma_k
\right)
\end{aligned}

\$\$

\textbf{Parameter Definitions:} * \textbf{Observed Data:} * \(x_i\): The
\(i\)-th observed D-dimensional data point.

\begin{itemize}
\tightlist
\item
  \textbf{Latent Variables (Inferred):}

  \begin{itemize}
  \tightlist
  \item
    \(z_i\): The integer cluster assignment for the \(i\)-th data point.
  \item
    \(w\): The K-dimensional vector of mixture weights.
  \item
    \(\mu_k\): The D-dimensional mean vector of the \(k\)-th cluster.
  \item
    \(\Sigma_k\): The DxD covariance matrix of the \(k\)-th cluster
    (composed of \(\sigma_k\) and \(L_{\text{corr},k}\)).
  \end{itemize}
\item
  \textbf{Hyperparameters (Fixed):}

  \begin{itemize}
  \tightlist
  \item
    \(K\): The total number of clusters.
  \item
    \(\alpha_0\): The concentration parameter vector for the Dirichlet
    prior on weights (e.g., \texttt{{[}1,\ 1,\ ...,\ 1{]}}).
  \item
    \(\mu_0\): The prior mean for the cluster centers.
  \item
    \(\Sigma_0\): The prior covariance for the cluster centers.
  \end{itemize}
\end{itemize}

\subsection{Notes}\label{notes}

\begin{tcolorbox}[enhanced jigsaw, title=\textcolor{quarto-callout-note-color}{\faInfo}\hspace{0.5em}{Note}, toprule=.15mm, breakable, bottomtitle=1mm, opacitybacktitle=0.6, leftrule=.75mm, colframe=quarto-callout-note-color-frame, bottomrule=.15mm, coltitle=black, left=2mm, rightrule=.15mm, toptitle=1mm, titlerule=0mm, arc=.35mm, colbacktitle=quarto-callout-note-color!10!white, opacityback=0, colback=white]

The primary challenge of the GMM compared to the DPMM is the need to
\textbf{manually specify the number of clusters \texttt{K}}. If the
chosen \texttt{K} is too small, the model may merge distinct clusters.
If \texttt{K} is too large, it may split natural clusters into
meaningless sub-groups. Therefore, applying a GMM often involves an
outer loop of model selection where one fits the model for a range of
\texttt{K} values and uses a scoring metric to select the best one.

\end{tcolorbox}

\subsection{Reference(s)}\label{references}

C. M. Bishop (2006). \emph{Pattern Recognition and Machine Learning}.
Springer. (Chapter 9)




\end{document}
