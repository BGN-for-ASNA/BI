% Options for packages loaded elsewhere
\PassOptionsToPackage{unicode}{hyperref}
\PassOptionsToPackage{hyphens}{url}
\PassOptionsToPackage{dvipsnames,svgnames,x11names}{xcolor}
%
\documentclass[
]{article}

\usepackage{amsmath,amssymb}
\usepackage{iftex}
\ifPDFTeX
  \usepackage[T1]{fontenc}
  \usepackage[utf8]{inputenc}
  \usepackage{textcomp} % provide euro and other symbols
\else % if luatex or xetex
  \usepackage{unicode-math}
  \defaultfontfeatures{Scale=MatchLowercase}
  \defaultfontfeatures[\rmfamily]{Ligatures=TeX,Scale=1}
\fi
\usepackage{lmodern}
\ifPDFTeX\else  
    % xetex/luatex font selection
\fi
% Use upquote if available, for straight quotes in verbatim environments
\IfFileExists{upquote.sty}{\usepackage{upquote}}{}
\IfFileExists{microtype.sty}{% use microtype if available
  \usepackage[]{microtype}
  \UseMicrotypeSet[protrusion]{basicmath} % disable protrusion for tt fonts
}{}
\makeatletter
\@ifundefined{KOMAClassName}{% if non-KOMA class
  \IfFileExists{parskip.sty}{%
    \usepackage{parskip}
  }{% else
    \setlength{\parindent}{0pt}
    \setlength{\parskip}{6pt plus 2pt minus 1pt}}
}{% if KOMA class
  \KOMAoptions{parskip=half}}
\makeatother
\usepackage{xcolor}
\setlength{\emergencystretch}{3em} % prevent overfull lines
\setcounter{secnumdepth}{5}
% Make \paragraph and \subparagraph free-standing
\makeatletter
\ifx\paragraph\undefined\else
  \let\oldparagraph\paragraph
  \renewcommand{\paragraph}{
    \@ifstar
      \xxxParagraphStar
      \xxxParagraphNoStar
  }
  \newcommand{\xxxParagraphStar}[1]{\oldparagraph*{#1}\mbox{}}
  \newcommand{\xxxParagraphNoStar}[1]{\oldparagraph{#1}\mbox{}}
\fi
\ifx\subparagraph\undefined\else
  \let\oldsubparagraph\subparagraph
  \renewcommand{\subparagraph}{
    \@ifstar
      \xxxSubParagraphStar
      \xxxSubParagraphNoStar
  }
  \newcommand{\xxxSubParagraphStar}[1]{\oldsubparagraph*{#1}\mbox{}}
  \newcommand{\xxxSubParagraphNoStar}[1]{\oldsubparagraph{#1}\mbox{}}
\fi
\makeatother


\providecommand{\tightlist}{%
  \setlength{\itemsep}{0pt}\setlength{\parskip}{0pt}}\usepackage{longtable,booktabs,array}
\usepackage{calc} % for calculating minipage widths
% Correct order of tables after \paragraph or \subparagraph
\usepackage{etoolbox}
\makeatletter
\patchcmd\longtable{\par}{\if@noskipsec\mbox{}\fi\par}{}{}
\makeatother
% Allow footnotes in longtable head/foot
\IfFileExists{footnotehyper.sty}{\usepackage{footnotehyper}}{\usepackage{footnote}}
\makesavenoteenv{longtable}
\usepackage{graphicx}
\makeatletter
\def\maxwidth{\ifdim\Gin@nat@width>\linewidth\linewidth\else\Gin@nat@width\fi}
\def\maxheight{\ifdim\Gin@nat@height>\textheight\textheight\else\Gin@nat@height\fi}
\makeatother
% Scale images if necessary, so that they will not overflow the page
% margins by default, and it is still possible to overwrite the defaults
% using explicit options in \includegraphics[width, height, ...]{}
\setkeys{Gin}{width=\maxwidth,height=\maxheight,keepaspectratio}
% Set default figure placement to htbp
\makeatletter
\def\fps@figure{htbp}
\makeatother

\usepackage{arxiv}
\usepackage{orcidlink}
\usepackage{amsmath}
\usepackage[T1]{fontenc}
\makeatletter
\@ifpackageloaded{caption}{}{\usepackage{caption}}
\AtBeginDocument{%
\ifdefined\contentsname
  \renewcommand*\contentsname{Table of contents}
\else
  \newcommand\contentsname{Table of contents}
\fi
\ifdefined\listfigurename
  \renewcommand*\listfigurename{List of Figures}
\else
  \newcommand\listfigurename{List of Figures}
\fi
\ifdefined\listtablename
  \renewcommand*\listtablename{List of Tables}
\else
  \newcommand\listtablename{List of Tables}
\fi
\ifdefined\figurename
  \renewcommand*\figurename{Figure}
\else
  \newcommand\figurename{Figure}
\fi
\ifdefined\tablename
  \renewcommand*\tablename{Table}
\else
  \newcommand\tablename{Table}
\fi
}
\@ifpackageloaded{float}{}{\usepackage{float}}
\floatstyle{ruled}
\@ifundefined{c@chapter}{\newfloat{codelisting}{h}{lop}}{\newfloat{codelisting}{h}{lop}[chapter]}
\floatname{codelisting}{Listing}
\newcommand*\listoflistings{\listof{codelisting}{List of Listings}}
\makeatother
\makeatletter
\makeatother
\makeatletter
\@ifpackageloaded{caption}{}{\usepackage{caption}}
\@ifpackageloaded{subcaption}{}{\usepackage{subcaption}}
\makeatother

\ifLuaTeX
  \usepackage{selnolig}  % disable illegal ligatures
\fi
\usepackage{bookmark}

\IfFileExists{xurl.sty}{\usepackage{xurl}}{} % add URL line breaks if available
\urlstyle{same} % disable monospaced font for URLs
\hypersetup{
  pdftitle={Manuscript BI},
  colorlinks=true,
  linkcolor={blue},
  filecolor={Maroon},
  citecolor={Blue},
  urlcolor={Blue},
  pdfcreator={LaTeX via pandoc}}


\newcommand{\runninghead}{A Preprint }
\title{Manuscript BI}
\def\asep{\\\\\\ } % default: all authors on same column
\author{}
\date{}
\begin{document}
\maketitle


\section{Title}\label{title}

\subsubsection{Abstract}\label{abstract}

\subsection{Introduction}\label{introduction}

Bayesian modeling has emerged as a vital tool in modern statistics and
machine learning, providing a framework for robust inference under
uncertainty. Despite its potential, the current ecosystem of Bayesian
software presents significant challenges for researchers, particularly
due to limited formal training in Bayesian methods and model
development, coding languages diversity, scalability, and the
specialized nature of many Bayesian frameworks.

One of the key obstacles in Bayesian modeling is the limited formal
training many researchers receive in Bayesian methods and model
development. Although Bayesian inference has gained traction, the
complexity of model formulation, coupled with the need for a deeper
understanding of probabilistic frameworks, can be overwhelming. The gap
between theoretical knowledge and practical application often results in
users defaulting to more simplistic models that may not fully leverage
the power of Bayesian analysis. For instance, while accessible
frameworks like BRMS exist, they are typically constrained to linear
regression models with flat priors. This reflects the predominance of
linear regression in many fields, where statistical analysis often
relies heavily on techniques that are commonly emphasized in educational
settings. As a result, the potential of Bayesian methods remains
underutilized in more complex, real-world applications.

Another significant barrier to the adoption of Bayesian tools is the
diversity of coding languages used across various frameworks.
Researchers often face the challenge of learning multiple programming
languages, each associated with a specific Bayesian tool, which can
complicate the model development process and create inefficiencies. For
example, while Stan remains a gold standard in Bayesian inference due to
its highly optimized sampling algorithms and broad support for various
models, it requires users to learn and write models in its own
specialized syntax (Stan code). For researchers accustomed to working in
Python or R, the need to switch to a different programming environment
can be a considerable hurdle. Additionally, Stan faces challenges
related to scaling, particularly in the parallelization of operations
and the utilization of GPU resources, which can limit its efficiency
when handling large datasets or complex models. For researchers
accustomed to working in Python or R, the need to switch to a different
programming environment can be a considerable hurdle. This added
cognitive load makes the modeling process more labor-intensive,
particularly for those without extensive expertise in multiple
programming languages. As a result, users may be discouraged from fully
exploring the capabilities of Bayesian methods due to the friction
introduced by these language barriers.

Many Bayesian modeling tools are specialized for specific types of
models or applications, which can limit their flexibility and broader
utility. Although these frameworks often excel in their respective
domains, their specialization can introduce barriers for users looking
to apply Bayesian inference to a wider array of problems. For example,
TensorFlow Probability (TFP) provides a robust framework for
probabilistic modeling within the TensorFlow ecosystem. However, a
significant challenge arises from the requirement that users frequently
need to build their own Bayesian samplers---algorithms such as
Metropolis-Hastings or Hamiltonian Monte Carlo, which are critical for
generating samples from the posterior distribution. This complexity,
combined with the steep learning curve of the TensorFlow infrastructure,
can deter researchers who are not already familiar with the system's
intricacies.

Similarly, specialized tools like STRAND and BISON highlight the
trade-offs inherent in current Bayesian software. STRAND and BISON
excels at modeling network, yet its narrow focus can result in limited
flexibility when adapting to more general Bayesian modeling needs such
as computing network metrics within the same model. And as both are
based on STAN, they also face challenges related to scalability and
parallelization, which can limit their efficiency when handling large
datasets which in the context of network can arrive at the order of
magnitude of the number of nodes.

Given these limitations, there is a clear need for a more user-friendly,
scalable, and versatile Bayesian modeling tool that can address the gaps
in the current software landscape. Our proposed Python software aims to
meet this need by providing a platform that combines the power of
Bayesian inference with greater accessibility and efficiency, making it
an ideal choice for both novice and experienced users. This software
leverages the computational efficiency of JAX by using either TensorFlow
Probability (TFP) or NumPyro as its backbone structure, while
simplifying the process of model specification and retaining their
flexibility and power (including parallelization of operations and the
utilization of GPU resources).

\subsection{Software Presentation}\label{software-presentation}

Our Bayesian Inference (BI) software comes in two versions: one relying
on NumPyro and the other on TFP. We have simplified the process of model
specification by providing a more user-friendly interface for defining
Bayesian models, making it accessible to both novice users and
experienced practitioners. Since both NumPyro and TFP rely on JAX, we
inherit the advantages of JAX, including speed, parallelization,
scalability, and GPU support. The user code remains consistent for
running parallelized mathematical operations and utilizing either GPU or
CPU. From start to finish, users can employ a class object to: 1) handle
data, 2) define models, 3) specify inference algorithms, 4) run
inference, 5) compute posterior distributions, 6) compute model
predictions, and 7) plot results. Additionally, as one of the major
issue in the research area is the lack of training on bayesian modeling,
we present 20 different models to cover wide range of research
questions. All presented models are accompanied by detailed
explanations, making it easier for users to understand the underlying
assumptions and apply the models to their specific research questions.

\subsubsection{User-Friendly Interface for Defining Bayesian
Models}\label{user-friendly-interface-for-defining-bayesian-models}

A primary feature of our Bayesian Inference (BI) software is its
user-friendly interface designed for defining Bayesian models. We
understand that Bayesian modeling can be daunting, especially for those
who may lack formal training in the subject. To mitigate this, our
interface offers a simplified version of the NumPyro and TFP functions.
Typically, to define a parameter in a model, both libraries require
three different functions. We have merged these functions into one,
along with the sampling functions that are separated in NumPyro and TFP.
The code below illustrates the differences between NumPyro, TFP, and BI
for declaring a parameter of length 10 that follows a Normal
distribution within a model.

\begin{verbatim}
numpyro.sample("mu", dist.Normal(0, 1)).expand([10])

yield Root(tfd.Sample(tfd.Normal(loc=1.0, scale=1.0), sample_shape=10))

bi.dist.normal("mu", 0, 1, shape = (10,))
\end{verbatim}

Additionally, while NumPyro and TFP provide different functions to
sample a parameter compared to those used to declare it, we have
combined these into the same function. The code below demonstrates the
differences between NumPyro, TFP, and BI for sampling a parameter of
length 10 that follows a Normal distribution outside of a model.

\begin{verbatim}
numpyro.sample('samples', dist.Normal(0, 1).expand([10]), rng_key=jax.random.PRNGKey(0))

tfp.distributions.Normal(loc=0., scale=1.).sample(10)

bi.dist.normal("mu", 0, 1, shape = (10,), sample = True)
\end{verbatim}

By combining functions to declare parameters within a model and those
used to sample parameters, we allow users to easily test their models by
first building them with the sampling options turned on. This approach
enables users to quickly see the shapes of the objects obtained and
better manage and combine model parameters. Finally, we include several
custom functions tailored for advanced statistical and network modeling.
These functions support specialized tasks such as centered random
effects, block modeling, and the computation of network measures. The
code below demonstrates the differences between NumPyro, TFP, and BI for
declaring a random center effect in a model:

\begin{verbatim}
# Numpyro version of rendom centered effect--------------------------------------------------
    a = numpyro.sample("a", dist.Normal(5, 2))
        b = numpyro.sample("b", dist.Normal(-1, 0.5))
        sigma_cafe = numpyro.sample("sigma_cafe", dist.Exponential(1).expand([2]))
        sigma = numpyro.sample("sigma", dist.Exponential(1))
        Rho = numpyro.sample("Rho", dist.LKJ(2, 2))
        cov = jnp.outer(sigma_cafe, sigma_cafe) * Rho
        a_cafe_b_cafe = numpyro.sample(
            "a_cafe,b_cafe", dist.MultivariateNormal(jnp.stack([a, b]), cov).expand([20])
        )
    a_cafe, b_cafe = a_cafe_b_cafe[:, 0], a_cafe_b_cafe[:, 1]

# TFP version of rendom centered effect--------------------------------------------------
    alpha = yield Root(tfd.Sample(tfd.Normal(loc=5.0, scale=2.0), sample_shape=1))
    beta = yield Root(tfd.Sample(tfd.Normal(loc=-1.0, scale=0.5), sample_shape=1))

    sigma = yield Root(tfd.Sample(tfd.Exponential(rate=1.0), sample_shape=1))
    sigma_alpha_beta = yield Root(
        tfd.Sample(tfd.Exponential(rate=1.0), sample_shape=2)
    )

    Rho = yield Root(tfd.LKJ(dimension=2, concentration=2.0))
    Mu = tf.concat([alpha, beta], axis=-1)
    scale = tf.linalg.LinearOperatorDiag(sigma_alpha_beta).matmul(tf.squeeze(Rho))

# BI version of rendom centered effect--------------------------------------------------
    Sigma_individual = bi.exponential('Sigma_individual', [ni], 1 )
    L_individual = bi.lkjcholesky('L_individual', [], ni, 1) 
    z_individual = bi.normal('z_individual', [ni,K], 0, 1)
    alpha = random_centered2(Sigma_individual, L_individual, z_individual)
\end{verbatim}

By generating these custom functions, we eliminate some data
manipulation steps for users, allowing them to more easily adhere to the
model's mathematical formula. All custom functions are discussed in
their respective model descriptions within the BI documentation.

Finally, The architecture of our software is built around a versatile
class object that encapsulates the entire modeling process. The code
below demonstrates how user can 1) handle data, 2) define models, 3)
specify inference algorithms, 4) run inference, 5) compute posterior
distributions, 6) compute model predictions, and 7) plot results from a
linear regression.

\begin{verbatim}
# Setup device & call BI object------------------------------------------------
m = bi(platform='cpu')

# Import data ------------------------------------------------
m.data('../data/Howell1.csv', sep=';') 
m.df = m.df[m.df.age > 18]
m.scale(['weight']) # Sacle continuous data
m.data_to_model(['weight', 'height']) # Cnnvert data to JAX arrays

# Define model ------------------------------------------------
def model(height, weight):    
    alpha = dist.normal( 178, 20, name = 'alpha')
    beta = dist.normal(  0, 1, name = 'beta')   
    sigma = dist.uniform( 0, 50, name = 'sigma')
    lk("Y", Normal(alpha + beta * weight , sigma), obs = height)

# Run mcmc ------------------------------------------------
m.run(model) 

# Summary ------------------------------------------------
m.sampler.print_summary(0.89)

# Plot posterior distributions ------------------------------------------------ 

#TODO
\end{verbatim}

This cohesive structure allows users to efficiently manage different
stages of their analysis, from data handling to model evaluation without
the need to switch from on library to another. BI laverage data handling
through pandas library and postrior evaluation througth library Arviz.

\subsubsection{Comprehensive Documentation of
Models}\label{comprehensive-documentation-of-models}

To further support users in their Bayesian modeling journey, our BI
software provides comprehensive documentation for 20 different models
specifically designed to address a wide array of research questions.
This extensive repertoire not only equips users with essential tools for
tackling diverse analytical challenges but also serves as a valuable
educational resource. By presenting various modeling options alongside
detailed guidance, we empower users to make informed decisions and
effectively apply Bayesian methods in their research. The documentation
for each model encompasses 1) general principles, 2) underlying
assumptions, 3) code snippets, and 4) mathematical details, enabling
users to gain a deeper understanding of the modeling process and its
nuances. Whether users are interested in hierarchical models,
time-series analysis, or cutting-edge network modeling approaches, our
library caters to a variety of analytical needs. This accessibility
fosters an environment where users can confidently explore and implement
Bayesian methods, ultimately enhancing their research capabilities.

\textbf{Table 1}: Table of documented models.

\begin{longtable}[]{@{}l@{}}
\toprule\noalign{}
Documented Models \\
\midrule\noalign{}
\endhead
\bottomrule\noalign{}
\endlastfoot
Linear Regression for continuous variable \\
Multiple continuous Variables.qmd \\
Interaction between continuous variables \\
Categorical variable \\
Binomial model \\
Beta binomial model \\
Poisson model \\
Gamma-Poisson \\
Multinomial model \\
Dirichlet model \\
Zero inflated \\
Varying intercepts \\
Varying slopes \\
Gaussian processes \\
Measuring error \\
Missing data \\
Network model \\
Sender receiver network model \\
Block model network \\
Network Computations \\
Network Based diffusion approach \\
Multiplex network model \\
Multiplex temporal network model \\
Multilayer network model \\
\end{longtable}

For example, the state-of-the-art network models implemented in BI
include methods for censoring and exposure control, block modeling,
multiplex and multilayer networks, network-based diffusion approaches,
and computation of network metrics. Each approach is presented in detail
in the documentation, complete with their custom functions. This
comprehensive documentation ensures that users can easily navigate the
software and identify the models that best suit their research
requirements. Furthermore, these diverse approaches can be combined to
create a single model, such as a multiplex model with controls for
censoring and exposure biases, along with the computation of social
network measures (code snippets bellow). This versatility showcases the
flexibility and power of the software, allowing users to tailor their
analyses to meet specific research needs, which is not possible with
other specialized Bayesian network frameworks such as STRAND and BISON.
These frameworks feature prebuilt models that either do not allow for
direct computation of network measures within the model (BISON requires
two separate models) or do not support network measure computation at
all (STRAND).

\begin{verbatim}
# Setup device & call BI object------------------------------------------------
m = bi(platform='cpu')

# Import data ------------------------------------------------
m.data('../data/Howell1.csv', sep=';') 
m.df = m.df[m.df.age > 18]
m.scale(['weight']) # Sacle continuous data
m.data_to_model(['weight', 'height']) # Cnnvert data to JAX arrays

# Define model ------------------------------------------------
def model(height, weight):    
    alpha = dist.normal( 178, 20, name = 'alpha')
    beta = dist.normal(  0, 1, name = 'beta')   
    sigma = dist.uniform( 0, 50, name = 'sigma')
    lk("Y", Normal(alpha + beta * weight , sigma), obs = height)

# Run mcmc ------------------------------------------------
m.run(model) 

# Summary ------------------------------------------------
m.sampler.print_summary(0.89)

# Plot posterior distributions ------------------------------------------------ 

#TODO
\end{verbatim}

\subsection{Discussion}\label{discussion}

The XXX framework is built on top of the popular Python programming
language, with a focus on providing a user-friendly interface for model
development and interpretation. Our framework is designed to be modular
and extensible, allowing users to easily incorporate their own custom
models and data types into the framework. We have also developed a set
of tutorials and examples to help users get started with the framework
and to demonstrate its capabilities.

One of the key features of this software is its comprehensive library of
16 predefined Bayesian models, covering a wide range of common
applications and use cases. These models are accompanied by detailed
explanations, making it easier for users to understand the underlying
assumptions and apply the models to their specific research questions.
In addition to these built-in models, the software includes several
custom functions tailored for advanced statistical and network modeling.
These functions support specialized tasks such as centered random
effects, block modeling, and the computation of network measures. \#\#
Aknowledgements \#\# References




\end{document}
